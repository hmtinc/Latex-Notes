%Notes by Harsh Mistry 
%CS 245
%based on Template from : https://www.cs.cmu.edu/~ggordon/10725-F12/template.tex

\documentclass{article}
\setlength{\oddsidemargin}{0.25 in}
\setlength{\evensidemargin}{-0.25 in}
\setlength{\topmargin}{-0.6 in}
\setlength{\textwidth}{6.5 in}
\setlength{\textheight}{8.5 in}
\setlength{\headsep}{0.75 in}
\setlength{\parindent}{0 in}
\setlength{\parskip}{0.1 in}
\usepackage{amsfonts,graphicx, amssymb}
\usepackage[fleqn]{amsmath}
\usepackage{fixltx2e}
\usepackage{xcolor}
\usepackage{color}
\usepackage{tcolorbox}
\usepackage{lipsum}
\usepackage{listings}
\usepackage{scrextend}
\usepackage{enumerate}
\tcbuselibrary{skins,breakable}
\usetikzlibrary{shadings,shadows}
\newcounter{lecnum}
\renewcommand{\thepage}{\thelecnum-\arabic{page}}
\renewcommand{\thesection}{\thelecnum.\arabic{section}}
\renewcommand{\theequation}{\thelecnum.\arabic{equation}}
\renewcommand{\thefigure}{\thelecnum.\arabic{figure}}
\renewcommand{\thetable}{\thelecnum.\arabic{table}}
\newcommand{\lecture}[4]{
   \pagestyle{myheadings}
   \thispagestyle{plain}
   \newpage
   \setcounter{lecnum}{#1}
   \setcounter{page}{1}
   
   
%Info Box 
   \begin{center}
   \framebox{
      \vbox{\vspace{2mm}
    \hbox to 6.28in { {\bf CS 245 - Logic and Computation 
	\hfill Fall 2016} }
       \vspace{4mm}
       \hbox to 6.28in { {\Large \hfill Lecture #1: #2  \hfill} }
       \vspace{2mm}
       \hbox to 6.28in { {\it Lecturer: #3 \hfill Notes By: #4} }
      \vspace{2mm}}
   }
   \end{center}
   
   \markboth{Lecture #1: #2}{Lecture #1: #2}



 
}

\renewcommand{\cite}[1]{[#1]}
\def\beginrefs{\begin{list}%
        {[\arabic{equation}]}{\usecounter{equation}
         \setlength{\leftmargin}{2.0truecm}\setlength{\labelsep}{0.4truecm}%
         \setlength{\labelwidth}{1.6truecm}}}
\def\endrefs{\end{list}}
\def\bibentry#1{\item[\hbox{[#1]}]}

\newcommand{\fig}[3]{
			\vspace{#2}
			\begin{center}
			Figure \thelecnum.#1:~#3
			\end{center}
	}

\newtheorem{theorem}{Theorem}[lecnum]
\newtheorem{lemma}[theorem]{Lemma}
\newtheorem{ex}[theorem]{Example}
\newtheorem{proposition}[theorem]{Proposition}
\newtheorem{claim}[theorem]{Claim}
\newtheorem{corollary}[theorem]{Corollary}
\newtheorem{definition}[theorem]{Definition}
\newenvironment{proof}{{\bf Proof:}}{\hfill\rule{2mm}{2mm}}
\newcommand\E{\mathbb{E}}

%color definitions :
\definecolor{darkred}{rgb}{0.55, 0.0, 0.0}
\definecolor{lightcoral}{rgb}{0.94, 0.5, 0.5}
\definecolor{tomato}{rgb}{1.0, 0.39, 0.28}
\definecolor{lightgray}{rgb}{.9,.9,.9}
\definecolor{darkgray}{rgb}{.4,.4,.4}
\definecolor{purple}{rgb}{0.65, 0.12, 0.82}
\definecolor{lightgreen}{rgb}{0.56, 0.93, 0.56}
\definecolor{darkgreen}{rgb}{0.0, 0.2, 0.13}
\definecolor{limegreen}{rgb}{0.2, 0.8, 0.2}
\definecolor{lightblue}{rgb}{0.68, 0.85, 0.9}
\definecolor{darkblue}{rgb}{0.0, 0.0, 0.55}


%Environments
\newenvironment{exblock}[1]{%
    \tcolorbox[beamer,%
    noparskip,breakable,
    colback=lightgreen,colframe=darkgreen,%
    colbacklower=limegreen!75!lightgreen,%
    title=#1]}%
    {\endtcolorbox}

\newenvironment{ablock}[1]{%
    \tcolorbox[beamer,%
    noparskip,breakable,
    colback=lightcoral,colframe=darkred,%
    colbacklower=tomato!75!lightcoral,%
    title=#1]}%
    {\endtcolorbox}

\newenvironment{cblock}[1]{%
    \tcolorbox[beamer,%
    noparskip,breakable,
    colback=lightblue,colframe=darkblue,%
    colbacklower=darkblue!75!lightblue,%
    title=#1]}%
    {\endtcolorbox}


%Languages
\lstdefinelanguage{JavaScript}{
  keywords={typeof, new, true, false, catch, function, return, null, catch, switch, var, if, in, while, do, else, case, break},
  keywordstyle=\color{blue}\bfseries,
  ndkeywords={class, export, boolean, throw, implements, import, this},
  ndkeywordstyle=\color{darkgray}\bfseries,
  identifierstyle=\color{black},
  sensitive=false,
  comment=[l]{//},
  morecomment=[s]{/*}{*/},
  commentstyle=\color{purple}\ttfamily,
  stringstyle=\color{red}\ttfamily,
  morestring=[b]',
  morestring=[b]"
}

%Listings
\lstset{
   language=JavaScript,
   backgroundcolor=\color{lightgray},
   extendedchars=true,
   basicstyle=\footnotesize\ttfamily,
   showstringspaces=false,
   showspaces=false,
   numbers=left,
   numberstyle=\footnotesize,
   numbersep=9pt,
   tabsize=2,
   breaklines=true,
   showtabs=false,
   captionpos=b
}


%Start of Document 
\begin{document}

\lecture{15 - 18}{November 1 - 10, 2016}{Kevin Lanctot}{Harsh Mistry}
\section{Axioms}
\begin{definition}
A \textbf{Axiom} is a formula that is assumed as a premise in any proof. An \textbf{Axiom Schema} is a set of axioms defined by a pattern or rule.
Axioms Often behave like additional inference rules 
\end{definition}

\subsection{Peano Axioms}
Fix the domain as \(\mathbb{N}\), the natural numbers. Interpret the constant symbol 0 as zerp and the unary function symbol s as success. \(s(x) \rightarrow x + 1\)

Thus each number in \(\mathbb{N}\) has a term : \(0, s(0), s(s(0)) , \ldots \)

Zero and successor satisfy the following axioms 
\begin{itemize}
\item PA1 : \(\forall x s(x) \neq 0 \), "zero is not a successor"
\item PA2 : \(\forall x \forall y ((s(x) = s(y) \rightarrow x = y ).\), "nothing has two predecessors 
\end{itemize}

\textbf{Addition and Multiplication Axioms : }
\begin{itemize}
\item PA3 : \(\forall x (x + 0 = x) \), Adding zero to any number yields the same number
\item PA4 : \(\forall x \forall y (x + s(y) = s(x + y))\), Adding a successor yields the successor of adding the number 
\item PA5 : \(\forall (x \times 0) = 0\), multiplying by zero yields zero
\item PA6 : \(\forall x \forall y ( x \times s(y) = x \times y + x) \)
\end{itemize}

\textbf{Induction Axiom :}
\begin{itemize}
\item PA7 : For each formula \(\varphi\) and variable x 
\[ \varphi[0/x] \rightarrow (\forall x (\varphi \rightarrow \varphi[s(x)/x]) \rightarrow \forall x \varphi) \]
\end{itemize}
\subsection{Properties of Peano Axioms}
\textit{The Peano Axioms imply all of the familiar properties of the natural numbers}

\begin{theorem} \(\vdash_{PA} \forall x \forall y ( x + y = y + x ) \)
\end{theorem}

\begin{lemma} \(\forall y (x + y = y + x) \vdash_{PA} \forall y ( s(x) + y = y + s(x))\)
\end{lemma}

\subsection{Definability}
Let the formula \(\varphi\) have free variables \(x_1 \ldots x_k\).

Given an interpretation I, a formula \(\varphi\) defined the k-ary relation of tuples that make \(\varphi\) true - that is, the relation 
$$\{ \langle a_1 \ldots a_k \rangle \in dom(I) \mid \varphi^{(I, \theta [x_1 \rightarrow a_1] \ldots [x_k \rightarrow a_k]))} = T$$

A relation R is definable (In I) if and only if \(R = R_\varphi\) for some formula \(\varphi\)

\subsubsection{Properties of Defined Relations}
The PA axioms allow one to show that the defined relation \(\leq\) has the usual properties 
\begin{itemize}
\item \(x \leq y\) and \(y \leq z\) imply \(x \leq z\) (Transitivity)
\item If \(x \leq y \) and \(y \leq x \) then \(x = y\) 
\end{itemize}

\section{Lists}

\textbf{Vocabulary for Lists :}
\begin{itemize}
\item a constant symbol \textbf{e}, represents empty list 
\item a binary function symbol \textbf{cons}, connects two lists
\end{itemize}

\textbf{Short-hand Notation :}
\begin{itemize}
\item \(\langle \rangle\) denotes the empty list e 
\item \(\langle a \rangle \) denotes a list with a single item a
\item If \(\langle \gamma \rangle\) denotes a non-empty list. Then \(\langle a, \gamma \rangle\) denotes a list with a and the rest of the items in \(\gamma\). 
\end{itemize}

\subsection{Axioms of Basic Lists}
\begin{itemize}
\item List 1 : \(\forall x \forall y cons (x, y) \neq e \)
\item List 2 : \(\forall x \forall y \forall z \forall w (cons(x,y) = cons(z, w) \rightarrow (x = z \wedge y = w ))\)
\item List 3 : For each formula \(\varphi (x) \) and each variable y not free in \(\varphi\)
\[\varphi[e/x] \rightarrow (\forall x (\varphi \rightarrow \forall y \varphi [cons (y,x) / x ] ) \rightarrow \forall x \varphi) \]
\end{itemize}


\end{document}