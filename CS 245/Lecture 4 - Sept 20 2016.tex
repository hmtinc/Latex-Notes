%Notes by Harsh Mistry 
%CS 245
%based on Template from : https://www.cs.cmu.edu/~ggordon/10725-F12/template.tex

\documentclass{article}
\setlength{\oddsidemargin}{0.25 in}
\setlength{\evensidemargin}{-0.25 in}
\setlength{\topmargin}{-0.6 in}
\setlength{\textwidth}{6.5 in}
\setlength{\textheight}{8.5 in}
\setlength{\headsep}{0.75 in}
\setlength{\parindent}{0 in}
\setlength{\parskip}{0.1 in}
\usepackage{amsfonts,graphicx, amssymb}
\usepackage[fleqn]{amsmath}
\usepackage{fixltx2e}
\usepackage{xcolor}
\usepackage{color}
\usepackage{tcolorbox}
\usepackage{lipsum}
\usepackage{listings}
\usepackage{scrextend}
\usepackage{enumerate}
\tcbuselibrary{skins,breakable}
\usetikzlibrary{shadings,shadows}
\newcounter{lecnum}
\renewcommand{\thepage}{\thelecnum-\arabic{page}}
\renewcommand{\thesection}{\thelecnum.\arabic{section}}
\renewcommand{\theequation}{\thelecnum.\arabic{equation}}
\renewcommand{\thefigure}{\thelecnum.\arabic{figure}}
\renewcommand{\thetable}{\thelecnum.\arabic{table}}
\newcommand{\lecture}[4]{
   \pagestyle{myheadings}
   \thispagestyle{plain}
   \newpage
   \setcounter{lecnum}{#1}
   \setcounter{page}{1}
   
   
%Info Box 
   \begin{center}
   \framebox{
      \vbox{\vspace{2mm}
    \hbox to 6.28in { {\bf CS 245 - Logic and Computation 
	\hfill Fall 2016} }
       \vspace{4mm}
       \hbox to 6.28in { {\Large \hfill Lecture #1: #2  \hfill} }
       \vspace{2mm}
       \hbox to 6.28in { {\it Lecturer: #3 \hfill Notes By: #4} }
      \vspace{2mm}}
   }
   \end{center}
   
   \markboth{Lecture #1: #2}{Lecture #1: #2}



 
}

\renewcommand{\cite}[1]{[#1]}
\def\beginrefs{\begin{list}%
        {[\arabic{equation}]}{\usecounter{equation}
         \setlength{\leftmargin}{2.0truecm}\setlength{\labelsep}{0.4truecm}%
         \setlength{\labelwidth}{1.6truecm}}}
\def\endrefs{\end{list}}
\def\bibentry#1{\item[\hbox{[#1]}]}

\newcommand{\fig}[3]{
			\vspace{#2}
			\begin{center}
			Figure \thelecnum.#1:~#3
			\end{center}
	}

\newtheorem{theorem}{Theorem}[lecnum]
\newtheorem{lemma}[theorem]{Lemma}
\newtheorem{ex}[theorem]{Example}
\newtheorem{proposition}[theorem]{Proposition}
\newtheorem{claim}[theorem]{Claim}
\newtheorem{corollary}[theorem]{Corollary}
\newtheorem{definition}[theorem]{Definition}
\newenvironment{proof}{{\bf Proof:}}{\hfill\rule{2mm}{2mm}}
\newcommand\E{\mathbb{E}}

%color definitions :
\definecolor{darkred}{rgb}{0.55, 0.0, 0.0}
\definecolor{lightcoral}{rgb}{0.94, 0.5, 0.5}
\definecolor{tomato}{rgb}{1.0, 0.39, 0.28}
\definecolor{lightgray}{rgb}{.9,.9,.9}
\definecolor{darkgray}{rgb}{.4,.4,.4}
\definecolor{purple}{rgb}{0.65, 0.12, 0.82}
\definecolor{lightgreen}{rgb}{0.56, 0.93, 0.56}
\definecolor{darkgreen}{rgb}{0.0, 0.2, 0.13}
\definecolor{limegreen}{rgb}{0.2, 0.8, 0.2}
\definecolor{lightblue}{rgb}{0.68, 0.85, 0.9}
\definecolor{darkblue}{rgb}{0.0, 0.0, 0.55}


%Environments
\newenvironment{exblock}[1]{%
    \tcolorbox[beamer,%
    noparskip,breakable,
    colback=lightgreen,colframe=darkgreen,%
    colbacklower=limegreen!75!lightgreen,%
    title=#1]}%
    {\endtcolorbox}

\newenvironment{ablock}[1]{%
    \tcolorbox[beamer,%
    noparskip,breakable,
    colback=lightcoral,colframe=darkred,%
    colbacklower=tomato!75!lightcoral,%
    title=#1]}%
    {\endtcolorbox}

\newenvironment{cblock}[1]{%
    \tcolorbox[beamer,%
    noparskip,breakable,
    colback=lightblue,colframe=darkblue,%
    colbacklower=darkblue!75!lightblue,%
    title=#1]}%
    {\endtcolorbox}


%Languages
\lstdefinelanguage{JavaScript}{
  keywords={typeof, new, true, false, catch, function, return, null, catch, switch, var, if, in, while, do, else, case, break},
  keywordstyle=\color{blue}\bfseries,
  ndkeywords={class, export, boolean, throw, implements, import, this},
  ndkeywordstyle=\color{darkgray}\bfseries,
  identifierstyle=\color{black},
  sensitive=false,
  comment=[l]{//},
  morecomment=[s]{/*}{*/},
  commentstyle=\color{purple}\ttfamily,
  stringstyle=\color{red}\ttfamily,
  morestring=[b]',
  morestring=[b]"
}

%Listings
\lstset{
   language=JavaScript,
   backgroundcolor=\color{lightgray},
   extendedchars=true,
   basicstyle=\footnotesize\ttfamily,
   showstringspaces=false,
   showspaces=false,
   numbers=left,
   numberstyle=\footnotesize,
   numbersep=9pt,
   tabsize=2,
   breaklines=true,
   showtabs=false,
   captionpos=b
}


%Start of Document 
\begin{document}

\lecture{4}{September 20, 2016}{Kevin Lanctot}{Harsh Mistry}

\section{Classifying Formulas}
\begin{itemize}
\item A formula \(\alpha\) is a tautology if any only if for every truth valuation t, \(\alpha^{t} = T\) 
\begin{itemize}
\item \((p \vee (\neg p))\) is a tautology 
\end{itemize}

\item A formula \(\alpha\) is a contradiction if and only if for every truth valuation t, \(\alpha^{t} = T\)
\begin{itemize}
\item \((p \wedge (\neg P)\) is a contradiction
\end{itemize}

\item A formula \(\alpha\) is satisfiable if and only if there is some truth valuation t such that \(\alpha^t = T\) 
\begin{itemize}
\item \((p \implies q)\) is satisfiable if you set both variables to T.
\end{itemize}
\end{itemize}

\textbf{Note : } A formula is satisfiable if and only if it is not a contradiction

\section{"Short-Cutting" a Truth Table}
Instead of filling an entire truth table, we can observer what happens if we set a variable to T or F in order to simply the formula. We can use this to evaluate formulas, be creating a valuation tree

\section{Equivalence of Formulas}
Two formulas \(\alpha \textbf{And} \beta\) are said to be equivalent if they share the same final column in their respective truth tables. To indicate this we use teh following notion :

\begin{center}
\( \alpha \equiv \beta\)
\end{center}

\begin{lemma} Suppose that \( \alpha \equiv \beta\). Then for any formula \(\gamma\) and any connective *, the formulas \((\alpha * \gamma)\) and \((\beta * \gamma)\) are equivalent : \((\alpha * \gamma) \equiv (\beta * \gamma)\)
\end{lemma}

\subsection{Algebra of Formulas}
Many equivalences of formulas look much like rules of ordinary
arithmetic.
\begin{itemize}
\item Commutativity 
\begin{itemize}
\item \((\alpha \wedge \beta) \equiv (\beta \wedge \alpha)\)
\item \((\alpha \vee \beta) \equiv (\beta \vee \alpha)\)
\end{itemize}

\item Associativity
\begin{itemize}
\item \((\alpha \wedge (\beta \wedge \gamma)) \equiv ((\alpha \wedge \beta) \wedge \gamma)\)
\item \((\alpha \vee (\beta \vee \gamma)) \equiv ((\alpha \vee \beta) \vee \gamma)\)
\end{itemize}

\item Distributivity
\begin{itemize}
\item \((\alpha \vee (\beta \wedge \gamma) \equiv ((\alpha \vee \beta) \wedge (\alpha \vee \gamma)\) 
\item \((\alpha \wedge (\beta \vee \gamma) \equiv ((\alpha \wedge \beta) \vee (\alpha \wedge \gamma)\) 
\end{itemize}

\item Idempotence 
\begin{itemize}
\item \((\alpha \vee \alpha) \equiv \alpha\)
\item \((\alpha \wedge \alpha) \equiv \alpha\)
\end{itemize}

\item Double Negation 
\begin{itemize}
\item \((\neg (\neg \alpha)) \equiv \alpha\)
\end{itemize}

\item De Morgan's Laws 
\begin{itemize}
\item \(\neg (\alpha \wedge \beta) \equiv (\neg \alpha \vee \neg \beta)\)
\item \(\neg (\alpha \vee \beta) \equiv (\neg \alpha \wedge \neg \beta)\)
\end{itemize}

\item Simplification I
\begin{itemize}
\item \((\alpha \wedge T) \equiv \alpha\)
\item \((\alpha \vee T) \equiv T\)
\item \((\alpha \wedge F) \equiv F\)
\item \((\alpha \vee F) \equiv \alpha\)
\end{itemize}

\item Simplification II
\begin{itemize}
\item \((\alpha \vee (\alpha \wedge \beta)) \equiv \alpha\)
\item \((\alpha \wedge (\alpha \vee \beta)) \equiv \alpha\)
\end{itemize}

\item Implication 
\begin{itemize}
\item \((\alpha \implies \beta) \equiv ((\neg \alpha) \vee \beta)\)
\end{itemize}

\item Contrapositive 
\begin{itemize}
\item \((\alpha \implies \beta) \equiv ((\neg \beta) \implies (\neg \alpha))\)
\end{itemize}

\item Equivalence
\begin{itemize}
\item \((\alpha \iff \beta) \equiv ((\alpha \implies \beta) \wedge (\beta \implies \alpha))\)
\end{itemize}

\item Excluded Middle 
\begin{itemize}
\item \((\alpha \vee (\neg \alpha)) \equiv T\)
\end{itemize}

\item Contradiction 
\begin{itemize}
\item \((\alpha \wedge (\neg \alpha)) \equiv F \)
\end{itemize}
\end{itemize}

\section{Satisfiability of Sets of Formulas}
Let \(\sum\) denote a set of formulas and t a valuation define : \(\sum^t = \begin{cases} T    \text{     if for each } \beta \in \sum, \beta^t = T \\F \text{     otherwise }\end{cases}\)

When \(\sum^t = T\), we say that t \textbf{satisfies} \(sum\)

A Set \(\sum\) is \textbf{Satisfiable} iff there is some valuation t such that \(\sum^t = T\)

\section{Logical Consequence (Entailment)}
Let \(sum\) be a set of formulas and let \(\alpha\) be a formula. We say that 
\begin{itemize}
\item \(\alpha\) is a logical consequence of \(sum\), or 
\item \(\sum\) entails \(\alpha\), or 
\item \(sum \models \alpha\) 
\end{itemize}

if and only if for any truth valuation t, if \(\sum^t = T \text{ then also } \alpha^t = T\)

\subsection{Equivalence and Entailment}
Equivalence can be expressed using the notion of entailment.

\begin{lemma} \(\alpha \equiv \beta\) if and only if both\(\{ \alpha \} \models \beta\) and \(\{\beta\} \models \alpha \)
\end{lemma}
\end{document}
