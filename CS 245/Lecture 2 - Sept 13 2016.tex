%Notes by Harsh Mistry 
%CS 245
%based on Template from : https://www.cs.cmu.edu/~ggordon/10725-F12/template.tex

\documentclass{article}
\setlength{\oddsidemargin}{0.25 in}
\setlength{\evensidemargin}{-0.25 in}
\setlength{\topmargin}{-0.6 in}
\setlength{\textwidth}{6.5 in}
\setlength{\textheight}{8.5 in}
\setlength{\headsep}{0.75 in}
\setlength{\parindent}{0 in}
\setlength{\parskip}{0.1 in}
\usepackage{amsfonts,graphicx, amssymb}
\usepackage[fleqn]{amsmath}
\usepackage{fixltx2e}
\usepackage{xcolor}
\usepackage{color}
\usepackage{tcolorbox}
\usepackage{lipsum}
\usepackage{listings}
\usepackage{scrextend}
\tcbuselibrary{skins,breakable}
\usetikzlibrary{shadings,shadows}
\newcounter{lecnum}
\renewcommand{\thepage}{\thelecnum-\arabic{page}}
\renewcommand{\thesection}{\thelecnum.\arabic{section}}
\renewcommand{\theequation}{\thelecnum.\arabic{equation}}
\renewcommand{\thefigure}{\thelecnum.\arabic{figure}}
\renewcommand{\thetable}{\thelecnum.\arabic{table}}
\newcommand{\lecture}[4]{
   \pagestyle{myheadings}
   \thispagestyle{plain}
   \newpage
   \setcounter{lecnum}{#1}
   \setcounter{page}{1}
   
   
%Info Box 
   \begin{center}
   \framebox{
      \vbox{\vspace{2mm}
    \hbox to 6.28in { {\bf CS 245 - Logic and Computation 
	\hfill Fall 2016} }
       \vspace{4mm}
       \hbox to 6.28in { {\Large \hfill Lecture #1: #2  \hfill} }
       \vspace{2mm}
       \hbox to 6.28in { {\it Lecturer: #3 \hfill Notes By: #4} }
      \vspace{2mm}}
   }
   \end{center}
   
   \markboth{Lecture #1: #2}{Lecture #1: #2}



 
}

\renewcommand{\cite}[1]{[#1]}
\def\beginrefs{\begin{list}%
        {[\arabic{equation}]}{\usecounter{equation}
         \setlength{\leftmargin}{2.0truecm}\setlength{\labelsep}{0.4truecm}%
         \setlength{\labelwidth}{1.6truecm}}}
\def\endrefs{\end{list}}
\def\bibentry#1{\item[\hbox{[#1]}]}

\newcommand{\fig}[3]{
			\vspace{#2}
			\begin{center}
			Figure \thelecnum.#1:~#3
			\end{center}
	}

\newtheorem{theorem}{Theorem}[lecnum]
\newtheorem{lemma}[theorem]{Lemma}
\newtheorem{ex}[theorem]{Example}
\newtheorem{proposition}[theorem]{Proposition}
\newtheorem{claim}[theorem]{Claim}
\newtheorem{corollary}[theorem]{Corollary}
\newtheorem{definition}[theorem]{Definition}
\newenvironment{proof}{{\bf Proof:}}{\hfill\rule{2mm}{2mm}}
\newcommand\E{\mathbb{E}}

%color definitions :
\definecolor{darkred}{rgb}{0.55, 0.0, 0.0}
\definecolor{lightcoral}{rgb}{0.94, 0.5, 0.5}
\definecolor{tomato}{rgb}{1.0, 0.39, 0.28}
\definecolor{lightgray}{rgb}{.9,.9,.9}
\definecolor{darkgray}{rgb}{.4,.4,.4}
\definecolor{purple}{rgb}{0.65, 0.12, 0.82}
\definecolor{lightgreen}{rgb}{0.56, 0.93, 0.56}
\definecolor{darkgreen}{rgb}{0.0, 0.2, 0.13}
\definecolor{limegreen}{rgb}{0.2, 0.8, 0.2}
\definecolor{lightblue}{rgb}{0.68, 0.85, 0.9}
\definecolor{darkblue}{rgb}{0.0, 0.0, 0.55}


%Environments
\newenvironment{exblock}[1]{%
    \tcolorbox[beamer,%
    noparskip,breakable,
    colback=lightgreen,colframe=darkgreen,%
    colbacklower=limegreen!75!lightgreen,%
    title=#1]}%
    {\endtcolorbox}

\newenvironment{ablock}[1]{%
    \tcolorbox[beamer,%
    noparskip,breakable,
    colback=lightcoral,colframe=darkred,%
    colbacklower=tomato!75!lightcoral,%
    title=#1]}%
    {\endtcolorbox}

\newenvironment{cblock}[1]{%
    \tcolorbox[beamer,%
    noparskip,breakable,
    colback=lightblue,colframe=darkblue,%
    colbacklower=darkblue!75!lightblue,%
    title=#1]}%
    {\endtcolorbox}


%Languages
\lstdefinelanguage{JavaScript}{
  keywords={typeof, new, true, false, catch, function, return, null, catch, switch, var, if, in, while, do, else, case, break},
  keywordstyle=\color{blue}\bfseries,
  ndkeywords={class, export, boolean, throw, implements, import, this},
  ndkeywordstyle=\color{darkgray}\bfseries,
  identifierstyle=\color{black},
  sensitive=false,
  comment=[l]{//},
  morecomment=[s]{/*}{*/},
  commentstyle=\color{purple}\ttfamily,
  stringstyle=\color{red}\ttfamily,
  morestring=[b]',
  morestring=[b]"
}

%Listings
\lstset{
   language=JavaScript,
   backgroundcolor=\color{lightgray},
   extendedchars=true,
   basicstyle=\footnotesize\ttfamily,
   showstringspaces=false,
   showspaces=false,
   numbers=left,
   numberstyle=\footnotesize,
   numbersep=9pt,
   tabsize=2,
   breaklines=true,
   showtabs=false,
   captionpos=b
}


%Start of Document 
\begin{document}

\lecture{2}{September 13, 2016}{Kevin Lanctot}{Harsh Mistry}


\section{Propositional Logic}

\begin{cblock}{What is a proposition?}
A \textit{\textbf{proposition}} is a declarative sentence that is either true or false. 
\end{cblock}

\subsection{English to Propositional Logic}
\begin{itemize}
\item \(\urcorner \ p\) : Not P 
\item \(p \wedge q\) : P and Q 
\item \(p \vee q\) : P or Q 
\item \(p \implies q\) : P then Q
\item \(p \iff q\) : P if and only if Q 
\end{itemize}

\begin{exblock}{Examples}
\begin{enumerate}
\item She is clever and hard working : \(P \wedge Q\)
\item He is clever but not hardworking : \(P \wedge Q\)
\item If he does not study then he will fail : \((\urcorner \ S) \implies F\)
\item He must study hard; otherwise he will fail : \((\urcorner \ S) \implies F\)
\item He will fail unless he studies hard : \(F \vee S\)
\item He will not fail only if he studies hard : \((\urcorner \ F ) \implies S\)  
\end{enumerate}

\textbf{Advanced Examples}
\begin{enumerate}
\item If it rains. he will be at home; otherwise he will go to the market or to school.  \\  \((R \implies ) \wedge ((\urcorner \ R) \implies (M \vee S))\)

\item If the sum of two numbers is even if an only if both numbers are even or both numbers are odd. \\
\( S \iff (E \vee O)\)
\end{enumerate}
\end{exblock}

\textbf{Note : } Some sentences are not propositions, as not all sentences evaluate to true or false. 

\subsection{Aspects of Logic}
Propositional Logic is a form of \textbf{symbolic} logic. By extension symbolic logic is formalized by the following.

\begin{itemize}
\item \textbf{Syntax : }The statements we consider.
\item \textbf{Semantics : } The meaning of the statement.
\item \textbf{Proof Procedures : } Can we prove the given statement? 
\end{itemize}

\subsection{Syntax}

In propositional logic, simple \textbf{atomic propositions} are the basic building blocks. These atomic propositions can be connected to form \textbf{compound propositions}. 

\begin{ablock}{Questions to consider}
\begin{itemize}
\item Does a given sequence of propositions form a valid argument? 
\item Can all propositions in a given set be true simultaneously? 
\end{itemize}
\end{ablock} 

Propositions are represented by formulas. 
A formula consists of a sequence of symbols. The three kinds of symbols are : 
\begin{itemize}
\item Propositional Variables : \textbf{p, q, r}
\item Connectives : \(\urcorner\) , \(\wedge, \vee, \implies, \iff\)
\item Punctuation  : \textbf{ '(' and ')'}
\end{itemize}

\subsubsection{Expressions}

\begin{cblock}{Meta-Symbols}
We often use a letter that is not formally a symbol in order to namean expression. For example, we might denote a expression as \(\alpha\)
This is an example of a \textbf{meta-symbol}. It is \textbf{NOT} a symbol!
\end{cblock}

\begin{itemize}
\item Two expression \(\alpha and \beta\) are equal if and only if they are the are same length 
\item We write \(\alpha \beta\) to mean the concatenation of two expressions.
\end{itemize}

\begin{definition}Concatenation : 
If \(\alpha\) is an expression of length i and \(\beta\) is an expression of length j then \(\alpha \beta\) is an expression of length i + j. We have\\

\[ \text{The kth symbol of } \alpha \beta \text{ is } \begin{cases} \text{the kth symbol of } \alpha \textbf{     if  } k \leq 1 \\
\text{the } (k-i)^{\text{th}} \text{ symbol of } \beta \textbf{     if  } k > i 
\end{cases}\] 
\end{definition}

\subsubsection{Well-formed formula}
Let \(P\) be a set of propositional variables. We define the set of well-formed formulas over p inductively as follows. 

\begin{enumerate}
\item A expression consisting of a single symbol of \(P\) is  a well-formed formula 
\item If \(\alpha\) is a well-formed formula, then (\(\neg \alpha\)) is a well formed formula
\item If \(\alpha\) and \(\beta\) are well formed then, \((\alpha \wedge \beta), (\alpha \vee \beta), (\alpha \implies \beta), \text{ and } (\alpha \iff \beta) \) are well-formed
\item Nothing else is well-formed
\end{enumerate}

\subsubsection{Kinds of Formulas}
\begin{itemize}
\item A propositional variable is called an atom 
\item (\(\neg \alpha\)) : Negation 
\item (\(\alpha \wedge \beta\)) : Conjunction
\item (\(\alpha \vee \beta\)) : Disjunction 
\item (\(\alpha \implies \beta\)) : Implication 
\item (\(\alpha \iff \beta\)) : Equivalence 
\end{itemize}

\section{Semantics of Propositional Logic}
The semantics of logic describes how to interpret the well-formed formulas of the logic.  Since semantics of propositional logic is compositional, the meaning of the whole formula derives from the meaning of its parts. 

\subsection{Valuations}

\begin{definition}
A \textbf{truth valuation} is a function with the set of all proposition symbols as domain and {F,T} as range. Basically, a truth valuation assigns a value to every propositional variable. 
\end{definition}

\subsection{Semantics of Connectives}
A connective represents a function from truth values to truth values. 
The two types of connectives are : \textbf{Unary} and \textbf{binary}.  

\begin{itemize}
\item Unary connectives map one value to one value. 
\item Binary connectives map two values to one value. 
\end{itemize}

\end{document}
