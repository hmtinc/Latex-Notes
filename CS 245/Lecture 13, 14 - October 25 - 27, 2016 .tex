%Notes by Harsh Mistry 
%CS 245
%based on Template from : https://www.cs.cmu.edu/~ggordon/10725-F12/template.tex

\documentclass{article}
\setlength{\oddsidemargin}{0.25 in}
\setlength{\evensidemargin}{-0.25 in}
\setlength{\topmargin}{-0.6 in}
\setlength{\textwidth}{6.5 in}
\setlength{\textheight}{8.5 in}
\setlength{\headsep}{0.75 in}
\setlength{\parindent}{0 in}
\setlength{\parskip}{0.1 in}
\usepackage{amsfonts,graphicx, amssymb}
\usepackage[fleqn]{amsmath}
\usepackage{fixltx2e}
\usepackage{xcolor}
\usepackage{color}
\usepackage{tcolorbox}
\usepackage{lipsum}
\usepackage{listings}
\usepackage{scrextend}
\usepackage{enumerate}
\tcbuselibrary{skins,breakable}
\usetikzlibrary{shadings,shadows}
\newcounter{lecnum}
\renewcommand{\thepage}{\thelecnum-\arabic{page}}
\renewcommand{\thesection}{\thelecnum.\arabic{section}}
\renewcommand{\theequation}{\thelecnum.\arabic{equation}}
\renewcommand{\thefigure}{\thelecnum.\arabic{figure}}
\renewcommand{\thetable}{\thelecnum.\arabic{table}}
\newcommand{\lecture}[4]{
   \pagestyle{myheadings}
   \thispagestyle{plain}
   \newpage
   \setcounter{lecnum}{#1}
   \setcounter{page}{1}
   
   
%Info Box 
   \begin{center}
   \framebox{
      \vbox{\vspace{2mm}
    \hbox to 6.28in { {\bf CS 245 - Logic and Computation 
	\hfill Fall 2016} }
       \vspace{4mm}
       \hbox to 6.28in { {\Large \hfill Lecture #1: #2  \hfill} }
       \vspace{2mm}
       \hbox to 6.28in { {\it Lecturer: #3 \hfill Notes By: #4} }
      \vspace{2mm}}
   }
   \end{center}
   
   \markboth{Lecture #1: #2}{Lecture #1: #2}



 
}

\renewcommand{\cite}[1]{[#1]}
\def\beginrefs{\begin{list}%
        {[\arabic{equation}]}{\usecounter{equation}
         \setlength{\leftmargin}{2.0truecm}\setlength{\labelsep}{0.4truecm}%
         \setlength{\labelwidth}{1.6truecm}}}
\def\endrefs{\end{list}}
\def\bibentry#1{\item[\hbox{[#1]}]}

\newcommand{\fig}[3]{
			\vspace{#2}
			\begin{center}
			Figure \thelecnum.#1:~#3
			\end{center}
	}

\newtheorem{theorem}{Theorem}[lecnum]
\newtheorem{lemma}[theorem]{Lemma}
\newtheorem{ex}[theorem]{Example}
\newtheorem{proposition}[theorem]{Proposition}
\newtheorem{claim}[theorem]{Claim}
\newtheorem{corollary}[theorem]{Corollary}
\newtheorem{definition}[theorem]{Definition}
\newenvironment{proof}{{\bf Proof:}}{\hfill\rule{2mm}{2mm}}
\newcommand\E{\mathbb{E}}

%color definitions :
\definecolor{darkred}{rgb}{0.55, 0.0, 0.0}
\definecolor{lightcoral}{rgb}{0.94, 0.5, 0.5}
\definecolor{tomato}{rgb}{1.0, 0.39, 0.28}
\definecolor{lightgray}{rgb}{.9,.9,.9}
\definecolor{darkgray}{rgb}{.4,.4,.4}
\definecolor{purple}{rgb}{0.65, 0.12, 0.82}
\definecolor{lightgreen}{rgb}{0.56, 0.93, 0.56}
\definecolor{darkgreen}{rgb}{0.0, 0.2, 0.13}
\definecolor{limegreen}{rgb}{0.2, 0.8, 0.2}
\definecolor{lightblue}{rgb}{0.68, 0.85, 0.9}
\definecolor{darkblue}{rgb}{0.0, 0.0, 0.55}


%Environments
\newenvironment{exblock}[1]{%
    \tcolorbox[beamer,%
    noparskip,breakable,
    colback=lightgreen,colframe=darkgreen,%
    colbacklower=limegreen!75!lightgreen,%
    title=#1]}%
    {\endtcolorbox}

\newenvironment{ablock}[1]{%
    \tcolorbox[beamer,%
    noparskip,breakable,
    colback=lightcoral,colframe=darkred,%
    colbacklower=tomato!75!lightcoral,%
    title=#1]}%
    {\endtcolorbox}

\newenvironment{cblock}[1]{%
    \tcolorbox[beamer,%
    noparskip,breakable,
    colback=lightblue,colframe=darkblue,%
    colbacklower=darkblue!75!lightblue,%
    title=#1]}%
    {\endtcolorbox}


%Languages
\lstdefinelanguage{JavaScript}{
  keywords={typeof, new, true, false, catch, function, return, null, catch, switch, var, if, in, while, do, else, case, break},
  keywordstyle=\color{blue}\bfseries,
  ndkeywords={class, export, boolean, throw, implements, import, this},
  ndkeywordstyle=\color{darkgray}\bfseries,
  identifierstyle=\color{black},
  sensitive=false,
  comment=[l]{//},
  morecomment=[s]{/*}{*/},
  commentstyle=\color{purple}\ttfamily,
  stringstyle=\color{red}\ttfamily,
  morestring=[b]',
  morestring=[b]"
}

%Listings
\lstset{
   language=JavaScript,
   backgroundcolor=\color{lightgray},
   extendedchars=true,
   basicstyle=\footnotesize\ttfamily,
   showstringspaces=false,
   showspaces=false,
   numbers=left,
   numberstyle=\footnotesize,
   numbersep=9pt,
   tabsize=2,
   breaklines=true,
   showtabs=false,
   captionpos=b
}


%Start of Document 
\begin{document}

\lecture{13 and 14}{October 25 - 27, 2016}{Kevin Lanctot}{Harsh Mistry}

\section{Proofs in First-Order Logic using Natural Deduction}

\subsection{New Natural Deduction Rules}
\begin{itemize}
\item \(\forall - Elimination\): if \(\sum \vdash \forall x \alpha \) then \(\sum  \vdash \alpha [t/x]\)
$$ \frac{\forall x \alpha}{\alpha[t/x]}$$
\item \(\exists - Introduction \) : if \(\sum \vdash \alpha [t/x]\) then \(\sum \vdash \exists x \alpha\) 
$$ \frac{\alpha[t/x]}{\exists x \alpha}$$
\item \(\forall - Introduction\) : if \(\sum \vdash \alpha[y/x]\) and 
y not free in \(\sum\) or \(\alpha\), then \(\sum \vdash \forall x \alpha\) 
$$ \frac{\begin{matrix}
y \text{fresh}\\ \vdots \\ \alpha[y/x] 
\end{matrix}}{\forall x \alpha}$$
\begin{itemize}
\item A variable is fresh in a sub proof if it occurs nowhere outside the box of the subproof. (i.e not a free variable outside the subproof.
\end{itemize}

\item \(\exists\) - Elimination : if \(\sum, \alpha[u/x] \vdash \beta\) with u fresh, then \(\sum, \exists x \alpha \vdash \beta\) 
$$ \frac{\exists x \alpha \hspace{0.5cm} \begin{matrix}
\alpha [u/x], \text{u fresh} \\ \vdots \\ \beta 
\end{matrix}}{\beta }$$

\end{itemize}

\subsection{Soundness of \(\forall\) Elimination and \(\exists\) Introduction}

\begin{itemize}
\item For any formula \(\varphi\), variable x and term t
$$ \forall x \varphi \models \varphi[t/x] \text{   and    } \varphi[t/x] \models \exists x \varphi $$
\item For every formula \(\varphi\), variable x and t 
$$\varphi[t/x]^{(I,E)} = \varphi^{(I,E[x \rightarrow t^{(I,E)}])}$$
\end{itemize}

\subsection{Defining Substitution}
For a variable x and term t 
\begin{enumerate}
\item If \(\alpha\) is \((Q x \beta)\), then \(\alpha[t/x]\) is \(\alpha\) 
\item If \(\alpha\) is \((Q y \beta)\) for some other variable y, then 
\begin{itemize}
\item if y does not occur in t, then \(\alpha[t/x]\) is \((Qy\beta [t/x])\) 
\item Otherwise, let z be a variable that occurs in neither \(\alpha\) not t; then \(\alpha[t/x]\) is \((Qz(\beta[z/y])[t/x])\)
\end{itemize}
\end{enumerate}
With this definition, we always get that, as required
$$ (Qy \alpha)^{(I,E[x \rightarrow t ^{(I,E)}])} = ((Qy\alpha) [t/x])^{(I,E)} $$

\subsection{FOL with Equality}
First order logic with equality is first order logic with the restriction that the symbol "=" must be interpreted as equality on the domain 
$$ (=)^I = \{(d,d) \mid d \in dom(I)\} $$

There are two way to account for this restriction in our proofs.
\begin{enumerate}
\item Add deduction rules for symbol = 
\begin{itemize}
\item Introduction \[ \frac{}{t=t} = i\]
\item Elimination \[ \frac{t_1 = t_2 \hspace{0.7cm} \alpha[t_1/x]}{\alpha[t_2/x]}\]
\end{itemize}
\item Use axioms rather than deduction rules
\begin{enumerate}
\item \(\forall \alpha \alpha = \alpha\) is an axiom 
\item For Each formula \(\alpha\) and variable z :  \(\forall\alpha \forall y ( \alpha = y \rightarrow (\alpha [\alpha/z] \rightarrow \alpha[y/z]))\) is an axiom 
\end{enumerate}
These axioms imply 
\begin{itemize}
\item Symmetry of =: \(\vdash \forall x \forall y (x = y \rightarrow y = x)\)
\item Transitivity of =: \(\vdash \forall w \forall x \forall y (x = y \rightarrow (y = w \rightarrow x = w))\)
\end{itemize}
\end{enumerate}

\subsection{Derived rules for Equality}
\begin{itemize}
\item EQtrans(k) : $$ \frac{t_1 = t_2 \hspace{0.7cm} \ldots \hspace{0.7cm} t_k = t_{k+1}}{t_1 = t_{k+1}}$$
\item EQsubs(r) : $$ \frac{t_1 = t_2}{r[t_1/z] = r[t_2/z]}$$
\end{itemize}


\end{document}