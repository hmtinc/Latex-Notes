%Notes by Harsh Mistry 
%CS 245
%based on Template from : https://www.cs.cmu.edu/~ggordon/10725-F12/template.tex

\documentclass{article}
\setlength{\oddsidemargin}{0.25 in}
\setlength{\evensidemargin}{-0.25 in}
\setlength{\topmargin}{-0.6 in}
\setlength{\textwidth}{6.5 in}
\setlength{\textheight}{8.5 in}
\setlength{\headsep}{0.75 in}
\setlength{\parindent}{0 in}
\setlength{\parskip}{0.1 in}
\usepackage{amsfonts,graphicx, amssymb}
\usepackage[fleqn]{amsmath}
\usepackage{fixltx2e}
\usepackage{xcolor}
\usepackage{color}
\usepackage{tcolorbox}
\usepackage{lipsum}
\usepackage{listings}
\usepackage{scrextend}
\usepackage{enumerate}
\tcbuselibrary{skins,breakable}
\usetikzlibrary{shadings,shadows}
\newcounter{lecnum}
\renewcommand{\thepage}{\thelecnum-\arabic{page}}
\renewcommand{\thesection}{\thelecnum.\arabic{section}}
\renewcommand{\theequation}{\thelecnum.\arabic{equation}}
\renewcommand{\thefigure}{\thelecnum.\arabic{figure}}
\renewcommand{\thetable}{\thelecnum.\arabic{table}}
\newcommand{\lecture}[4]{
   \pagestyle{myheadings}
   \thispagestyle{plain}
   \newpage
   \setcounter{lecnum}{#1}
   \setcounter{page}{1}
   
   
%Info Box 
   \begin{center}
   \framebox{
      \vbox{\vspace{2mm}
    \hbox to 6.28in { {\bf CS 245 - Logic and Computation 
	\hfill Fall 2016} }
       \vspace{4mm}
       \hbox to 6.28in { {\Large \hfill Lecture #1: #2  \hfill} }
       \vspace{2mm}
       \hbox to 6.28in { {\it Lecturer: #3 \hfill Notes By: #4} }
      \vspace{2mm}}
   }
   \end{center}
   
   \markboth{Lecture #1: #2}{Lecture #1: #2}



 
}

\renewcommand{\cite}[1]{[#1]}
\def\beginrefs{\begin{list}%
        {[\arabic{equation}]}{\usecounter{equation}
         \setlength{\leftmargin}{2.0truecm}\setlength{\labelsep}{0.4truecm}%
         \setlength{\labelwidth}{1.6truecm}}}
\def\endrefs{\end{list}}
\def\bibentry#1{\item[\hbox{[#1]}]}

\newcommand{\fig}[3]{
			\vspace{#2}
			\begin{center}
			Figure \thelecnum.#1:~#3
			\end{center}
	}

\newtheorem{theorem}{Theorem}[lecnum]
\newtheorem{lemma}[theorem]{Lemma}
\newtheorem{ex}[theorem]{Example}
\newtheorem{proposition}[theorem]{Proposition}
\newtheorem{claim}[theorem]{Claim}
\newtheorem{corollary}[theorem]{Corollary}
\newtheorem{definition}[theorem]{Definition}
\newenvironment{proof}{{\bf Proof:}}{\hfill\rule{2mm}{2mm}}
\newcommand\E{\mathbb{E}}

%color definitions :
\definecolor{darkred}{rgb}{0.55, 0.0, 0.0}
\definecolor{lightcoral}{rgb}{0.94, 0.5, 0.5}
\definecolor{tomato}{rgb}{1.0, 0.39, 0.28}
\definecolor{lightgray}{rgb}{.9,.9,.9}
\definecolor{darkgray}{rgb}{.4,.4,.4}
\definecolor{purple}{rgb}{0.65, 0.12, 0.82}
\definecolor{lightgreen}{rgb}{0.56, 0.93, 0.56}
\definecolor{darkgreen}{rgb}{0.0, 0.2, 0.13}
\definecolor{limegreen}{rgb}{0.2, 0.8, 0.2}
\definecolor{lightblue}{rgb}{0.68, 0.85, 0.9}
\definecolor{darkblue}{rgb}{0.0, 0.0, 0.55}


%Environments
\newenvironment{exblock}[1]{%
    \tcolorbox[beamer,%
    noparskip,breakable,
    colback=lightgreen,colframe=darkgreen,%
    colbacklower=limegreen!75!lightgreen,%
    title=#1]}%
    {\endtcolorbox}

\newenvironment{ablock}[1]{%
    \tcolorbox[beamer,%
    noparskip,breakable,
    colback=lightcoral,colframe=darkred,%
    colbacklower=tomato!75!lightcoral,%
    title=#1]}%
    {\endtcolorbox}

\newenvironment{cblock}[1]{%
    \tcolorbox[beamer,%
    noparskip,breakable,
    colback=lightblue,colframe=darkblue,%
    colbacklower=darkblue!75!lightblue,%
    title=#1]}%
    {\endtcolorbox}


%Languages
\lstdefinelanguage{JavaScript}{
  keywords={typeof, new, true, false, catch, function, return, null, catch, switch, var, if, in, while, do, else, case, break},
  keywordstyle=\color{blue}\bfseries,
  ndkeywords={class, export, boolean, throw, implements, import, this},
  ndkeywordstyle=\color{darkgray}\bfseries,
  identifierstyle=\color{black},
  sensitive=false,
  comment=[l]{//},
  morecomment=[s]{/*}{*/},
  commentstyle=\color{purple}\ttfamily,
  stringstyle=\color{red}\ttfamily,
  morestring=[b]',
  morestring=[b]"
}

%Listings
\lstset{
   language=JavaScript,
   backgroundcolor=\color{lightgray},
   extendedchars=true,
   basicstyle=\footnotesize\ttfamily,
   showstringspaces=false,
   showspaces=false,
   numbers=left,
   numberstyle=\footnotesize,
   numbersep=9pt,
   tabsize=2,
   breaklines=true,
   showtabs=false,
   captionpos=b
}


%Start of Document 
\begin{document}

\lecture{19, 20}{November 15 - 17, 2016}{Kevin Lanctot}{Harsh Mistry}

\section{Program Verification}
Formal Verification is when you state a specification and prove a program satisfies the specification for all inputs. We formally specify and verify programs to : 
\begin{itemize}
\item Reduce Bugs
\item To test safety critical software or important components 
\item To document individual component. 
\item To adopt a better development and verification process .
\end{itemize}

\subsection{Framework for Software Verification}
The steps of formal verification: 
\begin{enumerate}
\item Convert teh informal description R of requirements for an application domain into an "equivalent" formula \(\phi_R\) of some symbolic logic. 
\item Write a program P which is meant to realise \(\phi_R\) in some given programming environment, and
\item Prove that the program P satisfies the formula \(\phi_R\)
\end{enumerate}


\textbf{Note :} Only part 3 is covered in CS245

\subsection{Hoare Triples}
Our assertions about programs will have the form 
\begin{itemize}
\item (P) - Precondition 
\item C - Program or code
\item (Q) - Postcondition
\end{itemize}

The meaning of the triple (P) C (Q) is : 
If program C is run starting in a state that satisfies P, then the resulting state after execution of C will satisfy Q. 

An assertion (P) C (Q) is called a Hoare Triple. 

\subsubsection{Specification of a Program}

A specification of a program C is a Hoare triple with C as the second component. 

\textbf{Note : } Specification is not behaviour. 

\subsection{Partial Correctness}
A triple (P) C (Q) is satisfied under partial correctness, denoted 
$$ \models_{par} (P) C (Q), $$

if and only if , for every state s that satisfies condition P, if execution of C starting from state s terminates then the resulting state s' satisfies condition Q

\textbf{Note : } Partial correctness \textbf{does not imply termination.}

\subsection{Total correctness}
A triple (P) C (Q) is satisfied under total correctness denoted 
$$ \models_{par} (P) C (Q) $$
if and only if, for every state S that satisfies condition P, execution of C starting from state S terminates, and the resulting state s' satisfies Q.

$$\text{Total Correctness  = Partial Correctness + Termination}$$

\subsection{Logical variables}

Sometimes the preconditions and postconditions require additional
variables that do not appear in the program.
These are called logical variables.

The logical variable allows the postcondition to be expressed in
terms of the value of ‘ at the beginning of the code.

\subsection{Proving Correctness}

Our Goal: Prove total correctness

Our Approach: We usually prove it by proving partial correctness
and termination separately.

\begin{itemize}
\item For partial correctness: we shall introduce sound inference rules.
\item For total correctness: we shall use ad hoc reasoning, which
suffices for our examples.
\item In general, total correctness is undecidable. i.e. it cannot be
decided in an automated fashion that always classifiies
programs correctly.
\end{itemize}

\subsection{Inference Rules}
\begin{itemize}
\item Rule for Assignment
$$ \frac{}{(Q [E/x])\hspace{0.5cm} X = E \hspace{0.5cm} (Q)} $$
\item Inference Rule of Precondition Strengthening 
$$ \frac{P \rightarrow P^\prime \hspace{0.5cm} (P^\prime) C (Q)}{(P) C (Q)}$$
\item Inference Rule of Postcondition Weakening
$$ \frac{(P)C(Q^\prime) \hspace{0.5cm} Q^\prime \rightarrow Q}{(P) C (Q)}$$
\item Inference Rule of Composition 
$$ \frac{(P)C_1 (Q), \hspace{0.5cm} (Q) C_2 (R) }{(P) C_1 ; C_2 (R)}$$
\item Inference Rule for if-then-else : 
$$ \frac{(P \wedge B) C_1 (Q) \hspace{0.5cm} (P \wedge \neg B) C_2 (Q)}{(P) \textit{ if \{B\} } C_1 \textit{ else } C_2 (Q)}$$
\item Inference Rule for if-then (without else) 
$$ \frac{(P \wedge B) C (Q) \hspace{0.5cm} (P \wedge \neg B) \rightarrow Q }{(P) \textit{ if \{B\} C } (Q)}$$
\end{itemize}

\subsection{The Steps of Creating a Proof}
\begin{enumerate}
\item First \textbf{annotate the code} using the appropriate inference rules.
\item Then \textbf{move bottom-up in the proof} : adding an assertion before
each assignment statement, based on the assertion after the
assignment (as we have always done for assignments).
\item Finally prove any justifications labelled “implied”:
\begin{itemize}
\item Annotations from (1) above containing implications
\item Adjacent assertions created in step (2).
\end{itemize}
\end{enumerate}

Proofs here can use predicate logic, basic arithmetic, or other
appropriate reasoning

\end{document}
