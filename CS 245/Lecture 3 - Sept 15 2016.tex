%Notes by Harsh Mistry 
%CS 245
%based on Template from : https://www.cs.cmu.edu/~ggordon/10725-F12/template.tex

\documentclass{article}
\setlength{\oddsidemargin}{0.25 in}
\setlength{\evensidemargin}{-0.25 in}
\setlength{\topmargin}{-0.6 in}
\setlength{\textwidth}{6.5 in}
\setlength{\textheight}{8.5 in}
\setlength{\headsep}{0.75 in}
\setlength{\parindent}{0 in}
\setlength{\parskip}{0.1 in}
\usepackage{amsfonts,graphicx, amssymb}
\usepackage[fleqn]{amsmath}
\usepackage{fixltx2e}
\usepackage{xcolor}
\usepackage{color}
\usepackage{tcolorbox}
\usepackage{lipsum}
\usepackage{listings}
\usepackage{scrextend}
\usepackage{enumerate}
\tcbuselibrary{skins,breakable}
\usetikzlibrary{shadings,shadows}
\newcounter{lecnum}
\renewcommand{\thepage}{\thelecnum-\arabic{page}}
\renewcommand{\thesection}{\thelecnum.\arabic{section}}
\renewcommand{\theequation}{\thelecnum.\arabic{equation}}
\renewcommand{\thefigure}{\thelecnum.\arabic{figure}}
\renewcommand{\thetable}{\thelecnum.\arabic{table}}
\newcommand{\lecture}[4]{
   \pagestyle{myheadings}
   \thispagestyle{plain}
   \newpage
   \setcounter{lecnum}{#1}
   \setcounter{page}{1}
   
   
%Info Box 
   \begin{center}
   \framebox{
      \vbox{\vspace{2mm}
    \hbox to 6.28in { {\bf CS 245 - Logic and Computation 
	\hfill Fall 2016} }
       \vspace{4mm}
       \hbox to 6.28in { {\Large \hfill Lecture #1: #2  \hfill} }
       \vspace{2mm}
       \hbox to 6.28in { {\it Lecturer: #3 \hfill Notes By: #4} }
      \vspace{2mm}}
   }
   \end{center}
   
   \markboth{Lecture #1: #2}{Lecture #1: #2}



 
}

\renewcommand{\cite}[1]{[#1]}
\def\beginrefs{\begin{list}%
        {[\arabic{equation}]}{\usecounter{equation}
         \setlength{\leftmargin}{2.0truecm}\setlength{\labelsep}{0.4truecm}%
         \setlength{\labelwidth}{1.6truecm}}}
\def\endrefs{\end{list}}
\def\bibentry#1{\item[\hbox{[#1]}]}

\newcommand{\fig}[3]{
			\vspace{#2}
			\begin{center}
			Figure \thelecnum.#1:~#3
			\end{center}
	}

\newtheorem{theorem}{Theorem}[lecnum]
\newtheorem{lemma}[theorem]{Lemma}
\newtheorem{ex}[theorem]{Example}
\newtheorem{proposition}[theorem]{Proposition}
\newtheorem{claim}[theorem]{Claim}
\newtheorem{corollary}[theorem]{Corollary}
\newtheorem{definition}[theorem]{Definition}
\newenvironment{proof}{{\bf Proof:}}{\hfill\rule{2mm}{2mm}}
\newcommand\E{\mathbb{E}}

%color definitions :
\definecolor{darkred}{rgb}{0.55, 0.0, 0.0}
\definecolor{lightcoral}{rgb}{0.94, 0.5, 0.5}
\definecolor{tomato}{rgb}{1.0, 0.39, 0.28}
\definecolor{lightgray}{rgb}{.9,.9,.9}
\definecolor{darkgray}{rgb}{.4,.4,.4}
\definecolor{purple}{rgb}{0.65, 0.12, 0.82}
\definecolor{lightgreen}{rgb}{0.56, 0.93, 0.56}
\definecolor{darkgreen}{rgb}{0.0, 0.2, 0.13}
\definecolor{limegreen}{rgb}{0.2, 0.8, 0.2}
\definecolor{lightblue}{rgb}{0.68, 0.85, 0.9}
\definecolor{darkblue}{rgb}{0.0, 0.0, 0.55}


%Environments
\newenvironment{exblock}[1]{%
    \tcolorbox[beamer,%
    noparskip,breakable,
    colback=lightgreen,colframe=darkgreen,%
    colbacklower=limegreen!75!lightgreen,%
    title=#1]}%
    {\endtcolorbox}

\newenvironment{ablock}[1]{%
    \tcolorbox[beamer,%
    noparskip,breakable,
    colback=lightcoral,colframe=darkred,%
    colbacklower=tomato!75!lightcoral,%
    title=#1]}%
    {\endtcolorbox}

\newenvironment{cblock}[1]{%
    \tcolorbox[beamer,%
    noparskip,breakable,
    colback=lightblue,colframe=darkblue,%
    colbacklower=darkblue!75!lightblue,%
    title=#1]}%
    {\endtcolorbox}


%Languages
\lstdefinelanguage{JavaScript}{
  keywords={typeof, new, true, false, catch, function, return, null, catch, switch, var, if, in, while, do, else, case, break},
  keywordstyle=\color{blue}\bfseries,
  ndkeywords={class, export, boolean, throw, implements, import, this},
  ndkeywordstyle=\color{darkgray}\bfseries,
  identifierstyle=\color{black},
  sensitive=false,
  comment=[l]{//},
  morecomment=[s]{/*}{*/},
  commentstyle=\color{purple}\ttfamily,
  stringstyle=\color{red}\ttfamily,
  morestring=[b]',
  morestring=[b]"
}

%Listings
\lstset{
   language=JavaScript,
   backgroundcolor=\color{lightgray},
   extendedchars=true,
   basicstyle=\footnotesize\ttfamily,
   showstringspaces=false,
   showspaces=false,
   numbers=left,
   numberstyle=\footnotesize,
   numbersep=9pt,
   tabsize=2,
   breaklines=true,
   showtabs=false,
   captionpos=b
}


%Start of Document 
\begin{document}

\lecture{3}{September 15, 2016}{Kevin Lanctot}{Harsh Mistry}

\section{Review}
\begin{itemize}
\item Propositions : True / False
\item Syntax : Format 
\item Symbols :
\begin{itemize}
\item p, q, r
\item \(\wedge, \vee, \neg, \implies, \iff\)
\item () 
\end{itemize}
\item Expression : \(((P \wedge Q) \implies R)\)
\end{itemize}

\section{Unique Readability of Formulas}
\begin{theorem}
Every well-formed formula has a unique derivation as a well-formed formula. So, each well-formed formula has exactly one the following forms: 
\begin{enumerate}
\item An atom 
\item (\(\neg \alpha\))
\item \((\alpha \wedge \beta)\)
\item \((\alpha \vee \beta)\)
\item \((\alpha \implies \beta)\)
\item \((\alpha \iff \beta)\)
\end{enumerate}
\end{theorem}

In each case, it is of that form in exactly one way. In essence, there is only one way of reading the formula and breaking it down.

\begin{ablock}{Simple and Strong Induction Review}
\begin{center}
\begin{tabular}{|c|c|c|}
\hline 
• & Simple Induction & Strong Induction (Course of value) \\ 
\hline 
Basis  & Show P(0) & Show P(0) \\ 
\hline 
Ind. Hypothesis & P(k) holds & \textbf{ P(m) holds for every M \(\leq\) k } \\ 
\hline 
Ind. Step & Show P(k+1) holds & Show P(k+1) Holds \\ 
\hline 
Conclusion  & P(k) holds for every k  & P(k) holds for every k \\ 
\hline 
\end{tabular} 
\end{center}
\end{ablock}

\subsection{Structural Induction}

A formula is not a natural number, but it suffices to prove any one of the following

\begin{itemize}
\item For every natural number n, every formula with n or fewer
symbols has property P.
\item For every natural number n, every formula with n or fewer
connectives has property P.
\item For every natural number n, every formula whose parse tree
has height n or less has property P.
\item or , For every natural number n, every formula whose parse tree
has height n or less has property P.
\end{itemize}

Induction applied tp any of the formulations above, requires steps that show: \\
If \(P(\alpha)\) and \(P(\beta)\) then \(P((\neg \alpha))\) and \(P(\alpha * \beta)\). \\  

Formulas \(\alpha\) and \(\beta\) have smaller n values that \(\neg \alpha\) and \(\alpha * \beta\) do. Where * represents a connective operator , 

\subsection{Principal of Structural Induction}
\begin{theorem}
Let R be a property. Suppose that 
\begin{enumerate}
\item for each atomic formula p, we have R(p); and
\item for each formula \(\alpha\), if \(R(\alpha)\)  then \(R((\neg \alpha))\) ; and 
\item for each pair of formulas \(\alpha\) and \(\beta\) and each binary connective *, if \(R(\alpha)\) and \(R(\beta)\) then \(R(\alpha * \beta)\) 
\end{enumerate}

Then \(R(\alpha)\) for every formula \(\alpha\) 

Use of this principle is called \textbf{\textcolor{red}{Structural Induction}}, which is a special case of mathematical induction
\end{theorem}  



\subsection{Proof of Unique Readability}
\begin{proof}
Property P(n) : 

Every Formula \(\varphi\) contain at most n connectives satisfies all there of the following. 

\begin{enumerate}[A:]
\item The first symbol of \(\varphi\) is either '(' or variable 
\item \(\varphi\) has an equal number of '(' and ')' and each proper prefix of \(\varphi\) has more '(' than  ')'.
\item \(\varphi\) has a unique construction as a formula 
\end{enumerate}

Note : A \textbf{Proper Prefix} of \(\varphi\) is a non-empty expression x such that \(\varphi\) is xy for some non-empty expression y.

\begin{center}
We can prove property P for all n by induction 
\end{center}

\textbf{Inductive Hypothesis : } P(k) holds for some natural number k 

\textbf{We Must Prove :} P(k+1) holds, to do this we let formula \(\alpha\) have k+1 connectives 

\textbf{A key case : } \(\alpha\) is \(\beta * \gamma\). For property C, we must show that \(\alpha\) is  \(\beta^{\prime} *^{\prime} \gamma^{\prime} \) for \textbf{Formulas} \(\beta^{\prime}\) and \(\gamma^{\prime}\), then \(\beta = \beta^{\prime}\), \(* = *^{\prime}\), and \(\gamma = \gamma^{\prime}\), 

If \(\beta^{\prime}\) has the same length as \(\beta\), then they must be the same string

Otherwise, either \(\beta^{\prime}\) is a proper prefix of \(\beta\)  or \(\beta\) is a proper prefix of \(\beta^{\prime}\) . But since \(\beta\) and \(\beta^{\prime}\)  are formulas with at most k
connectives, the inductive hypothesis applies to them.
In particular, each has property B, and thus neither can be a
proper prefix of the other.


Thus \(\beta\) has a unique derivation, as required by property C.

\end{proof}
\subsection{Options for a Proof}

\begin{enumerate}
\item Prove A, B and C simultaneously, as a single “compound”
property. (Section 3.2.2)
\item Prove them separately: first A, then B, and finally C.
\end{enumerate}

\section{Truth Tables}
Columns: List all the propositional variables on left and all the
subformulas (in increasing order of the number of connectives) on
the right.

Rows: Create a row for every possible combination of truth
valuations for the propositional variables.

\end{document}
