%Notes by Harsh Mistry 
%CS 245
%based on Template from : https://www.cs.cmu.edu/~ggordon/10725-F12/template.tex

\documentclass{article}
\setlength{\oddsidemargin}{0.25 in}
\setlength{\evensidemargin}{-0.25 in}
\setlength{\topmargin}{-0.6 in}
\setlength{\textwidth}{6.5 in}
\setlength{\textheight}{8.5 in}
\setlength{\headsep}{0.75 in}
\setlength{\parindent}{0 in}
\setlength{\parskip}{0.1 in}
\usepackage{amsfonts,graphicx, amssymb}
\usepackage[fleqn]{amsmath}
\usepackage{fixltx2e}
\usepackage{xcolor}
\usepackage{color}
\usepackage{tcolorbox}
\usepackage{lipsum}
\usepackage{listings}
\usepackage{scrextend}
\usepackage{enumerate}
\tcbuselibrary{skins,breakable}
\usetikzlibrary{shadings,shadows}
\newcounter{lecnum}
\renewcommand{\thepage}{\thelecnum-\arabic{page}}
\renewcommand{\thesection}{\thelecnum.\arabic{section}}
\renewcommand{\theequation}{\thelecnum.\arabic{equation}}
\renewcommand{\thefigure}{\thelecnum.\arabic{figure}}
\renewcommand{\thetable}{\thelecnum.\arabic{table}}
\newcommand{\lecture}[4]{
   \pagestyle{myheadings}
   \thispagestyle{plain}
   \newpage
   \setcounter{lecnum}{#1}
   \setcounter{page}{1}
   
   
%Info Box 
   \begin{center}
   \framebox{
      \vbox{\vspace{2mm}
    \hbox to 6.28in { {\bf CS 245 - Logic and Computation 
	\hfill Fall 2016} }
       \vspace{4mm}
       \hbox to 6.28in { {\Large \hfill Lecture #1: #2  \hfill} }
       \vspace{2mm}
       \hbox to 6.28in { {\it Lecturer: #3 \hfill Notes By: #4} }
      \vspace{2mm}}
   }
   \end{center}
   
   \markboth{Lecture #1: #2}{Lecture #1: #2}



 
}

\renewcommand{\cite}[1]{[#1]}
\def\beginrefs{\begin{list}%
        {[\arabic{equation}]}{\usecounter{equation}
         \setlength{\leftmargin}{2.0truecm}\setlength{\labelsep}{0.4truecm}%
         \setlength{\labelwidth}{1.6truecm}}}
\def\endrefs{\end{list}}
\def\bibentry#1{\item[\hbox{[#1]}]}

\newcommand{\fig}[3]{
			\vspace{#2}
			\begin{center}
			Figure \thelecnum.#1:~#3
			\end{center}
	}

\newtheorem{theorem}{Theorem}[lecnum]
\newtheorem{lemma}[theorem]{Lemma}
\newtheorem{ex}[theorem]{Example}
\newtheorem{proposition}[theorem]{Proposition}
\newtheorem{claim}[theorem]{Claim}
\newtheorem{corollary}[theorem]{Corollary}
\newtheorem{definition}[theorem]{Definition}
\newenvironment{proof}{{\bf Proof:}}{\hfill\rule{2mm}{2mm}}
\newcommand\E{\mathbb{E}}

%color definitions :
\definecolor{darkred}{rgb}{0.55, 0.0, 0.0}
\definecolor{lightcoral}{rgb}{0.94, 0.5, 0.5}
\definecolor{tomato}{rgb}{1.0, 0.39, 0.28}
\definecolor{lightgray}{rgb}{.9,.9,.9}
\definecolor{darkgray}{rgb}{.4,.4,.4}
\definecolor{purple}{rgb}{0.65, 0.12, 0.82}
\definecolor{lightgreen}{rgb}{0.56, 0.93, 0.56}
\definecolor{darkgreen}{rgb}{0.0, 0.2, 0.13}
\definecolor{limegreen}{rgb}{0.2, 0.8, 0.2}
\definecolor{lightblue}{rgb}{0.68, 0.85, 0.9}
\definecolor{darkblue}{rgb}{0.0, 0.0, 0.55}


%Environments
\newenvironment{exblock}[1]{%
    \tcolorbox[beamer,%
    noparskip,breakable,
    colback=lightgreen,colframe=darkgreen,%
    colbacklower=limegreen!75!lightgreen,%
    title=#1]}%
    {\endtcolorbox}

\newenvironment{ablock}[1]{%
    \tcolorbox[beamer,%
    noparskip,breakable,
    colback=lightcoral,colframe=darkred,%
    colbacklower=tomato!75!lightcoral,%
    title=#1]}%
    {\endtcolorbox}

\newenvironment{cblock}[1]{%
    \tcolorbox[beamer,%
    noparskip,breakable,
    colback=lightblue,colframe=darkblue,%
    colbacklower=darkblue!75!lightblue,%
    title=#1]}%
    {\endtcolorbox}


%Languages
\lstdefinelanguage{JavaScript}{
  keywords={typeof, new, true, false, catch, function, return, null, catch, switch, var, if, in, while, do, else, case, break},
  keywordstyle=\color{blue}\bfseries,
  ndkeywords={class, export, boolean, throw, implements, import, this},
  ndkeywordstyle=\color{darkgray}\bfseries,
  identifierstyle=\color{black},
  sensitive=false,
  comment=[l]{//},
  morecomment=[s]{/*}{*/},
  commentstyle=\color{purple}\ttfamily,
  stringstyle=\color{red}\ttfamily,
  morestring=[b]',
  morestring=[b]"
}

%Listings
\lstset{
   language=JavaScript,
   backgroundcolor=\color{lightgray},
   extendedchars=true,
   basicstyle=\footnotesize\ttfamily,
   showstringspaces=false,
   showspaces=false,
   numbers=left,
   numberstyle=\footnotesize,
   numbersep=9pt,
   tabsize=2,
   breaklines=true,
   showtabs=false,
   captionpos=b
}


%Start of Document 
\begin{document}

\lecture{6 and 7}{September 25, 29, 2016}{Kevin Lanctot}{Harsh Mistry}

\section{Proofs and Entailment}
We want show : 
\begin{enumerate}
\item Soundness : \(\sum \vdash_{Res} \varphi \) implies that \(\sum \models \varphi\), that is, if we prove something using Resolution then it is true 
\item Completeness : \(Sum \models \varphi\) implies that \(\sum \vdash \varphi\), that is, every entailment has a proof 
\end{enumerate}

\subsection{Resolution is Sound}
For resolution to be meaningful, we need the following.

\begin{theorem}
Suppose that \(\{\alpha_1, \ldots, \alpha_2\} \vdash_{Res} \perp \); that is, there is a resolution refutation with premises \(\alpha_1, \ldots, \alpha_n\) ans conclusion \(\perp\). Then the set \(\{\alpha1, \ldots, \alpha_n\}\) is unsatisfiable (Contradictory)

That is, \(\sum \cup \{\neg \varphi\} \vdash_{Res} \perp, \text{ then } \sum \cup \{\neg \varphi\} \)is a contradiction. Therefore, \(\sum \models \varphi\). 
\end{theorem}

In essence, the solution proof system is sound. 

\subsection{Resolution Can Fail}
\begin{definition}
A proof system S is complete if every entailment has a proof; that is, if every entailment has a proof; that is, if 
$$ \sum \models \alpha \textbf{ Implies } \sum \vdash_{S} \alpha $$
\end{definition}

\begin{theorem}
Resolution is complete refutation system for CNF formulas. That is, if there is no proof of \(\perp\) from a set of premises in CNF, then \(\sum\) satisfiable. 
\end{theorem}

\begin{claim}
Suppose that a resolution proof “reaches a dead end”—that
is, no new clause can be obtained, and yet \(\perp\) has not been derived. Then the entire set of formulas (including the premises!) is satisfiable. 
\end{claim}

The resolution method yields an algorithm to determine whether a
given formula, or set of formulas, is satisfiable or contradictory.

\begin{itemize}
\item Convert to CNF
\item Form resolvents (the result of applying resolution’s inference
rule) until either \(\perp\) is derived, or no more derivations are
possible.
\end{itemize}

If \(\perp\)  is derived, the original formula/set is contradictory. Otherwise,
the preceding proof describes how to find a satisfying valuation.

\section{Proofs in Propositional Logic : Natural Deduction}

\begin{itemize}
\item \(\wedge\)-Introduction Rule : If \(\sum \vdash \varphi\) and \(\sum \vdash \alpha\), then \(\sum \vdash \varphi \wedge \alpha\)
\begin{itemize}
\item $$ \frac{\varphi \text{    } \alpha }{\varphi \wedge 
\alpha}$$
\end{itemize}

\item \(\wedge\)- Elimination Rule : If \(\sum \vdash \varphi \wedge \alpha. \text{ then } \sum \vdash \varphi \text{ and } \sum \vdash \alpha \)
\begin{itemize}
\item $$\frac{\varphi \wedge \alpha}{\varphi}  \frac{\varphi \wedge \alpha}{\alpha}$$
\end{itemize}

\item \(\rightarrow\)-elimination rule : If \(sum \vdash \varphi \rightarrow \alpha \) and \( \sum \vdash \varphi\),then \(\sum \vdash a\)
\begin{itemize}
\item $$\frac{\varphi \rightarrow \alpha \text{    } \varphi}{\alpha}$$
\end{itemize}

\item \(\rightarrow\)-Introduction rule : If \(\sum \cup \{ \varphi \} \vdash \alpha\), then \(\sum \vdash \varphi \rightarrow \alpha\)
\begin{itemize}
\item $$ \frac{\begin{matrix} \varphi \\ \vdots \\ \alpha \end{matrix}}{\varphi \rightarrow \alpha}$$
\end{itemize}
\end{itemize}

\end{document}
