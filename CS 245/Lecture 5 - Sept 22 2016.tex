%Notes by Harsh Mistry 
%CS 245
%based on Template from : https://www.cs.cmu.edu/~ggordon/10725-F12/template.tex

\documentclass{article}
\setlength{\oddsidemargin}{0.25 in}
\setlength{\evensidemargin}{-0.25 in}
\setlength{\topmargin}{-0.6 in}
\setlength{\textwidth}{6.5 in}
\setlength{\textheight}{8.5 in}
\setlength{\headsep}{0.75 in}
\setlength{\parindent}{0 in}
\setlength{\parskip}{0.1 in}
\usepackage{amsfonts,graphicx, amssymb}
\usepackage[fleqn]{amsmath}
\usepackage{fixltx2e}
\usepackage{xcolor}
\usepackage{color}
\usepackage{tcolorbox}
\usepackage{lipsum}
\usepackage{listings}
\usepackage{scrextend}
\usepackage{enumerate}
\tcbuselibrary{skins,breakable}
\usetikzlibrary{shadings,shadows}
\newcounter{lecnum}
\renewcommand{\thepage}{\thelecnum-\arabic{page}}
\renewcommand{\thesection}{\thelecnum.\arabic{section}}
\renewcommand{\theequation}{\thelecnum.\arabic{equation}}
\renewcommand{\thefigure}{\thelecnum.\arabic{figure}}
\renewcommand{\thetable}{\thelecnum.\arabic{table}}
\newcommand{\lecture}[4]{
   \pagestyle{myheadings}
   \thispagestyle{plain}
   \newpage
   \setcounter{lecnum}{#1}
   \setcounter{page}{1}
   
   
%Info Box 
   \begin{center}
   \framebox{
      \vbox{\vspace{2mm}
    \hbox to 6.28in { {\bf CS 245 - Logic and Computation 
	\hfill Fall 2016} }
       \vspace{4mm}
       \hbox to 6.28in { {\Large \hfill Lecture #1: #2  \hfill} }
       \vspace{2mm}
       \hbox to 6.28in { {\it Lecturer: #3 \hfill Notes By: #4} }
      \vspace{2mm}}
   }
   \end{center}
   
   \markboth{Lecture #1: #2}{Lecture #1: #2}



 
}

\renewcommand{\cite}[1]{[#1]}
\def\beginrefs{\begin{list}%
        {[\arabic{equation}]}{\usecounter{equation}
         \setlength{\leftmargin}{2.0truecm}\setlength{\labelsep}{0.4truecm}%
         \setlength{\labelwidth}{1.6truecm}}}
\def\endrefs{\end{list}}
\def\bibentry#1{\item[\hbox{[#1]}]}

\newcommand{\fig}[3]{
			\vspace{#2}
			\begin{center}
			Figure \thelecnum.#1:~#3
			\end{center}
	}

\newtheorem{theorem}{Theorem}[lecnum]
\newtheorem{lemma}[theorem]{Lemma}
\newtheorem{ex}[theorem]{Example}
\newtheorem{proposition}[theorem]{Proposition}
\newtheorem{claim}[theorem]{Claim}
\newtheorem{corollary}[theorem]{Corollary}
\newtheorem{definition}[theorem]{Definition}
\newenvironment{proof}{{\bf Proof:}}{\hfill\rule{2mm}{2mm}}
\newcommand\E{\mathbb{E}}

%color definitions :
\definecolor{darkred}{rgb}{0.55, 0.0, 0.0}
\definecolor{lightcoral}{rgb}{0.94, 0.5, 0.5}
\definecolor{tomato}{rgb}{1.0, 0.39, 0.28}
\definecolor{lightgray}{rgb}{.9,.9,.9}
\definecolor{darkgray}{rgb}{.4,.4,.4}
\definecolor{purple}{rgb}{0.65, 0.12, 0.82}
\definecolor{lightgreen}{rgb}{0.56, 0.93, 0.56}
\definecolor{darkgreen}{rgb}{0.0, 0.2, 0.13}
\definecolor{limegreen}{rgb}{0.2, 0.8, 0.2}
\definecolor{lightblue}{rgb}{0.68, 0.85, 0.9}
\definecolor{darkblue}{rgb}{0.0, 0.0, 0.55}


%Environments
\newenvironment{exblock}[1]{%
    \tcolorbox[beamer,%
    noparskip,breakable,
    colback=lightgreen,colframe=darkgreen,%
    colbacklower=limegreen!75!lightgreen,%
    title=#1]}%
    {\endtcolorbox}

\newenvironment{ablock}[1]{%
    \tcolorbox[beamer,%
    noparskip,breakable,
    colback=lightcoral,colframe=darkred,%
    colbacklower=tomato!75!lightcoral,%
    title=#1]}%
    {\endtcolorbox}

\newenvironment{cblock}[1]{%
    \tcolorbox[beamer,%
    noparskip,breakable,
    colback=lightblue,colframe=darkblue,%
    colbacklower=darkblue!75!lightblue,%
    title=#1]}%
    {\endtcolorbox}


%Languages
\lstdefinelanguage{JavaScript}{
  keywords={typeof, new, true, false, catch, function, return, null, catch, switch, var, if, in, while, do, else, case, break},
  keywordstyle=\color{blue}\bfseries,
  ndkeywords={class, export, boolean, throw, implements, import, this},
  ndkeywordstyle=\color{darkgray}\bfseries,
  identifierstyle=\color{black},
  sensitive=false,
  comment=[l]{//},
  morecomment=[s]{/*}{*/},
  commentstyle=\color{purple}\ttfamily,
  stringstyle=\color{red}\ttfamily,
  morestring=[b]',
  morestring=[b]"
}

%Listings
\lstset{
   language=JavaScript,
   backgroundcolor=\color{lightgray},
   extendedchars=true,
   basicstyle=\footnotesize\ttfamily,
   showstringspaces=false,
   showspaces=false,
   numbers=left,
   numberstyle=\footnotesize,
   numbersep=9pt,
   tabsize=2,
   breaklines=true,
   showtabs=false,
   captionpos=b
}


%Start of Document 
\begin{document}

\lecture{5}{September 22, 2016}{Kevin Lanctot}{Harsh Mistry}

\section{Definability of Connectives}

Formulas \(\alpha \implies \beta\) and \(\neg \alpha \vee \beta\) are equivalent. This, \(\implies\) is said to be \textbf{Definable} in terms of \(\neg\) and \(\vee\)

\section{Adequate Sets}
A set of connectives is said to be adequate iff any n-ary \((n \geq 1)\)
connective can be defined in terms of the ones in the set.

\section{Proof In Propositional Logic : Resolution}
We notate “there is a proof with assumptions \(sum\) and conclusion \(\varphi\) by
\begin{center}
\(\sum \vdash \varphi\)
\end{center}
We can be read as \(\sum\) proves \(\varphi\)

\subsection{Inference Rules}
In general, an inference rule is written as  : \(\frac{\alpha_1 \alpha_2 \ldots \alpha_i}{\beta}\)

The notation means if \(\alpha_1 \alpha_2 \ldots \alpha_i\) already appears in the proof, then one may infer the formula \(\beta\)

\subsection{Approaches}
\begin{itemize}
\item Direct Proofs : Establish \(\sum \vdash \varphi\) by stating the assumptions and from their derive \(\varphi\)
\item Refutation (Contradiction) : Give a direct proof of \(\sum \cup \{ \neg \varphi\} \models \perp\) 
\end{itemize}

\subsection{The "Resolution" System and Rule}
Resolution is a refutation system, with the following inference rule:
$$ \frac{(\alpha \vee p) (\neg p \vee \beta)}{a \vee B} $$

for any variable p and formulas \(\alpha\) and \(\beta\).

We consider the following as special cases:
$$\textbf{Unit Resolution (Eliminate P) } \frac{(\alpha \vee P) (\neg p)}{\alpha} $$
$$\textbf{Contradiction (Refuation is complete) } \frac{p (\neg p)}{\perp} $$

A proof is complete when one derives a contradiction \(\perp\).
In this case, the original assumptions are refuted.

\subsection{Connective Normal Form}

The Resolution rule can only be used successfully on formulas of a
restricted form.

\textbf{CNF :}
\begin{itemize}
\item A \textbf{Literal} is a variable or the negation of a variable 
\item A \textbf{Clause} is a disjunction of literals
\item A formula in \textbf{CNF} if it is a conjunction of clauses
\end{itemize}

In essence, a formula is in CNF if and only if 
\begin{itemize}
\item its only connectives are \(\neg, \vee, \textbf{ and/or } \wedge,\)
\item \(\neg\) applies only to variables, and 
\item \(\vee\) applies only to sub formulas with no occurrence of \(\wedge\)
\end{itemize}

\subsection{Converting to CNF}
\begin{enumerate}
\item Eliminate Implication and Equivalence 
\begin{itemize}
\item Replace \(\alpha \implies \beta\) by \(\neg \alpha \vee \beta\)
\item Replace \(\alpha \iff \beta\) by \((\neg \alpha \vee \beta) \wedge (\alpha \vee \neg \beta)\)
\end{itemize}
\item Apply De Morgan’s and double-negation laws as often as
possible.
\begin{itemize}
\item Replace \(\neg (\alpha \vee \beta)\) with \(\neg \alpha \wedge  \neg \beta\)
\item Replace \(\neg (\alpha \wedge \beta)\) with \(\neg \alpha \vee  \neg \beta\)
\item Replace \(\neg \neg \alpha\) with \(\alpha\)
\end{itemize}
\item Transform into a conjunction of clauses using distributivity 
\begin{itemize}
\item Replace \((\alpha \vee (\beta \wedge \gamma))\) with \((\alpha \vee \beta) \wedge (\alpha \vee \alpha)\)
\end{itemize}
\item Simplify using idempotence, contradiction, excluded middle and
Simplification I \& II.
\end{enumerate}

\subsection{The Resolution Proof Procedure}
\begin{enumerate}
\item Convert each formula in \(\sum\) to CNF
\item Convert \(\neg \varphi\) to CNF
\item Split the CNF formulas at the \(\wedge\)'s, yielding a set of clauses 
\item Form the resulting set of clauses, keep applying the resolution until either :
\begin{itemize}
\item The empty clause \(\perp\) results. In this case \(\varphi\) is a theorem 
\item The rule can no longer be applies to give a new formula. In this case, \(\varphi\) is not a theorem 
\end{itemize}
\end{enumerate}

\end{document}
