%Notes by Harsh Mistry 
%Math 128
%based on template from : https://www.cs.cmu.edu/~ggordon/10725-F12/

\documentclass{article}
\setlength{\oddsidemargin}{0.25 in}
\setlength{\evensidemargin}{-0.25 in}
\setlength{\topmargin}{-0.6 in}
\setlength{\textwidth}{6.5 in}
\setlength{\textheight}{8.5 in}
\setlength{\headsep}{0.75 in}
\setlength{\parindent}{0 in}
\setlength{\parskip}{0.1 in}
\usepackage{amsfonts,graphicx, amssymb}
\usepackage[fleqn]{amsmath}
\usepackage{blindtext}
\newcounter{lecnum}
\renewcommand{\thepage}{\thelecnum-\arabic{page}}
\renewcommand{\thesection}{\thelecnum.\arabic{section}}
\renewcommand{\theequation}{\thelecnum.\arabic{equation}}
\renewcommand{\thefigure}{\thelecnum.\arabic{figure}}
\renewcommand{\thetable}{\thelecnum.\arabic{table}}
\newcommand{\lecture}[4]{
   \pagestyle{myheadings}
   \thispagestyle{plain}
   \newpage
   \setcounter{lecnum}{#1}
   \setcounter{page}{1}
   
   
%Info Box 
   \begin{center}
   \framebox{
      \vbox{\vspace{2mm}
    \hbox to 6.28in { {\bf Math 128 :  Calculus 2 for the Sciences
	\hfill Winter 2016} }
       \vspace{4mm}
       \hbox to 6.28in { {\Large \hfill Lecture #1: #2  \hfill} }
       \vspace{2mm}
       \hbox to 6.28in { {\it Lecturer: #3 \hfill Notes By: #4} }
      \vspace{2mm}}
   }
   \end{center}
   
   \markboth{Lecture #1: #2}{Lecture #1: #2}



 
}

\renewcommand{\cite}[1]{[#1]}
\def\beginrefs{\begin{list}%
        {[\arabic{equation}]}{\usecounter{equation}
         \setlength{\leftmargin}{2.0truecm}\setlength{\labelsep}{0.4truecm}%
         \setlength{\labelwidth}{1.6truecm}}}
\def\endrefs{\end{list}}
\def\bibentry#1{\item[\hbox{[#1]}]}

\newcommand{\fig}[3]{
			\vspace{#2}
			\begin{center}
			Figure \thelecnum.#1:~#3
			\end{center}
	}

\newtheorem{theorem}{Theorem}[lecnum]
\newtheorem{lemma}[theorem]{Lemma}
\newtheorem{ex}[theorem]{Example}
\newtheorem{proposition}[theorem]{Proposition}
\newtheorem{claim}[theorem]{Claim}
\newtheorem{corollary}[theorem]{Corollary}
\newtheorem{definition}[theorem]{Definition}
\newenvironment{proof}{{\bf Proof:}}{\hfill\rule{2mm}{2mm}}
\newcommand\E{\mathbb{E}}


%Start of Document 
\begin{document}
\lecture{17-20}{February 10 - 24 , 2016}{Jen Nelson}{Harsh Mistry}

%
\section{Parametric Equations}
Up to now we have defined curves by 
\begin{itemize}
\item y = f(x), x = f(y)
\item F(x,y) = 0 
\end{itemize}

For many applications this repersentation is not ideal (or even posssible) \\
Another way to define a curve : \\
Suppose x and y are given as functions of a third variable \\
\[x = f(t)\]
\[y = g(t)\]
t us the parameter that links x and y 


\section{Tangents}
Given a curve with parametric equations x = f(t) , y = g(t), can we find \(\frac{dy}{dx} \)
\[\frac{dy}{dt} = \frac{dy}{dt} = \frac{dy}{dx} \frac{dx}{dt}\]
\[\frac{dy}{dx} = \frac{\frac{dy}{dt}}{\frac{dx}{dt}} \text{ provided } \frac{dx}{dt} \neq 0 \]

So the curve will have: 
\begin{itemize}
\item Horizontal tangents where \(\frac{dy}{dt} = 0 \text{ and }    \frac{dx}{dt} \neq 0\)
\item vertical tangents where \(\frac{dx}{dt} = 0 \text{ and } \frac{dy}{dt} \neq 0 \)
\end{itemize}

If both \(\frac{dy}{dt} =0 \) and \(\frac{dx}{dt} = 0 \) at some \(t^*\), then you need to look at \(\lim_{t\to t^*} \frac{\frac{dy}{dt}}{\frac{dx}{dt}}\)

\section{Area} 
Given \(y = g(t) , x = f(t) , \alpha \leq t \leq \beta \)
\[ A = \int_a^b F(x)dx = \int_a^b ydx = \int_{\alpha}^{\beta} g(t) f'(t)dt \]


\section{Arc Length}
If C is a curve describe by parametric equations \(y = g(t) , x = f(t) , \alpha \leq t \leq \beta \), and C is travered exactly once as t increases from \(\alpha \to \beta \), then the length of the curve is given by 
\[ L = \int_{\alpha}^{\beta} \sqrt{(\frac{dx}{dt})^2 + (\frac{dy}{dt})^2} dt\]

\section{Polar Coordinates}
Given point (x,y), \((r, \theta)\) is a polar coordinante\\
\begin{itemize}
\item r is the distance from O (origin) to P (point) 
\item \(\theta\) is the angle OP makes with the polar axis 
\end{itemize}

\section{Polar Coordinates \(\longleftrightarrow\) Cartesian coodinates}
In general, if P has polar coordinates \((r,\theta)\), its cartiesian coordinates are given by, 
$$ x = r \cos \theta  ,\ \ y = r \sin \theta $$

In general, if P has Cartiesian coordinates (x,y), its polar coordinates \((r,\theta)\) are given by, 
$$ r^2 = x^2 + y^2,\ \ \tan \theta = \frac{y}{x}$$

\section{Symmetry}
if the polar equation is unchanged when 
\begin{itemize}
\item  \(\theta\) is replaced by \(-\theta\) : Then the equation is symmetric about polar axis
\item \(\theta\) is replaced by \(\pi - \theta\) : Then the equation is symmetric about vertical line \(\theta = \frac{\pi}{2}\) 
\end{itemize}

\begin{center}
\textbf{End of Lecture Notes} \\
\textbf{Notes By : Harsh Mistry}
\end{center}
\end{document}
