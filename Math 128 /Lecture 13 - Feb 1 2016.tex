%Notes by Harsh Mistry 
%Math 128
%based on template from : https://www.cs.cmu.edu/~ggordon/10725-F12/

\documentclass{article}
\setlength{\oddsidemargin}{0.25 in}
\setlength{\evensidemargin}{-0.25 in}
\setlength{\topmargin}{-0.6 in}
\setlength{\textwidth}{6.5 in}
\setlength{\textheight}{8.5 in}
\setlength{\headsep}{0.75 in}
\setlength{\parindent}{0 in}
\setlength{\parskip}{0.1 in}
\usepackage{amsfonts,graphicx, amssymb}
\usepackage[fleqn]{amsmath}
\usepackage{blindtext}
\newcounter{lecnum}
\renewcommand{\thepage}{\thelecnum-\arabic{page}}
\renewcommand{\thesection}{\thelecnum.\arabic{section}}
\renewcommand{\theequation}{\thelecnum.\arabic{equation}}
\renewcommand{\thefigure}{\thelecnum.\arabic{figure}}
\renewcommand{\thetable}{\thelecnum.\arabic{table}}
\newcommand{\lecture}[4]{
   \pagestyle{myheadings}
   \thispagestyle{plain}
   \newpage
   \setcounter{lecnum}{#1}
   \setcounter{page}{1}
   
   
%Info Box 
   \begin{center}
   \framebox{
      \vbox{\vspace{2mm}
    \hbox to 6.28in { {\bf Math 128 :  Calculus 2 for the Sciences
	\hfill Winter 2016} }
       \vspace{4mm}
       \hbox to 6.28in { {\Large \hfill Lecture #1: #2  \hfill} }
       \vspace{2mm}
       \hbox to 6.28in { {\it Lecturer: #3 \hfill Notes By: #4} }
      \vspace{2mm}}
   }
   \end{center}
   
   \markboth{Lecture #1: #2}{Lecture #1: #2}



 
}

\renewcommand{\cite}[1]{[#1]}
\def\beginrefs{\begin{list}%
        {[\arabic{equation}]}{\usecounter{equation}
         \setlength{\leftmargin}{2.0truecm}\setlength{\labelsep}{0.4truecm}%
         \setlength{\labelwidth}{1.6truecm}}}
\def\endrefs{\end{list}}
\def\bibentry#1{\item[\hbox{[#1]}]}

\newcommand{\fig}[3]{
			\vspace{#2}
			\begin{center}
			Figure \thelecnum.#1:~#3
			\end{center}
	}

\newtheorem{theorem}{Theorem}[lecnum]
\newtheorem{lemma}[theorem]{Lemma}
\newtheorem{ex}[theorem]{Example}
\newtheorem{proposition}[theorem]{Proposition}
\newtheorem{claim}[theorem]{Claim}
\newtheorem{corollary}[theorem]{Corollary}
\newtheorem{definition}[theorem]{Definition}
\newenvironment{proof}{{\bf Proof:}}{\hfill\rule{2mm}{2mm}}
\newcommand\E{\mathbb{E}}


%Start of Document 
\begin{document}

\lecture{13}{February 1, 2016}{Jen Nelson}{Harsh Mistry}

%
\section{Differential equations}
A differential equations (DE) is an equation involving an unknown function and one or more of its derivatives

\begin{ex}
\[\frac{dp}{dt} = kp \leftarrow \text{ Law of Natural Growth} \]
\[\frac{dp}{dt}  = kp ( 1- \frac{p}{m}) \leftarrow  \text{ Logistic Model for Growth}\]
\[ \frac{md^2x}{dt^2} = -kx \leftarrow  \text{ Spring Motion} \]
\[\frac{d^2\theta}{dt^2} + \frac{g}{r} \sin \theta = 0  \leftarrow \text{ Motion of a Pendulum} \]
\end{ex}

\begin{itemize}
\item The order of a DE is the order of the highest derivative 
\item By solving a DE we mean finding the function which makes the equation
\end{itemize}

\begin{ex}
Verify that y = \( \frac{x^2}{2} + 4x \) is a solution to the DE \( \frac{dy}{dx} = x^2 + \frac{y}{x} \) \\ 
To verify, simply just check the left and right side. 
\end{ex}

\begin{ex}
Solve \(\frac{dy}{dx} = 6x^2 + 2x \) \\
\[ y = 2x^3 + x^2 + C \]
\end{ex}

\begin{ex}
Solve \(\frac{dy}{dx} = y \) \\
\[ y = e^x\]
But there can be can multiple solutions to this DE. So we reprsent the solution using a general solution. These general solutions repersent a family of solutions. The general solution for this DE is : 
\[ y = ce^x \]
\end{ex}



\begin{center}
\textbf{End of Lecture Notes} \\
\textbf{Notes By : Harsh Mistry}
\end{center}
\end{document}
