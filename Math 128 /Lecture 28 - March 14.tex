%Notes by Harsh Mistry 
%Math 128
%Based on template from: https://www.cs.cmu.edu/~ggordon/10725-F12/template.tex

\documentclass{article}
\setlength{\oddsidemargin}{0.25 in}
\setlength{\evensidemargin}{-0.25 in}
\setlength{\topmargin}{-0.6 in}
\setlength{\textwidth}{6.5 in}
\setlength{\textheight}{8.5 in}
\setlength{\headsep}{0.75 in}
\setlength{\parindent}{0 in}
\setlength{\parskip}{0.1 in}
\usepackage{amsfonts,graphicx, amssymb}
\usepackage[fleqn]{amsmath}
\usepackage{enumerate}
\usepackage{blindtext}
\newcounter{lecnum}
\renewcommand{\thepage}{\thelecnum-\arabic{page}}
\renewcommand{\thesection}{\thelecnum.\arabic{section}}
\renewcommand{\theequation}{\thelecnum.\arabic{equation}}
\renewcommand{\thefigure}{\thelecnum.\arabic{figure}}
\renewcommand{\thetable}{\thelecnum.\arabic{table}}
\newcommand{\lecture}[4]{
   \pagestyle{myheadings}
   \thispagestyle{plain}
   \newpage
   \setcounter{lecnum}{#1}
   \setcounter{page}{1}
   
   
%Info Box 
   \begin{center}
   \framebox{
      \vbox{\vspace{2mm}
    \hbox to 6.28in { {\bf Math 128 :  Calculus 2 for the Sciences
	\hfill Winter 2016} }
       \vspace{4mm}
       \hbox to 6.28in { {\Large \hfill Lecture #1: #2  \hfill} }
       \vspace{2mm}
       \hbox to 6.28in { {\it Lecturer: #3 \hfill Notes By: #4} }
      \vspace{2mm}}
   }
   \end{center}
   
   \markboth{Lecture #1: #2}{Lecture #1: #2}



 
}

\renewcommand{\cite}[1]{[#1]}
\def\beginrefs{\begin{list}%
        {[\arabic{equation}]}{\usecounter{equation}
         \setlength{\leftmargin}{2.0truecm}\setlength{\labelsep}{0.4truecm}%
         \setlength{\labelwidth}{1.6truecm}}}
\def\endrefs{\end{list}}
\def\bibentry#1{\item[\hbox{[#1]}]}

\newcommand{\fig}[3]{
			\vspace{#2}
			\begin{center}
			Figure \thelecnum.#1:~#3
			\end{center}
	}

\newtheorem{theorem}{Theorem}[lecnum]
\newtheorem{lemma}[theorem]{Lemma}
\newtheorem{ex}[theorem]{Example}
\newtheorem{proposition}[theorem]{Proposition}
\newtheorem{claim}[theorem]{Claim}
\newtheorem{corollary}[theorem]{Corollary}
\newtheorem{definition}[theorem]{Definition}
\newenvironment{proof}{{\bf Proof:}}{\hfill\rule{2mm}{2mm}}
\newcommand\E{\mathbb{E}}


%Start of Document 
\begin{document}

\lecture{28}{March 14, 2016}{Jen Nelson}{Harsh Mistry}

%
\section{Absolute/Conditional Convergence}
\begin{definition}
A series \(\sum_{n=1}^{\infty}\) is 
\begin{itemize}
\item Convergent if \(\sum a_n = a_1 + a_2 + \ldots \) is finite 
\item Absolutely convergent if the series of absolute values \(\sum \mid a_n \mid \) is convergent 
\item Conditionally convergent if it is convergent but not absolutely convergent : \(\sum a_n\) converges, but \(\sum \mid a_n \mid \) does not 
\end{itemize}
\end{definition}

\begin{theorem}
Absolute Convergence \(\implies\) Convergence 
\end{theorem}

Why? 
\begin{itemize}
\item If an alternating series converges absolutely, there is no need for AST 
\item Any rearrangements of the series results in the same sum
\item Gives us a way to test convergence of a series that has negative terms, but is not alternating 
\end{itemize}

\section{Ratio Test}
\begin{theorem}- \\
\begin{enumerate} [(a)]
\item If \(\lim_{n \to \infty} |\frac{a_n+1}{a_n}| = L < 1\), then \(\sum a_n\) is absolutely convergent 
\item If \(\lim_{n \to \infty} |\frac{a_n+1}{a_n}| = L > 1\) or  \(\lim_{n \to \infty} |\frac{a_n+1}{a_n}| = \infty\), then \(\sum a_n\) diverges 
\item If \(\lim_{n \to \infty} |\frac{a_n+1}{a_n}| = 1\), then \textbf{NO CONCLUSION} about the convergence or divergence of \(\sum a_n\) can be made 
\end{enumerate}
\end{theorem}

\begin{center}
\textbf{End of Lecture Notes} \\
\textbf{Notes By : Harsh Mistry}
\end{center}
\end{document}
