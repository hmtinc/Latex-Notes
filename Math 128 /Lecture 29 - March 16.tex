%Notes by Harsh Mistry 
%Math 128
%Based on template from: https://www.cs.cmu.edu/~ggordon/10725-F12/template.tex

\documentclass{article}
\setlength{\oddsidemargin}{0.25 in}
\setlength{\evensidemargin}{-0.25 in}
\setlength{\topmargin}{-0.6 in}
\setlength{\textwidth}{6.5 in}
\setlength{\textheight}{8.5 in}
\setlength{\headsep}{0.75 in}
\setlength{\parindent}{0 in}
\setlength{\parskip}{0.1 in}
\usepackage{amsfonts,graphicx, amssymb}
\usepackage[fleqn]{amsmath}
\usepackage{enumerate}
\usepackage{blindtext}
\newcounter{lecnum}
\renewcommand{\thepage}{\thelecnum-\arabic{page}}
\renewcommand{\thesection}{\thelecnum.\arabic{section}}
\renewcommand{\theequation}{\thelecnum.\arabic{equation}}
\renewcommand{\thefigure}{\thelecnum.\arabic{figure}}
\renewcommand{\thetable}{\thelecnum.\arabic{table}}
\newcommand{\lecture}[4]{
   \pagestyle{myheadings}
   \thispagestyle{plain}
   \newpage
   \setcounter{lecnum}{#1}
   \setcounter{page}{1}
   
   
%Info Box 
   \begin{center}
   \framebox{
      \vbox{\vspace{2mm}
    \hbox to 6.28in { {\bf Math 128 :  Calculus 2 for the Sciences
	\hfill Winter 2016} }
       \vspace{4mm}
       \hbox to 6.28in { {\Large \hfill Lecture #1: #2  \hfill} }
       \vspace{2mm}
       \hbox to 6.28in { {\it Lecturer: #3 \hfill Notes By: #4} }
      \vspace{2mm}}
   }
   \end{center}
   
   \markboth{Lecture #1: #2}{Lecture #1: #2}



 
}

\renewcommand{\cite}[1]{[#1]}
\def\beginrefs{\begin{list}%
        {[\arabic{equation}]}{\usecounter{equation}
         \setlength{\leftmargin}{2.0truecm}\setlength{\labelsep}{0.4truecm}%
         \setlength{\labelwidth}{1.6truecm}}}
\def\endrefs{\end{list}}
\def\bibentry#1{\item[\hbox{[#1]}]}

\newcommand{\fig}[3]{
			\vspace{#2}
			\begin{center}
			Figure \thelecnum.#1:~#3
			\end{center}
	}

\newtheorem{theorem}{Theorem}[lecnum]
\newtheorem{lemma}[theorem]{Lemma}
\newtheorem{ex}[theorem]{Example}
\newtheorem{proposition}[theorem]{Proposition}
\newtheorem{claim}[theorem]{Claim}
\newtheorem{corollary}[theorem]{Corollary}
\newtheorem{definition}[theorem]{Definition}
\newenvironment{proof}{{\bf Proof:}}{\hfill\rule{2mm}{2mm}}
\newcommand\E{\mathbb{E}}


%Start of Document 
\begin{document}

\lecture{29}{March 16, 2016}{Jen Nelson}{Harsh Mistry}

%
\section{Ratio Test Con't}
\textbf{Notes:}
\begin{itemize}
\item Works well when \(a_n\) contains \(n!\) or \((constant)^n\)
\item Does not work well if \(a_n\) is rational 
\item Does not require all positive terms
\item The term that goes in the numerator matters
\end{itemize}

\section{Power Series}
A power series is an infinite series of the form 
$$ \sum_{n=0}^{\infty} c_nx^n = c_0 + c_1x + c_2x^2 + c_3x^3 + \ldots $$
where x is a variable and c's are constants, called the coefficients of the series.\\
More generally, a power series centered at a or about a is 
$$ \sum_{n=0}^{\infty} c_n(x-a)^n = c_0 c_1(x-a) + c_2(x-a)^2 + c_3(x-a)^3 + \ldots $$
Also, we know the geometric series \(\sum x^{n-1} \) is a power series which diverges when \(|x| \leq 1\)\\
So this tells that \(\sum x^{n-1} = \frac{1}{1 - x}\) on the interval (-1,1)\\

We say that (-1,1) is the interval of convergence and R = 1 is the radius of convergence for the series \\

For any power series we are often interested in determining for which values of x the power series converges. We can use the ratio test to do so. 

\begin{ex}
For what values of x does \(\sum_{n=0}^{\infty} \frac{1}{n+1}(x-2)^n\) converge?
\begin{itemize}
\item Using the ratio test we get \(\lim_{n \to \infty} = |x-2|\)
\item It converges absolutely if \(|x-2| < 2 \iff 1 < x < 3\)
\item It diverges if \(|x-2| > 1 \iff x > 3 or x < 1 \) 
\item At x = 1, the series converges and At x = 3, the series diverges
\item So, the power series converges for \(1 \leq x < 3\) 
\end{itemize}

\textbf{Thus,} \\
The Interval of convergence is [1,3)\\
The Radius of convergence is 1 \\
The center of convergence is 2\\
\end{ex}

\begin{center}
\textbf{End of Lecture Notes} \\
\textbf{Notes By : Harsh Mistry}
\end{center}
\end{document}
