%Notes by Harsh Mistry 
%Math 128
%Template Taken from : https://www.cs.cmu.edu/~ggordon/10725-F12/template.tex

\documentclass{article}
\setlength{\oddsidemargin}{0.25 in}
\setlength{\evensidemargin}{-0.25 in}
\setlength{\topmargin}{-0.6 in}
\setlength{\textwidth}{6.5 in}
\setlength{\textheight}{8.5 in}
\setlength{\headsep}{0.75 in}
\setlength{\parindent}{0 in}
\setlength{\parskip}{0.1 in}
\usepackage{amsfonts,graphicx, amssymb}
\usepackage[fleqn]{amsmath}
\usepackage{blindtext}
\newcounter{lecnum}
\renewcommand{\thepage}{\thelecnum-\arabic{page}}
\renewcommand{\thesection}{\thelecnum.\arabic{section}}
\renewcommand{\theequation}{\thelecnum.\arabic{equation}}
\renewcommand{\thefigure}{\thelecnum.\arabic{figure}}
\renewcommand{\thetable}{\thelecnum.\arabic{table}}
\newcommand{\lecture}[4]{
   \pagestyle{myheadings}
   \thispagestyle{plain}
   \newpage
   \setcounter{lecnum}{#1}
   \setcounter{page}{1}
   
   
%Info Box 
   \begin{center}
   \framebox{
      \vbox{\vspace{2mm}
    \hbox to 6.28in { {\bf Math 128 :  Calculus 2 for the Sciences
	\hfill Winter 2016} }
       \vspace{4mm}
       \hbox to 6.28in { {\Large \hfill Lecture #1: #2  \hfill} }
       \vspace{2mm}
       \hbox to 6.28in { {\it Lecturer: #3 \hfill Notes By: #4} }
      \vspace{2mm}}
   }
   \end{center}
   
   \markboth{Lecture #1: #2}{Lecture #1: #2}



 
}

\renewcommand{\cite}[1]{[#1]}
\def\beginrefs{\begin{list}%
        {[\arabic{equation}]}{\usecounter{equation}
         \setlength{\leftmargin}{2.0truecm}\setlength{\labelsep}{0.4truecm}%
         \setlength{\labelwidth}{1.6truecm}}}
\def\endrefs{\end{list}}
\def\bibentry#1{\item[\hbox{[#1]}]}

\newcommand{\fig}[3]{
			\vspace{#2}
			\begin{center}
			Figure \thelecnum.#1:~#3
			\end{center}
	}

\newtheorem{theorem}{Theorem}[lecnum]
\newtheorem{lemma}[theorem]{Lemma}
\newtheorem{ex}[theorem]{Example}
\newtheorem{proposition}[theorem]{Proposition}
\newtheorem{claim}[theorem]{Claim}
\newtheorem{corollary}[theorem]{Corollary}
\newtheorem{definition}[theorem]{Definition}
\newenvironment{proof}{{\bf Proof:}}{\hfill\rule{2mm}{2mm}}
\newcommand\E{\mathbb{E}}


%Start of Document 
\begin{document}

\lecture{4}{January 11, 2016}{Jen Nelson}{Harsh Mistry}

%
\section{Half Angle Identities}
\begin{itemize}
\item \( \cos ^2 x = \frac{1}{2} (1 + \cos 2x) \)
\item \( \sin ^2 x  = \frac{1}{2} (1 - \cos 2x) \)
\end{itemize}

\begin{ex}
\[ \begin{aligned} \int_{0}^{\frac{\pi}{3}} \cos ^2 x dx & = \int_{0}^{\frac{\pi}{3}} \frac{1}{2} (1 + \cos 2x) dx \\
& = \frac{1}{2} (x + \frac{\sin 2x}{2}) \mid_{0}^{\frac{\pi}{3}}  \\
& = \frac{1}{2} (\frac{\pi}{3} + \frac{\frac{\sqrt{3}}{2}}{2} ) \\
& = \frac{\pi}{6} + \frac{\sqrt{3}}{8}\end{aligned} \]
\end{ex}

\begin{ex}
\[ \begin{aligned} \int \sin ^2x  \cos ^2x dx\ & = \frac{1}{4} \int (1 - \cos 2x) (1 + \cos 2x) dx \\
& =  \frac{1}{4} \int (1 - \cos ^2 2x) \\
& = \frac{1}{4} \int (1 - \frac{1}{2} ( 1+ cos ^4 x)) dx \\
& = \frac{1}{4} \int (\frac{1}{2} - \frac{1}{2} \cos 4x) \\
& = \frac{1}{8} ( x - \frac{\sin 4x}{4}) + C \end{aligned} \]
\end{ex}

\section{Integrals Involving Tan and Sec} 
$$ \int \tan ^m x \sec ^n x dx , m , n \in \mathbb{Z} , m , n \geq 0  $$

\newpage

\textbf{Known Derivatives}
$$ \frac{d}{dx} \tan  x = \sec ^2 x  \ \ \ \frac{d}{dx} \sec x = \sec x \tan x $$
$$ \frac{d}{dx} \tan ^m x = m \tan ^{m-1} x \sec ^2 x \ \ \ \ \frac{d}{dx} \sec ^m = n \sec ^{n-1} x \sec x \tan x = n \sec ^n x \tan x $$

\begin{itemize}
\item Try to manipulate expression, so we have \( \tan x \text{ or} \sec ^2 x \)
\item Use identity \( 1 + \tan ^2 x = \sec ^2 x \) 
\end{itemize}

\begin{ex} - \\
\begin{minipage}{.5\textwidth} %
$$
\begin{aligned}
\int \sec ^9 x \tan ^5 x dx & = \int \sec ^9 x \tan ^4 x \tan x dx \\
& = \int \sec ^9 x (\sec ^2 x -1) ^2 \tan x dx \\
& = \int \sec ^9 x (\sec 3x - 1) ^2 \tan x dx \\
& = \int u^8 (u^2 -1 ) ^2 du \\
& = \int u^8 (u^4 - 2u^2 + 1) du \\
& = \frac{u^13}{13} - \frac{2u^11}{11} + \frac{u^9 }{8 } + C \\
& = \frac{\sec ^13 x}{13} - \frac{2\sec ^11 x}{11} + \frac{\sec ^9 x}{9} + C 
\end{aligned}
$$
\end{minipage} %
\begin{minipage}{.5\textwidth} %
$$ u = \sec x \implies du = \sec x \tan x $$
\end{minipage}
\end{ex}

\begin{ex} - \\
\begin{minipage}{.5\textwidth} %
$$
\begin{aligned}
\int \sec ^4 x \tan ^2 x dx & = \int \sec 3x \tan ^2 x \sec ^2 x dx \\ 
& = \int (1 + tan^2 x ) \tan 3x \sec ^2 x dx \\
& = \int ( 1 + u^2) u^2 du \\
& = \frac{u^3}{3} + \frac{u^5}{5} + C \\
& = \frac{\tan ^3 x}{3} + \frac{ \tan ^5 x}{5} + C 
\end{aligned}
$$
\end{minipage} %
\begin{minipage}{.5\textwidth} %
$$ u = \tan x \implies  du = \sec ^2 x$$
\end{minipage}
\end{ex}

\begin{ex} - \\
\begin{minipage}{.5\textwidth} %
$$
\begin{aligned}
\int \sec ^3 x dx & = \sec x \tan x - \int \tan ^2 x \sec x dx \\
& = \sec x \tan x  - \int (\sec ^2 x - 1) \sec x dx \\
& = \sec x \tan x - \int \sec ^3 x dx + \int \sec x dx \\
\int \sec ^3 x dx  & = \frac{1}{2} \sec x \tan x + \frac{1}{2} \ln \mid \sec x + \tan x \mid + C
\end{aligned}
$$
\end{minipage} %
\begin{minipage}{.5\textwidth} %

$$ u = \sec x \implies  du = \sec x \tan x $$
$$ dv = \sec ^2 x dx \implies v = \tan x $$
\end{minipage}
\end{ex}

\textbf{Note : } Integrals involving Cot and Csc share a similar strategy 

\section{Trig Substitution}
Interested in integrals containing \( \sqrt{a^2 - b^2 x^2} , \sqrt{a^2+b^2x^2} \text { or } \sqrt{b^2x^2 - a^2} \)

\begin{ex} -\\
\[ \int \frac{1}{\sqrt{1 -x^2}} dx  ( = \arcsin x + C ) \] \\
Another Way to Solve : \\
Inverse Subsitution \( \begin{cases} x = \sin \beta \\ dx = \cos \beta d\beta \end{cases} \) \\
\[ \int \frac{\cos \beta}{\sqrt{1 - sin ^2 \beta}} = \int \frac{\cos}{\sqrt{\cos 3x}} \]
\end{ex}


\begin{center}
\textbf{End of Lecture Notes} \\
\textbf{Notes By : Harsh Mistry}
\end{center}
\end{document}
