%Notes by Harsh Mistry 
%Math 128
%Template Taken from : https://www.cs.cmu.edu/~ggordon/10725-F12/template.tex

\documentclass[twoside]{article}
\setlength{\oddsidemargin}{0.25 in}
\setlength{\evensidemargin}{-0.25 in}
\setlength{\topmargin}{-0.6 in}
\setlength{\textwidth}{6.5 in}
\setlength{\textheight}{8.5 in}
\setlength{\headsep}{0.75 in}
\setlength{\parindent}{0 in}
\setlength{\parskip}{0.1 in}
\usepackage{amsmath,amsfonts,graphicx}
\newcounter{lecnum}
\renewcommand{\thepage}{\thelecnum-\arabic{page}}
\renewcommand{\thesection}{\thelecnum.\arabic{section}}
\renewcommand{\theequation}{\thelecnum.\arabic{equation}}
\renewcommand{\thefigure}{\thelecnum.\arabic{figure}}
\renewcommand{\thetable}{\thelecnum.\arabic{table}}
\newcommand{\lecture}[4]{
   \pagestyle{myheadings}
   \thispagestyle{plain}
   \newpage
   \setcounter{lecnum}{#1}
   \setcounter{page}{1}
   
   
%Info Box 
   \begin{center}
   \framebox{
      \vbox{\vspace{2mm}
    \hbox to 6.28in { {\bf Math 128 :  Calculus 2 for the Sciences
	\hfill Winter 2016} }
       \vspace{4mm}
       \hbox to 6.28in { {\Large \hfill Lecture #1: #2  \hfill} }
       \vspace{2mm}
       \hbox to 6.28in { {\it Lecturer: #3 \hfill Notes By: #4} }
      \vspace{2mm}}
   }
   \end{center}
   
   \markboth{Lecture #1: #2}{Lecture #1: #2}



 
}

\renewcommand{\cite}[1]{[#1]}
\def\beginrefs{\begin{list}%
        {[\arabic{equation}]}{\usecounter{equation}
         \setlength{\leftmargin}{2.0truecm}\setlength{\labelsep}{0.4truecm}%
         \setlength{\labelwidth}{1.6truecm}}}
\def\endrefs{\end{list}}
\def\bibentry#1{\item[\hbox{[#1]}]}

\newcommand{\fig}[3]{
			\vspace{#2}
			\begin{center}
			Figure \thelecnum.#1:~#3
			\end{center}
	}

\newtheorem{theorem}{Theorem}[lecnum]
\newtheorem{lemma}[theorem]{Lemma}
\newtheorem{ex}[theorem]{Example}
\newtheorem{proposition}[theorem]{Proposition}
\newtheorem{claim}[theorem]{Claim}
\newtheorem{corollary}[theorem]{Corollary}
\newtheorem{definition}[theorem]{Definition}
\newenvironment{proof}{{\bf Proof:}}{\hfill\rule{2mm}{2mm}}
\newcommand\E{\mathbb{E}}


%Start of Document 
\begin{document}

\lecture{1}{January 4, 2016}{Jen Nelson}{Harsh Mistry}


% **** YOUR NOTES GO HERE:

% Admin Info
\begin{center}
Admin Info \\
Jen Nelson
Office : MC 6222 \\
Office Hours : M 12:30pm - 2:00pm  and W 2:00pm - 3:00pm\\
Email: jen.nelson@uwaterloo.ca \\
Midterm date : Feb 22nd, 2016 

\end{center}

%
\section{Review : Definite Integral }
\begin{definition}
$$\int_{a}^{b} f(x) dx  =  \lim_{n\to\infty} \sum_{i=1}^{n}  \Delta x f(x_i) $$
\end{definition}

\section{Review : The Fundamental Theorem of Calculus}
If f is continuous on [a,b] then, \\

Part A, 
\[\frac{d}{dx} \int_{a}^{x} f(x) dx  =  F(x) \]

Part B,
$$\int_{a}^{b} f(x) dx  =  F(b) - F(a)  \ $$ 

\section{Review : The Indefinite Integral}
\textbf{General Antiderivative}
$$ \int f(x) dx = F(x) + c $$

The indefinite integral is a family of functions and its repersents all functions woose derivatives are f


\textbf{Relationship : }
$$\int_{a}^{b} f(x) dx  = \begin{bmatrix} \int f(x) dx \end{bmatrix}_{a}^{b}$$ 

\section{Review : U-Subsitution}
if u = f(x) is differentiable on interval I and f is continuous on the range of g, then 
$$ \int f(g(x)) g\prime(x) dx = \int f(u) du = F(u) + c  $$

\textbf{Sample Problem}
$$\int \frac{x^3}{(x+5)^2}$$ 


\begin{center}
\textbf{End of Lecture Notes} \\
\textbf{Notes By : Harsh Mistry}
\end{center}
\end{document}





