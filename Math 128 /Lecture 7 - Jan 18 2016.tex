%Notes by Harsh Mistry 
%Math 128
%Template Taken from : https://www.cs.cmu.edu/~ggordon/10725-F12/template.tex

\documentclass{article}
\setlength{\oddsidemargin}{0.25 in}
\setlength{\evensidemargin}{-0.25 in}
\setlength{\topmargin}{-0.6 in}
\setlength{\textwidth}{6.5 in}
\setlength{\textheight}{8.5 in}
\setlength{\headsep}{0.75 in}
\setlength{\parindent}{0 in}
\setlength{\parskip}{0.1 in}
\usepackage{amsfonts,graphicx, amssymb}
\usepackage[fleqn]{amsmath}
\usepackage{blindtext}
\newcounter{lecnum}
\renewcommand{\thepage}{\thelecnum-\arabic{page}}
\renewcommand{\thesection}{\thelecnum.\arabic{section}}
\renewcommand{\theequation}{\thelecnum.\arabic{equation}}
\renewcommand{\thefigure}{\thelecnum.\arabic{figure}}
\renewcommand{\thetable}{\thelecnum.\arabic{table}}
\newcommand{\lecture}[4]{
   \pagestyle{myheadings}
   \thispagestyle{plain}
   \newpage
   \setcounter{lecnum}{#1}
   \setcounter{page}{1}
   
   
%Info Box 
   \begin{center}
   \framebox{
      \vbox{\vspace{2mm}
    \hbox to 6.28in { {\bf Math 128 :  Calculus 2 for the Sciences
	\hfill Winter 2016} }
       \vspace{4mm}
       \hbox to 6.28in { {\Large \hfill Lecture #1: #2  \hfill} }
       \vspace{2mm}
       \hbox to 6.28in { {\it Lecturer: #3 \hfill Notes By: #4} }
      \vspace{2mm}}
   }
   \end{center}
   
   \markboth{Lecture #1: #2}{Lecture #1: #2}



 
}

\renewcommand{\cite}[1]{[#1]}
\def\beginrefs{\begin{list}%
        {[\arabic{equation}]}{\usecounter{equation}
         \setlength{\leftmargin}{2.0truecm}\setlength{\labelsep}{0.4truecm}%
         \setlength{\labelwidth}{1.6truecm}}}
\def\endrefs{\end{list}}
\def\bibentry#1{\item[\hbox{[#1]}]}

\newcommand{\fig}[3]{
			\vspace{#2}
			\begin{center}
			Figure \thelecnum.#1:~#3
			\end{center}
	}

\newtheorem{theorem}{Theorem}[lecnum]
\newtheorem{lemma}[theorem]{Lemma}
\newtheorem{ex}[theorem]{Example}
\newtheorem{proposition}[theorem]{Proposition}
\newtheorem{claim}[theorem]{Claim}
\newtheorem{corollary}[theorem]{Corollary}
\newtheorem{definition}[theorem]{Definition}
\newenvironment{proof}{{\bf Proof:}}{\hfill\rule{2mm}{2mm}}
\newcommand\E{\mathbb{E}}


%Start of Document 
\begin{document}

\lecture{7}{January 18, 2016}{Jen Nelson}{Harsh Mistry}

%
\section{Partial Fraction Examples}


\begin{ex} - \\
\[\int \frac{4x}{3x^2-5x^2+x+1} dx \]
\[\frac{4x}{3x^2-5x+4} = \frac{A}{x-1} + \frac{B}{(x-1)^2} + \frac{C}{3x+1}\]
\[\implies 4x = A(x+1)(3x+1) + B(3x+1) + c(x-1)^2 \]     
\begin{itemize}
\item x = 1 :  \( 4(1) = B(3(1) + 1) \implies B = 1\)
\item x = \( \frac{1}{3} \) : \(\frac{4}{3} = c(\frac{-4}{3}) ^2 \implies C = \frac{-3}{4}\)
\item x = 0 : \( 0 = -A + B + C \implies A = \frac{1}{4}\)
\end{itemize}
\end{ex}

\begin{ex}
\[ \int \frac{x^3-2x^2+8x-4}{x(x^2+2)^2} dx = \int \frac{A}{x} + \frac{Bx+C}{x^2 +2 } + \frac{Dx + E}{(x^2 + 2)^2} dx \]
Expand, Multiply, and then compare coefficients (Do as practice)\\
A = -1 B = -1 C = 1, D = 0, E = 6 \\
\[ \begin{aligned} \int \frac{x^3-2x^2+8x-4}{x(x^2+2)^2} dx 7 & = \int \frac{-1}{x} + \frac{x+1}{x^2 +2 } + \frac{6}{(x^2 + 2)^2} dx \\
& =  \int \frac{-1}{x} dx + \int \frac{x+1}{x^2 +2 } dx + \int \frac{6}{(x^2 + 2)^2} dx  \\
\vdots\\
& = -\ln \mid x \mid + \frac{1}{2} \ln \mid x^2 + 2 \mid + \frac{5\sqrt{2}}{4} \arctan (\frac{x}{\sqrt{2}}) + \frac{3x}{4(x^2+2)}  \end{aligned}\]
\end{ex}

\begin{ex} Determine the form of the partial fraction decomposition of : \\
\[ \frac{x+1}{(x) (x-1)^2 (x+2) (x^2 - 2x + 4) (x^2+1)^2} \]
\[= \frac{A}{x} + \frac{B}{x+2} + \frac{C}{x-1} + \frac{D}{(x-1)^2} + \frac{Ex + F}{x^2 -2x + 4} + \frac{Gx+H}{x^2 + 1} + \frac{Jx+K}{(x^2+1)^2} \]
\end{ex}

\section{Improper Integral (Type 1)}
\textbf{Infinite Interval } \( ( a = \infty \text{ and/or } b = \infty) \) \\
Consider \( \int_{1}^{\infty} \frac{1}{x^2} dx \) This integeral is improper since the upper limit is not finite. Think of the integral in terms of area. \\

Consider the area under \( \frac{1}{x^2} on \) [1, t] where t is finite.\\
We can then use FTC II: 
\(\int_{1}^{t} \frac{1}{x^2} dx = \frac{1}{x} \mid_{1}^{t} = 1 - \frac{1}{t} \) \\
We can determine the area on [1, \(\infty \)] by letting \( t \rightarrow \infty\) : \\

$$ \int_{1}^{\infty} \frac{1}{x^2} dx = \lim{t \to \infty} \int_{1}^{t} \frac{1}{x^2} dx = \lim_{t \to \infty} (1 - \frac{1}{t}) = 1 $$




\begin{center}
\textbf{End of Lecture Notes} \\
\textbf{Notes By : Harsh Mistry}
\end{center}
\end{document}
