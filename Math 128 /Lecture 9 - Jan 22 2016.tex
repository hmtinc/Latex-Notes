%Notes by Harsh Mistry 
%Math 128
%Template Taken from : https://www.cs.cmu.edu/~ggordon/10725-F12/template.tex

\documentclass{article}
\setlength{\oddsidemargin}{0.25 in}
\setlength{\evensidemargin}{-0.25 in}
\setlength{\topmargin}{-0.6 in}
\setlength{\textwidth}{6.5 in}
\setlength{\textheight}{8.5 in}
\setlength{\headsep}{0.75 in}
\setlength{\parindent}{0 in}
\setlength{\parskip}{0.1 in}
\usepackage{amsfonts,graphicx, amssymb}
\usepackage[fleqn]{amsmath}
\usepackage{blindtext}
\newcounter{lecnum}
\renewcommand{\thepage}{\thelecnum-\arabic{page}}
\renewcommand{\thesection}{\thelecnum.\arabic{section}}
\renewcommand{\theequation}{\thelecnum.\arabic{equation}}
\renewcommand{\thefigure}{\thelecnum.\arabic{figure}}
\renewcommand{\thetable}{\thelecnum.\arabic{table}}
\newcommand{\lecture}[4]{
   \pagestyle{myheadings}
   \thispagestyle{plain}
   \newpage
   \setcounter{lecnum}{#1}
   \setcounter{page}{1}
   
   
%Info Box 
   \begin{center}
   \framebox{
      \vbox{\vspace{2mm}
    \hbox to 6.28in { {\bf Math 128 :  Calculus 2 for the Sciences
	\hfill Winter 2016} }
       \vspace{4mm}
       \hbox to 6.28in { {\Large \hfill Lecture #1: #2  \hfill} }
       \vspace{2mm}
       \hbox to 6.28in { {\it Lecturer: #3 \hfill Notes By: #4} }
      \vspace{2mm}}
   }
   \end{center}
   
   \markboth{Lecture #1: #2}{Lecture #1: #2}



 
}

\renewcommand{\cite}[1]{[#1]}
\def\beginrefs{\begin{list}%
        {[\arabic{equation}]}{\usecounter{equation}
         \setlength{\leftmargin}{2.0truecm}\setlength{\labelsep}{0.4truecm}%
         \setlength{\labelwidth}{1.6truecm}}}
\def\endrefs{\end{list}}
\def\bibentry#1{\item[\hbox{[#1]}]}

\newcommand{\fig}[3]{
			\vspace{#2}
			\begin{center}
			Figure \thelecnum.#1:~#3
			\end{center}
	}

\newtheorem{theorem}{Theorem}[lecnum]
\newtheorem{lemma}[theorem]{Lemma}
\newtheorem{ex}[theorem]{Example}
\newtheorem{proposition}[theorem]{Proposition}
\newtheorem{claim}[theorem]{Claim}
\newtheorem{corollary}[theorem]{Corollary}
\newtheorem{definition}[theorem]{Definition}
\newenvironment{proof}{{\bf Proof:}}{\hfill\rule{2mm}{2mm}}
\newcommand\E{\mathbb{E}}


%Start of Document 
\begin{document}

\lecture{9}{January 22, 2016}{Jen Nelson}{Harsh Mistry}

%
\section{The Comparison Test} if f and g are continous and \( 0 \leq g(x) \leq f(x) \) for \(x \geq a \)
\begin{itemize}
\item \(\int_{a}^{\infty} f(x) dx \text{ converges then } \int_{a}^{\infty} g(x) dx\) converges
\item \(\int_{a}^{\infty} f(x) dx \text{ diverges then } \int_{a}^{\infty} g(x) dx\) diverges
\end{itemize}

\begin{ex}
Does \(\int_{0}^{\infty} \frac{x+5}{1+x^2} \) converge or diverge ? \\
Note : \( \frac{x+5}{1+x^2} \geq \frac{x}{1+x^2} \)\\
\[\int_{0}^{\infty} \frac{x}{1+x^2} dx = \lim_{b\to\infty} \int_{0}^{b} \frac{x}{1+x^2} = \lim_{b\to\infty} \begin{bmatrix} \frac{\ln\mid 1+x^2 \mid }{2}\end{bmatrix}_{0}^{b} =  \lim_{b\to\infty}  \frac{\ln(1+b^2)}{2} - \frac{\ln (1)}{2} = \infty \]
\( \int_{0}^{\infty} \frac{x}{1+x^2} dx\) diverges so, \( \int_{0}^{\infty} \frac{x+5}{1+x^2}\) diverges by the comparison test
\end{ex}

\begin{ex}
Does \(\int_{1}^{\infty} \frac{\sin ^2 x}{x^2} dx \) converge or diverge ? \\
\[ \begin{aligned} -1 & \leq \sin x \leq 1 \\
\implies 0 & \leq \sin ^2 x \leq 1 \\
\implies 0 & \leq \frac{\sin ^2 x}{x} \leq \frac{1}{x^2} \\
& \int_{1}^{\infty} \frac{1}{x^2} dx \text{ converges since } (P = 2 > 1)  \end{aligned} \]
\( \therefore \int_{1}^{\infty} \frac{\sin ^2 x}{x^2}\) dx converges by the comparison test
\end{ex}

\section{Review : Area Between Curves} Consider \( f(x) g(x) on\) [a, b] with \( f(x) \geq g(x) \). The area bounded by \( f(x) \& g(x) \) between a and b can be repersented as. 
$$ A = \int_{a}^{b} f(x) dx - \int_{a}^{b} g(x) dx  $$





\begin{center}
\textbf{End of Lecture Notes} \\
\textbf{Notes By : Harsh Mistry}
\end{center}
\end{document}
