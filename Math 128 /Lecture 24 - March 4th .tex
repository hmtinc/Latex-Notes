%Notes by Harsh Mistry 
%Math 128
%Based on template from: https://www.cs.cmu.edu/~ggordon/10725-F12/template.tex

\documentclass{article}
\setlength{\oddsidemargin}{0.25 in}
\setlength{\evensidemargin}{-0.25 in}
\setlength{\topmargin}{-0.6 in}
\setlength{\textwidth}{6.5 in}
\setlength{\textheight}{8.5 in}
\setlength{\headsep}{0.75 in}
\setlength{\parindent}{0 in}
\setlength{\parskip}{0.1 in}
\usepackage{amsfonts,graphicx, amssymb}
\usepackage[fleqn]{amsmath}
\usepackage{blindtext}
\newcounter{lecnum}
\renewcommand{\thepage}{\thelecnum-\arabic{page}}
\renewcommand{\thesection}{\thelecnum.\arabic{section}}
\renewcommand{\theequation}{\thelecnum.\arabic{equation}}
\renewcommand{\thefigure}{\thelecnum.\arabic{figure}}
\renewcommand{\thetable}{\thelecnum.\arabic{table}}
\newcommand{\lecture}[4]{
   \pagestyle{myheadings}
   \thispagestyle{plain}
   \newpage
   \setcounter{lecnum}{#1}
   \setcounter{page}{1}
   
   
%Info Box 
   \begin{center}
   \framebox{
      \vbox{\vspace{2mm}
    \hbox to 6.28in { {\bf Math 128 :  Calculus 2 for the Sciences
	\hfill Winter 2016} }
       \vspace{4mm}
       \hbox to 6.28in { {\Large \hfill Lecture #1: #2  \hfill} }
       \vspace{2mm}
       \hbox to 6.28in { {\it Lecturer: #3 \hfill Notes By: #4} }
      \vspace{2mm}}
   }
   \end{center}
   
   \markboth{Lecture #1: #2}{Lecture #1: #2}



 
}

\renewcommand{\cite}[1]{[#1]}
\def\beginrefs{\begin{list}%
        {[\arabic{equation}]}{\usecounter{equation}
         \setlength{\leftmargin}{2.0truecm}\setlength{\labelsep}{0.4truecm}%
         \setlength{\labelwidth}{1.6truecm}}}
\def\endrefs{\end{list}}
\def\bibentry#1{\item[\hbox{[#1]}]}

\newcommand{\fig}[3]{
			\vspace{#2}
			\begin{center}
			Figure \thelecnum.#1:~#3
			\end{center}
	}

\newtheorem{theorem}{Theorem}[lecnum]
\newtheorem{lemma}[theorem]{Lemma}
\newtheorem{ex}[theorem]{Example}
\newtheorem{proposition}[theorem]{Proposition}
\newtheorem{claim}[theorem]{Claim}
\newtheorem{corollary}[theorem]{Corollary}
\newtheorem{definition}[theorem]{Definition}
\newenvironment{proof}{{\bf Proof:}}{\hfill\rule{2mm}{2mm}}
\newcommand\E{\mathbb{E}}


%Start of Document 
\begin{document}

\lecture{24}{March 4, 2016}{Jen Nelson}{Harsh Mistry}

%
\section{Series Continued}
\begin{definition}
We say that a series \( \sum_{n=1}^{\infty} a_n \)  converges if the sequence of partial sums \( \{S_n\} \) converges \\
The limits of \(\{S_n\}\) is called the sum of the series.
$$ \lim_{n \to \infty} S_n = S = \sum_{n=1}^{\infty} a_n $$
\end{definition}

\begin{theorem}
The Geometric series \\
$$ \sum_{n = 1}^{\infty} ar^{n-1} = a + ar + ar^2 + ar^3 + \ldots + ar^(n-1) + \ldots $$
is convergent if \(\mid r \mid \leq 1 \) and its sum is 
$$ \sum_{n = 1}^{\infty} ar^{n-1} = \frac{a}{1-r} $$
if \(\mid r \mid \geq 1\), the geometric series is divergent 
\end{theorem}

$$ \begin{aligned} s_n & = & a + & ar + \ldots + ar^{n-1} \\
rs_n &  = & & ar + ar^2 + ar^{n-1} + ar^n \end{aligned} $$

$$ s_n - rs_n = a - ar^n \implies s_n = \frac{a(1-r^n)}{1-r} , \text{ provided } r \neq 1 $$

If \( -1 \leq r \leq 1\) , then \( \lim_{n \to \infty} s_n = \frac{a}{1-r}\)\\
If \( r > 1\) or \( r \leq -1 \), the limits do not exist, so the series diverges \\
If \( r = 1\), then \( s_n = a + \ldots + a = na \) and \(\lim_{n\to \infty} s_n = \lim_{n\to \infty} na = \infty \), so the series diverges \\

\textbf{Note: } Its better to think of the formula as \(\frac{\text{"First term"}}{\text{1 - common factor}} \)
 

\section{Telescoping Series }
For some series it will not be possible to find a closed formula for \(\{S_n\}\) \\
We can however, determine whether or not the series converges. We rely on a number of tests to achieve this.

\newpage

\begin{theorem}
\( \text{If } \sum_{n=1}^{\infty} a_n\) converges then \(\lim_{n \to \infty} a_n = 0 \)
\end{theorem}

\begin{corollary} The \(n^{th}\)-Term Test/Test for Divergence \\
If \( \lim_{n \to \infty} \neq 0\) or DNE, the series \(\sum_{n=1}^{\infty} a_n \) diverges\end{corollary}

\begin{center}
\textbf{End of Lecture Notes} \\
\textbf{Notes By : Harsh Mistry}
\end{center}
\end{document}
