%Notes by Harsh Mistry 
%Math 128
%Based on template from: https://www.cs.cmu.edu/~ggordon/10725-F12/template.tex

\documentclass{article}
\setlength{\oddsidemargin}{0.25 in}
\setlength{\evensidemargin}{-0.25 in}
\setlength{\topmargin}{-0.6 in}
\setlength{\textwidth}{6.5 in}
\setlength{\textheight}{8.5 in}
\setlength{\headsep}{0.75 in}
\setlength{\parindent}{0 in}
\setlength{\parskip}{0.1 in}
\usepackage{amsfonts,graphicx, amssymb}
\usepackage[fleqn]{amsmath}
\usepackage{blindtext}
\newcounter{lecnum}
\renewcommand{\thepage}{\thelecnum-\arabic{page}}
\renewcommand{\thesection}{\thelecnum.\arabic{section}}
\renewcommand{\theequation}{\thelecnum.\arabic{equation}}
\renewcommand{\thefigure}{\thelecnum.\arabic{figure}}
\renewcommand{\thetable}{\thelecnum.\arabic{table}}
\newcommand{\lecture}[4]{
   \pagestyle{myheadings}
   \thispagestyle{plain}
   \newpage
   \setcounter{lecnum}{#1}
   \setcounter{page}{1}
   
   
%Info Box 
   \begin{center}
   \framebox{
      \vbox{\vspace{2mm}
    \hbox to 6.28in { {\bf Math 128 :  Calculus 2 for the Sciences
	\hfill Winter 2016} }
       \vspace{4mm}
       \hbox to 6.28in { {\Large \hfill Lecture #1: #2  \hfill} }
       \vspace{2mm}
       \hbox to 6.28in { {\it Lecturer: #3 \hfill Notes By: #4} }
      \vspace{2mm}}
   }
   \end{center}
   
   \markboth{Lecture #1: #2}{Lecture #1: #2}



 
}

\renewcommand{\cite}[1]{[#1]}
\def\beginrefs{\begin{list}%
        {[\arabic{equation}]}{\usecounter{equation}
         \setlength{\leftmargin}{2.0truecm}\setlength{\labelsep}{0.4truecm}%
         \setlength{\labelwidth}{1.6truecm}}}
\def\endrefs{\end{list}}
\def\bibentry#1{\item[\hbox{[#1]}]}

\newcommand{\fig}[3]{
			\vspace{#2}
			\begin{center}
			Figure \thelecnum.#1:~#3
			\end{center}
	}

\newtheorem{theorem}{Theorem}[lecnum]
\newtheorem{lemma}[theorem]{Lemma}
\newtheorem{ex}[theorem]{Example}
\newtheorem{proposition}[theorem]{Proposition}
\newtheorem{claim}[theorem]{Claim}
\newtheorem{corollary}[theorem]{Corollary}
\newtheorem{definition}[theorem]{Definition}
\newenvironment{proof}{{\bf Proof:}}{\hfill\rule{2mm}{2mm}}
\newcommand\E{\mathbb{E}}


%Start of Document 
\begin{document}

\lecture{25}{March 7, 2016}{Jen Nelson}{Harsh Mistry}

%
\section{Harmonic Series}

In general : \(S_{2^n} > 1 + \frac{n}{2}\) \\
Therefore,  \( \lim_{n\to \infty} S_{2^n} = \infty \implies \lim_{n\to \infty} S_{n} = \infty \) 
Therefore, \( \{s_n\}\), the sequence of a partial sum diverges and thus \( \sum_{n=1}^{\infty} \frac{1}{n} \) diverges 

\section{Divergence Test} 
If \(\lim_{n \to \infty} a_n \neq 0\) or DNE, then the series \(\sum_{n = 1}^{\infty} a_n \) diverges \\
If the limit does equal zero, we cannot conclude anything

\section{Integral Test}
\begin{theorem}
Suppose the terms in the series \(\sum_{n = 1}^{\infty} a_n \) are denoted by \(a_n = f(x)\) and \(f(x)\) is continuous, positive, and decreasing on \( x \geq 1\) \\
Then \(\sum_{n = 1}^{\infty} a_n \) converges if and only if the improper integral \(\int_{1}^{\infty} f(x)dx\) converges
\end{theorem}

\textbf{Notes : } Despite this link, the values are not equal 

\begin{theorem}
The p-series \(\sum_{n=1}^{\infty} \frac{1}{n^p}\) is convergent if \(p > 1\) and divergent if \(p \neq 1\) 
\end{theorem}


\begin{center}
\textbf{End of Lecture Notes} \\
\textbf{Notes By : Harsh Mistry}
\end{center}
\end{document}
