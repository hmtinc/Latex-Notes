%Notes by Harsh Mistry 
%Math 128
%Template Taken from : https://www.cs.cmu.edu/~ggordon/10725-F12/template.tex

\documentclass{article}
\setlength{\oddsidemargin}{0.25 in}
\setlength{\evensidemargin}{-0.25 in}
\setlength{\topmargin}{-0.6 in}
\setlength{\textwidth}{6.5 in}
\setlength{\textheight}{8.5 in}
\setlength{\headsep}{0.75 in}
\setlength{\parindent}{0 in}
\setlength{\parskip}{0.1 in}
\usepackage{amsfonts,graphicx, amssymb}
\usepackage[fleqn]{amsmath}
\usepackage{blindtext}
\newcounter{lecnum}
\renewcommand{\thepage}{\thelecnum-\arabic{page}}
\renewcommand{\thesection}{\thelecnum.\arabic{section}}
\renewcommand{\theequation}{\thelecnum.\arabic{equation}}
\renewcommand{\thefigure}{\thelecnum.\arabic{figure}}
\renewcommand{\thetable}{\thelecnum.\arabic{table}}
\newcommand{\lecture}[4]{
   \pagestyle{myheadings}
   \thispagestyle{plain}
   \newpage
   \setcounter{lecnum}{#1}
   \setcounter{page}{1}
   
   
%Info Box 
   \begin{center}
   \framebox{
      \vbox{\vspace{2mm}
    \hbox to 6.28in { {\bf Math 128 :  Calculus 2 for the Sciences
	\hfill Winter 2016} }
       \vspace{4mm}
       \hbox to 6.28in { {\Large \hfill Lecture #1: #2  \hfill} }
       \vspace{2mm}
       \hbox to 6.28in { {\it Lecturer: #3 \hfill Notes By: #4} }
      \vspace{2mm}}
   }
   \end{center}
   
   \markboth{Lecture #1: #2}{Lecture #1: #2}



 
}

\renewcommand{\cite}[1]{[#1]}
\def\beginrefs{\begin{list}%
        {[\arabic{equation}]}{\usecounter{equation}
         \setlength{\leftmargin}{2.0truecm}\setlength{\labelsep}{0.4truecm}%
         \setlength{\labelwidth}{1.6truecm}}}
\def\endrefs{\end{list}}
\def\bibentry#1{\item[\hbox{[#1]}]}

\newcommand{\fig}[3]{
			\vspace{#2}
			\begin{center}
			Figure \thelecnum.#1:~#3
			\end{center}
	}

\newtheorem{theorem}{Theorem}[lecnum]
\newtheorem{lemma}[theorem]{Lemma}
\newtheorem{ex}[theorem]{Example}
\newtheorem{proposition}[theorem]{Proposition}
\newtheorem{claim}[theorem]{Claim}
\newtheorem{corollary}[theorem]{Corollary}
\newtheorem{definition}[theorem]{Definition}
\newenvironment{proof}{{\bf Proof:}}{\hfill\rule{2mm}{2mm}}
\newcommand\E{\mathbb{E}}


%Start of Document 
\begin{document}

\lecture{5}{January 13, 2016}{Jen Nelson}{Harsh Mistry}

%
\section{Trig Substitution}
Interested in integrals containing \( \sqrt{a^2 - b^2 x^2} , \sqrt{a^2+b^2x^2} \text { or } \sqrt{b^2x^2 - a^2} \) 
(also powers like \( (a^2 - b^2x^2) ^\frac{5}{2} \) ) 

\begin{ex} - \\
\begin{minipage}{.5\textwidth} %
$$
\begin{aligned}
\int \frac{1}{\sqrt{1-x^2}} dx & = \int \frac{1}{\sqrt{1- \sin ^2 \theta}} \cos \theta d\theta \\
& = \int \frac{\cos \theta}{\sqrt{\cos ^2 \theta}} d\theta \\
& = \int \frac{\cos \theta}{\mid \cos \theta \mid} d\theta \\
& = \int 1 d \theta \\
& = \theta + c \\
& = \arcsin x + c
\end{aligned}
$$
\end{minipage} %
\begin{minipage}{.5\textwidth} %
$$ x = \sin\theta \implies dx = \cos x \implies \theta = \arcsin x  $$
\end{minipage}
\end{ex}

\begin{ex} - \\
\begin{minipage}{.5\textwidth} %
$$
\begin{aligned}
\int \frac{1}{\sqrt{4-9x^2}} dx & = \int \frac{1}{\sqrt{4- 9(\frac{4}{9}\sin ^2 \theta)}} \frac{2}{3}\cos \theta d\theta \\
& = \frac{2}{3}\int \frac{\cos \theta}{2\sqrt{\cos ^2 \theta}} d\theta \\
& = \frac{1}{3}\int \frac{\cos \theta}{\mid \cos \theta \mid} d\theta \\
& = \frac{1}{3}\int 1 d \theta \\
& = \frac{1}{2}\theta + c \\
& = \frac{\arcsin \frac{3x}{2}} {3} + c
\end{aligned}
$$
\end{minipage} %
\begin{minipage}{.5\textwidth} %
$$ x = \frac{2}{3}\sin\theta \implies dx = \frac{2}{3 }\cos x $$
\end{minipage}
\end{ex}
\newpage

\begin{ex} - \\
\begin{minipage}{.5\textwidth} %
$$
\begin{aligned}
\int \frac{\sqrt{25x^2-4}}{x} dx & = \frac{\sqrt{25(\frac{4}{25} \sec \theta )^2-4}}{\frac{2}{5} \sec \theta} \frac{2}{5} \sec \theta \tan \theta d\theta \\
& = \int 2 \sqrt{\sec ^2 \theta -1} \tan \theta d\theta \\
& = 2 \int  \sqrt{\tan ^2 \theta} \tan \theta d\theta \\
& = 2 \int  \mid \tan \theta \mid  \tan \theta d\theta \\
& = 2 \int  \tan ^2 \theta d\theta  \ (\text{ since } \pi \leq \theta \leq \frac{3\pi}{2} )\\
& = 2 \int  \sec ^2 \theta - 1 d\theta \\
& = 2 \tan ^2 (\cos ^{-1} \frac{5x}{2}) - 2 \sec ^ {-1} \frac{5x}{2} + C\\
& = 2 (\frac{\sqrt{25x^2-4}}{2}) - 2 \sec ^ {-1} \frac{5x}{2} + C\\
\end{aligned}
$$
\end{minipage} %
\begin{minipage}{.5\textwidth} %
$$ x = \frac{2}{5}\sec\theta \implies dx = \frac{2}{5}\sec \theta \tan \theta d \theta $$
$$ \theta = \sec ^{-1} \frac{5x}{2} $$
\end{minipage}
\end{ex}

\textbf{In General } \\
\(\sqrt{a^2 - b^2 x^2}\)  sub in \(x = \frac{a}{b} \sin \theta \)  and use the identity \(\cos ^2 \theta = 1  - \sin ^2 \theta  \) \\
\(\sqrt{a^2 + b^2 x^2}\)  sub in \(x = \frac{a}{b} \tan \theta \)  and use the identity \(\sec ^2 \theta = 1  + \tan ^2 \theta  \) \\
\(\sqrt{b^2 x^2 - a^2}\)  sub in \(c = \frac{a}{b} \sec \theta \)  and use the identity \(\tan ^2 \theta = \sec ^2 -1  \theta  \)


\begin{center}
\textbf{End of Lecture Notes} \\
\textbf{Notes By : Harsh Mistry}
\end{center}
\end{document}
