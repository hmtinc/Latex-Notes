%Notes by Harsh Mistry 
%Math 128
%Template Taken from : https://www.cs.cmu.edu/~ggordon/10725-F12/template.tex

\documentclass{article}
\setlength{\oddsidemargin}{0.25 in}
\setlength{\evensidemargin}{-0.25 in}
\setlength{\topmargin}{-0.6 in}
\setlength{\textwidth}{6.5 in}
\setlength{\textheight}{8.5 in}
\setlength{\headsep}{0.75 in}
\setlength{\parindent}{0 in}
\setlength{\parskip}{0.1 in}
\usepackage{amsfonts,graphicx, amssymb}
\usepackage[fleqn]{amsmath}
\newcounter{lecnum}
\renewcommand{\thepage}{\thelecnum-\arabic{page}}
\renewcommand{\thesection}{\thelecnum.\arabic{section}}
\renewcommand{\theequation}{\thelecnum.\arabic{equation}}
\renewcommand{\thefigure}{\thelecnum.\arabic{figure}}
\renewcommand{\thetable}{\thelecnum.\arabic{table}}
\newcommand{\lecture}[4]{
   \pagestyle{myheadings}
   \thispagestyle{plain}
   \newpage
   \setcounter{lecnum}{#1}
   \setcounter{page}{1}
   
   
%Info Box 
   \begin{center}
   \framebox{
      \vbox{\vspace{2mm}
    \hbox to 6.28in { {\bf Math 128 :  Calculus 2 for the Sciences
	\hfill Winter 2016} }
       \vspace{4mm}
       \hbox to 6.28in { {\Large \hfill Lecture #1: #2  \hfill} }
       \vspace{2mm}
       \hbox to 6.28in { {\it Lecturer: #3 \hfill Notes By: #4} }
      \vspace{2mm}}
   }
   \end{center}
   
   \markboth{Lecture #1: #2}{Lecture #1: #2}



 
}

\renewcommand{\cite}[1]{[#1]}
\def\beginrefs{\begin{list}%
        {[\arabic{equation}]}{\usecounter{equation}
         \setlength{\leftmargin}{2.0truecm}\setlength{\labelsep}{0.4truecm}%
         \setlength{\labelwidth}{1.6truecm}}}
\def\endrefs{\end{list}}
\def\bibentry#1{\item[\hbox{[#1]}]}

\newcommand{\fig}[3]{
			\vspace{#2}
			\begin{center}
			Figure \thelecnum.#1:~#3
			\end{center}
	}

\newtheorem{theorem}{Theorem}[lecnum]
\newtheorem{lemma}[theorem]{Lemma}
\newtheorem{ex}[theorem]{Example}
\newtheorem{proposition}[theorem]{Proposition}
\newtheorem{claim}[theorem]{Claim}
\newtheorem{corollary}[theorem]{Corollary}
\newtheorem{definition}[theorem]{Definition}
\newenvironment{proof}{{\bf Proof:}}{\hfill\rule{2mm}{2mm}}
\newcommand\E{\mathbb{E}}


%Start of Document 
\begin{document}

\lecture{2}{January 6, 2016}{Jen Nelson}{Harsh Mistry}


% **** YOUR NOTES GO HERE:


%
\section{Integration by Subsitution Examples }
\begin{ex}
\begin{equation}
\begin{aligned}
\int \frac{x^3}{(x+5)^2}  & = \int \frac{(u-5)^3}{u} du   \ \textbf{ Let u = x + 5} \\
& = \int \frac{u^3 - 15u^2 + 75u - 125}{u^2} \\
& = \int u - 15 + \frac{75}{u} - \frac{125}{u^2}\\
& = \begin{bmatrix} \frac{u^2}{2} - 15u + 75 \ln u + \frac{125}{u} + c \end{bmatrix} \\
& = \begin{bmatrix} \frac{(x+5)^2}{2} - 15(x+5) + 75 \ln (x+5) + \frac{125}{x+5} + c \end{bmatrix} 
\end{aligned}
\end{equation}
\end{ex}

\begin{ex}
\begin{equation}
\begin{aligned}
\int_{1}^{2} \frac{e^\frac{1}{x}}{x^2} dx & = -1 \int_{1}^{\frac{1}{2}} e^u du \ \textbf{ Let u = } \frac{1}{x} \\
& = - \begin{bmatrix} e^u \end{bmatrix}_{1}^{\frac{1}{2}} \\
& = -e^\frac{1}{2} + e
\end{aligned}
\end{equation}
\end{ex}

\begin{ex}
\begin{equation}
\begin{aligned}
\int \tan x dx = \int \frac{\sin x}{\cos x} & = - \int \frac{1}{u} du \ \textbf{ Let u = } \cos x\\
& = - \ln \mid u \mid\ + \ c  \\
& = - \ln \mid \cos x \mid\ + \ c  \ \textbf{ or }  \ln \mid \sec \mid\ + \  c  
\end{aligned}
\end{equation}
\end{ex}

\textbf{Practice Problem}
\[ \int \sec x \ dx   \text{ Hint : Multiply by } \frac{\sec x + \tan x}{\sec x + \tan x} \]

\begin{ex}
\begin{equation}
\begin{aligned}
\int_{-1}^{1} \frac{\sin x}{1 + x^2} dx   =  0  \textbf{ Integral is 0 because the function is odd }
\end{aligned}
\end{equation}
\end{ex}





\section{Integration by Parts (IBP)}
\textbf{IBP Formula}
\[\int udv = uv - \int v du \]

\textbf{Why?} \\
Reverse Engineering the product rule results in the IBP formula

\begin{proof}
\[ \text{The product rule : } \frac{d}{dx} (f(x)g(x))  = f \prime (x) g(x) + f(x) g \prime (x) \]
Integrate both sides with respect to x 
$$\int \frac{d}{dx} (f(x)g(x)) dx  = \int f \prime (x) g(x) + f(x) g \prime (x) dx $$
$$\implies  f(x)g(x) = \int f \prime (x) g(x)  dx + \int f(x) g \prime (x) dx$$
$$\implies  \int f(x) g \prime (x) dx = f(x)g(x) - \int f \prime (x) g(x)$$
\begin{center}
let u = f(x) and v = g(x) 
\end{center}
$$\therefore \int udv = uv - \int vdu $$
\end{proof}

\textbf{Process : }\\
So, divide the integrand into 2 parts u \& dv. Then apply the IBP formula.  
\[\text{You'll find that calculating } \int udv \text{ reduces } \int vdu\]

\begin{ex}

\[\int xe^x \]
\[\textbf{Suppose } u = x  \ \&  \ dv = e^x \ dx \ \rightarrow du = dx \ \& \ v = e^x \]
\[\int xe^x \ dx  = xe^x  - \int e^x dx \]
\[= xe^x - e^x + c \]

\end{ex}
\textbf{Practice Problem}
$$\int x^2 \cos x \ dx $$


\begin{center}
\textbf{End of Lecture Notes} \\
\textbf{Notes By : Harsh Mistry}
\end{center}
\end{document}
