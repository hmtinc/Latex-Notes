%Notes by Harsh Mistry 
%Math 128
%based on templatefrom : https://www.cs.cmu.edu/~ggordon/10725-F12/

\documentclass{article}
\setlength{\oddsidemargin}{0.25 in}
\setlength{\evensidemargin}{-0.25 in}
\setlength{\topmargin}{-0.6 in}
\setlength{\textwidth}{6.5 in}
\setlength{\textheight}{8.5 in}
\setlength{\headsep}{0.75 in}
\setlength{\parindent}{0 in}
\setlength{\parskip}{0.1 in}
\usepackage{amsfonts,graphicx, amssymb}
\usepackage[fleqn]{amsmath}
\usepackage{blindtext}
\newcounter{lecnum}
\renewcommand{\thepage}{\thelecnum-\arabic{page}}
\renewcommand{\thesection}{\thelecnum.\arabic{section}}
\renewcommand{\theequation}{\thelecnum.\arabic{equation}}
\renewcommand{\thefigure}{\thelecnum.\arabic{figure}}
\renewcommand{\thetable}{\thelecnum.\arabic{table}}
\newcommand{\lecture}[4]{
   \pagestyle{myheadings}
   \thispagestyle{plain}
   \newpage
   \setcounter{lecnum}{#1}
   \setcounter{page}{1}
   
   
%Info Box 
   \begin{center}
   \framebox{
      \vbox{\vspace{2mm}
    \hbox to 6.28in { {\bf Math 128 :  Calculus 2 for the Sciences
	\hfill Winter 2016} }
       \vspace{4mm}
       \hbox to 6.28in { {\Large \hfill Lecture #1: #2  \hfill} }
       \vspace{2mm}
       \hbox to 6.28in { {\it Lecturer: #3 \hfill Notes By: #4} }
      \vspace{2mm}}
   }
   \end{center}
   
   \markboth{Lecture #1: #2}{Lecture #1: #2}



 
}

\renewcommand{\cite}[1]{[#1]}
\def\beginrefs{\begin{list}%
        {[\arabic{equation}]}{\usecounter{equation}
         \setlength{\leftmargin}{2.0truecm}\setlength{\labelsep}{0.4truecm}%
         \setlength{\labelwidth}{1.6truecm}}}
\def\endrefs{\end{list}}
\def\bibentry#1{\item[\hbox{[#1]}]}

\newcommand{\fig}[3]{
			\vspace{#2}
			\begin{center}
			Figure \thelecnum.#1:~#3
			\end{center}
	}

\newtheorem{theorem}{Theorem}[lecnum]
\newtheorem{lemma}[theorem]{Lemma}
\newtheorem{ex}[theorem]{Example}
\newtheorem{proposition}[theorem]{Proposition}
\newtheorem{claim}[theorem]{Claim}
\newtheorem{corollary}[theorem]{Corollary}
\newtheorem{definition}[theorem]{Definition}
\newenvironment{proof}{{\bf Proof:}}{\hfill\rule{2mm}{2mm}}
\newcommand\E{\mathbb{E}}


%Start of Document 
\begin{document}

\lecture{11}{January 27, 2016}{Jen Nelson}{Harsh Mistry}

%
\section{Volumes Continued}

\begin{ex}
Find the volume of the solid obtained by rotating the area bounded by \( x = y^2 \) and \( x + y = 2\) about y = -3 \\

y = 1, - 2 \\
volume of one shell = \( 2\pi (y+3) (2-y-y^2) dy \)\\
\[\begin{aligned} \text{Total Volume} & = \int_{-2}^1 2\pi (y+3)(2-y-y^2)dy \\ & = \frac{45\pi}{2} \end{aligned}\]
\end{ex}

\textbf{Practice : } Find the volume enclosed by \(x = y^3 , y = \sqrt{2-x} , y = 0\) rotated about y = 1 \\
Try to use both vertical rectangles and horizontal rectangles.

\textbf{Tips} : 
\begin{itemize}
\item Draw a diagram 
\item Check rotating both vertical and horizontal rectangles to determine which orientation to use 
\item Be careful with radius of cylinder/disk/washer if not rotating about x
\item Limits of integration must be for same variable that is in the integral
\end{itemize}

\section{Arc Length} Consider  y = f(x) \\
If f(x) is continuous on [a,b], then the length of the curve y = f(x) on \( a \leq x \leq b  \)  is
$$ L = \int_a^b \sqrt{1 + (\frac{dy}{dx})^2}$$
If instead we have x = g(y) where \(c \leq y \leq d \), then we use
$$ L = \int_c^d \sqrt{1 + (\frac{dx}{dy})^2} $$

\begin{center}
\textbf{End of Lecture Notes} \\
\textbf{Notes By : Harsh Mistry}
\end{center}
\end{document}
