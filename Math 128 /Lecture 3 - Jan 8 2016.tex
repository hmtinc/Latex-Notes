%Notes by Harsh Mistry 
%Math 128
%Template Taken from : https://www.cs.cmu.edu/~ggordon/10725-F12/template.tex

\documentclass{article}
\setlength{\oddsidemargin}{0.25 in}
\setlength{\evensidemargin}{-0.25 in}
\setlength{\topmargin}{-0.6 in}
\setlength{\textwidth}{6.5 in}
\setlength{\textheight}{8.5 in}
\setlength{\headsep}{0.75 in}
\setlength{\parindent}{0 in}
\setlength{\parskip}{0.1 in}
\usepackage{amsfonts,graphicx, amssymb}
\usepackage[fleqn]{amsmath}
\usepackage{blindtext}
\newcounter{lecnum}
\renewcommand{\thepage}{\thelecnum-\arabic{page}}
\renewcommand{\thesection}{\thelecnum.\arabic{section}}
\renewcommand{\theequation}{\thelecnum.\arabic{equation}}
\renewcommand{\thefigure}{\thelecnum.\arabic{figure}}
\renewcommand{\thetable}{\thelecnum.\arabic{table}}
\newcommand{\lecture}[4]{
   \pagestyle{myheadings}
   \thispagestyle{plain}
   \newpage
   \setcounter{lecnum}{#1}
   \setcounter{page}{1}
   
   
%Info Box 
   \begin{center}
   \framebox{
      \vbox{\vspace{2mm}
    \hbox to 6.28in { {\bf Math 128 :  Calculus 2 for the Sciences
	\hfill Winter 2016} }
       \vspace{4mm}
       \hbox to 6.28in { {\Large \hfill Lecture #1: #2  \hfill} }
       \vspace{2mm}
       \hbox to 6.28in { {\it Lecturer: #3 \hfill Notes By: #4} }
      \vspace{2mm}}
   }
   \end{center}
   
   \markboth{Lecture #1: #2}{Lecture #1: #2}



 
}

\renewcommand{\cite}[1]{[#1]}
\def\beginrefs{\begin{list}%
        {[\arabic{equation}]}{\usecounter{equation}
         \setlength{\leftmargin}{2.0truecm}\setlength{\labelsep}{0.4truecm}%
         \setlength{\labelwidth}{1.6truecm}}}
\def\endrefs{\end{list}}
\def\bibentry#1{\item[\hbox{[#1]}]}

\newcommand{\fig}[3]{
			\vspace{#2}
			\begin{center}
			Figure \thelecnum.#1:~#3
			\end{center}
	}

\newtheorem{theorem}{Theorem}[lecnum]
\newtheorem{lemma}[theorem]{Lemma}
\newtheorem{ex}[theorem]{Example}
\newtheorem{proposition}[theorem]{Proposition}
\newtheorem{claim}[theorem]{Claim}
\newtheorem{corollary}[theorem]{Corollary}
\newtheorem{definition}[theorem]{Definition}
\newenvironment{proof}{{\bf Proof:}}{\hfill\rule{2mm}{2mm}}
\newcommand\E{\mathbb{E}}


%Start of Document 
\begin{document}

\lecture{3}{January 8, 2016}{Jen Nelson}{Harsh Mistry}

%
\section{IBP}
\begin{ex} - \\
\begin{minipage}{.5\textwidth} %
$$
\begin{aligned}
\int x^2 \cos 3x dx  & = \frac{x^2 \sin 3x}{3} - \frac{2}{3} \int x \sin 3x dx \\
& = \frac{x^2 \sin 3x}{3} - \frac{2}{3} \left( \frac{-x^2 \cos 3x}{3}  + \frac{1}{3} \int \cos 3x \right) \\
& = \frac{x^2 \sin 3x}{3} + \frac{2}{9} x \cos 3x -  \frac{2}{27} \sin 3x + C 
\end{aligned}
$$
\end{minipage} %
\begin{minipage}{.5\textwidth} %
$$ u = x^2 \ dv = \cos 3x $$
$$ du = 2x dx  \  v = \frac{\sin 3x}{3} $$
$$ u = x \ dv = \sin x $$
$$ du = dx  \  v = \frac{- \cos 3x}{3} $$
\end{minipage}
\end{ex}

\section{IBP for Definite Integrals} 
The IBP Formula can also be used for definite integrals
$$ \int_{a}^{b} udv = uv \mid_{a}^{b} - \int_{a}^{b} vdu $$

\begin{ex} - \\
\begin{minipage}{.5\textwidth} %
$$
\begin{aligned}
\int_{0}^{\frac{1}{2}} \cos ^{-1} x dx & = x \cos ^{-1} x \mid_{0}^{\frac{1}{2}} + \int_{0}^{\frac{1}{2}}
\frac{x}{\sqrt{1 - x^2}} \\
& = x \cos ^{-1} x \mid_{0}^{\frac{1}{2}} + \int_{1}^{\frac{3}{4}}
\frac{\frac{dt}{-2}}{\sqrt{2}} \\
& = \ldots \\
& = \frac{\pi}{6} - \frac{\sqrt{3}}{2} + 1
\end{aligned}
$$
\end{minipage} %
\begin{minipage}{.5\textwidth} %
$$ u = cos^{-1} \ du = \frac{-1}{\sqrt{-x^2}} $$
$$ dv = 2x dx  \  v = \frac{\sin 3x}{3} $$
$$ \text{Let } t = 1 - x^2 $$
$$ dt = 2x dx $$
\end{minipage}
\end{ex}


\textbf{Practice} : Sometimes IBP or U-sub will work
\[ \int x \sqrt{x + 1 }\]
Evalute using : 
\begin{enumerate}
  \item IBP
  \item U-Sub
\end{enumerate}


\begin{ex} - \\
\begin{minipage}{.5\textwidth} %
$$
\begin{aligned}
\int e^{-1} \sin 2x \ dx & = \frac{-e^{-x}2x}{2} - \frac{1}{2} \int e^{-x} \cos 2x dx \\
& = \frac{-e^{-x}2x}{2} - \frac{1}{2} \left( \frac{e^{-x}2x}{2} + \frac{1}{2} \int e^{-1} \sin 2x \ dx \right) \\
\frac{5}{4} \int e^{-1} \sin 2x \ dx & =   \frac{-e^{-x}2x}{2} - \frac{1}{4} e^{-x} \sin 2x\\
\therefore \int e^{-1} \sin 2x \ dx & = \frac{-2}{5} e^{-x} \cos 2x - \frac{e^{-x} \sin 2x}{5}  +  C 
\end{aligned}
$$
\end{minipage} %
\begin{minipage}{.5\textwidth} %
$$ u = e^{-x} \ du = -e^{-x}$$
$$ dv = \sin 2x \ dx \ v = \frac{- \cos 2x}{2} $$
$$ u = e^{-x} \ du = -e^{-x}$$
$$ dv = \cos 2x \ dx \ v = \frac{ \sin 2x}{2} $$
\end{minipage}
\end{ex}
This is called Integration by reproduction 

\textbf{Practice : }
\[ \int \sin (\ln x) dx \ \text{ Hint : let u = } \ln x \text{ then use IBP } \] 



\section{Trig Integrals}
Recall : Basic Trig Derivatives \\
$$ \frac{d}{x} \sin x \ \frac{d}{x} \cos x \ \frac{d}{x} tan \ \frac{d}{x} \sec $$
There is a pairing between sin/cos and tan/sec  

\subsection{Integrals Involving Sin and Cos}
$$ \int \sin ^m x  \cos ^n x \ dx \text{ , m, n } \in \mathbb{Z} \text{ m, n} \geq 0 $$

\begin{ex} - \\
\begin{minipage}{.5\textwidth} %
$$
\begin{aligned}
\int \sin ^5 x \cos x \ dx & = \int u^5 \ dx \\
& = \frac{u^6}{6} + C \\
& = \frac{(\sin x)^6}{6} + C
\end{aligned}
$$
\end{minipage} %
\begin{minipage}{.5\textwidth} %
$$ \text{Let u = } \sin x $$
$$ du = \cos x \ dx $$
\end{minipage}
\end{ex}

\newpage 

\begin{ex} 
-\\
\begin{minipage}{.5\textwidth} %
$$\int \sin ^5 x \ dx $$
This integral does not have a extra Cos like 3.4, which allowed for u-sub to work 
$$
\begin{aligned}
\int \sin ^4 x  \ \sin x  \ dx & = \int ( 1 - \cos ^2 x) ^2 \sin x \ dx \\
& = - \int (1 - u) ^ 2 du \\
& = - \int 1 - 2u^2 + u^4  \ du \\
& = u + \frac{2u^3}{3} - \frac{u}{5} + C \\
& = \cos x + \frac{2(\cos x)^3}{3} - \frac{\cos x}{5} + C
\end{aligned}
$$
\end{minipage} %
\begin{minipage}{.5\textwidth} %
$$ Let \ \ u = \cos x $$
$$ du = -sinx dx $$
\end{minipage}
\end{ex}
The Method in 3.5 worked because the power was odd 

\begin{ex} 
-\\
\begin{minipage}{.5\textwidth} %
$$\int \sin ^2 x \cos ^5 x \ dx $$
The exponent of Cos is odd, so same method can be used
$$
\begin{aligned}
\int \sin ^2 x  \ \cos ^4 x \ \cos x  \ dx & = \int \sin ^2 x (1 - sin ^2 x) ^2 \cos x \ dx \\
& = \int u^2 (1-u^2) du \\
& = \int (u^2 - 2u^4 + u^6) du \\
& = \frac{u^3}{3} - \frac{2u^5}{5} + \frac{u^7}{7} + C \\
& = \frac{\sin ^3 x}{3} - \frac{2 \sin ^5 x}{5} + \frac{\sin ^7 x}{7} + + C
\end{aligned}
$$
\end{minipage} %
\begin{minipage}{.5\textwidth} %
$$ Let \ \ u = \sin x $$
$$ du = cos  x \ dx $$
\end{minipage}
\end{ex}

\begin{center}
\textbf{End of Lecture Notes} \\
\textbf{Notes By : Harsh Mistry}
\end{center}
\end{document}
