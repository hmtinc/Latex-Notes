%Notes by Harsh Mistry 
%Math 128
%Template Taken from : https://www.cs.cmu.edu/~ggordon/10725-F12/template.tex

\documentclass{article}
\setlength{\oddsidemargin}{0.25 in}
\setlength{\evensidemargin}{-0.25 in}
\setlength{\topmargin}{-0.6 in}
\setlength{\textwidth}{6.5 in}
\setlength{\textheight}{8.5 in}
\setlength{\headsep}{0.75 in}
\setlength{\parindent}{0 in}
\setlength{\parskip}{0.1 in}
\usepackage{amsfonts,graphicx, amssymb}
\usepackage[fleqn]{amsmath}
\usepackage{blindtext}
\newcounter{lecnum}
\renewcommand{\thepage}{\thelecnum-\arabic{page}}
\renewcommand{\thesection}{\thelecnum.\arabic{section}}
\renewcommand{\theequation}{\thelecnum.\arabic{equation}}
\renewcommand{\thefigure}{\thelecnum.\arabic{figure}}
\renewcommand{\thetable}{\thelecnum.\arabic{table}}
\newcommand{\lecture}[4]{
   \pagestyle{myheadings}
   \thispagestyle{plain}
   \newpage
   \setcounter{lecnum}{#1}
   \setcounter{page}{1}
   
   
%Info Box 
   \begin{center}
   \framebox{
      \vbox{\vspace{2mm}
    \hbox to 6.28in { {\bf Math 128 :  Calculus 2 for the Sciences
	\hfill Winter 2016} }
       \vspace{4mm}
       \hbox to 6.28in { {\Large \hfill Lecture #1: #2  \hfill} }
       \vspace{2mm}
       \hbox to 6.28in { {\it Lecturer: #3 \hfill Notes By: #4} }
      \vspace{2mm}}
   }
   \end{center}
   
   \markboth{Lecture #1: #2}{Lecture #1: #2}



 
}

\renewcommand{\cite}[1]{[#1]}
\def\beginrefs{\begin{list}%
        {[\arabic{equation}]}{\usecounter{equation}
         \setlength{\leftmargin}{2.0truecm}\setlength{\labelsep}{0.4truecm}%
         \setlength{\labelwidth}{1.6truecm}}}
\def\endrefs{\end{list}}
\def\bibentry#1{\item[\hbox{[#1]}]}

\newcommand{\fig}[3]{
			\vspace{#2}
			\begin{center}
			Figure \thelecnum.#1:~#3
			\end{center}
	}

\newtheorem{theorem}{Theorem}[lecnum]
\newtheorem{lemma}[theorem]{Lemma}
\newtheorem{ex}[theorem]{Example}
\newtheorem{proposition}[theorem]{Proposition}
\newtheorem{claim}[theorem]{Claim}
\newtheorem{corollary}[theorem]{Corollary}
\newtheorem{definition}[theorem]{Definition}
\newenvironment{proof}{{\bf Proof:}}{\hfill\rule{2mm}{2mm}}
\newcommand\E{\mathbb{E}}


%Start of Document 
\begin{document}

\lecture{8}{January 20, 2016}{Jen Nelson}{Harsh Mistry}

%
\section{Improper Integral Type 1} Infinite Interval \( a = -\infty \) and/or \( b = \infty \) \\
In General : 
\begin{itemize}
\item \(\int_{a}^{\infty} f(x) dx = \lim_{b \to \infty} \int_{a}^{b} f(x) dx, \) provided f is continuous on [a , \( \infty\)) and the limit exists
\item \(\int_{-\infty}^{b} f(x)dx = \lim_{a \to - \infty} \int {a}^{b} f(x)dx, \) provided f is continuons on \(-\infty , b \) ] and the limit exists 
\item \( \int_{-\infty}^{\infty} f(x) dx =  \lim_{a \to -\infty} \int_{a}^{c} f(x)dx + \lim_{b \to \infty} \int_{c}^{b} f(x)dx \text{ 
provided that both limits exists} \)
\end{itemize}

If the limit exists we say the improper integral is equal to the limt or is Convergent and converges. If the limit does not exist, the 
improper integral has no value and is divergent 

\section{Improper Integeral Type 2} Discontinuous Integrand (\(f\) discontinuous on [a,b])\\
In General :l 
\begin{itemize}
\item If \(f\) is continuous on [a,b) but not at b then \\
\(\int_{a}^{b} f(x)dx = \lim_{c\to b^{-}} \int_{a}^{c} f(x) dx,\) provided the limit exists
\item If \(f\) is continuous on (a,b] but not at a then \\
\(\int_{a}^{b} f(x)dx = \lim_{c\to a^{+}} \int_{c}^{b} f(x) dx,\) provided the limit exists
\item If \(f\) is continuous on (a,b] but not at d, where \( a < d < b \) , then \\
\(\int_{a}^{b} f(x)dx = \lim_{c\to d^{-}} \int_{a}^{c} f(x) dx + \lim_{c\to d^{+}} \int_{c}^{b} f(x) dx \) provided both limit exist.
\end{itemize}




\begin{center}
\textbf{End of Lecture Notes} \\
\textbf{Notes By : Harsh Mistry}
\end{center}
\end{document}
