%Notes by Harsh Mistry 
%Math 128
%Template Taken from : https://www.cs.cmu.edu/~ggordon/10725-F12/template.tex

\documentclass{article}
\setlength{\oddsidemargin}{0.25 in}
\setlength{\evensidemargin}{-0.25 in}
\setlength{\topmargin}{-0.6 in}
\setlength{\textwidth}{6.5 in}
\setlength{\textheight}{8.5 in}
\setlength{\headsep}{0.75 in}
\setlength{\parindent}{0 in}
\setlength{\parskip}{0.1 in}
\usepackage{amsfonts,graphicx, amssymb}
\usepackage[fleqn]{amsmath}
\usepackage{blindtext}
\newcounter{lecnum}
\renewcommand{\thepage}{\thelecnum-\arabic{page}}
\renewcommand{\thesection}{\thelecnum.\arabic{section}}
\renewcommand{\theequation}{\thelecnum.\arabic{equation}}
\renewcommand{\thefigure}{\thelecnum.\arabic{figure}}
\renewcommand{\thetable}{\thelecnum.\arabic{table}}
\newcommand{\lecture}[4]{
   \pagestyle{myheadings}
   \thispagestyle{plain}
   \newpage
   \setcounter{lecnum}{#1}
   \setcounter{page}{1}
   
   
%Info Box 
   \begin{center}
   \framebox{
      \vbox{\vspace{2mm}
    \hbox to 6.28in { {\bf Math 128 :  Calculus 2 for the Sciences
	\hfill Winter 2016} }
       \vspace{4mm}
       \hbox to 6.28in { {\Large \hfill Lecture #1: #2  \hfill} }
       \vspace{2mm}
       \hbox to 6.28in { {\it Lecturer: #3 \hfill Notes By: #4} }
      \vspace{2mm}}
   }
   \end{center}
   
   \markboth{Lecture #1: #2}{Lecture #1: #2}



 
}

\renewcommand{\cite}[1]{[#1]}
\def\beginrefs{\begin{list}%
        {[\arabic{equation}]}{\usecounter{equation}
         \setlength{\leftmargin}{2.0truecm}\setlength{\labelsep}{0.4truecm}%
         \setlength{\labelwidth}{1.6truecm}}}
\def\endrefs{\end{list}}
\def\bibentry#1{\item[\hbox{[#1]}]}

\newcommand{\fig}[3]{
			\vspace{#2}
			\begin{center}
			Figure \thelecnum.#1:~#3
			\end{center}
	}

\newtheorem{theorem}{Theorem}[lecnum]
\newtheorem{lemma}[theorem]{Lemma}
\newtheorem{ex}[theorem]{Example}
\newtheorem{proposition}[theorem]{Proposition}
\newtheorem{claim}[theorem]{Claim}
\newtheorem{corollary}[theorem]{Corollary}
\newtheorem{definition}[theorem]{Definition}
\newenvironment{proof}{{\bf Proof:}}{\hfill\rule{2mm}{2mm}}
\newcommand\E{\mathbb{E}}


%Start of Document 
\begin{document}

\lecture{6}{January 15, 2016}{Jen Nelson}{Harsh Mistry}

%
\section{Trig Substitution}
Interested in integrals containing \( \sqrt{a^2 - b^2 x^2} , \sqrt{a^2+b^2x^2} \text { or } \sqrt{b^2x^2 - a^2} \) 
(also powers like \( (a^2 - b^2x^2) ^\frac{5}{2} \) ) 

\begin{ex} - \\
\begin{minipage}{.5\textwidth} %
\text {we can use subsitution to solve if we complete the sqauare} 
$$
\begin{aligned}
\int \frac{1}{\sqrt{x^2+4x+5}} dx  & = \int \frac{1}{\sqrt{(x+2)^2 - 4 + 5}} dx \\
& = \int \frac{1}{\sqrt{(x+2)^2 + 1}} dx \\
& = \int \frac{1}{\sqrt{(\tan ^2 \theta + 1}} \sec ^2 \theta d \theta  \\ 
& = \int \frac{1}{\sqrt{(\sec ^2 \theta}} \sec ^2 \theta d \theta  \\ 
& = \int  \sec \theta d \theta  \\ 
& = \ln \mid \tan \theta + \sec \theta \mid + c \\
& = \ln \mid (x+2) \sqrt{x^2 + 4x +5} \mid + c
\end{aligned}
$$
\end{minipage} %
\begin{minipage}{.5\textwidth} %
$$ x + 2 = \tan\theta \implies dx = \sec ^2  \theta$$
\end{minipage}
\end{ex}

\section{Partial Fractions}
Used for integrating rational functions which are in the forn \( f(x) = \frac{p(x)}{q(x)} \) where p(x) and c(x) are polynomials 

\begin{ex} - \\
\(\int \frac{6x+8}{x^2+3x+2} dx \) can be evaulated with regular subsitution, but takes to long.\\
Using partial fractions we find that \(\int \frac{6x+8}{x^2+3x+2} dx = \int \frac{2}{x+1} + \frac{4}{x+2} \) which is easier to integrate and results in \( 2 \ln \mid x+1 \mid 4 \ln \mid x+1 \mid + c \)
\end{ex}

\textbf{Steps}
\begin{enumerate}
\item If degree p(x) is greater that degree q(x), then divide q(x) into p(x) using long divison
\item Factor q(x)
\item For every linear factor \( (ax+b)^ n\)  in q(x) include the following terms \( \frac{A_1}{ax+B} + \ldots + \frac{A_n}{(ax+B)^n} \)
\item Multiply by all factors to get rid of the denominators 
\item Compare coefficients
\item Solve for A , B , C , etc 
\item Sub Terms back in and integrate
\end{enumerate}



\begin{center}
\textbf{End of Lecture Notes} \\
\textbf{Notes By : Harsh Mistry}
\end{center}
\end{document}
