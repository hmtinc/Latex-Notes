%Notes by Harsh Mistry 
%CS 458
%Based on Template From  https://www.cs.cmu.edu/~ggordon/10725-F12/template.tex

\documentclass[twoside]{article}
\setlength{\oddsidemargin}{0.25 in}
\setlength{\evensidemargin}{-0.25 in}
\setlength{\topmargin}{-0.6 in}
\setlength{\textwidth}{6.5 in}
\setlength{\textheight}{8.5 in}
\setlength{\headsep}{0.75 in}
\setlength{\parindent}{0 in}
\setlength{\parskip}{0.1 in}
\usepackage{amsmath,amsfonts,graphicx}
\newcounter{lecnum}
\renewcommand{\thepage}{\thelecnum-\arabic{page}}
\renewcommand{\thesection}{\thelecnum.\arabic{section}}
\renewcommand{\theequation}{\thelecnum.\arabic{equation}}
\renewcommand{\thefigure}{\thelecnum.\arabic{figure}}
\renewcommand{\thetable}{\thelecnum.\arabic{table}}
\newcommand{\lecture}[4]{
   \pagestyle{myheadings}
   \thispagestyle{plain}
   \newpage
   \setcounter{lecnum}{#1}
   \setcounter{page}{1}
   
   \graphicspath{ {images/} }
   
%Info Box 
   \begin{center}
   \framebox{
      \vbox{\vspace{2mm}
    \hbox to 6.28in { {\bf CS 458/658 - Computer Security and Privacy
	\hfill Fall 2018} }
       \vspace{4mm}
       \hbox to 6.28in { {\Large \hfill Lecture #1: #2  \hfill} }
       \vspace{2mm}
       \hbox to 6.28in { {\it Lecturer: #3 \hfill Notes By: #4} }
      \vspace{2mm}}
   }
   \end{center}
   
   \markboth{Lecture #1: #2}{Lecture #1: #2}



 
}

\renewcommand{\cite}[1]{[#1]}
\def\beginrefs{\begin{list}%
        {[\arabic{equation}]}{\usecounter{equation}
         \setlength{\leftmargin}{2.0truecm}\setlength{\labelsep}{0.4truecm}%
         \setlength{\labelwidth}{1.6truecm}}}
\def\endrefs{\end{list}}
\def\bibentry#1{\item[\hbox{[#1]}]}

\newcommand{\fig}[3]{
			\vspace{#2}
			\begin{center}
			Figure \thelecnum.#1:~#3
			\end{center}
	}

\newtheorem{theorem}{Theorem}[lecnum]
\newtheorem{lemma}[theorem]{Lemma}
\newtheorem{ex}[theorem]{Example}
\newtheorem{proposition}[theorem]{Proposition}
\newtheorem{claim}[theorem]{Claim}
\newtheorem{corollary}[theorem]{Corollary}
\newtheorem{definition}[theorem]{Definition}
\newenvironment{proof}{{\bf Proof:}}{\hfill\rule{2mm}{2mm}}
\newcommand\E{\mathbb{E}}


%Start of Document 
\begin{document}

\lecture{7}{September 20, 2018}{Ian Goldberg }{Harsh Mistry}

\section{Operating Systems Continued}

\subsection{User Authentication Continued}

\subsubsection{Graphical Passwords}
\begin{itemize}
\item Graphical passwords are an alternative to text based passwords
\item There are multiple techniques such as 
\begin{itemize}
\item User choosing a picture to login 
\item User choosing a set of places in a picture
\end{itemize}
\end{itemize}
\subsubsection{Server Authentication}
\begin{itemize}
\item With the help of a password, system authenticates user (client)
\item But user should also authenticate system (server) else might end up with attacker
\end{itemize}

\subsubsection{Biometrics}
\begin{itemize}
\item Biometrics have been hailed as a way to get rid of the problems with password and token-based authentication
\item Unfortunately, they have their own problems
\item Biometrics are based on the concept of using physical characteristics
\item If observed trait is sufficiently close to previously stored trait, the system must accept the user
\item Biometrics can't be changed if compromised
\item Biometrics work well for local authentication, but are less suited for remote authentication or for identification. 
\item With local authentication, a guard can ensure that the person is indeed the individual and not someone trying to fool the system
\item Authentication of biometrics is ensuring a captured trait correspond to a particular stored trait.
\item Identification is ensuring a capture trait corresponds to any of the stored traits
\item False positives can make biometrics-based identification useless
\end{itemize}


\subsection{Security Policies and Models}
\subsubsection{Trusted Operating Systems}
\begin{itemize}
\item Trusting an entity means that if this entity misbehaves, the security of the system fails.
\item We trust an OS if we have confidence that is provides security services
\item Trusted operating systems typically builds on four factors 
\begin{itemize}
\item Policy : A set of rules outlining what is secured and why
\item Model : A model that implements the policy and that can be used for reasoning about the policy
\item Design : A specification of how the OS implements the model 
\item Trust: Assurance that the OS is implemented according to design 
\end{itemize}
\item Trusted software means it does what its expected and \textbf{Nothing More!}
\begin{itemize}
\item Functional correctness : Software works correctly
\item Enforcement of integrity : Wrong inputs don't impact correctness of data
\item Limit Privilege : Access rights are minimized and not passed to others
\item Appropriate confidence level : Software has been rated as required by environment
\end{itemize}
\item Trust can change over time
\end{itemize}
\subsubsection{Security Policies}
\begin{itemize}
\item Many OS security policies have their roots in military security policies
\item Each object has a sensitivity level 
\item Each object might also be assigned to one or more compartments

\end{itemize}

\end{document}





