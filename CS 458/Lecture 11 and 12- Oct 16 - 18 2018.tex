%Notes by Harsh Mistry 
%CS 458
%Based on Template From  https://www.cs.cmu.edu/~ggordon/10725-F12/template.tex

\documentclass[twoside]{article}
\setlength{\oddsidemargin}{0.25 in}
\setlength{\evensidemargin}{-0.25 in}
\setlength{\topmargin}{-0.6 in}
\setlength{\textwidth}{6.5 in}
\setlength{\textheight}{8.5 in}
\setlength{\headsep}{0.75 in}
\setlength{\parindent}{0 in}
\setlength{\parskip}{0.1 in}
\usepackage{amsmath,amsfonts,graphicx}
\newcounter{lecnum}
\renewcommand{\thepage}{\thelecnum-\arabic{page}}
\renewcommand{\thesection}{\thelecnum.\arabic{section}}
\renewcommand{\theequation}{\thelecnum.\arabic{equation}}
\renewcommand{\thefigure}{\thelecnum.\arabic{figure}}
\renewcommand{\thetable}{\thelecnum.\arabic{table}}
\newcommand{\lecture}[4]{
   \pagestyle{myheadings}
   \thispagestyle{plain}
   \newpage
   \setcounter{lecnum}{#1}
   \setcounter{page}{1}
   
   \graphicspath{ {images/} }
   
%Info Box 
   \begin{center}
   \framebox{
      \vbox{\vspace{2mm}
    \hbox to 6.28in { {\bf CS 458/658 - Computer Security and Privacy
	\hfill Fall 2018} }
       \vspace{4mm}
       \hbox to 6.28in { {\Large \hfill Lecture #1: #2  \hfill} }
       \vspace{2mm}
       \hbox to 6.28in { {\it Lecturer: #3 \hfill Notes By: #4} }
      \vspace{2mm}}
   }
   \end{center}
   
   \markboth{Lecture #1: #2}{Lecture #1: #2}



 
}

\renewcommand{\cite}[1]{[#1]}
\def\beginrefs{\begin{list}%
        {[\arabic{equation}]}{\usecounter{equation}
         \setlength{\leftmargin}{2.0truecm}\setlength{\labelsep}{0.4truecm}%
         \setlength{\labelwidth}{1.6truecm}}}
\def\endrefs{\end{list}}
\def\bibentry#1{\item[\hbox{[#1]}]}

\newcommand{\fig}[3]{
			\vspace{#2}
			\begin{center}
			Figure \thelecnum.#1:~#3
			\end{center}
	}

\newtheorem{theorem}{Theorem}[lecnum]
\newtheorem{lemma}[theorem]{Lemma}
\newtheorem{ex}[theorem]{Example}
\newtheorem{proposition}[theorem]{Proposition}
\newtheorem{claim}[theorem]{Claim}
\newtheorem{corollary}[theorem]{Corollary}
\newtheorem{definition}[theorem]{Definition}
\newenvironment{proof}{{\bf Proof:}}{\hfill\rule{2mm}{2mm}}
\newcommand\E{\mathbb{E}}


%Start of Document 
\begin{document}

\lecture{11  and 12}{October 16 - 18, 2018}{Ian Goldberg }{Harsh Mistry}

\section{Network Security Continued} 
\subsection{Network Concepts}
\subsubsection{Web Site Vulnerabilities}
\begin{itemize}
\item Web site defacements
\item Attacked can examine returned HTML code for vulnerabilities
\item Attacked can send malicious  URL to web server
\item HTTP protocol is stateless, so client is often asked to keep the state in the form of a cookie or URL. This can be abused by submitting modified state information to the server
\item Cross-Site Scripting or Request Forgery attacked which are a form of code injection can be used to add HTML code to somebody else's page
\item XSS : code steals sensitive information contained in the web page and sends it to attacker
\item CSRF : Code performs malicious action at some web site if user is currently logged in there
\end{itemize}


\subsection{Denial of Service}
\begin{itemize}
\item Exploit knowledge of implementation details about a node to make node perform poorly
\item A SYN flood is when a attacker initiates a TCP condition and doesn't send any acks
\item DNS attacks are also common where a users cache can be filled with incorrect host info
\end{itemize}

\subsubsection{Reflection and Amplification of DDoS Attack}
\begin{itemize}
\item An attack where the victim is flooded with legitimate-looking traffic that originates from unsuspecting network nodes on the internet
\item Amplification : A vulnerable network node runs a service that responds to queries with much more data than the query itself
\item Reflection : The attacker spoofs the source address of the queries to that of the victim so that the vulnerable network nodes send responses to the victim
\end{itemize}

\subsubsection{Distributed Denial of Service}
\begin{itemize}
\item Similar to DoS expect the attack is spread across a series of devices to mask the source
\item This networks of devices are often refereed to as a botnets
\end{itemize}

\subsection{Active code}
\begin{itemize}
\item To reduce load on server, a server might ask a client to execute code on its behalf
\item Obviously, this can be dangerous for clients as this code could possibly be malicious
\item To combat this, its ideal to run untrusted code in a sandbox
\end{itemize}


\subsection{Network Security Controls}
\begin{itemize}
\item Ensure the design validates user inputs
\item Separate services across multiple devices
\item Ensure services are duplicated to ensure redundancy
\item Use Firewalls to filter traffic
\item Attempt to pass traffic through a few routers to enable easier monitors
\end{itemize}

\subsection{Firewalls}
\begin{itemize}
\item Firewalls allow for all traffic to be passed through choke points
\item Types of firewalls
\begin{itemize}
\item Packet filtering gateways : The simplest type which make decisions based on just the header
\item Stateful inspection firewalls: Keeps state to identify packets that belong together
\item Application proxy : All traffic for a specific application is passed through a proxy
\item Personal firewalls : Typically forbid everything unless explcity allowed and run on a users computer
\end{itemize}
\end{itemize}

\subsection{DMZ}
\begin{itemize}
\item Sub-network that contains an organizations external services
\item Often is set-up between a internal and external firewall
\end{itemize}

\subsection{Intrusion detection systems}
\begin{itemize}
\item Firewalls do not protect agaonst inside attackers or insiders making mistakes can be subverted
\item IDSs are next line of defense
\item IDss monitor activity to identify malicious or suspicious events
\item Host Based IDSs
\begin{itemize}
\item Run on a host to protect the host 
\item Can exploit lots of information, and misses out on information available to other hosts
\end{itemize}
\item Network Based IDSs
\begin{itemize}
\item Runs on dedicated node to protect all hosts on a network 
\item Has to rely on on information available in monitored packets
\end{itemize}
\item Distributed IDSs combine both Network and Host based approaches
\item Signature Based IDSs
\begin{itemize}
\item Attack signatures are compared aganist a collection of known signatures. 
\end{itemize}
\item Heuristic/anomaly Based IDSs
\begin{itemize}
\item Looks for behaviour that is out the ordinary by modelling good behaviour and raising alerts when system activity no longer resembles this model
\item  All activity is classified as good or benign, suspicious, or unknown
\item  The primary disadvantage is that even goof IDs using this method take time to learn and classify unknown events. 
\end{itemize}
\end{itemize}

\end{document}





