%Notes by Harsh Mistry 
%CS 458
%Based on Template From  https://www.cs.cmu.edu/~ggordon/10725-F12/template.tex

\documentclass[twoside]{article}
\setlength{\oddsidemargin}{0.25 in}
\setlength{\evensidemargin}{-0.25 in}
\setlength{\topmargin}{-0.6 in}
\setlength{\textwidth}{6.5 in}
\setlength{\textheight}{8.5 in}
\setlength{\headsep}{0.75 in}
\setlength{\parindent}{0 in}
\setlength{\parskip}{0.1 in}
\usepackage{amsmath,amsfonts,graphicx}
\newcounter{lecnum}
\renewcommand{\thepage}{\thelecnum-\arabic{page}}
\renewcommand{\thesection}{\thelecnum.\arabic{section}}
\renewcommand{\theequation}{\thelecnum.\arabic{equation}}
\renewcommand{\thefigure}{\thelecnum.\arabic{figure}}
\renewcommand{\thetable}{\thelecnum.\arabic{table}}
\newcommand{\lecture}[4]{
   \pagestyle{myheadings}
   \thispagestyle{plain}
   \newpage
   \setcounter{lecnum}{#1}
   \setcounter{page}{1}
   
   \graphicspath{ {images/} }
   
%Info Box 
   \begin{center}
   \framebox{
      \vbox{\vspace{2mm}
    \hbox to 6.28in { {\bf CS 458/658 - Computer Security and Privacy
	\hfill Fall 2018} }
       \vspace{4mm}
       \hbox to 6.28in { {\Large \hfill Lecture #1: #2  \hfill} }
       \vspace{2mm}
       \hbox to 6.28in { {\it Lecturer: #3 \hfill Notes By: #4} }
      \vspace{2mm}}
   }
   \end{center}
   
   \markboth{Lecture #1: #2}{Lecture #1: #2}



 
}

\renewcommand{\cite}[1]{[#1]}
\def\beginrefs{\begin{list}%
        {[\arabic{equation}]}{\usecounter{equation}
         \setlength{\leftmargin}{2.0truecm}\setlength{\labelsep}{0.4truecm}%
         \setlength{\labelwidth}{1.6truecm}}}
\def\endrefs{\end{list}}
\def\bibentry#1{\item[\hbox{[#1]}]}

\newcommand{\fig}[3]{
			\vspace{#2}
			\begin{center}
			Figure \thelecnum.#1:~#3
			\end{center}
	}

\newtheorem{theorem}{Theorem}[lecnum]
\newtheorem{lemma}[theorem]{Lemma}
\newtheorem{ex}[theorem]{Example}
\newtheorem{proposition}[theorem]{Proposition}
\newtheorem{claim}[theorem]{Claim}
\newtheorem{corollary}[theorem]{Corollary}
\newtheorem{definition}[theorem]{Definition}
\newenvironment{proof}{{\bf Proof:}}{\hfill\rule{2mm}{2mm}}
\newcommand\E{\mathbb{E}}


%Start of Document 
\begin{document}

\lecture{7}{September 20, 2018}{Ian Goldberg }{Harsh Mistry}

\section{Operating Systems Continued}

\section{Access Control Continued}

\subsubsection{Role-based access control (RBAC)}
\begin{itemize}
\item In a company, objects that a user can access often do not depend on the identity of the user.
\item RBAC involves an administrator assigning a role to a user which grants access certain rights to a rolw
\item RBAC Extensions 
\begin{itemize}
\item Hierarchical roles
\item Multiple Roles
\item Separation of Duty - Rights for a task are split across multiple roles
\end{itemize}
\end{itemize}

\section{User Authentication}
\begin{itemize}
\item Computer systems often have to identify and authenticate users before authorizing them
\item \textbf{Identification} is determining who you are
\item \textbf{Authentication} is proving the individuals identity
\end{itemize}

\subsection{Authentication Factors}
\begin{itemize}
\item Factors the user knows (PIN, Password, etc)
\item Factors the user has (Card, Badge, etc)
\item Factors impacting who the user is (Biometrics, etc) 
\item Factors impacting the users context (Location, time, device proximity, etc) 
\end{itemize}

\subsubsection{Combination of Authentication Factors}
\begin{itemize}
\item Different classes of authentication factors can be combined for more solid authentication
\item User multiple factors from the same class might not provide better authentication
\end{itemize}

\subsubsection{Passwords}
\begin{itemize}
\item Probably oldest authentication mechanism used in computer systems
\item Many usability problems, such as 
\begin{itemize}
\item Entering passwords is inconvenient
\item Password composition/change rules
\item Forgotten passwords may not be recoverable 
\item If password is shared among many people, password update becomes difficult
\end{itemize}
\item Security Problems
\begin{itemize}
\item If password is disclosed to unauthorized individual
\item Shoulder surfing 
\item Keystroke logging
\item Interface illusions / Phishing 
\item Passwords could be reused across sites
\item Passwords can be guessed
\end{itemize}
\end{itemize}


\end{document}





