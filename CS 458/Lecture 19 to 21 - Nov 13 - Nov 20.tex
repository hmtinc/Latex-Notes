%Notes by Harsh Mistry 
%CS 458
%Based on Template From  https://www.cs.cmu.edu/~ggordon/10725-F12/template.tex

\documentclass[twoside]{article}
\setlength{\oddsidemargin}{0.25 in}
\setlength{\evensidemargin}{-0.25 in}
\setlength{\topmargin}{-0.6 in}
\setlength{\textwidth}{6.5 in}
\setlength{\textheight}{8.5 in}
\setlength{\headsep}{0.75 in}
\setlength{\parindent}{0 in}
\setlength{\parskip}{0.1 in}
\usepackage{amsmath,amsfonts,graphicx}
\newcounter{lecnum}
\renewcommand{\thepage}{\thelecnum-\arabic{page}}
\renewcommand{\thesection}{\thelecnum.\arabic{section}}
\renewcommand{\theequation}{\thelecnum.\arabic{equation}}
\renewcommand{\thefigure}{\thelecnum.\arabic{figure}}
\renewcommand{\thetable}{\thelecnum.\arabic{table}}
\newcommand{\lecture}[4]{
   \pagestyle{myheadings}
   \thispagestyle{plain}
   \newpage
   \setcounter{lecnum}{#1}
   \setcounter{page}{1}
   
   \graphicspath{ {images/} }
   
%Info Box 
   \begin{center}
   \framebox{
      \vbox{\vspace{2mm}
    \hbox to 6.28in { {\bf CS 458/658 - Computer Security and Privacy
	\hfill Fall 2018} }
       \vspace{4mm}
       \hbox to 6.28in { {\Large \hfill Lecture #1: #2  \hfill} }
       \vspace{2mm}
       \hbox to 6.28in { {\it Lecturer: #3 \hfill Notes By: #4} }
      \vspace{2mm}}
   }
   \end{center}
   
   \markboth{Lecture #1: #2}{Lecture #1: #2}



 
}

\renewcommand{\cite}[1]{[#1]}
\def\beginrefs{\begin{list}%
        {[\arabic{equation}]}{\usecounter{equation}
         \setlength{\leftmargin}{2.0truecm}\setlength{\labelsep}{0.4truecm}%
         \setlength{\labelwidth}{1.6truecm}}}
\def\endrefs{\end{list}}
\def\bibentry#1{\item[\hbox{[#1]}]}

\newcommand{\fig}[3]{
			\vspace{#2}
			\begin{center}
			Figure \thelecnum.#1:~#3
			\end{center}
	}

\newtheorem{theorem}{Theorem}[lecnum]
\newtheorem{lemma}[theorem]{Lemma}
\newtheorem{ex}[theorem]{Example}
\newtheorem{proposition}[theorem]{Proposition}
\newtheorem{claim}[theorem]{Claim}
\newtheorem{corollary}[theorem]{Corollary}
\newtheorem{definition}[theorem]{Definition}
\newenvironment{proof}{{\bf Proof:}}{\hfill\rule{2mm}{2mm}}
\newcommand\E{\mathbb{E}}


%Start of Document 
\begin{document}

\lecture{19 - 21}{November 13 - 20, 2018}{Ian Goldberg }{Harsh Mistry}

\section{Database Security and Privacy} 

\subsection{Security Requirements}
\begin{itemize}
\item Physical Database Integrity
\begin{itemize}
\item Protect against database corruption
\item Allow only authorized individuals to perform updates
\item Recover from physical problems by performing backups or using transactions
\end{itemize}
\item Logical Database Integrity 
\item Element Integrity 
\begin{itemize}
\item Ensure correctness/accuracy of database elements
\item Access control to limit who can update element 
\item Element checks to validate correctness
\item Change log or shadow field to undo erroneous changes
\item Error detection codes to protect against OS or hard disk problems
\item Two-phase update 
\begin{itemize}
\item Phase 1 : Gather information required for changes
\item Phase 2 : Make changes permanent 
\item Repeat each step if problem arises
\end{itemize}
\item Concurrency Control - Operations need to be atomic
\end{itemize}
\item Referential Integrity
\begin{itemize}
\item There should be no dangling foreign keys
\end{itemize}
\item Auditability
\begin{itemize}
\item keep audit log of all database accesses
\item Access control can be difficult, audit log allows to retroactively identify users
\end{itemize}
\item User authentication 
\item Availability
\end{itemize}


\subsection{Data disclosure and inference}

\subsubsection{Types of data disclosure}
\begin{itemize}
\item Exact data
\item Bounds 
\item Negative result
\item Existence
\item Probable value
\end{itemize}

\subsubsection{Security vs. precision}
\begin{itemize}
\item Security : Forbid any queries that access sensitive data, even if (aggregated) result is no longer sensitive
\item Precision : Aggregated result should reveal as much non-sensitive data as possible
\end{itemize}

\subsubsection{Data inference}
\begin{itemize}
\item When sensitive data is derived from sensitive data
\item Controls for statistical inference attacks
\begin{itemize}
\item Apply control to query or to data items 
\item Suppress sensitive data from result
\item Conceal answer by providing a value close to the actual value
\item n-item k percent rule : For the set of records that were included in the result, if there is a subset of n records that is responsible for over
k percent of the result, omit the n records from result
\item Combined results : report set or range of possible values
\item Random sample : Compute result on random sample of database
\item Random data perturbation : Add or subtract small random error to/from each value before computing result
\item Query analysis :  Maintain history of user’s queries and observe possible
inferences
\end{itemize}
\end{itemize}

\subsubsection{Data aggregation}
\begin{itemize}
\item Related to data inference
\item Building sensitive results from less sensitive inputs
\item Aggregation can take place outside of a DBMS, which makes it difficult to control
\end{itemize}

\subsection{Multilevel Security (MLS) Databases}
\begin{itemize}
\item Support classification/compartmentalization of information according to its confidentiality
\item At element level if necessary
\item In an MLS database, each object has a sensitivity classification and maybe a set of compartments
\end{itemize}

\subsubsection{*-Property}
\begin{itemize}
\item Implementing the *-property (no read up, no write down) in an MLS database is difficult
\item DBMS must have read and write access at all levels to answer user queries, perform back-ups, optimize database,. 
\end{itemize}

\subsubsection{Confidentiality}
\begin{itemize}
\item Depending on a user’s clearance, he/she might get different answers for a query
\item Existence of a record itself could be confidential
\item Keeping existence hidden can lead to having multiple records with the same primary key, but different sensitivity (polyinstantiation)
\end{itemize}

\subsubsection{Partitioning}
\begin{itemize}
\item Have separate database for each classification level
\item Might lead to data stored redundantly in multiple databases
\item Doesn’t address the problem of a high-level user needing access to low-level data combined with high-level data
\end{itemize}

\subsubsection{Encryption}
\begin{itemize}
\item Separate data by encrypting it with a key unique to its classification level
\item Must be careful to use encryption scheme in the right way
\item Processing of a query becomes expensive, many records might have to be decrypted
\end{itemize}

\subsubsection{Integrity lock}
\begin{itemize}
\item Provides both integrity and access control
\item Each data item consist of, the actual item, an integrity level, and a cryptographic signature
\item Signature protects against attacks on the above fields, such as attacks trying to modify the sensitivity label, and attacks trying to move/copy the item in the database
\item This scheme does not protect against replay attacks
\item Any (untrusted) database can be used to store data items and their integrity locks
\item (Trusted) procedure handles access control and manages integrity locks
\item Have to encrypt items and locks if there are other ways to get access to data in database
\end{itemize}

\subsection{Designs of secure databases}
\begin{itemize}
\item Trusted front end 
\begin{itemize}
\item Front end authenticates a user and forwards user query to old-style DBMS
\item Front end gets result from DBMS and removes data items that user is not allowed to see
\item Inefficient if DBMS returns lots of items and most of them are being dropped by front end
\end{itemize}
\item Commutative filters
\begin{itemize}
\item Front end re-writes user query according to a user’s classification
\item Benefits from DBMS’ superior query processing capabilities and discards forbidden data items early on
\item Front end might still have to do some post processing
\end{itemize}
\item Distributed/federated databases
\begin{itemize}
\item Based on partitioning
\item Front end forwards user query only to databases that user can access based on classification
\item Front end might have to combine the results from multiple databases
\end{itemize}
\item Views
\begin{itemize}
\item Many DBMS support views
\item A view is logical database that represents a subset of some other database
\end{itemize}
\item Truman vs. non-Truman semantics
\begin{itemize}
\item Truman semantics: the DBMS pretends that the data the user is allowed to access is all the data there is
\item Non-Truman semantics: the DBMS can reject queries that ask for data the user is not allowed to access
\end{itemize}
\end{itemize}

\subsection{Data Mining and Data release}
\begin{itemize}
\item Multilevel databases weren’t a commercial success
\item data miners actively gather additional data from third parties
\item Data mining tries to automatically find interesting patterns in data using a plethora of technologies
\item Data mining can be useful for security purposes
\item Security Problems
\begin{itemize}
\item Confidentiality - Derivation of sensitive information
\item Integrity - Mistakes in data
\item Availability - Incompatibility of different databases  
\end{itemize}
\end{itemize}
\end{document}





