%Notes by Harsh Mistry 
%CS 350
%Based on Template From  https://www.cs.cmu.edu/~ggordon/10725-F12/template.tex

\documentclass[twoside]{article}
\setlength{\oddsidemargin}{0.25 in}
\setlength{\evensidemargin}{-0.25 in}
\setlength{\topmargin}{-0.6 in}
\setlength{\textwidth}{6.5 in}
\setlength{\textheight}{8.5 in}
\setlength{\headsep}{0.75 in}
\setlength{\parindent}{0 in}
\setlength{\parskip}{0.1 in}
\usepackage{amsmath,amsfonts,graphicx}
\newcounter{lecnum}
\renewcommand{\thepage}{\thelecnum-\arabic{page}}
\renewcommand{\thesection}{\thelecnum.\arabic{section}}
\renewcommand{\theequation}{\thelecnum.\arabic{equation}}
\renewcommand{\thefigure}{\thelecnum.\arabic{figure}}
\renewcommand{\thetable}{\thelecnum.\arabic{table}}
\newcommand{\lecture}[4]{
   \pagestyle{myheadings}
   \thispagestyle{plain}
   \newpage
   \setcounter{lecnum}{#1}
   \setcounter{page}{1}
   
   \graphicspath{ {images/} }
   
%Info Box 
   \begin{center}
   \framebox{
      \vbox{\vspace{2mm}
    \hbox to 6.28in { {\bf CS 458/658 - Computer Security and Privacy
	\hfill Fall 2018} }
       \vspace{4mm}
       \hbox to 6.28in { {\Large \hfill Lecture #1: #2  \hfill} }
       \vspace{2mm}
       \hbox to 6.28in { {\it Lecturer: #3 \hfill Notes By: #4} }
      \vspace{2mm}}
   }
   \end{center}
   
   \markboth{Lecture #1: #2}{Lecture #1: #2}



 
}

\renewcommand{\cite}[1]{[#1]}
\def\beginrefs{\begin{list}%
        {[\arabic{equation}]}{\usecounter{equation}
         \setlength{\leftmargin}{2.0truecm}\setlength{\labelsep}{0.4truecm}%
         \setlength{\labelwidth}{1.6truecm}}}
\def\endrefs{\end{list}}
\def\bibentry#1{\item[\hbox{[#1]}]}

\newcommand{\fig}[3]{
			\vspace{#2}
			\begin{center}
			Figure \thelecnum.#1:~#3
			\end{center}
	}

\newtheorem{theorem}{Theorem}[lecnum]
\newtheorem{lemma}[theorem]{Lemma}
\newtheorem{ex}[theorem]{Example}
\newtheorem{proposition}[theorem]{Proposition}
\newtheorem{claim}[theorem]{Claim}
\newtheorem{corollary}[theorem]{Corollary}
\newtheorem{definition}[theorem]{Definition}
\newenvironment{proof}{{\bf Proof:}}{\hfill\rule{2mm}{2mm}}
\newcommand\E{\mathbb{E}}


%Start of Document 
\begin{document}

\lecture{3}{September 13, 2018}{Ian Goldberg }{Harsh Mistry}

\section{Program Security Continued}
\subsection{Unintentional Flaws Continued }
\subsubsection{TOCTOU Errors}
\begin{itemize}
\item Time of Check to Time of Use errors or otherwise known as race-conditions are errors caused by a change of state after a verification check
\item The fundamental problem is the after a verification is done, certain values can be modified to allow for access to disallowed behaviour. 
\item Defences 
\begin{itemize}
\item When performing a privileged action on behalf of anoteh rparty, make sure all information is constant
\item Keep a private copy of request, so it can't be altered. 
\item Where possible act on the object itself, and not on some level of indirection
\item If above cases do not apply, use locks to ensure the object is not changed during run time
\end{itemize}
\end{itemize}

\subsection{Intentional Malicious Flaws}

Various forms of software are written with malicious intent, but a common characteristic is all malware needs to be executed in order to cause harm. 
The types of malware are Viruses, Worms, Trojans, and Logic Bombs

\subsubsection{Viruses}
\begin{itemize}
\item A virus is malware that infects other files in attempt to replicate it self. 
\item Viruses typically exuctables are modified to include jump instructions to the virus code in effort to ensure the virus propagates across the system.
\item Viruses will also try to infect the computer itself such as writing it self to the boot sector or running the entire system in a hyper-visor.
\item Viruses often contain a payload which is to be activated at a future date to execute the intended action of the virus. 
\item Finding viruses can be done in two ways
\begin{itemize}
\item Detect from time to time, scan the entire system. 
\item Detect viruses when new files are added to the computer
\end{itemize}
\item Signature based detection 
\begin{itemize}
\item A unique portion/characteristic of the program is used to form a signature of the virus which can be compared against a running list. 
\item Viruses can be polymorphic and modify it self in order to ensure the signature does not match. As a result, signature doesn't really effect polymorphic protection
\end{itemize}
\item Behaviour based detection
\begin{itemize}
\item Behaviour based systems detect viruses by analysing its core-behaviour and purpose. They usually do this within a sandboxed environment. 
\end{itemize}
\item False Negative/Positives
\begin{itemize}
\item Behaviour Detection has a tendency to have higher false positives.  
\item Signature Detection has a tendency to have higher false negatives.
\end{itemize}
\item False Positives must be lower than the base rate (The actual true percentage of viruses)
\end{itemize}

\subsubsection{Worms}
\begin{itemize}
\item A worm is a self-contained piece of code that can replicate with little or no user involvement
\item Worms often use security flaws in widely deployed software as a path to infection
\item Worms typically start searching for other computers to infect. Additionally, there may or may not be a payload that activates at a certain time or by another trigger.
\end{itemize}

\subsubsection{Trojans}
\begin{itemize}
\item Trojan horses are programs which claim to do something innocuous (and usually do), but which also hide malicious behaviour
\item Often Gain control by getting the user to run code of the attacker’s choice, usually by also providing some code the user wants to run. 
\item Usually are marked as \verb|PUP| (potentially unwanted programs)
\item Trojan horses usually do not themselves spread between computers; they rely on multiple users executing the “trojaned” software
\end{itemize}

\end{document}





