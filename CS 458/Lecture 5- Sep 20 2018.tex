%Notes by Harsh Mistry 
%CS 458
%Based on Template From  https://www.cs.cmu.edu/~ggordon/10725-F12/template.tex

\documentclass[twoside]{article}
\setlength{\oddsidemargin}{0.25 in}
\setlength{\evensidemargin}{-0.25 in}
\setlength{\topmargin}{-0.6 in}
\setlength{\textwidth}{6.5 in}
\setlength{\textheight}{8.5 in}
\setlength{\headsep}{0.75 in}
\setlength{\parindent}{0 in}
\setlength{\parskip}{0.1 in}
\usepackage{amsmath,amsfonts,graphicx}
\newcounter{lecnum}
\renewcommand{\thepage}{\thelecnum-\arabic{page}}
\renewcommand{\thesection}{\thelecnum.\arabic{section}}
\renewcommand{\theequation}{\thelecnum.\arabic{equation}}
\renewcommand{\thefigure}{\thelecnum.\arabic{figure}}
\renewcommand{\thetable}{\thelecnum.\arabic{table}}
\newcommand{\lecture}[4]{
   \pagestyle{myheadings}
   \thispagestyle{plain}
   \newpage
   \setcounter{lecnum}{#1}
   \setcounter{page}{1}
   
   \graphicspath{ {images/} }
   
%Info Box 
   \begin{center}
   \framebox{
      \vbox{\vspace{2mm}
    \hbox to 6.28in { {\bf CS 458/658 - Computer Security and Privacy
	\hfill Fall 2018} }
       \vspace{4mm}
       \hbox to 6.28in { {\Large \hfill Lecture #1: #2  \hfill} }
       \vspace{2mm}
       \hbox to 6.28in { {\it Lecturer: #3 \hfill Notes By: #4} }
      \vspace{2mm}}
   }
   \end{center}
   
   \markboth{Lecture #1: #2}{Lecture #1: #2}



 
}

\renewcommand{\cite}[1]{[#1]}
\def\beginrefs{\begin{list}%
        {[\arabic{equation}]}{\usecounter{equation}
         \setlength{\leftmargin}{2.0truecm}\setlength{\labelsep}{0.4truecm}%
         \setlength{\labelwidth}{1.6truecm}}}
\def\endrefs{\end{list}}
\def\bibentry#1{\item[\hbox{[#1]}]}

\newcommand{\fig}[3]{
			\vspace{#2}
			\begin{center}
			Figure \thelecnum.#1:~#3
			\end{center}
	}

\newtheorem{theorem}{Theorem}[lecnum]
\newtheorem{lemma}[theorem]{Lemma}
\newtheorem{ex}[theorem]{Example}
\newtheorem{proposition}[theorem]{Proposition}
\newtheorem{claim}[theorem]{Claim}
\newtheorem{corollary}[theorem]{Corollary}
\newtheorem{definition}[theorem]{Definition}
\newenvironment{proof}{{\bf Proof:}}{\hfill\rule{2mm}{2mm}}
\newcommand\E{\mathbb{E}}


%Start of Document 
\begin{document}

\lecture{4}{September 20, 2018}{Ian Goldberg }{Harsh Mistry}

\section{Program Security Continued}

\subsection{Non Malicious Flaws Continued}
\subsubsection{Side Channels}
\begin{itemize}
\item Side Channel Attacks are powerful attacks that simply involve observing how computers behave in response to sensitive data.
\item The attacker just has to abuse the nature of how the computer works or how users interact with them. 
\item \textbf{Constant time} implementations are programs that minimize number of side channels
\end{itemize}

\subsection{Designing Programs for Security}
\subsubsection{Modularity}
\begin{itemize}
\item Break the module into a number of small pieces each responsible for a specific portion
\item Modules should also have low coupling, as in they should not be tightly related to other modules.
\item This helps localize errors and flaws to a specific module.
\end{itemize}
\subsubsection{Encapsulation}
\begin{itemize}
\item Ensure all information is contained, only share information necessary
\item Ideally, developers for each module should not need to know how the other modules functions. They should just need to know the API
\end{itemize}
\subsubsection{Information Hiding}
\begin{itemize}
\item Expanding on encapsulation, external modules should not have access to a modules internal information.
\item In essence, internal module data should be hidden
\end{itemize}
\subsubsection{Mutual Suspicion}
\begin{itemize}
\item Even though modules work together, each individual module should validate input
\item This is done to ensure that security flaws don't propagate through the applications. 
\item Basically, each module needs to protect against attack vectors exists in other modules
\end{itemize}
\subsubsection{Confinement}
\begin{itemize}
\item Expanding on mutual suspicion, modules should also be suspicious any other modules called. 
\item An example of this is \textbf{Sandboxing}
\end{itemize}
\subsection{Security Controls}
\subsubsection{Static code analysis}
\begin{itemize}
\item Static code analysis tools are applications that attempt to scan code for common security flaws.
\item Often looks for buffer overflows, but some can point out TOCTTOU errors and other flaws.
\end{itemize}
\subsubsection{Formal Methods}
\begin{itemize}
\item Utilize program verification methods to formally prove an application performs to the specification
\end{itemize}
\subsubsection{Genetic Diversity}
\begin{itemize}
\item The reason worms and viruses are able to propagate is due to the fact alot of computers are running the same vulnerable code 
\item This could potentially be avoided by using different implementations. 
\end{itemize}
\subsubsection{Code Review}
\begin{itemize}
\item Code review is the single most effective way to find faults once the code has been written
\item Kinds of code review
\begin{itemize}
\item Someone writes the code and another dedicated person review the code. This is the least effective method
\item Guided walk-through. Extremely common for safety-critical systems.
\item Easter Egg Review. Author inserts intentional security flaws to ensure the reviewer actually reviewed it. 
\end{itemize}
\end{itemize}
\end{document}





