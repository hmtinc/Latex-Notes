%Notes by Harsh Mistry 
%CS 458
%Based on Template From  https://www.cs.cmu.edu/~ggordon/10725-F12/template.tex

\documentclass[twoside]{article}
\setlength{\oddsidemargin}{0.25 in}
\setlength{\evensidemargin}{-0.25 in}
\setlength{\topmargin}{-0.6 in}
\setlength{\textwidth}{6.5 in}
\setlength{\textheight}{8.5 in}
\setlength{\headsep}{0.75 in}
\setlength{\parindent}{0 in}
\setlength{\parskip}{0.1 in}
\usepackage{amsmath,amsfonts,graphicx}
\newcounter{lecnum}
\renewcommand{\thepage}{\thelecnum-\arabic{page}}
\renewcommand{\thesection}{\thelecnum.\arabic{section}}
\renewcommand{\theequation}{\thelecnum.\arabic{equation}}
\renewcommand{\thefigure}{\thelecnum.\arabic{figure}}
\renewcommand{\thetable}{\thelecnum.\arabic{table}}
\newcommand{\lecture}[4]{
   \pagestyle{myheadings}
   \thispagestyle{plain}
   \newpage
   \setcounter{lecnum}{#1}
   \setcounter{page}{1}
   
   \graphicspath{ {images/} }
   
%Info Box 
   \begin{center}
   \framebox{
      \vbox{\vspace{2mm}
    \hbox to 6.28in { {\bf CS 458/658 - Computer Security and Privacy
	\hfill Fall 2018} }
       \vspace{4mm}
       \hbox to 6.28in { {\Large \hfill Lecture #1: #2  \hfill} }
       \vspace{2mm}
       \hbox to 6.28in { {\it Lecturer: #3 \hfill Notes By: #4} }
      \vspace{2mm}}
   }
   \end{center}
   
   \markboth{Lecture #1: #2}{Lecture #1: #2}



 
}

\renewcommand{\cite}[1]{[#1]}
\def\beginrefs{\begin{list}%
        {[\arabic{equation}]}{\usecounter{equation}
         \setlength{\leftmargin}{2.0truecm}\setlength{\labelsep}{0.4truecm}%
         \setlength{\labelwidth}{1.6truecm}}}
\def\endrefs{\end{list}}
\def\bibentry#1{\item[\hbox{[#1]}]}

\newcommand{\fig}[3]{
			\vspace{#2}
			\begin{center}
			Figure \thelecnum.#1:~#3
			\end{center}
	}

\newtheorem{theorem}{Theorem}[lecnum]
\newtheorem{lemma}[theorem]{Lemma}
\newtheorem{ex}[theorem]{Example}
\newtheorem{proposition}[theorem]{Proposition}
\newtheorem{claim}[theorem]{Claim}
\newtheorem{corollary}[theorem]{Corollary}
\newtheorem{definition}[theorem]{Definition}
\newenvironment{proof}{{\bf Proof:}}{\hfill\rule{2mm}{2mm}}
\newcommand\E{\mathbb{E}}


%Start of Document 
\begin{document}

\lecture{10}{October 11, 2018}{Ian Goldberg }{Harsh Mistry}

\section{Network Security} 
\subsection{Network Concepts}
\begin{itemize}
\item Internet is a network of network where all components communicate via TCP/IP
\end{itemize}

\subsubsection{TCP/IP protocol suite}
\begin{itemize}
\item Transport and network layer designed in the 1970's to connect local networks at different universities and research labs
\item Participants knew and trusted each other 
\item Design addressed non-malicious errors, but no delicious errors.
\end{itemize}

\subsubsection{Threats in networks}
\begin{itemize}
\item Intelligence
\item Attacks on confidentiality
\item Impersonation and spoofing
\item Attacks on integrity
\item Protocols failures
\item Web site vulnerabilities
\item Denial of Service
\item Botnets
\item Threats in active/mobile code
\item Script Kiddies
\end{itemize}

\subsubsection{Port Scan}
\begin{itemize}
\item To distinguish between multiple applications running on the server, each application runs on a port
\item Attacker sends queries to ports on target machine and tries to identify whether and what kind of application is running on a port
\item Identification based on loose-lipped applications or how exactly implements a protocol
\item Loose-lopped systems reveal information that could facilitate an attack
\item Nmap tool can identify many applications
\item Goal of attacker is to find application with remotely exploitable flaw
\end{itemize}

\subsubsection{Intelligence}
\begin{itemize}
\item Social Engineering (Attacker gathers sensitive information from a person)
\item Dumpster diving
\item Eavesdropping on oral communication
\begin{itemize}
\item Owner of node can always monitor communication flowing through node
\item Can also eavesdrop while communication is flowing across a link 
\item Eavesdropping can also occur if secure communications are mistakenly sent to the wrong recipient.  
\end{itemize}
\item Social media and cloud data can be used to collect alot of senstive information as we share more details online
\end{itemize}

\subsubsection{Impersonation}
\begin{itemize}
\item Impersonate a person by stealing his/her password
\begin{itemize}
\item Guessing attack
\item Exploit default passwords that have not been changed
\item Sniff password while it is being transmitted two nodes
\end{itemize}
\item Exploit trust relationship between machines/accounts
\begin{itemize}
\item Rhosts/rlogin allows user A on machine X to specify that user B on machine Y can act as A on X without having to enter password
\item Rlogin is trust based on encrypted or reliant on passwords
\end{itemize}
\end{itemize}
\subsubsection{Spoofing}
\begin{itemize}
\item Object masquerades as another object
\item URL spoofing
\item Web page spoofing and URL spoofing are used in Phishing attacks 
\item \textbf{Evil Twin} attack for Wifi access points
\item Spoofing is also used in session hijacking and man-in-the-middle attacks
\end{itemize}

\subsubsection{Session Hijacking}
\begin{itemize}
\item TCP protocol sets up state at sender and receiver end nodes and uses the state while exchanging packets
\item Web servers sometimes have client keep a little piece of data "cookies" to re-identify client for future visits 
\begin{itemize}
\item Attacker can sniff or steal cookie and masquerade as client
\end{itemize}
\item Man in the middle attacks can be executed to capture sensitive data
\end{itemize}

\subsubsection{Integrity Attacks}
\begin{itemize}
\item Attacker can modify packets while they are being transmitted
\begin{itemize}
\item Change  payload of packets
\item Change address of sender of receiver end node
\item Replay previously seen packets
\item Delete or create packets
\end{itemize}
\item Line noise, network congestion, or software errors, could also cause these problems.
\item DNS cache poisoning is an excellent example of an integrity attack
\begin{itemize}
\item DNS will keep a cache of mappings between domain names and destination addresses. 
\item An attacker can modify these mappings or create new wrong ones to point the user to a different end location. 
\end{itemize}
\end{itemize}

\subsubsection{Protocol Failures}
\begin{itemize}
\item TCP/IP assumes that all nodes implement protocols faithfully
\item E.g TCP includes a mechanism that ass a sender node to slow down if the network is congested. 
\item Some implementations do no check whether a packet is well formatted
\item Protocols can be very complicated, behaviour in rare cases may not be uniquely defined
\item Some protocols include broken security
\end{itemize}

\end{document}





