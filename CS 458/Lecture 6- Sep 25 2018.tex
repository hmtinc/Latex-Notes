%Notes by Harsh Mistry 
%CS 458
%Based on Template From  https://www.cs.cmu.edu/~ggordon/10725-F12/template.tex

\documentclass[twoside]{article}
\setlength{\oddsidemargin}{0.25 in}
\setlength{\evensidemargin}{-0.25 in}
\setlength{\topmargin}{-0.6 in}
\setlength{\textwidth}{6.5 in}
\setlength{\textheight}{8.5 in}
\setlength{\headsep}{0.75 in}
\setlength{\parindent}{0 in}
\setlength{\parskip}{0.1 in}
\usepackage{amsmath,amsfonts,graphicx}
\newcounter{lecnum}
\renewcommand{\thepage}{\thelecnum-\arabic{page}}
\renewcommand{\thesection}{\thelecnum.\arabic{section}}
\renewcommand{\theequation}{\thelecnum.\arabic{equation}}
\renewcommand{\thefigure}{\thelecnum.\arabic{figure}}
\renewcommand{\thetable}{\thelecnum.\arabic{table}}
\newcommand{\lecture}[4]{
   \pagestyle{myheadings}
   \thispagestyle{plain}
   \newpage
   \setcounter{lecnum}{#1}
   \setcounter{page}{1}
   
   \graphicspath{ {images/} }
   
%Info Box 
   \begin{center}
   \framebox{
      \vbox{\vspace{2mm}
    \hbox to 6.28in { {\bf CS 458/658 - Computer Security and Privacy
	\hfill Fall 2018} }
       \vspace{4mm}
       \hbox to 6.28in { {\Large \hfill Lecture #1: #2  \hfill} }
       \vspace{2mm}
       \hbox to 6.28in { {\it Lecturer: #3 \hfill Notes By: #4} }
      \vspace{2mm}}
   }
   \end{center}
   
   \markboth{Lecture #1: #2}{Lecture #1: #2}



 
}

\renewcommand{\cite}[1]{[#1]}
\def\beginrefs{\begin{list}%
        {[\arabic{equation}]}{\usecounter{equation}
         \setlength{\leftmargin}{2.0truecm}\setlength{\labelsep}{0.4truecm}%
         \setlength{\labelwidth}{1.6truecm}}}
\def\endrefs{\end{list}}
\def\bibentry#1{\item[\hbox{[#1]}]}

\newcommand{\fig}[3]{
			\vspace{#2}
			\begin{center}
			Figure \thelecnum.#1:~#3
			\end{center}
	}

\newtheorem{theorem}{Theorem}[lecnum]
\newtheorem{lemma}[theorem]{Lemma}
\newtheorem{ex}[theorem]{Example}
\newtheorem{proposition}[theorem]{Proposition}
\newtheorem{claim}[theorem]{Claim}
\newtheorem{corollary}[theorem]{Corollary}
\newtheorem{definition}[theorem]{Definition}
\newenvironment{proof}{{\bf Proof:}}{\hfill\rule{2mm}{2mm}}
\newcommand\E{\mathbb{E}}


%Start of Document 
\begin{document}

\lecture{4}{September 20, 2018}{Ian Goldberg }{Harsh Mistry}

\section{Operating Systems}

\begin{itemize}
\item An operating system allows different users to access different resources in a "shared way"
\item The operating system needs to control this sharing and provide an interface to allow this access
\item Identification and authentication are required for this access control
\end{itemize}

\subsection{Protected Objects}
\begin{itemize}
\item Memory
\item Data
\item CPU
\item Programs
\item I/I devices
\item Networks
\item OS
\end{itemize}

\subsection{Separation}
\begin{itemize}
\item Physical Separation - Use different physical resources for different users
\item Temporal Separation - Execute different users' programs at different times
\item Logical Separation - User is given the impression that no other users exists
\item Cryptographic Separation - Encrypt data and make it unintelligible to outsiders    
\end{itemize}

\subsection{Sharing}
\begin{itemize}
\item Sometimes users do want to share resources
\item As a result, OS should allow "flexible" sharing"
\end{itemize}

\subsection{Memory and Address protection}
\begin{itemize}
\item Prevent one program from corrupting other programs or data, operating system and maybe itself
\item Often, the OS can exploit hardware support for this protection, so it’s cheap
\item Memory protection is part of translation from virtual to physical addresses
\item Protection Techniques
\begin{itemize}
\item Fence Register 
\begin{itemize}
\item Exception if memory access below address in fence
register
\item Protects operating system from user programs
\item Single-user OS only
\end{itemize}
\item Base/Bounds register pair
\begin{itemize}
\item Exception if memory access below/above address in
base/bounds register
\item Different values for each user program
\item Maintained by operating system during context switch
\item Limited flexibility
\end{itemize}
\item Tagged architecture
\begin{itemize}
\item Each memory word has one or more extra bits that
identify access rights to word
\item Very flexible
\item Large overhead
\item Difficult to port OS from/to other hardware
architectures
\end{itemize}
\item Segmentation 
\item Paging
\end{itemize}
\end{itemize}

\subsubsection{Segmentation}
\begin{itemize}
\item Each program has multiple address spaces
\item Different segments for code, data, and stack
\item Virtual addresses consist of two parts: Segment Name and offset within Segment 
\item OS keeps mapping from segment name to its base physical address in Segment Table
\item OS can (transparently) relocate or resize segments
and share them between processes
\item Segment table also keeps protection attributes
\item Advantages 
\begin{itemize}
\item Each address reference is checked for protection by
hardware
\item Many different classes of data items can be assigned
different levels of protection
\item Users can share access to a segment, with potentially
different access rights
\item Users cannot access an unpermitted segment
\end{itemize}
\item Disadvantages 
\begin{itemize}
\item External fragmentation
\item Dynamic length of segments requires costly
out-of-bounds check for generated physical addresses
\item Segment names are difficult to implement efficiently
\end{itemize}
\end{itemize}

\subsubsection{Paging}
\begin{itemize}
\item Program (i.e., virtual address space) is divided into equal-sized chunks (pages)
\item Physical memory is divided into equal-sized chunks (frames)
\item Frame size equals page size
\item Virtual addresses consist of two parts: Page number and offset  within page
\item OS keeps mapping from page \(\#\) to its base physical address in Page Table
\item Page table also keeps memory protection attributes
\item Advantages
\begin{itemize}
\item Each address reference is checked for protection by
hardware
\item Users can share access to a page, with potentially
different access rights
\item Users cannot access an unpermitted page
\item Unpopular pages can be moved to disk to free memory
\end{itemize}
\item Disadvantages
\begin{itemize}
\item Internal fragmentation
\item Assigning different levels of protection to different
classes of data items not feasible
\end{itemize}
\end{itemize}


\subsubsection{x86}
\begin{itemize}
\item x86 architecture has both segmentation and paging
\item Memory protection bits indicate no access, read/write access or read-only access
\item Most processors also include NX (No eXecute) bit, forbidding execution of instructions stored in page
\end{itemize}

\section{Access Control}
Memory is only one of many objects for which OS has to run access control. In general, access control has three goals :
\begin{itemize}
\item Check every access: Else OS might fail to notice that
access has been revoked
\item Enforce least privilege: Grant program access only to
smallest number of objects required to perform a task
\item Verify acceptable use: Limit types of activity that can be performed on an object
\end{itemize}

\subsection{Access Control Structures}
There are 4 different types of control structures
\begin{itemize}
\item Access control matrix
\item Access control lists
\item Capabilities
\item Role-based access control
\end{itemize}

\subsubsection{Access Control Matrix}
\begin{itemize}
\item Set of protected objects : O
\item Set of subjects : S
\item Set of rights : R
\item Access control matrix consists of entries \(a[s,o]\), where \(s \in S, o \in O\) and \(a[s,o] \subseteq R\)
\item Access control matrix is rarely implemented as a matrix, instead an access control matric is typically implemented as 
\begin{itemize}
\item A set of access control lists
\item A set of capabilities 
\item or some combination
\end{itemize}
\end{itemize}

\end{document}





