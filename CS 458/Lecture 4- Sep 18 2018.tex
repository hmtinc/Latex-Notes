%Notes by Harsh Mistry 
%CS 350
%Based on Template From  https://www.cs.cmu.edu/~ggordon/10725-F12/template.tex

\documentclass[twoside]{article}
\setlength{\oddsidemargin}{0.25 in}
\setlength{\evensidemargin}{-0.25 in}
\setlength{\topmargin}{-0.6 in}
\setlength{\textwidth}{6.5 in}
\setlength{\textheight}{8.5 in}
\setlength{\headsep}{0.75 in}
\setlength{\parindent}{0 in}
\setlength{\parskip}{0.1 in}
\usepackage{amsmath,amsfonts,graphicx}
\newcounter{lecnum}
\renewcommand{\thepage}{\thelecnum-\arabic{page}}
\renewcommand{\thesection}{\thelecnum.\arabic{section}}
\renewcommand{\theequation}{\thelecnum.\arabic{equation}}
\renewcommand{\thefigure}{\thelecnum.\arabic{figure}}
\renewcommand{\thetable}{\thelecnum.\arabic{table}}
\newcommand{\lecture}[4]{
   \pagestyle{myheadings}
   \thispagestyle{plain}
   \newpage
   \setcounter{lecnum}{#1}
   \setcounter{page}{1}
   
   \graphicspath{ {images/} }
   
%Info Box 
   \begin{center}
   \framebox{
      \vbox{\vspace{2mm}
    \hbox to 6.28in { {\bf CS 458/658 - Computer Security and Privacy
	\hfill Fall 2018} }
       \vspace{4mm}
       \hbox to 6.28in { {\Large \hfill Lecture #1: #2  \hfill} }
       \vspace{2mm}
       \hbox to 6.28in { {\it Lecturer: #3 \hfill Notes By: #4} }
      \vspace{2mm}}
   }
   \end{center}
   
   \markboth{Lecture #1: #2}{Lecture #1: #2}



 
}

\renewcommand{\cite}[1]{[#1]}
\def\beginrefs{\begin{list}%
        {[\arabic{equation}]}{\usecounter{equation}
         \setlength{\leftmargin}{2.0truecm}\setlength{\labelsep}{0.4truecm}%
         \setlength{\labelwidth}{1.6truecm}}}
\def\endrefs{\end{list}}
\def\bibentry#1{\item[\hbox{[#1]}]}

\newcommand{\fig}[3]{
			\vspace{#2}
			\begin{center}
			Figure \thelecnum.#1:~#3
			\end{center}
	}

\newtheorem{theorem}{Theorem}[lecnum]
\newtheorem{lemma}[theorem]{Lemma}
\newtheorem{ex}[theorem]{Example}
\newtheorem{proposition}[theorem]{Proposition}
\newtheorem{claim}[theorem]{Claim}
\newtheorem{corollary}[theorem]{Corollary}
\newtheorem{definition}[theorem]{Definition}
\newenvironment{proof}{{\bf Proof:}}{\hfill\rule{2mm}{2mm}}
\newcommand\E{\mathbb{E}}


%Start of Document 
\begin{document}

\lecture{3}{September 13, 2018}{Ian Goldberg }{Harsh Mistry}

\section{Program Security Continued}

\subsection{Intentional Malicious Flaws}

\subsubsection{Logic Bombs}
\begin{itemize}
\item A logic bomb is malicious code hiding in the software already on your computer, waiting for a certain trigger to "go off"
\item Often placed by an insider and the trigger is often something the insider has influence over after leaving. 
\end{itemize}

\subsubsection{Spotting Trojan horses and logic bombs}
\begin{itemize}
\item Spotting trojan horses and logic bombs is difficult because the end-user is intentionally running code. 
\item To avoid trojans you could not run untrusted code, but thats not always a guarantee. 
\item A better solution for avoiding malicious code is to prevent software from doing harmful things 
\end{itemize}


\subsection{Other Malicious Flaws}
\subsubsection{Web Bugs}
\begin{itemize}
\item A web bug is a an object (usually a 1x1 pixel) embedded in a page which is fetched from a different server from the on that served the web page itself.
\item Information about you can be sent to a third-party server which could be used in various ways such as tracking for advertisements
\item Web bugs are considered malicious code because web bugs are an privacy more than security, as it violates the concept of \textit{informational self-determination}
\end{itemize}
\subsubsection{Back Doors}
\begin{itemize}
\item A back door is a set of instruction designed to bypass the normal authentication mechanism and allow access to the system 
\end{itemize}

\subsubsection{Salami Attacks}
\begin{itemize}
\item A salami attack is an attack that is made up of many smaller, often considered inconsequential
\item A classic example would be, stealing fraction of cents of round-off from many accounts.
\end{itemize}

\subsubsection{Privilege Escalation}

A privilege escalation is an attack which raises the
privilege level of the attacker (beyond that to
which he would ordinarily be entitled)

\subsubsection{Rootkits}
\begin{itemize}
\item A rootkit is a tool often used by \textit{script kiddies}
\item It has two main parts
\begin{itemize}
\item A method for gaining unauthorized root privileges.
\item A way to hide its own existence, so the root kit can't be detecting 
\end{itemize}
\end{itemize}

\subsection{Non Malicious Flaws}
\subsection{Covert Channels}
\begin{itemize}
\item An attacker creates a capability to transfer
sensitive/unauthorized information through a
channel that is not supposed to transmit that
information
\item Basically, malware sneaks information out through valid communication
\end{itemize}

\end{document}





