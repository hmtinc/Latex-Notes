%Notes by Harsh Mistry 
%CS 341
%Based on Template From  https://www.cs.cmu.edu/~ggordon/10725-F12/template.tex

\documentclass[twoside]{article}
\setlength{\oddsidemargin}{0.25 in}
\setlength{\evensidemargin}{-0.25 in}
\setlength{\topmargin}{-0.6 in}
\setlength{\textwidth}{6.5 in}
\setlength{\textheight}{8.5 in}
\setlength{\headsep}{0.75 in}
\setlength{\parindent}{0 in}
\setlength{\parskip}{0.1 in}
\usepackage{amsmath,amsfonts,graphicx}
\newcounter{lecnum}
\renewcommand{\thepage}{\thelecnum-\arabic{page}}
\renewcommand{\thesection}{\thelecnum.\arabic{section}}
\renewcommand{\theequation}{\thelecnum.\arabic{equation}}
\renewcommand{\thefigure}{\thelecnum.\arabic{figure}}
\renewcommand{\thetable}{\thelecnum.\arabic{table}}
\newcommand{\lecture}[4]{
   \pagestyle{myheadings}
   \thispagestyle{plain}
   \newpage
   \setcounter{lecnum}{#1}
   \setcounter{page}{1}
   \graphicspath{ {images/} }
   
   
%Info Box 
   \begin{center}
   \framebox{
      \vbox{\vspace{2mm}
    \hbox to 6.28in { {\bf CS 341 -  Algorithms
	\hfill Winter 2018} }
       \vspace{4mm}
       \hbox to 6.28in { {\Large \hfill Lecture #1: #2  \hfill} }
       \vspace{2mm}
       \hbox to 6.28in { {\it Lecturer: #3 \hfill Notes By: #4} }
      \vspace{2mm}}
   }
   \end{center}
   
   \markboth{Lecture #1: #2}{Lecture #1: #2}



 
}

\renewcommand{\cite}[1]{[#1]}
\def\beginrefs{\begin{list}%
        {[\arabic{equation}]}{\usecounter{equation}
         \setlength{\leftmargin}{2.0truecm}\setlength{\labelsep}{0.4truecm}%
         \setlength{\labelwidth}{1.6truecm}}}
\def\endrefs{\end{list}}
\def\bibentry#1{\item[\hbox{[#1]}]}

\newcommand{\fig}[3]{
			\vspace{#2}
			\begin{center}
			Figure \thelecnum.#1:~#3
			\end{center}
	}

\newtheorem{theorem}{Theorem}[lecnum]
\newtheorem{lemma}[theorem]{Lemma}
\newtheorem{ex}[theorem]{Example}
\newtheorem{proposition}[theorem]{Proposition}
\newtheorem{claim}[theorem]{Claim}
\newtheorem{corollary}[theorem]{Corollary}
\newtheorem{definition}[theorem]{Definition}
\newenvironment{proof}{{\bf Proof:}}{\hfill\rule{2mm}{2mm}}
\newcommand\E{\mathbb{E}}


%Start of Document 
\begin{document}

\lecture{4}{January 16, 2018}{Bin Ma}{Harsh Mistry}
 \section{Recursion Continued}

\subsection{Master Theorem}
\[T(n) = \begin{cases}\Theta(n^c), \text{if } c  > \log_b a \\
\Theta(n^c \cdot \log n), \text{ if } c = \log_b a \\
\Theta(n^{\log_b a}) , \text{if } c < \log_b a 
\end{cases}\]

\begin{proof}
Proof for case 1. 
\begin{itemize}
\item[\textbf{Base Case : }] Base case is when n is a small constant  where the theorem is obviously true 
\item[\textbf{I.H. : }] \(T(n) \leq \gamma \cdot n^c\)
\item[\textbf{I.S : }]
\[\begin{aligned}
T(n) & = a \cdot T (\frac{n}{b}) + n^c \\
& \leq a \cdot \gamma \frac{n}{b}^x + n^c \\
& = (\gamma \cdot a \cdot b^{-c} + 1) \cdot n^c
\end{aligned}\]
Let \(\gamma = \frac{1}{1 - \frac{n}{b^c}}\), it is esy to check that \[ (\gamma \cdot a \cdot b^{-c} + 1) \cdot n^c \leq \gamma n^c\]
\end{itemize}
\end{proof}

\subsubsection{Modify Induction conclusion}
\(T(n) \leq 2 T(\frac{n}{2}) + \sqrt{n} \). We guess \(T(n) = O(n)\). In induction, we have \(T(n) \leq 2c \cdot \frac{n}{2} + \sqrt{n} \leq c \cdot n + \sqrt{n}\)
. This does not suffice to prove \(T(n) \leq c \cdot n\).

This can be solved by proving a slightly modified property. We want to introduce some \(- \sqrt{n}\) in \(T(\frac{n}{2})\) in order to cancel out the \(\sqrt{n}\). Specifically, we will prove \(T(n) \leq c \cdot n - 3 \sqrt{n}\) for a sightly larger c. 

This does not change our conclusion. However, during induction, we have 
\[T(n) \leq 2T(\frac{n}{2}) + \sqrt{n} \leq 2\left(c \cdot \frac{n}{2} - 3 \sqrt{\frac{n}{2}} \right) + \sqrt{n} \leq cn - \left(\frac{6}{\sqrt{2}} - 1\right) \sqrt{n} \leq c \cdot n -  3\sqrt{n}\]

\subsubsection{Variable Substitution}
To solve \(T(n) = 2T(\sqrt{n}) + \log_2 n\). 

Let \(S(m) = T(2^{m/2}) + m = 2S(\frac{m}{2} + m)\), then \(S(m) = m \log m \). Therefore, \(T(n) = S(\log_2 n) = \log_2 n \cdot log_2 n \cdot \log_2 \log_2 n)\)2. 

\section{Divide and Conquer}
The divide and conquer portion of class was all example. Refer to course slides or notes for additional examples

\end{document}





