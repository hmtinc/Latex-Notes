%Notes by Harsh Mistry 
%CS 341
%Based on Template From  https://www.cs.cmu.edu/~ggordon/10725-F12/template.tex

\documentclass[twoside]{article}
\setlength{\oddsidemargin}{0.25 in}
\setlength{\evensidemargin}{-0.25 in}
\setlength{\topmargin}{-0.6 in}
\setlength{\textwidth}{6.5 in}
\setlength{\textheight}{8.5 in}
\setlength{\headsep}{0.75 in}
\setlength{\parindent}{0 in}
\setlength{\parskip}{0.1 in}
\usepackage{amsmath,amsfonts,graphicx}
\newcounter{lecnum}
\renewcommand{\thepage}{\thelecnum-\arabic{page}}
\renewcommand{\thesection}{\thelecnum.\arabic{section}}
\renewcommand{\theequation}{\thelecnum.\arabic{equation}}
\renewcommand{\thefigure}{\thelecnum.\arabic{figure}}
\renewcommand{\thetable}{\thelecnum.\arabic{table}}
\newcommand{\lecture}[4]{
   \pagestyle{myheadings}
   \thispagestyle{plain}
   \newpage
   \setcounter{lecnum}{#1}
   \setcounter{page}{1}
   
   
%Info Box 
   \begin{center}
   \framebox{
      \vbox{\vspace{2mm}
    \hbox to 6.28in { {\bf CS 341 -  Algorithms
	\hfill Winter 2018} }
       \vspace{4mm}
       \hbox to 6.28in { {\Large \hfill Lecture #1: #2  \hfill} }
       \vspace{2mm}
       \hbox to 6.28in { {\it Lecturer: #3 \hfill Notes By: #4} }
      \vspace{2mm}}
   }
   \end{center}
   
   \markboth{Lecture #1: #2}{Lecture #1: #2}



 
}

\renewcommand{\cite}[1]{[#1]}
\def\beginrefs{\begin{list}%
        {[\arabic{equation}]}{\usecounter{equation}
         \setlength{\leftmargin}{2.0truecm}\setlength{\labelsep}{0.4truecm}%
         \setlength{\labelwidth}{1.6truecm}}}
\def\endrefs{\end{list}}
\def\bibentry#1{\item[\hbox{[#1]}]}

\newcommand{\fig}[3]{
			\vspace{#2}
			\begin{center}
			Figure \thelecnum.#1:~#3
			\end{center}
	}

\newtheorem{theorem}{Theorem}[lecnum]
\newtheorem{lemma}[theorem]{Lemma}
\newtheorem{ex}[theorem]{Example}
\newtheorem{proposition}[theorem]{Proposition}
\newtheorem{claim}[theorem]{Claim}
\newtheorem{corollary}[theorem]{Corollary}
\newtheorem{definition}[theorem]{Definition}
\newenvironment{proof}{{\bf Proof:}}{\hfill\rule{2mm}{2mm}}
\newcommand\E{\mathbb{E}}


%Start of Document 
\begin{document}

\lecture{1}{January 4, 2018}{Bin Ma}{Harsh Mistry}

\section{Admin Info}
\begin{center}
Bin Ma\\
TA : Hong Zhou (h76zhou@uwaterloo.ca)
\end{center}

\subsection{Mark Breakdown}
\begin{itemize}
\item Assignments (30 \%)
\item Midterm - February 27th (25 \%)
\item Final (45 \%)
\end{itemize}

\section{Introduction}

The goal of the course is to lean to design an algorithm. More specifically, you will learn 
\begin{itemize}
\item Well-known algorithms
\item Skills to analyse complexities
\item Skills to adapt existing solutions to new problems
\item Skills to design new algorithms
\end{itemize}

 \textbf{What is a problem?} 

\begin{itemize}
\item A problem defines the format of input desired output
\item For the purpose of this course, input size is not bounded
\item A problem does not specify an algorithm
\end{itemize}

\textbf{What is an algorithm?}

\begin{itemize}
\item A defined an finite procedure that solves a problem : taking any input of the problem and produces the desired output.
\end{itemize}

\textbf{How to evaluate an algorithm?}
\begin{itemize}
\item Time and Space complexity
\item Easiness of implementation
\end{itemize}

\subsection{Course contents}
\begin{itemize}
\item Algorithm analysis : correctness and time complexity
\item Algorithm design techniques : 
\begin{itemize}
\item Reduction
\item Recursion
\item Divide-and-Conquer
\item Greedy
\item Dynamic programming
\item Exhaustive search 
\item local search (not studied in this course)
\item Linear programming (not studied in this course)
\end{itemize}
\item Intractability : Not every problem has an efficient algorithm
\item Undecidabiliy : Not every problem has an algorithm
\end{itemize}

\end{document}





