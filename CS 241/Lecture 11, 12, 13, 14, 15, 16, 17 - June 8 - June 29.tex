%Notes by Harsh Mistry 
%CS 241
%based on Template from : https://www.cs.cmu.edu/~ggordon/10725-F12/template.tex

\documentclass{article}
\setlength{\oddsidemargin}{0.25 in}
\setlength{\evensidemargin}{-0.25 in}
\setlength{\topmargin}{-0.6 in}
\setlength{\textwidth}{6.5 in}
\setlength{\textheight}{8.5 in}
\setlength{\headsep}{0.75 in}
\setlength{\parindent}{0 in}
\setlength{\parskip}{0.1 in}
\usepackage{amsfonts,graphicx, amssymb}
\usepackage[fleqn]{amsmath}
\usepackage{fixltx2e}
\usepackage{color}
\usepackage{tcolorbox}
\usepackage{lipsum}
\usepackage{listings}
\usepackage{scrextend}
\tcbuselibrary{skins,breakable}
\usetikzlibrary{shadings,shadows}
\newcounter{lecnum}
\renewcommand{\thepage}{\thelecnum-\arabic{page}}
\renewcommand{\thesection}{\thelecnum.\arabic{section}}
\renewcommand{\theequation}{\thelecnum.\arabic{equation}}
\renewcommand{\thefigure}{\thelecnum.\arabic{figure}}
\renewcommand{\thetable}{\thelecnum.\arabic{table}}
\newcommand{\lecture}[4]{
   \pagestyle{myheadings}
   \thispagestyle{plain}
   \newpage
   \setcounter{lecnum}{#1}
   \setcounter{page}{1}
   
   
%Info Box 
   \begin{center}
   \framebox{
      \vbox{\vspace{2mm}
    \hbox to 6.28in { {\bf CS 241 - Foundations of Sequential Programs
	\hfill Spring 2017} }
       \vspace{4mm}
       \hbox to 6.28in { {\Large \hfill Lecture #1: #2  \hfill} }
       \vspace{2mm}
       \hbox to 6.28in { {\it Lecturer: #3 \hfill Notes By: #4} }
      \vspace{2mm}}
   }
   \end{center}
   
   \markboth{Lecture #1: #2}{Lecture #1: #2}



 
}

\renewcommand{\cite}[1]{[#1]}
\def\beginrefs{\begin{list}%
        {[\arabic{equation}]}{\usecounter{equation}
         \setlength{\leftmargin}{2.0truecm}\setlength{\labelsep}{0.4truecm}%
         \setlength{\labelwidth}{1.6truecm}}}
\def\endrefs{\end{list}}
\def\bibentry#1{\item[\hbox{[#1]}]}

\newcommand{\fig}[3]{
			\vspace{#2}
			\begin{center}
			Figure \thelecnum.#1:~#3
			\end{center}
	}
	
	\newcommand{\bs}{
		\texttt{\char`\\ \hspace{0.1cm}} 
	}
	

	
\newcommand{\pipe}{\(\mid\)}
\newcommand{\ctr}{\(\wedge\)}

\newtheorem{theorem}{Theorem}[lecnum]
\newtheorem{lemma}[theorem]{Lemma}
\newtheorem{ex}[theorem]{Example}
\newtheorem{expx}[theorem]{Problem}
\newtheorem{prop}[theorem]{Proposition}
\newtheorem{claim}[theorem]{Claim}
\newtheorem{corollary}[theorem]{Corollary}
\newtheorem{definition}[theorem]{Definition}
\newenvironment{proof}{{\bf Proof:}}{\hfill\rule{2mm}{2mm}}
\newcommand\E{\mathbb{E}}

%color definitions :
\definecolor{darkred}{rgb}{0.55, 0.0, 0.0}
\definecolor{lightcoral}{rgb}{0.94, 0.5, 0.5}
\definecolor{tomato}{rgb}{1.0, 0.39, 0.28}
\definecolor{lightgray}{rgb}{.9,.9,.9}
\definecolor{darkgray}{rgb}{.4,.4,.4}
\definecolor{purple}{rgb}{0.65, 0.12, 0.82}
\definecolor{lightgreen}{rgb}{0.56, 0.93, 0.56}
\definecolor{darkgreen}{rgb}{0.0, 0.2, 0.13}
\definecolor{limegreen}{rgb}{0.2, 0.8, 0.2}
\definecolor{lightblue}{rgb}{0.68, 0.85, 0.9}
\definecolor{darkblue}{rgb}{0.0, 0.0, 0.55}


%Environments
\newenvironment{exblock}[1]{%
    \tcolorbox[beamer,%
    noparskip,breakable,
    colback=lightgreen,colframe=darkgreen,%
    colbacklower=limegreen!75!lightgreen,%
    title=#1]}%
    {\endtcolorbox}

\newenvironment{ablock}[1]{%
    \tcolorbox[beamer,%
    noparskip,breakable,
    colback=lightcoral,colframe=darkred,%
    colbacklower=tomato!75!lightcoral,%
    title=#1]}%
    {\endtcolorbox}

\newenvironment{cblock}[1]{%
    \tcolorbox[beamer,%
    noparskip,breakable,
    colback=lightblue,colframe=darkblue,%
    colbacklower=darkblue!75!lightblue,%
    title=#1]}%
    {\endtcolorbox}


%Languages
\lstdefinelanguage{JavaScript}{
  keywords={typeof, new, true, false, catch, function, return, null, catch, switch, var, if, in, while, do, else, case, break},
  keywordstyle=\color{blue}\bfseries,
  ndkeywords={class, export, boolean, throw, implements, import, this},
  ndkeywordstyle=\color{darkgray}\bfseries,
  identifierstyle=\color{black},
  sensitive=false,
  comment=[l]{//},
  morecomment=[s]{/*}{*/},
  commentstyle=\color{purple}\ttfamily,
  stringstyle=\color{red}\ttfamily,
  morestring=[b]',
  morestring=[b]"
}

%Listings
\lstset{
   language=JavaScript,
   backgroundcolor=\color{lightgray},
   extendedchars=true,
   basicstyle=\footnotesize\ttfamily,
   showstringspaces=false,
   showspaces=false,
   numbers=left,
   numberstyle=\footnotesize,
   numbersep=9pt,
   tabsize=2,
   breaklines=true,
   showtabs=false,
   captionpos=b
}


%Start
\begin{document}

\lecture{11, 12, 13, 14, 15, 16, 17}{June 8th - June 29th, 2017}{Kevin Lanctot}{Harsh Mistry}

\section{Context-Free Grammars}

\begin{ex} Grammar for strings starting with a's and ending with b's 
\begin{itemize}
\item \(S \rightarrow aSb\)
\item \(S \rightarrow \epsilon\)
\end{itemize}
\end{ex}

\begin{definition}
\begin{itemize}
\item \textbf{Language of a CFG} : The set of all valid strings (Sequences of terminals) that can be derived from the start symbol (S)
\item \textbf{G} is a context-free grammar 
\item \textbf{L(G)} is the language specified by G
\item a \textbf{derivation} : starting with the start symbol, applying a
sequence of rules until there are no more non-terminals
\item \textbf{Production Rules} capture Union, Concatenation, Recursion
\end{itemize}
\end{definition}

Context-free grammars consist of a four-tuple \verb|{N,T,P,S}|
\begin{itemize}
\item N is a finite set of non-terminals
\item T is a finite set of terminals
\item P is a finite set of production rules in the form A \(\rightarrow \beta\) where
\item S is the start symbol,
\end{itemize}

\subsection*{Derivations}
\begin{itemize}
\item Leftmost Derivation - Always expand the leftmost non-terminal first 
\item Rightmost Derivation - Always expand the rightmost non-terminal first. 
\end{itemize}

\subsection{Parse Trees}
Parse Trees (Also called derivations trees) visualize the entire derivation at once. 
The root of the tree is the start symbol, the children are given by subsequent rules, and the leafs are the terminals\\

Parse trees are processed \textbf{Post order} with a depth first traversal. 

\begin{itemize}
\item Depth first - visit your first child and all its descendants before visiting your second child. 
\item Post order - process all your children before you do any processing yourself
\end{itemize}

\section{Top-Down Parsing}
Parsing giving a grammar and a word. involves fining the derivation of w. The two strategies for parsing are : 
\begin{itemize}
\item Top-down : Find a non-terminal (e.g S) and replace it with the right-hand side 
\item Bottom-up : Replace a right-hand side with a non-terminal 
\end{itemize} 

\subsection*{Stack-Based Parsing}
For top-down parsing, we use a stack to remember information about derivations or processed input

\begin{definition}
The start symbol occurs as the LHS of exactly one rule
and that rule must begin and end with a terminal
\end{definition}

\textbf{Parsing Algorithms}
\begin{enumerate}
\item To start, push the start symbol, \(S^\prime\), on the stack
\item When a non-terminal is at the top of the stack : 
\begin{itemize}
\item expand the non-terminal using a production rule where the
RHS of the rule matches the input (e.g. if the rule is \(S^\prime \rightarrow  \vdash S \dashv \)
then pop S’ off the stack and push \(\vdash S \dashv\) onto the stack)
\end{itemize}
\item When it is a terminal at the top of the stack : match with input 
\begin{itemize}
\item Pop terminal 
\item read the next character from the input 
\end{itemize}
\end{enumerate}

\subsection*{LL(1) Parsing}
LL(1) means processing the input from Left to right while finding the Leftmost derivation and looking ahead by 1 token. 

To achieve this we implement a Predict table and consider three functions. \
\begin{itemize}
\item First()\\
Determine for each non-terminal what are the possible terminal they can derive on the leftmost side of the string
\item Follow()\\
For all rules of the form \(A \rightarrow \beta\) whenever \(Empty(\beta)\)
\item Empty() or Nullable()
\end{itemize} 

\subsection*{Non-LL(1) Grammars}
A Non-LL Grammar Is a grammar that is ambiguous and requires us to look ahead to the second symbol in order to tell which rule to use. 

\section{Bottom-up Parsing}
\begin{itemize}
\item the derivation progresses from the bottom of the parse tree
up to the top (i.e. S’), i.e. a bottom-up derivation
\item current step in derivation: stack + input to be read
-\item stack is read from bottom to top
\end{itemize}
\subsection*{LR Parsing}
LR parsing is when input is read normally but is analysed from the right. 

There are two operations in LR parsing 
\begin{itemize}
\item Shift 
\begin{itemize}
\item move a character from the input file to the stack
\item we’ll also include it in the “Read” column to keep track of
what has been read so far.
\end{itemize}
\item Reduce 
\begin{itemize}
\item If there is a production rule of the form S \(\rightarrow\) AyB and AyB is on
the stack then reduce (i.e. replace) AyB to S
\item this step is the act of applying a production rule to simplify
what is on the stack
\end{itemize}
\end{itemize}

\subsection*{Building an LR(0) Automation}
\begin{itemize}
\item Start state: make teh start state the first rule with a dot \((\cdot)\) in front of the left most symbol of the RHS
\item For each state create a transition out of teh state with the symbol "\(\cdot\)" 
\item For non-terminals : If "\(\cdot\)"  precedes a non-terminal, add all productions with that non-terminal on the LHS to the current state (and place the "\(\cdot\)"  in the leftmost position of those rules)
\end{itemize}

\subsection*{Using an LR(0) Automation}
\begin{lstlisting}
for each input token
	read the stack (from the bottom up) + the current input
	do the action indicated for the current input
		if there’s a transition out of current state with current input
			then shift (push) that input onto the stack
		if current state has only one item and the rightmost symbol is “.”
			then we know we can reduce i.e. {
				pop the RHS of the stack,
				reread the stack (from the bottom-up),
				follow the transition for the LHS,
				push the LHS onto the stack
			}
if S is on the stack when all input is read
	then ACCEPT
\end{lstlisting}

\subsection*{LR Parsing Limitations}
The time complexity is \(O(n^2)\), as for each of the n input chars, we move through the stack and automaton up to n times. 

To resolve this and to achieve \(O(n)\) complexity we store the automation state in the States stack. So if you pop i elements off the Symbol Stack and then pop i elements off the States Stack. It tells you what state to go to next 

\subsection{SLR(1) Parser}
\begin{itemize}
\item When we add one character of lookahead, we have an SLR(1)
(Simple LR with 1 character lookahead) parser
\item We modify our existing LR(0) automaton as follows
\item When you are in a state that has a rule of the form
A \(\rightarrow \alpha \cdot  \{X\}\) (which calls for a reduction) if the next
symbol is X reduce using that rule, otherwise shift.
\item Allowing for lookahead and improving time complexity by using
a state stack, we now have the following algorithm, where
\item the top of the state stack, \verb|state_stack.top|, is the current state
\item \(\delta(s, t)\) is the new state the automaton goes to when it is in
state s and it processes the next token, t
\end{itemize}

\subsection*{SLR(1) Parsing Algorithm}
\begin{lstlisting}
push start_symbol onto the symbol_stack
push state(q0, start) onto the state_stack
	for each token a in the input {
		while (there is a reduction A to g * {a} in state_stack.top ) {
		pop |g| symbols off the symbol_stack // reduce
		pop |g| symbols off the state_stack
		push A on the symbol_stack
		push stat(state_stack.top, A) onto the state_stack
}
	push a onto the symbol_stack // shift
	if (state(state_stack.top, a) == undefined) report parse error
	else push state(state_stack.top, a) onto the state_stack
}
if (end has been shifted)
then ACCEPT
\end{lstlisting}

\section{Context-Sensitive Analysis}
Context-Sensitive Analysis involves analysing code for errors that go beyond Syntax issues such as Undeclared Variables, Duplicate Procedures, invalid types, and out-of-scope variables.  

\subsection*{Solving Variable Declaration Issues}
To resolve Variable declaration issues we can use a symbol table like a MIPS assembler, but keep track of types of variables as well.
\begin{itemize}
\item When using a variable, make sure its is in the symbol table 
\item When declaring a variable
\begin{itemize}
\item Check that is not already in the symbol table 
\item If it is not, then add it
\item If it is, throw an error  
\end{itemize}
\end{itemize}

For Variable scopes, implement a global symbol table for procedure names and types and have pointers to separate symbol tables for each procedure to track its parameters and local variables. 

\subsection*{Type Checking}
To type check decorate the parse tree with types and ensure all the rules are followed. 
\begin{itemize}
\item \(\frac{}{\text{NUM : int}}\)
\item \(\frac{}{\text{NULL : int*}}\)
\item \(\frac{E : \tau}{(E) : \tau}\)
\item \(\frac{\text{E: int}}{\text{\& E : int*}}\)
\item \(\frac{\text{E: int*}}{\text{\* E : int*}}\)
\item \(\frac{\text{E: int}}{\text{new int[E] : int*}}\)
\item \(\frac{E_1 : \text{int}, E_2 : \text{int}}{E_1 * E_2 : \text{int}}\)
\item \(\frac{E_1 : \text{int}, E_2 : \text{int}}{E_1 / E_2 : \text{int}}\)
\item \(\frac{E_1 : \text{int}, E_2 : \text{int}}{E_1 \% E_2 : \text{int}}\)
\item \(\frac{E_1 : \text{int}, E_2 : \text{int}}{E_1 + E_2 : \text{int}}\)
\item \(\frac{E_1 : \text{int*}, E_2 : \text{int}}{E_1 + E_2 : \text{int*}}\)
\item \(\frac{E_1 : \text{int}, E_2 : \text{int*}}{E_1 + E_2 : \text{int*}}\)

\item \(\frac{E_1 : \text{int}, E_2 : \text{int}}{E_1 - E_2 : \text{int}}\)
\item \(\frac{E_1 : \text{int*}, E_2 : \text{int}}{E_1 - E_2 : \text{int*}}\)
\item \(\frac{E_1 : \text{int*}, E_2 : \text{int*}}{E_1 - E_2 : \text{int}}\)

\item \(\frac{E_1 : \tau, E_2 : \tau}{well-typed(E_1 ==  E_2)}\)
\item \(\frac{E_1 : \tau, E_2 : \tau}{well-typed(E_1 <  E_2)}\)
\item \(\frac{E_1 : \tau, E_2 : \tau}{well-typed(E_1 =  E_2)}\)
\item \(\frac{E_1 : \tau}{well-typed(delete [] E)}\)

\item \(\frac{well-typed(T) , well-typed(S_1)}{well-typed(while(T) \{S_1\})}\)
\item \(\frac{well-typed(T) , well-typed(S_1), well-typed(S_2)}{well-typed(if (T) \{S_1\} else \{s_2\})}\)
\end{itemize}
\end{document}
