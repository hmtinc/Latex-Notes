%Notes by Harsh Mistry 
%Econ 301
%Based on Template From  https://www.cs.cmu.edu/~ggordon/10725-F12/template.tex

\documentclass[twoside]{article}
\setlength{\oddsidemargin}{0.25 in}
\setlength{\evensidemargin}{-0.25 in}
\setlength{\topmargin}{-0.6 in}
\setlength{\textwidth}{6.5 in}
\setlength{\textheight}{8.5 in}
\setlength{\headsep}{0.75 in}
\setlength{\parindent}{0 in}
\setlength{\parskip}{0.1 in}
\usepackage{amsmath,amsfonts,graphicx}
\newcounter{lecnum}
\renewcommand{\thepage}{\thelecnum-\arabic{page}}
\renewcommand{\thesection}{\thelecnum.\arabic{section}}
\renewcommand{\theequation}{\thelecnum.\arabic{equation}}
\renewcommand{\thefigure}{\thelecnum.\arabic{figure}}
\renewcommand{\thetable}{\thelecnum.\arabic{table}}
\newcommand{\lecture}[4]{
   \pagestyle{myheadings}
   \thispagestyle{plain}
   \newpage
   \setcounter{lecnum}{#1}
   \setcounter{page}{1}
   
   
%Info Box 
   \begin{center}
   \framebox{
      \vbox{\vspace{2mm}
    \hbox to 6.28in { {\bf Econ 301 - Microeconomic Theory 2
	\hfill Winter 2016} }
       \vspace{4mm}
       \hbox to 6.28in { {\Large \hfill Lecture #1: #2  \hfill} }
       \vspace{2mm}
       \hbox to 6.28in { {\it Lecturer: #3 \hfill Notes By: #4} }
      \vspace{2mm}}
   }
   \end{center}
   
   \markboth{Lecture #1: #2}{Lecture #1: #2}



 
}

\renewcommand{\cite}[1]{[#1]}
\def\beginrefs{\begin{list}%
        {[\arabic{equation}]}{\usecounter{equation}
         \setlength{\leftmargin}{2.0truecm}\setlength{\labelsep}{0.4truecm}%
         \setlength{\labelwidth}{1.6truecm}}}
\def\endrefs{\end{list}}
\def\bibentry#1{\item[\hbox{[#1]}]}

\newcommand{\fig}[3]{
			\vspace{#2}
			\begin{center}
			Figure \thelecnum.#1:~#3
			\end{center}
	}

\newtheorem{theorem}{Theorem}[lecnum]
\newtheorem{lemma}[theorem]{Lemma}
\newtheorem{ex}[theorem]{Example}
\newtheorem{proposition}[theorem]{Proposition}
\newtheorem{claim}[theorem]{Claim}
\newtheorem{corollary}[theorem]{Corollary}
\newtheorem{definition}[theorem]{Definition}
\newenvironment{proof}{{\bf Proof:}}{\hfill\rule{2mm}{2mm}}
\newcommand\E{\mathbb{E}}


%Start of Document 
\begin{document}

\lecture{1}{January 3, 2018}{Jean Guillaume Forand}{Harsh Mistry}

\section{Admin Info}
Jean Guillaume Forand – HH 131 \\
Office Hours : Tuesday 3:30 – 5:00\\
Midterm 1 (Jan 31st)  - 20 \% \\
Midterm 2 - 20 \% \\
Final - 50\% \\
Assignments - 10 \%  \\

\subsection{Topics}
\begin{enumerate}
\item Consumer Choice
\item  General Equilibrium
\item Welfare 
\item Market Failure
\begin{itemize}
\item Externality
\item Public goods
\end{itemize}
\end{enumerate}

\section{Consumer Choice}
\begin{itemize}
\item Econ 201 with calculus
\item basic model of economic theory 
\end{itemize}

Economics is the study pf decision-making in social environments

\begin{itemize}
	\item Classics : micro, macro
	\item Also : Family economics, economics of crime, etc
	\item Economists construct models : abstract representation   
	\begin{itemize}
		\item Build stylized version of reality (I.e market for cars) 
		\item Deduct logically necessary features of the model (I.e what determines the price of cars) 
		\item Test whether a models predictions are consistent with empirical evidence. 
		\end{itemize}
\item Economic models are mathematical 
\end{itemize}

\subsection{A unifying model of decision-making}
\begin{enumerate}
	 \item A \underline{decision-maker}
      \item  Possible \underline{outcome} DM may face 
	\item \underline{Actions} available to DM, which affect outcomes.
\item \underline{Preferences} of DM over outcomes, which describe the motivations 
\end{enumerate}

Using Mathematics 
\begin{enumerate}
	 \item A \underline{decision-maker}
	 \item A set \(X\) of \underline{outcome}
	 \item a set of\underline{actions} on feasible choices \(B \subseteq X\)
	 \item A function \(u : X \rightarrow \mathbb{R} \), which assigns numbers \(u(x)\) to each \(x \in X\), interpreted as DM's utuility from \(x\)
\end{enumerate}

\subsubsection{Solution (Optimal Choice)}
Choices by DM that lead to outcomes that best with her preferences 
Using mathematics, the optimal choice is \(x^\triangle
 \in X\) which solves :  
\[ max u(x)_{x \in X} \text{ s.t  } x \in B \]

\end{document}





