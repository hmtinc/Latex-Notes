%Notes by Harsh Mistry 
%Stat231
%based on Template from : https://www.cs.cmu.edu/~ggordon/10725-F12/template.tex

\documentclass{article}
\setlength{\oddsidemargin}{0.25 in}
\setlength{\evensidemargin}{-0.25 in}
\setlength{\topmargin}{-0.6 in}
\setlength{\textwidth}{6.5 in}
\setlength{\textheight}{8.5 in}
\setlength{\headsep}{0.75 in}
\setlength{\parindent}{0 in}
\setlength{\parskip}{0.1 in}
\usepackage{amsfonts,graphicx, amssymb}
\usepackage[fleqn]{amsmath}
\usepackage{fixltx2e}
\usepackage{tikz}
\usepackage{color}
\usepackage{tcolorbox}
\usepackage{lipsum}
\usepackage{listings}
\usepackage{scrextend}
\tcbuselibrary{skins,breakable}
\usetikzlibrary{shadings,shadows}
\newcounter{lecnum}
\renewcommand{\thepage}{\thelecnum-\arabic{page}}
\renewcommand{\thesection}{\thelecnum.\arabic{section}}
\renewcommand{\theequation}{\thelecnum.\arabic{equation}}
\renewcommand{\thefigure}{\thelecnum.\arabic{figure}}
\renewcommand{\thetable}{\thelecnum.\arabic{table}}
\newcommand{\lecture}[4]{
   \pagestyle{myheadings}
   \thispagestyle{plain}
   \newpage
   \setcounter{lecnum}{#1}
   \setcounter{page}{1}
   
   
%Info Box 
   \begin{center}
   \framebox{
      \vbox{\vspace{2mm}
    \hbox to 6.28in { {\bf Stat 231 - Statistics 
	\hfill Spring 2017} }
       \vspace{4mm}
       \hbox to 6.28in { {\Large \hfill Lecture #1: #2  \hfill} }
       \vspace{2mm}
       \hbox to 6.28in { {\it Lecturer: #3 \hfill Notes By: #4} }
      \vspace{2mm}}
   }
   \end{center}
   
   \markboth{Lecture #1: #2}{Lecture #1: #2}



 
}

\renewcommand{\cite}[1]{[#1]}
\def\beginrefs{\begin{list}%
        {[\arabic{equation}]}{\usecounter{equation}
         \setlength{\leftmargin}{2.0truecm}\setlength{\labelsep}{0.4truecm}%
         \setlength{\labelwidth}{1.6truecm}}}
\def\endrefs{\end{list}}
\def\bibentry#1{\item[\hbox{[#1]}]}

\newcommand{\fig}[3]{
			\vspace{#2}
			\begin{center}
			Figure \thelecnum.#1:~#3
			\end{center}
	}
	
\newcommand{\pipe}{\(\mid\)}
\newcommand{\ctr}{\(\wedge\)}

\newtheorem{theorem}{Theorem}[lecnum]
\newtheorem{lemma}[theorem]{Lemma}
\newtheorem{ex}[theorem]{Example}
\newtheorem{proposition}[theorem]{Proposition}
\newtheorem{claim}[theorem]{Claim}
\newtheorem{corollary}[theorem]{Corollary}
\newtheorem{definition}[theorem]{Definition}
\newenvironment{proof}{{\bf Proof:}}{\hfill\rule{2mm}{2mm}}
\newcommand\E{\mathbb{E}}

%color definitions :
\definecolor{darkred}{rgb}{0.55, 0.0, 0.0}
\definecolor{lightcoral}{rgb}{0.94, 0.5, 0.5}
\definecolor{tomato}{rgb}{1.0, 0.39, 0.28}
\definecolor{lightgray}{rgb}{.9,.9,.9}
\definecolor{darkgray}{rgb}{.4,.4,.4}
\definecolor{purple}{rgb}{0.65, 0.12, 0.82}
\definecolor{lightgreen}{rgb}{0.56, 0.93, 0.56}
\definecolor{darkgreen}{rgb}{0.0, 0.2, 0.13}
\definecolor{limegreen}{rgb}{0.2, 0.8, 0.2}
\definecolor{lightblue}{rgb}{0.68, 0.85, 0.9}
\definecolor{darkblue}{rgb}{0.0, 0.0, 0.55}


%Environments
\newenvironment{exblock}[1]{%
    \tcolorbox[beamer,%
    noparskip,breakable,
    colback=lightgreen,colframe=darkgreen,%
    colbacklower=limegreen!75!lightgreen,%
    title=#1]}%
    {\endtcolorbox}

\newenvironment{ablock}[1]{%
    \tcolorbox[beamer,%
    noparskip,breakable,
    colback=lightcoral,colframe=darkred,%
    colbacklower=tomato!75!lightcoral,%
    title=#1]}%
    {\endtcolorbox}

\newenvironment{cblock}[1]{%
    \tcolorbox[beamer,%
    noparskip,breakable,
    colback=lightblue,colframe=darkblue,%
    colbacklower=darkblue!75!lightblue,%
    title=#1]}%
    {\endtcolorbox}


%Languages
\lstdefinelanguage{JavaScript}{
  keywords={typeof, new, true, false, catch, function, return, null, catch, switch, var, if, in, while, do, else, case, break},
  keywordstyle=\color{blue}\bfseries,
  ndkeywords={class, export, boolean, throw, implements, import, this},
  ndkeywordstyle=\color{darkgray}\bfseries,
  identifierstyle=\color{black},
  sensitive=false,
  comment=[l]{//},
  morecomment=[s]{/*}{*/},
  commentstyle=\color{purple}\ttfamily,
  stringstyle=\color{red}\ttfamily,
  morestring=[b]',
  morestring=[b]"
}

%Listings
\lstset{
   language=JavaScript,
   backgroundcolor=\color{lightgray},
   extendedchars=true,
   basicstyle=\footnotesize\ttfamily,
   showstringspaces=false,
   showspaces=false,
   numbers=left,
   numberstyle=\footnotesize,
   numbersep=9pt,
   tabsize=2,
   breaklines=true,
   showtabs=false,
   captionpos=b
}


%Start
\begin{document}

\lecture{25, 26, 27, 28, 29, 30}{June 26 - July 10, 2017}{Suryapratim Banerjee}{Harsh Mistry}

\section{Measurement Bias Testing}
\textbf{Objective : } To test whether a scale is biased


\textbf{Experiment :} We take an object of a known weight (10) and measure of using the scale n times. 
$$Y_i = \text{ith reading of the scale}$$
Let \(S\) = bias of our scale and \(R_i\) error on the ith measurement
$$ Y_i = \text{true weight} + S + R_i$$

\textbf{Conventions : }
\begin{itemize}
\item If d is to the right of the median of the \(x^2\) \\
\(p-value = 2P(D \geq d)\)
\item If d is to the left of the median 
\(p-value = 2P(D \leq d)\)
\end{itemize}

\section{Testing Using The Likelihood}
Suppose we are not able to find D fro some problem. 
Assume n is large. We can then use the likelihood ratio test statistic as our discrepancy measure. 
$$ \bigtriangleup (\theta) = -2log\frac{L(\theta)}{L(\tilde{\theta}}$$


$$ D = \mid \frac{\tilde{\theta} - \theta_o}{\sqrt{\frac{\theta  ( 1- \theta)}{n}}}\mid$$
\begin{enumerate}
\item Find \(L(\theta)\) and calculate \(\hat{\theta}\)
\item Calculate \(\bigtriangleup(\theta_o\)
\item Calculate the p-value where \( d = \bigtriangleup\)
\end{enumerate}

\section{Introduction To Regression Models}
A linear regression model assumes that E(y) is a linear function of x 
$$ E(Y) = \alpha + \beta X$$

\section{Simple Linear Regression Model}
\textbf{Assumptions}
\begin{itemize}
\item Given \(x_i\)'s, \(Y_i\)'s have a Normal distribution. 
\item The mean of the \(Y_i\)'s are a linear function of the \(x_i\)'s
\item The variance of the \(Y_I\)'s are constant \(\sigma^2\) (unknown) \\
\(\sigma^2\) is independent of x 
\item \( \sigma^2\) is not a function of x and \(Y_i\)'s are independent
\end{itemize}

\subsection{Model}
$$ Y_i \sim G(\alpha + \beta x_i , \sigma) $$
$$ \text{Alternate Form : } Y_i = \alpha + \beta x_i + R_i $$

\subsection{Gauss-Markov Assumptions}
\begin{itemize}
\item \(Y_i\)'s independent
\item \(Y_i\)'s are Gaussian 
\item \(E(Y_i = \alpha + \beta x_i\) 
\item \(Var(Y_i) = \sigma^2 \forall x\) 
\item \(E(Y_i)\) is not linear \(\longrightarrow\) Non-linear regression models
\item \(V(Y_i) \neq \sigma^2 \longrightarrow \) Hetroscedastic  Models
\item More than one explanatory variable \(\longrightarrow\) Multivariable Regression
\end{itemize}

\subsection{Questions }
\begin{itemize}
\item Interpretation of \(\alpha, \beta, \sigma\) ? \\
\(\beta\) is the increase in the mean of Y if x goes by 1 unit \\
\(\alpha = E(Y) \) when x = 0
\item MLE for \(\alpha, \beta, \sigma\)\\
\begin{itemize}
\item \(\hat{\alpha} = \bar{y} - \hat{\beta} \bar{x}\) where \(\hat{\beta} = \frac{S_{xy}}{S_{xx}} \)\\
\item \(\hat{\sigma}^2 = \frac{1}{n}[X_{yy} - \hat{\beta} S_{xy}] \)
\item \(S_{yy} = \sum (y_i - \bar{y})^2\)
\item \(S_{xx}  = \sum ( x_i - \bar{x})^2\)
\item \(S_{xy}  = \sum (x_i - \bar{x}) (y_i  - \bar{y}) \)
\end{itemize}
$$ S^2 = \frac{1}{n-2} [S_{yy} - \beta S_{xy} $$
\end{itemize}

Let \(a_i = \frac{1}{S_{xx}} \cdot (x_i - \bar{x}) \)
$$ \hat{p} = \sum a_i y_i $$

\subsection{Confidence interval for p}
\begin{itemize}
\item \(E(\tilde{p}) = \alpha \sigma + \beta = \beta \)
\item \(V(\beta) = \frac{\sigma^2}{S_{xx}}\)
\item \[ \frac{\tilde{\beta} - \beta}{\frac{S}{\sqrt{S_{xx}}}} \sim T_{n-2}\]
\item Coverage Interval 
\[\tilde{\beta} \pm t^* \frac{S}{\sqrt{S_{ss}}}\]
\item D value for hypothesis testing 
\[D = \mid \frac{\tilde{\beta} - \beta_0}{\frac{S}{\sqrt{S_{xx}}}} \mid \] 
\end{itemize}

\end{document}

