%Notes by Harsh Mistry 
%Stat231
%based on Template from : https://www.cs.cmu.edu/~ggordon/10725-F12/template.tex

\documentclass{article}
\setlength{\oddsidemargin}{0.25 in}
\setlength{\evensidemargin}{-0.25 in}
\setlength{\topmargin}{-0.6 in}
\setlength{\textwidth}{6.5 in}
\setlength{\textheight}{8.5 in}
\setlength{\headsep}{0.75 in}
\setlength{\parindent}{0 in}
\setlength{\parskip}{0.1 in}
\usepackage{amsfonts,graphicx, amssymb}
\usepackage[fleqn]{amsmath}
\usepackage{fixltx2e}
\usepackage{tikz}
\usepackage{color}
\usepackage{tcolorbox}
\usepackage{lipsum}
\usepackage{listings}
\usepackage{scrextend}
\tcbuselibrary{skins,breakable}
\usetikzlibrary{shadings,shadows}
\usepackage{accents}
\newcounter{lecnum}
\renewcommand{\thepage}{\thelecnum-\arabic{page}}
\renewcommand{\thesection}{\thelecnum.\arabic{section}}
\renewcommand{\theequation}{\thelecnum.\arabic{equation}}
\renewcommand{\thefigure}{\thelecnum.\arabic{figure}}
\renewcommand{\thetable}{\thelecnum.\arabic{table}}
\newcommand{\lecture}[4]{
   \pagestyle{myheadings}
   \thispagestyle{plain}
   \newpage
   \setcounter{lecnum}{#1}
   \setcounter{page}{1}
   
   
%Info Box 
   \begin{center}
   \framebox{
      \vbox{\vspace{2mm}
    \hbox to 6.28in { {\bf Stat 231 - Statistics 
	\hfill Spring 2017} }
       \vspace{4mm}
       \hbox to 6.28in { {\Large \hfill Lecture #1: #2  \hfill} }
       \vspace{2mm}
       \hbox to 6.28in { {\it Lecturer: #3 \hfill Notes By: #4} }
      \vspace{2mm}}
   }
   \end{center}
   
   \markboth{Lecture #1: #2}{Lecture #1: #2}



 
}

\renewcommand{\cite}[1]{[#1]}
\def\beginrefs{\begin{list}%
        {[\arabic{equation}]}{\usecounter{equation}
         \setlength{\leftmargin}{2.0truecm}\setlength{\labelsep}{0.4truecm}%
         \setlength{\labelwidth}{1.6truecm}}}
\def\endrefs{\end{list}}
\def\bibentry#1{\item[\hbox{[#1]}]}

\newcommand{\fig}[3]{
			\vspace{#2}
			\begin{center}
			Figure \thelecnum.#1:~#3
			\end{center}
	}
	
\newcommand{\pipe}{\(\mid\)}
\newcommand{\ctr}{\(\wedge\)}

\newtheorem{theorem}{Theorem}[lecnum]
\newtheorem{lemma}[theorem]{Lemma}
\newtheorem{ex}[theorem]{Example}
\newtheorem{proposition}[theorem]{Proposition}
\newtheorem{claim}[theorem]{Claim}
\newtheorem{corollary}[theorem]{Corollary}
\newtheorem{definition}[theorem]{Definition}
\newenvironment{proof}{{\bf Proof:}}{\hfill\rule{2mm}{2mm}}
\newcommand\E{\mathbb{E}}

%color definitions :
\definecolor{darkred}{rgb}{0.55, 0.0, 0.0}
\definecolor{lightcoral}{rgb}{0.94, 0.5, 0.5}
\definecolor{tomato}{rgb}{1.0, 0.39, 0.28}
\definecolor{lightgray}{rgb}{.9,.9,.9}
\definecolor{darkgray}{rgb}{.4,.4,.4}
\definecolor{purple}{rgb}{0.65, 0.12, 0.82}
\definecolor{lightgreen}{rgb}{0.56, 0.93, 0.56}
\definecolor{darkgreen}{rgb}{0.0, 0.2, 0.13}
\definecolor{limegreen}{rgb}{0.2, 0.8, 0.2}
\definecolor{lightblue}{rgb}{0.68, 0.85, 0.9}
\definecolor{darkblue}{rgb}{0.0, 0.0, 0.55}


%Environments
\newenvironment{exblock}[1]{%
    \tcolorbox[beamer,%
    noparskip,breakable,
    colback=lightgreen,colframe=darkgreen,%
    colbacklower=limegreen!75!lightgreen,%
    title=#1]}%
    {\endtcolorbox}

\newenvironment{ablock}[1]{%
    \tcolorbox[beamer,%
    noparskip,breakable,
    colback=lightcoral,colframe=darkred,%
    colbacklower=tomato!75!lightcoral,%
    title=#1]}%
    {\endtcolorbox}

\newenvironment{cblock}[1]{%
    \tcolorbox[beamer,%
    noparskip,breakable,
    colback=lightblue,colframe=darkblue,%
    colbacklower=darkblue!75!lightblue,%
    title=#1]}%
    {\endtcolorbox}


%Languages
\lstdefinelanguage{JavaScript}{
  keywords={typeof, new, true, false, catch, function, return, null, catch, switch, var, if, in, while, do, else, case, break},
  keywordstyle=\color{blue}\bfseries,
  ndkeywords={class, export, boolean, throw, implements, import, this},
  ndkeywordstyle=\color{darkgray}\bfseries,
  identifierstyle=\color{black},
  sensitive=false,
  comment=[l]{//},
  morecomment=[s]{/*}{*/},
  commentstyle=\color{purple}\ttfamily,
  stringstyle=\color{red}\ttfamily,
  morestring=[b]',
  morestring=[b]"
}

%Listings
\lstset{
   language=JavaScript,
   backgroundcolor=\color{lightgray},
   extendedchars=true,
   basicstyle=\footnotesize\ttfamily,
   showstringspaces=false,
   showspaces=false,
   numbers=left,
   numberstyle=\footnotesize,
   numbersep=9pt,
   tabsize=2,
   breaklines=true,
   showtabs=false,
   captionpos=b
}


%Start
\begin{document}

\lecture{31, 32, 33, 34, 35, 36}{July 12 - July 24, 2017}{Suryapratim Banerjee}{Harsh Mistry}

\section{Estimated Residuals}
\(\hat{r_i} = \) actual - predicted. 

\subsection*{Standardized Estimated Residual}
$$ \hat{r_i} = \frac{\hat{r_i}}{s} \sim Z $$
where s is the standard error

\section{Tests For The Regression Assumtion}
The tests are graphical (and subjective) comes with experience
\begin{itemize}
\item Scatter Plot : We draw a scatter plot and check whether a linear relationship is appropriate
\item Residual Plots
\begin{itemize}
\item \(\hat{r_i}\)'s should be in a "small" band around zero
\item Variability in the \(\hat{r_i}\)'s should be more or less constant 
\item Absence of any obvious patterns
\end{itemize}
\item The QQ-plot. : If the assumptions are right, the q-q plot should be a 45 degree line
\end{itemize}


\section{Two Population Problems}
\subsection*{Equality of means}
\begin{itemize}
\item Matched pair populations
\item Unmatched data : equal variance populations
\item Unmatched data : unequal variances and large sample sizes
\end{itemize}


\subsection*{Matched Pair}
Consider the following : 
$$ B_1, \ldots, B_n \sim N(\gamma_1, \sigma_1^2) $$
$$ A_1, \ldots, A_n \sim N(\gamma_2, \sigma_2^2) $$
\begin{definition}
\((B_i, A_i) \rightarrow \) is a matched pair in the population, Given this this the units are the same or there is a natural match between the two populations
\end{definition}

In this case the NULL hypothesis and Challenging view are
$$ H_0 : \gamma _1 = \gamma_2 $$
$$ H_1 : \gamma_1 \neq \gamma_2 $$

Define \(Y_i = A_i - B_i \) , so 
$$ Y_i \sim N(\gamma_2 - \gamma_1, \sigma_1^2 + \sigma_2^2)$$

Using this we find 
$$ D = \mid \frac{\bar{Y}}{S / \sqrt{n}} \mid $$
$$ d = \mid \frac{\bar{y}}{S / \sqrt{n}} \mid $$
$$ S = \frac{1}{n-1} \sum(Y_i - \bar{y}) $$
$$ p-value = P(D \geq d) = P(\mid T_{n-1} \mid \geq d) $$

\subsection*{Unmatched Data}
There is no natural pairing between the two populations. 
\subsubsection*{Model}
$$ Y_{1i} \sim N(\gamma_1, \sigma^2) $$
$$ Y_{2j} \sim N(\gamma_2, \sigma^2) $$

\subsubsection*{First Method for Equal Variance}
In this case, we're assuming two populations have the same variability. 
$$ \text{From 1 } \hat{Y_1} \sim N (\gamma_1, \sigma^2 / n_1) $$
$$ \text{From 2 } \hat{Y_1} \sim N (\gamma_2, \sigma^2 / n_2) $$
Thus, 
$$  \hat{Y_1} - \hat{Y_2} \sim N (\gamma_1 - \gamma_2, \sigma^2 (\frac{1}{n_1} + \frac{1}{n_1})) $$
So, 
$$ D = \mid \frac{\hat{Y_1} - \hat{Y_2} - 0}{S \sqrt{\frac{1}{n_1} + \frac{1}{n_2}}}\mid \sim T_{n_1 + n_2 - 2}$$
$$ S^2 = \frac{(n_1 -1) S_1^2 + (n_2 - 1) S_2^2}{n_1 + n_2 - 2} $$
$$p-value = P(D \geq d) = P(\mid T_{n_1 + n_2 - 2}  \mid \geq d)$$

\subsubsection*{Second Method for Equal Variance}

Define \(X = \begin{cases} 0 \text{ if population = 1}  \\ 1 \text{ if population = 2}\end{cases} \)

Using this you can draw a linear graph where \(\alpha = E(Y) \) and \(B =\) change in 1 unit, As a result the p-value can be found easily. 
$$ D = \mid \frac{\tilde{\beta} - 0}{S / \sqrt{S_{XX}}} \mid $$
$$ d = \mid \frac{\hat{\beta} - 0}{S / \sqrt{S_{XX}}} \mid $$
$$ p-value = P(D \geq d) = P(\mid T_{n_1 + n_2 - 2} \mid \geq d) $$

\subsubsection*{Unequal variance}
$$ Y_{1i} \sim G(\gamma_1, \sigma^2) $$
$$ Y_{2j} \sim G(\gamma_2, \sigma^2) $$
Assume \(n_1, n_2\) are large 
$$ D = \frac{(\bar{y_1} - \bar{y-2}) - (y_1 - y_2)}{\sqrt{(S_1^2 / n_1) + (S_2^2 / n_2)}}$$

\section{Test For Goodness Of Fit}
In so,e situations, the unknown parameter of interest is a vector 
$$ \underaccent{\tilde}{\theta} = (\theta_1, \ldots, \theta_m)$$
$$ H_0 =  \underaccent{\tilde}{\theta} =  \underaccent{\tilde}{\theta} ( \alpha) $$

Recall : 
$$ \bigtriangleup(\theta) = -2 \log \frac{L(\theta)}{L(\hat{\theta}) } \sim X^2_1$$

\begin{theorem}
If \( \underaccent{\tilde}{\theta}\) is a vector then 
$$ \bigtriangleup(\theta) = -2 \log \frac{L(\theta)}{L(\hat{\theta}) } \sim X^2_n $$
Where n = the number of independent unrestricted parameters of \(\theta\) + the number of parameters estimated under \(H_0\)
\end{theorem}

\subsubsection*{For a Multinomial Problem}
$$ \bigtriangleup = 2 \sum Y_i \cdot \ln \frac{Y_i}{E_i}  \sim X^2+{n - 1- 1} $$
where 
\begin{itemize}
\item \(Y_i\) = observation frequency of category i 
\item \(E_i\) = expected frequency of category i if \(H_0\) is true
\end{itemize}
Given this we can determine that 
\begin{itemize}
\item \(E_i = n \times p_i\) 
\item \(p_i\) = expected probabilities of category i under H
\item n = sample size  
\end{itemize}

\subsubsection*{Poisson Problem}
Divide the sata into categories and compute the observed frequency of each category
$$ \hat{p_i} = \frac{e^{\bar{x}} \bar{x}^i }{i!} $$ 
$$ E_i = n x p_i $$

\subsubsection*{Exponential Problem}
Produce a table consisting  of the frequency associated with the interval, then use the following to determine the expected values 
$$ \hat{\theta} = \bar{x} $$
$$ \hat{p_i} = \int_{interval_{min}}^{interval_{max}} \frac{1}{\bar{x}} e^{- \frac{2}{\bar{x}}} dx$$

\textbf{Restrictions :} 
\begin{itemize}
\item n needs to be large 
\item \(n_i \geq 5  \forall i\) 
\end{itemize}

\subsubsection*{Two Categorical Variables}
$$ e_{ij} = \frac{\text{ith Row Total} \times \text{jth Column Total}}{\text{Total sample size}}$$
$$ \lambda = 2 \sum_i \sum_j y_{ij} \ln \frac{y_{ij}}{e_{ij}}$$

\subsubsection*{Normal Problem}
$$ H_0 : X_i \sim G(\gamma, \sigma^2) $$
\begin{enumerate}
\item Divide the data into mutually exclusive and exhaustive categories and calculate the frequencies of each category.\\
We need atleast 3 categories
\item Assume the null hypothesis is true\\
Estimate \(\hat{p_i} = \) estimate probability of each category. 
$$ \hat{\gamma} = \bar{x} $$
$$ p_i = P(\frac{interval_{min} - \bar{x}}{\text{number of elements}} \leq x \leq  \frac{interval_{max} - \bar{x}}{\text{number of elements}}$$
\item Calculate \(e_i\)
$$ e_i = n \times \hat{p_i} $$
\item Compute \(\lambda\)
$$ \lambda = 2 \sum_i y_{i} \ln \frac{y_{i}}{e_{i}}$$
\item Compute the p-value 
$$ p-value = P(X^2 _{\text{number of categories}} \geq \lambda) $$
\end{enumerate}
\subsubsection*{General Problem}
$$ f(y_i; \theta) = \frac{2y}{\theta} e^{-y^2 / \theta}   y \geq 0 $$
$$ H_0 : Y_i \sim f(y_i ; \theta); $$
\begin{itemize}
\item Compute \(\theta = MLE\) for \(\theta\) from your data and use that to compute \(\hat{p_i}\) and thus \(e_i\)
\end{itemize}

\textbf{Pearson's Chi-Squared Statistic : }
$$ k = \sum_{i=1}^n \frac{(Y_i - E_i)^2)}{E_i} $$
This also follows \(X^2\) with the same degree of freedom, but is less powerful than the LRTS.  

\section{Design of Experiments}
We have to design the experiments in such as way that confounding variables are taken into account. 

\begin{itemize}
\item Blocking : we collect data holding the value if confounding variable constant, the problem is identifying all variables.
\item Randomization : we divide the data into two groups with the expectation that confounding factors cancel each other. 
\end{itemize}



\end{document}

