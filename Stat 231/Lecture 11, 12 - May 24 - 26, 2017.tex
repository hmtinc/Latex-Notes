%Notes by Harsh Mistry 
%Stat231
%based on Template from : https://www.cs.cmu.edu/~ggordon/10725-F12/template.tex

\documentclass{article}
\setlength{\oddsidemargin}{0.25 in}
\setlength{\evensidemargin}{-0.25 in}
\setlength{\topmargin}{-0.6 in}
\setlength{\textwidth}{6.5 in}
\setlength{\textheight}{8.5 in}
\setlength{\headsep}{0.75 in}
\setlength{\parindent}{0 in}
\setlength{\parskip}{0.1 in}
\usepackage{amsfonts,graphicx, amssymb}
\usepackage[fleqn]{amsmath}
\usepackage{fixltx2e}
\usepackage{tikz}
\usepackage{color}
\usepackage{tcolorbox}
\usepackage{lipsum}
\usepackage{listings}
\usepackage{scrextend}
\tcbuselibrary{skins,breakable}
\usetikzlibrary{shadings,shadows}
\newcounter{lecnum}
\renewcommand{\thepage}{\thelecnum-\arabic{page}}
\renewcommand{\thesection}{\thelecnum.\arabic{section}}
\renewcommand{\theequation}{\thelecnum.\arabic{equation}}
\renewcommand{\thefigure}{\thelecnum.\arabic{figure}}
\renewcommand{\thetable}{\thelecnum.\arabic{table}}
\newcommand{\lecture}[4]{
   \pagestyle{myheadings}
   \thispagestyle{plain}
   \newpage
   \setcounter{lecnum}{#1}
   \setcounter{page}{1}
   
   
%Info Box 
   \begin{center}
   \framebox{
      \vbox{\vspace{2mm}
    \hbox to 6.28in { {\bf Stat 231 - Statistics 
	\hfill Spring 2017} }
       \vspace{4mm}
       \hbox to 6.28in { {\Large \hfill Lecture #1: #2  \hfill} }
       \vspace{2mm}
       \hbox to 6.28in { {\it Lecturer: #3 \hfill Notes By: #4} }
      \vspace{2mm}}
   }
   \end{center}
   
   \markboth{Lecture #1: #2}{Lecture #1: #2}



 
}

\renewcommand{\cite}[1]{[#1]}
\def\beginrefs{\begin{list}%
        {[\arabic{equation}]}{\usecounter{equation}
         \setlength{\leftmargin}{2.0truecm}\setlength{\labelsep}{0.4truecm}%
         \setlength{\labelwidth}{1.6truecm}}}
\def\endrefs{\end{list}}
\def\bibentry#1{\item[\hbox{[#1]}]}

\newcommand{\fig}[3]{
			\vspace{#2}
			\begin{center}
			Figure \thelecnum.#1:~#3
			\end{center}
	}
	
\newcommand{\pipe}{\(\mid\)}
\newcommand{\ctr}{\(\wedge\)}

\newtheorem{theorem}{Theorem}[lecnum]
\newtheorem{lemma}[theorem]{Lemma}
\newtheorem{ex}[theorem]{Example}
\newtheorem{proposition}[theorem]{Proposition}
\newtheorem{claim}[theorem]{Claim}
\newtheorem{corollary}[theorem]{Corollary}
\newtheorem{definition}[theorem]{Definition}
\newenvironment{proof}{{\bf Proof:}}{\hfill\rule{2mm}{2mm}}
\newcommand\E{\mathbb{E}}

%color definitions :
\definecolor{darkred}{rgb}{0.55, 0.0, 0.0}
\definecolor{lightcoral}{rgb}{0.94, 0.5, 0.5}
\definecolor{tomato}{rgb}{1.0, 0.39, 0.28}
\definecolor{lightgray}{rgb}{.9,.9,.9}
\definecolor{darkgray}{rgb}{.4,.4,.4}
\definecolor{purple}{rgb}{0.65, 0.12, 0.82}
\definecolor{lightgreen}{rgb}{0.56, 0.93, 0.56}
\definecolor{darkgreen}{rgb}{0.0, 0.2, 0.13}
\definecolor{limegreen}{rgb}{0.2, 0.8, 0.2}
\definecolor{lightblue}{rgb}{0.68, 0.85, 0.9}
\definecolor{darkblue}{rgb}{0.0, 0.0, 0.55}


%Environments
\newenvironment{exblock}[1]{%
    \tcolorbox[beamer,%
    noparskip,breakable,
    colback=lightgreen,colframe=darkgreen,%
    colbacklower=limegreen!75!lightgreen,%
    title=#1]}%
    {\endtcolorbox}

\newenvironment{ablock}[1]{%
    \tcolorbox[beamer,%
    noparskip,breakable,
    colback=lightcoral,colframe=darkred,%
    colbacklower=tomato!75!lightcoral,%
    title=#1]}%
    {\endtcolorbox}

\newenvironment{cblock}[1]{%
    \tcolorbox[beamer,%
    noparskip,breakable,
    colback=lightblue,colframe=darkblue,%
    colbacklower=darkblue!75!lightblue,%
    title=#1]}%
    {\endtcolorbox}


%Languages
\lstdefinelanguage{JavaScript}{
  keywords={typeof, new, true, false, catch, function, return, null, catch, switch, var, if, in, while, do, else, case, break},
  keywordstyle=\color{blue}\bfseries,
  ndkeywords={class, export, boolean, throw, implements, import, this},
  ndkeywordstyle=\color{darkgray}\bfseries,
  identifierstyle=\color{black},
  sensitive=false,
  comment=[l]{//},
  morecomment=[s]{/*}{*/},
  commentstyle=\color{purple}\ttfamily,
  stringstyle=\color{red}\ttfamily,
  morestring=[b]',
  morestring=[b]"
}

%Listings
\lstset{
   language=JavaScript,
   backgroundcolor=\color{lightgray},
   extendedchars=true,
   basicstyle=\footnotesize\ttfamily,
   showstringspaces=false,
   showspaces=false,
   numbers=left,
   numberstyle=\footnotesize,
   numbersep=9pt,
   tabsize=2,
   breaklines=true,
   showtabs=false,
   captionpos=b
}


%Start
\begin{document}

\lecture{11, 12}{May 24th - 26th, 2017}{Suryapratim Banerjee}{Harsh Mistry}

\section*{Exponential Model}
\[Y_i \sim Exp(\gamma)  \]
\textbf{Density Function :}
\[f(y) = \frac{1}{\gamma}e^{-y / \gamma}\]
\textbf{Likelihood Function :}
\[L(\gamma) = \frac{1}{\gamma} e ^{-y_1 / \gamma} \cdot \ldots \cdot \frac{1}{y}e^{-y_n / \gamma} = \frac{1}{\gamma^n} e^{-\frac{1}{\gamma} \sum y_i}\]
\textbf{Log-Likelihood Function}
\[l(\gamma) = -n \ln \gamma - \frac{1}{\gamma} \sum y_i\]
\section*{Gaussian Distribution}
\[Y_i \sim G(\gamma, \sigma) \] 
\[f(y) = \frac{1}{\sqrt{2\pi}  \sigma} e^{-\frac{1}{2 \sigma^2} (y - \gamma)^2}\]
\textbf{Likelihood Function :}
\[L(\gamma, \sigma) = \frac{1}{(2\pi)^{n/2}  \sigma^n } e^{-\frac{1}{2 \sigma^2} (y_i - \gamma)^2}\]
\textbf{Log-Likelihood Function}
\[l(\gamma, \sigma) =   \frac{-n}{2} \cdot \ln(2\pi) - n \ln  \sigma - \frac{1}{2 \sigma^2} \sum (y_i - \sigma)^2\]
\section*{Invariance Property Of The MLE}
If \(\hat{\theta}\) is the MLE for \(\theta\), then \(g(\hat{\theta})\) in the MLE for \(g(\theta)\) if g is continuous. 
\section*{Uniform Distribution}
\[Y_i \sim  U[0, \theta] \]
\textbf{Density Function :}
\[f(y) = \begin{cases} \frac{1}{\theta}  & \text{ if } 0 \leq y \leq \theta \\ 0 & \text{elsewhere}\end{cases}\]
\textbf{Likelihood Function :}
\[L(\theta) = \begin{cases} \frac{1}{\theta^n} y  & \text{ if } 0 \leq y_i \leq \theta , \forall i \\
0 & \text{ if } \theta < max\{y_1, \ldots, y_n\} \end{cases} \]
\section{Model Selection}
Model : "Identify" the random variable from which \(\{y_1, \ldots, y_n\}\) is drawn 

\subsection*{Subjective Tests}
We run numerical and graphical tests on the data to select the "right" model.
\subsubsection*{Numerical Tests}
\begin{itemize}
\item Check whether the data set satisfies the theoretical properties of the distributions assumed in your model  
\end{itemize}
\subsubsection{Graphical Tools}
\begin{itemize}
\item Super impose the relative frequency histogram of your data set to the theoretical distirbution function assumed and see whether the shapes match 
\end{itemize}

\subsubsection*{The Q-Q plot} 
Typically used to check whether Gaussian is the "right mode"




\end{document}


