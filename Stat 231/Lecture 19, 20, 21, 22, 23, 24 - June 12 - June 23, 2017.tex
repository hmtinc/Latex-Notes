%Notes by Harsh Mistry 
%Stat231
%based on Template from : https://www.cs.cmu.edu/~ggordon/10725-F12/template.tex

\documentclass{article}
\setlength{\oddsidemargin}{0.25 in}
\setlength{\evensidemargin}{-0.25 in}
\setlength{\topmargin}{-0.6 in}
\setlength{\textwidth}{6.5 in}
\setlength{\textheight}{8.5 in}
\setlength{\headsep}{0.75 in}
\setlength{\parindent}{0 in}
\setlength{\parskip}{0.1 in}
\usepackage{amsfonts,graphicx, amssymb}
\usepackage[fleqn]{amsmath}
\usepackage{fixltx2e}
\usepackage{tikz}
\usepackage{color}
\usepackage{tcolorbox}
\usepackage{lipsum}
\usepackage{listings}
\usepackage{scrextend}
\tcbuselibrary{skins,breakable}
\usetikzlibrary{shadings,shadows}
\newcounter{lecnum}
\renewcommand{\thepage}{\thelecnum-\arabic{page}}
\renewcommand{\thesection}{\thelecnum.\arabic{section}}
\renewcommand{\theequation}{\thelecnum.\arabic{equation}}
\renewcommand{\thefigure}{\thelecnum.\arabic{figure}}
\renewcommand{\thetable}{\thelecnum.\arabic{table}}
\newcommand{\lecture}[4]{
   \pagestyle{myheadings}
   \thispagestyle{plain}
   \newpage
   \setcounter{lecnum}{#1}
   \setcounter{page}{1}
   
   
%Info Box 
   \begin{center}
   \framebox{
      \vbox{\vspace{2mm}
    \hbox to 6.28in { {\bf Stat 231 - Statistics 
	\hfill Spring 2017} }
       \vspace{4mm}
       \hbox to 6.28in { {\Large \hfill Lecture #1: #2  \hfill} }
       \vspace{2mm}
       \hbox to 6.28in { {\it Lecturer: #3 \hfill Notes By: #4} }
      \vspace{2mm}}
   }
   \end{center}
   
   \markboth{Lecture #1: #2}{Lecture #1: #2}



 
}

\renewcommand{\cite}[1]{[#1]}
\def\beginrefs{\begin{list}%
        {[\arabic{equation}]}{\usecounter{equation}
         \setlength{\leftmargin}{2.0truecm}\setlength{\labelsep}{0.4truecm}%
         \setlength{\labelwidth}{1.6truecm}}}
\def\endrefs{\end{list}}
\def\bibentry#1{\item[\hbox{[#1]}]}

\newcommand{\fig}[3]{
			\vspace{#2}
			\begin{center}
			Figure \thelecnum.#1:~#3
			\end{center}
	}
	
\newcommand{\pipe}{\(\mid\)}
\newcommand{\ctr}{\(\wedge\)}

\newtheorem{theorem}{Theorem}[lecnum]
\newtheorem{lemma}[theorem]{Lemma}
\newtheorem{ex}[theorem]{Example}
\newtheorem{proposition}[theorem]{Proposition}
\newtheorem{claim}[theorem]{Claim}
\newtheorem{corollary}[theorem]{Corollary}
\newtheorem{definition}[theorem]{Definition}
\newenvironment{proof}{{\bf Proof:}}{\hfill\rule{2mm}{2mm}}
\newcommand\E{\mathbb{E}}

%color definitions :
\definecolor{darkred}{rgb}{0.55, 0.0, 0.0}
\definecolor{lightcoral}{rgb}{0.94, 0.5, 0.5}
\definecolor{tomato}{rgb}{1.0, 0.39, 0.28}
\definecolor{lightgray}{rgb}{.9,.9,.9}
\definecolor{darkgray}{rgb}{.4,.4,.4}
\definecolor{purple}{rgb}{0.65, 0.12, 0.82}
\definecolor{lightgreen}{rgb}{0.56, 0.93, 0.56}
\definecolor{darkgreen}{rgb}{0.0, 0.2, 0.13}
\definecolor{limegreen}{rgb}{0.2, 0.8, 0.2}
\definecolor{lightblue}{rgb}{0.68, 0.85, 0.9}
\definecolor{darkblue}{rgb}{0.0, 0.0, 0.55}


%Environments
\newenvironment{exblock}[1]{%
    \tcolorbox[beamer,%
    noparskip,breakable,
    colback=lightgreen,colframe=darkgreen,%
    colbacklower=limegreen!75!lightgreen,%
    title=#1]}%
    {\endtcolorbox}

\newenvironment{ablock}[1]{%
    \tcolorbox[beamer,%
    noparskip,breakable,
    colback=lightcoral,colframe=darkred,%
    colbacklower=tomato!75!lightcoral,%
    title=#1]}%
    {\endtcolorbox}

\newenvironment{cblock}[1]{%
    \tcolorbox[beamer,%
    noparskip,breakable,
    colback=lightblue,colframe=darkblue,%
    colbacklower=darkblue!75!lightblue,%
    title=#1]}%
    {\endtcolorbox}


%Languages
\lstdefinelanguage{JavaScript}{
  keywords={typeof, new, true, false, catch, function, return, null, catch, switch, var, if, in, while, do, else, case, break},
  keywordstyle=\color{blue}\bfseries,
  ndkeywords={class, export, boolean, throw, implements, import, this},
  ndkeywordstyle=\color{darkgray}\bfseries,
  identifierstyle=\color{black},
  sensitive=false,
  comment=[l]{//},
  morecomment=[s]{/*}{*/},
  commentstyle=\color{purple}\ttfamily,
  stringstyle=\color{red}\ttfamily,
  morestring=[b]',
  morestring=[b]"
}

%Listings
\lstset{
   language=JavaScript,
   backgroundcolor=\color{lightgray},
   extendedchars=true,
   basicstyle=\footnotesize\ttfamily,
   showstringspaces=false,
   showspaces=false,
   numbers=left,
   numberstyle=\footnotesize,
   numbersep=9pt,
   tabsize=2,
   breaklines=true,
   showtabs=false,
   captionpos=b
}


%Start
\begin{document}

\lecture{19, 20, 21, 22, 23, 24}{June 12th - June 23nd, 2017}{Suryapratim Banerjee}{Harsh Mistry}

\section{T-Distribution Recap}
\(T_n\) is a continuous random variable in \((- \infty, \infty)\) that is said to follow a student's T distribution with n degrees of freedom if T is a ratio of two independent random variables. 
$$ T = \frac{Z}{W} $$ 
Where \(Z \sim G(0,1)\) and \(W = \sqrt{x^2(n) / n} \)

\subsection*{Properties}
\begin{itemize}
\item \(T_n\) is symmetric around zero for all n 
\item \(T_n\) looks like the Z-distribution but with a higher kurtosis.
\item n = parameter of the T-distribution 
\item As  ns approach infinity and \(T \rightarrow Z\), The T looks like the z 
\end{itemize} 

\section{T Table}

\begin{theorem}
Let \(y_1 \ldots y_n\) be independent Gaussian random variables with mean \(\mu \text{ and }  \sigma^2\)
Define 
$$ \bar{Y} = \frac{1}{n} \sum Y_i \text{ (estimator)} $$
$$  S^2 = \frac{1}{n-1} \sum (Y_i - \bar{Y})^2  \text{ (estimator corresponding to its sample variance)}$$
Then 
$$
\frac{\bar{Y} -  \mu}{S / \sqrt{n}} \sim T_{n-1}$$  
$$ \frac{(n-1)S^2}{\sigma^2} \sim X^2_{n-1}$$
\end{theorem}

\textbf{Confidence Interval for \(\mu\)}
\begin{enumerate}
\item \(Y_1 \sim G(\mu, r)\)
\item \(\hat{\mu} = \bar{y} \rightarrow \) Estimate 
\item \(\bar{Y} \rightarrow\) estimator
\item Construct the pivotal distribution 
$$ \frac{\bar{y} - \mu }{S / \sqrt{n}} \sim T_{n-1}$$
\item Find the end points of the pivot (Look in row (n-1) column = percentage)
\item Construct the coverage Interval
$$ P( -c < \frac{\bar{Y} - mu}{\frac{S}{\sqrt{n}}} < c ) = percentage $$
$$ P( \bar{Y} - c \frac{S}{\sqrt{n}} \leq \mu \leq \bar{Y} + c \frac{S}{\sqrt{n}}) = percentage$$
\item Replace \(\bar{Y}\) with \(\bar{y}\)
\end{enumerate}
\textbf{Confidence Interval for \(\sigma^2\) }
\begin{enumerate}
\item Find \(s^2\)
\item Estimator = \(S^2\) 
\item Pivotal 
$$\frac{(n-1)S^2}{\sigma^2} \sim X^2_{n-1}$$
\item Find Coverage interval from \(x^2\) table 
\item Find Confidence Interval  
\end{enumerate}

\section{Other Distributions}
\subsection*{Poison Problem}
\begin{itemize}
\item Find Estimate for \(\theta\)
\[\hat{\theta} = \bar{y}\]
\item Estimator of \(\bar{y} \rightarrow \bar{Y}\)
\item By the CLT 
\[\frac{\bar{Y} - \theta}{\sqrt{\bar{Y} / n}} \sim G(0,1) \]
\item Construct the coverage Interval
$$ P( -c < \frac{\bar{Y} - \theta}{\sqrt{\bar{Y} / n}} < c ) = percentage $$
$$  \bar{Y} \pm c \sqrt{\bar{Y} / n} $$
\subsection*{Exponential Problem}
Mean = \(\theta\); Variance \(\theta^2\) 
$$ \bar{Y} \sim G(\theta , \theta/ \sqrt{n}) $$
$$ \frac{\bar{Y} - \theta}{\theta / \sqrt{n}} = z $$
\end{itemize}

\subsection*{Exact Interval for the Exponential Problem}
If \(Y_i \sim Exp(\theta) \) then 
$$ \frac{2 Y_i}{\theta} \sim Exp(2) \sim X^2(2) $$


\section{Relationship between Likelihood Intervals and Confidence Intervals}

Assume N is large 

\begin{theorem}
If \(\theta\) is the true value of the parameter; \(\hat{\theta}\) is the MLE; Then 
$$ \delta (\theta) = - 2 log \frac{L(\theta)}{L(\tilde{\theta})} \sim X^2 (1)	$$
\(\delta\) = Likelihood Ratio Test Statistic 
\end{theorem}

\section{Hypothesis Testing}
\begin{definition}
A hypothesis is a statement made about some attribute of the population
$$ H_ 0 : \theta = \theta_0 $$
\end{definition}

Two claims are tested against each other 
\begin{itemize}
\item \(H_0\) : Null Hypothesis = Current conventional wisdom 
\item \(H_1\) : Alternate Hypothesis = Challenging view
\end{itemize}

\begin{definition}
a \textbf{p-value} is the probability of observing your evidence (or worse) given that the null hypothesis is true. 
\end{definition}

\begin{ablock}{Notes}
\(H_0\) and \(H_1\) are not treated symmetrically unless we have "overwhelming evidence against \(H_0\), we do not reject it. The burden of proof is on \(H_1\)
\end{ablock}

\begin{itemize}
\item \(p > 0.1 \implies\) No evidence against \(H_0\)
\item \(0. 05 < p \leq 0.1 \implies \) Weak evidence against \(H_0\)
\item \(0.01 <  p \leq 0.05 \implies \) Strong evidence against \(H_0\) 
\item \(p \leq 0.01 \implies\) Very string evidence against \(H_0\) 
\end{itemize}

"Statistically significant" \(\rightarrow\) p-value of the test \(\leq\) 0.05

\begin{itemize}
\item If \(p < 0.05 \implies\) Reject \(H_0\) at \(5\%\) level of significance
\item If \(p > 0.05 \implies\) Do not reject \(H_0\)
\end{itemize}

\subsection*{Type Errors}
\begin{itemize}
\item Type I error : Rejecting \(H_0\) when its actually true 
\item Type II error : Do not reject \(H_0\) when it is actually false
\end{itemize}
These Two errors may conflict with each other. 

\subsection*{Statistical Tests}
\begin{itemize}
\item Discrepancy measure : A random variable that measures the level of disagreement of the data with the null hypothesis. 
\begin{itemize}
\item The distribution of D is known 
\item \(D \geq 0\) and \(D = 0\), is the best evidence for the \(H_0\) 
\item p-value : \(P(D \geq d; H_0 \text{ is true } )\) where d - value of D in your sample 

\end{itemize}
\end{itemize}

\end{document}


