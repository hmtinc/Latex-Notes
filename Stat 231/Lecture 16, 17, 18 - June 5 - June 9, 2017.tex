%Notes by Harsh Mistry 
%Stat231
%based on Template from : https://www.cs.cmu.edu/~ggordon/10725-F12/template.tex

\documentclass{article}
\setlength{\oddsidemargin}{0.25 in}
\setlength{\evensidemargin}{-0.25 in}
\setlength{\topmargin}{-0.6 in}
\setlength{\textwidth}{6.5 in}
\setlength{\textheight}{8.5 in}
\setlength{\headsep}{0.75 in}
\setlength{\parindent}{0 in}
\setlength{\parskip}{0.1 in}
\usepackage{amsfonts,graphicx, amssymb}
\usepackage[fleqn]{amsmath}
\usepackage{fixltx2e}
\usepackage{tikz}
\usepackage{color}
\usepackage{tcolorbox}
\usepackage{lipsum}
\usepackage{listings}
\usepackage{scrextend}
\tcbuselibrary{skins,breakable}
\usetikzlibrary{shadings,shadows}
\newcounter{lecnum}
\renewcommand{\thepage}{\thelecnum-\arabic{page}}
\renewcommand{\thesection}{\thelecnum.\arabic{section}}
\renewcommand{\theequation}{\thelecnum.\arabic{equation}}
\renewcommand{\thefigure}{\thelecnum.\arabic{figure}}
\renewcommand{\thetable}{\thelecnum.\arabic{table}}
\newcommand{\lecture}[4]{
   \pagestyle{myheadings}
   \thispagestyle{plain}
   \newpage
   \setcounter{lecnum}{#1}
   \setcounter{page}{1}
   
   
%Info Box 
   \begin{center}
   \framebox{
      \vbox{\vspace{2mm}
    \hbox to 6.28in { {\bf Stat 231 - Statistics 
	\hfill Spring 2017} }
       \vspace{4mm}
       \hbox to 6.28in { {\Large \hfill Lecture #1: #2  \hfill} }
       \vspace{2mm}
       \hbox to 6.28in { {\it Lecturer: #3 \hfill Notes By: #4} }
      \vspace{2mm}}
   }
   \end{center}
   
   \markboth{Lecture #1: #2}{Lecture #1: #2}



 
}

\renewcommand{\cite}[1]{[#1]}
\def\beginrefs{\begin{list}%
        {[\arabic{equation}]}{\usecounter{equation}
         \setlength{\leftmargin}{2.0truecm}\setlength{\labelsep}{0.4truecm}%
         \setlength{\labelwidth}{1.6truecm}}}
\def\endrefs{\end{list}}
\def\bibentry#1{\item[\hbox{[#1]}]}

\newcommand{\fig}[3]{
			\vspace{#2}
			\begin{center}
			Figure \thelecnum.#1:~#3
			\end{center}
	}
	
\newcommand{\pipe}{\(\mid\)}
\newcommand{\ctr}{\(\wedge\)}

\newtheorem{theorem}{Theorem}[lecnum]
\newtheorem{lemma}[theorem]{Lemma}
\newtheorem{ex}[theorem]{Example}
\newtheorem{proposition}[theorem]{Proposition}
\newtheorem{claim}[theorem]{Claim}
\newtheorem{corollary}[theorem]{Corollary}
\newtheorem{definition}[theorem]{Definition}
\newenvironment{proof}{{\bf Proof:}}{\hfill\rule{2mm}{2mm}}
\newcommand\E{\mathbb{E}}

%color definitions :
\definecolor{darkred}{rgb}{0.55, 0.0, 0.0}
\definecolor{lightcoral}{rgb}{0.94, 0.5, 0.5}
\definecolor{tomato}{rgb}{1.0, 0.39, 0.28}
\definecolor{lightgray}{rgb}{.9,.9,.9}
\definecolor{darkgray}{rgb}{.4,.4,.4}
\definecolor{purple}{rgb}{0.65, 0.12, 0.82}
\definecolor{lightgreen}{rgb}{0.56, 0.93, 0.56}
\definecolor{darkgreen}{rgb}{0.0, 0.2, 0.13}
\definecolor{limegreen}{rgb}{0.2, 0.8, 0.2}
\definecolor{lightblue}{rgb}{0.68, 0.85, 0.9}
\definecolor{darkblue}{rgb}{0.0, 0.0, 0.55}


%Environments
\newenvironment{exblock}[1]{%
    \tcolorbox[beamer,%
    noparskip,breakable,
    colback=lightgreen,colframe=darkgreen,%
    colbacklower=limegreen!75!lightgreen,%
    title=#1]}%
    {\endtcolorbox}

\newenvironment{ablock}[1]{%
    \tcolorbox[beamer,%
    noparskip,breakable,
    colback=lightcoral,colframe=darkred,%
    colbacklower=tomato!75!lightcoral,%
    title=#1]}%
    {\endtcolorbox}

\newenvironment{cblock}[1]{%
    \tcolorbox[beamer,%
    noparskip,breakable,
    colback=lightblue,colframe=darkblue,%
    colbacklower=darkblue!75!lightblue,%
    title=#1]}%
    {\endtcolorbox}


%Languages
\lstdefinelanguage{JavaScript}{
  keywords={typeof, new, true, false, catch, function, return, null, catch, switch, var, if, in, while, do, else, case, break},
  keywordstyle=\color{blue}\bfseries,
  ndkeywords={class, export, boolean, throw, implements, import, this},
  ndkeywordstyle=\color{darkgray}\bfseries,
  identifierstyle=\color{black},
  sensitive=false,
  comment=[l]{//},
  morecomment=[s]{/*}{*/},
  commentstyle=\color{purple}\ttfamily,
  stringstyle=\color{red}\ttfamily,
  morestring=[b]',
  morestring=[b]"
}

%Listings
\lstset{
   language=JavaScript,
   backgroundcolor=\color{lightgray},
   extendedchars=true,
   basicstyle=\footnotesize\ttfamily,
   showstringspaces=false,
   showspaces=false,
   numbers=left,
   numberstyle=\footnotesize,
   numbersep=9pt,
   tabsize=2,
   breaklines=true,
   showtabs=false,
   captionpos=b
}


%Start
\begin{document}

\lecture{16, 17, 18}{May 29th - June 2nd, 2017}{Suryapratim Banerjee}{Harsh Mistry}

\section{Interval Estimation}
\subsection*{Interpretation}
\begin{itemize}
\item The confidence Interval is an Estimate of the r.v.s \(L, U\) where \([L, U]\) contains \(\theta\) with \(95 \%\) confidence 
\item If the experiment was repeated many times, approximately \(95\%\) of the intervals constructed would contain \(\theta\)
\subsection*{Distribution}
\textbf{Sampling Distribution of the Sample Mean}
$$ \bar{Y} \sim G(\gamma, \frac{\sigma}{\sqrt{n}}) $$ 
\end{itemize}

\begin{enumerate}
\item Find the estimate for the unknown parameter. \\
MLE for \(\gamma = \bar{y} = \frac{1}{n} \sum y_i \)\\
\(\hat{\gamma} = \bar{y}\)
\item Identify the estimator and its distribution\\
\(\bar{y} = \)r.v. from which \(\bar{y}\) is an outcome \\
\(\bar{y} = \) Estimator
\item Construct the pivotal quantity from the sampling distribution
\[\frac{\bar{y} - \gamma}{\frac{\sigma}{\sqrt{n}}} = z = G(0,1) \]
\item Find the extreme points of your pivotal distribution
\item Use step 4 to construct the coverage  Interval 
\item Use the coverage interval to construct your confidence interval. \\
Confidence Interval = \([\bar{y}  \pm z^* \frac{\sigma}{\sqrt{n}}]\)
\end{enumerate}

\section{Distribution Theory}
\subsection{The Chi-Squared Distribution}
\begin{definition} Let W be a random variable such that 
$$ W  = Z_1 ^2 + Z_2^2 + \ldots + Z_n^2 $$
Where \(Z_i \sim G(0,1)\) and \(Z_i\)'s are independent, Then \(W\) is said to follow a chi-squared distribution with n degrees of freedom 
$$ W \sim X^2_n $$ 
\end{definition}

\subsubsection*{Properties of Chi-Squared Distibutions}
\begin{itemize}
\item n - degrees of freedom parameter of the chi-squared distribution
\item As n changes the shape changes 
\item W can take values between 0 and \(\infty\)
\item E(W) = n and Var(W)= 2n
\item Suppose \(W_1 \sim X^2_n\) and \(W_2 \sim X^2_n\),  where \(W_1\) and \(W_2\) are independent 
$$ W_1 + W_2 \sim X^2_{n_1 + n_2} $$ 
\end{itemize}

\subsection{Students T-distribution}
\begin{definition}
A random variable T is said to follow a student's T distribution with n degrees of freedom if
$$ T = \frac{z}{W}$$
Where \(Z \sim G(0,1)\) , \(W = \sqrt{x^2_n / n}\), and Z/W are independent 
\end{definition}

\subsubsection*{Properties}
\begin{itemize}
\item T can take all values \((-\infty, \infty)\) 
\item T is symmetric around zero for any n 
\item For "small" n, T looks like the z distribution, but fatter tails \(K > 3\)
\item As n becomes larger \(T \longrightarrow Z\) and the \(pdf\) converges for \(n \longrightarrow \infty\)
\end{itemize}

\end{document}


