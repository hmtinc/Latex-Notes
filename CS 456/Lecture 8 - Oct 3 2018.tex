%Notes by Harsh Mistry 
%CS 456/656
%Based on Template From  https://www.cs.cmu.edu/~ggordon/10725-F12/template.tex

\documentclass[twoside]{article}
\setlength{\oddsidemargin}{0.25 in}
\setlength{\evensidemargin}{-0.25 in}
\setlength{\topmargin}{-0.6 in}
\setlength{\textwidth}{6.5 in}
\setlength{\textheight}{8.5 in}
\setlength{\headsep}{0.75 in}
\setlength{\parindent}{0 in}
\setlength{\parskip}{0.1 in}
\usepackage{amsmath,amsfonts,graphicx}
\newcounter{lecnum}
\renewcommand{\thepage}{\thelecnum-\arabic{page}}
\renewcommand{\thesection}{\thelecnum.\arabic{section}}
\renewcommand{\theequation}{\thelecnum.\arabic{equation}}
\renewcommand{\thefigure}{\thelecnum.\arabic{figure}}
\renewcommand{\thetable}{\thelecnum.\arabic{table}}
\newcommand{\lecture}[4]{
   \pagestyle{myheadings}
   \thispagestyle{plain}
   \newpage
   \setcounter{lecnum}{#1}
   \setcounter{page}{1}
   
   
%Info Box 
   \begin{center}
   \framebox{
      \vbox{\vspace{2mm}
    \hbox to 6.28in { {\bf CS 456/656 - Computer Networks
	\hfill Fall 2018} }
       \vspace{4mm}
       \hbox to 6.28in { {\Large \hfill Lecture #1: #2  \hfill} }
       \vspace{2mm}
       \hbox to 6.28in { {\it Lecturer: #3 \hfill Notes By: #4} }
      \vspace{2mm}}
   }
   \end{center}
   
   \markboth{Lecture #1: #2}{Lecture #1: #2}



 
}

\renewcommand{\cite}[1]{[#1]}
\def\beginrefs{\begin{list}%
        {[\arabic{equation}]}{\usecounter{equation}
         \setlength{\leftmargin}{2.0truecm}\setlength{\labelsep}{0.4truecm}%
         \setlength{\labelwidth}{1.6truecm}}}
\def\endrefs{\end{list}}
\def\bibentry#1{\item[\hbox{[#1]}]}

\newcommand{\fig}[3]{
			\vspace{#2}
			\begin{center}
			Figure \thelecnum.#1:~#3
			\end{center}
	}

\newtheorem{theorem}{Theorem}[lecnum]
\newtheorem{lemma}[theorem]{Lemma}
\newtheorem{ex}[theorem]{Example}
\newtheorem{proposition}[theorem]{Proposition}
\newtheorem{claim}[theorem]{Claim}
\newtheorem{corollary}[theorem]{Corollary}
\newtheorem{definition}[theorem]{Definition}
\newenvironment{proof}{{\bf Proof:}}{\hfill\rule{2mm}{2mm}}
\newcommand\E{\mathbb{E}}


%Start of Document 
\begin{document}

\lecture{8}{October 3, 2018}{Kamal Zille Huma}{Harsh Mistry}

\section{DNS}
\begin{itemize}
\item DNS (Domain Name Server) is a distributed hierachical database that keeps record of HostNames and correspondng IP addresses. 
\item If the DNS goes down you won't be able to connect 
\item DNS is distrbuted to avoid downtime and ensure speedy search results
\item Names are parsed Right to Left
\item Root servers given a request route requests to DNS servers for the specific top-level domain
\item Top-Level Servers will direct you to the DNS server for the specific site
\item Authoritive server is the last DNS server in the chain
\item Local DNS will have the location of the Root DNS server
\item Local DNS is checked first for possible cached values which might have the Authoritive server location
\item DNS is primarily UDP, but if no response is received, then your client will open a TCP connection
\item TCP is used as a backup
\item DNS Definitions
\begin{itemize}
\item DNS : Distributed database storing resource records (RR) 
\item RR format is (name, value, type, ttl) 
\end{itemize}
\item Types 
\begin{itemize}
\item Type = A 
\begin{itemize}
\item Name is the host name
\item Value is the IP address
\end{itemize}
\item Type = NS 
\begin{itemize}
\item Name is domain (i.e foo.com)
\item Value is host name of authoritative name server for this domain 
\end{itemize}
\item Type = CNAME
\begin{itemize}
\item Name is alias name for the real name 
\item Value is canonical name (real name)
\item example : ibm.com is really backup2.ibm.com
\end{itemize}
\item Type = MX
\begin{itemize}
\item Value is name of mailserver associated with name
\end{itemize}
\end{itemize}
\item DNS Protocols

\end{itemize}




\end{document}





