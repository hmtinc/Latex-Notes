%Notes by Harsh Mistry 
%CS 456/656
%Based on Template From  https://www.cs.cmu.edu/~ggordon/10725-F12/template.tex

\documentclass[twoside]{article}
\setlength{\oddsidemargin}{0.25 in}
\setlength{\evensidemargin}{-0.25 in}
\setlength{\topmargin}{-0.6 in}
\setlength{\textwidth}{6.5 in}
\setlength{\textheight}{8.5 in}
\setlength{\headsep}{0.75 in}
\setlength{\parindent}{0 in}
\setlength{\parskip}{0.1 in}
\usepackage{amsmath,amsfonts,graphicx}
\newcounter{lecnum}
\renewcommand{\thepage}{\thelecnum-\arabic{page}}
\renewcommand{\thesection}{\thelecnum.\arabic{section}}
\renewcommand{\theequation}{\thelecnum.\arabic{equation}}
\renewcommand{\thefigure}{\thelecnum.\arabic{figure}}
\renewcommand{\thetable}{\thelecnum.\arabic{table}}
\newcommand{\lecture}[4]{
   \pagestyle{myheadings}
   \thispagestyle{plain}
   \newpage
   \setcounter{lecnum}{#1}
   \setcounter{page}{1}
   
   
%Info Box 
   \begin{center}
   \framebox{
      \vbox{\vspace{2mm}
    \hbox to 6.28in { {\bf CS 456/656 - Computer Networks
	\hfill Fall 2018} }
       \vspace{4mm}
       \hbox to 6.28in { {\Large \hfill Lecture #1: #2  \hfill} }
       \vspace{2mm}
       \hbox to 6.28in { {\it Lecturer: #3 \hfill Notes By: #4} }
      \vspace{2mm}}
   }
   \end{center}
   
   \markboth{Lecture #1: #2}{Lecture #1: #2}



 
}

\renewcommand{\cite}[1]{[#1]}
\def\beginrefs{\begin{list}%
        {[\arabic{equation}]}{\usecounter{equation}
         \setlength{\leftmargin}{2.0truecm}\setlength{\labelsep}{0.4truecm}%
         \setlength{\labelwidth}{1.6truecm}}}
\def\endrefs{\end{list}}
\def\bibentry#1{\item[\hbox{[#1]}]}

\newcommand{\fig}[3]{
			\vspace{#2}
			\begin{center}
			Figure \thelecnum.#1:~#3
			\end{center}
	}

\newtheorem{theorem}{Theorem}[lecnum]
\newtheorem{lemma}[theorem]{Lemma}
\newtheorem{ex}[theorem]{Example}
\newtheorem{proposition}[theorem]{Proposition}
\newtheorem{claim}[theorem]{Claim}
\newtheorem{corollary}[theorem]{Corollary}
\newtheorem{definition}[theorem]{Definition}
\newenvironment{proof}{{\bf Proof:}}{\hfill\rule{2mm}{2mm}}
\newcommand\E{\mathbb{E}}


%Start of Document 
\begin{document}

\lecture{13 - 18}{October 29 - November 13, 2018}{Kamal Zille Huma}{Harsh Mistry}

\section{Transport Layer Continued}

\begin{itemize}
\item Ack indicates the expected packet from the other side 
\item Timeout Interval = EstimatedRTT + 4 * DEVRTT
\end{itemize}

\subsection{TCP}
\begin{itemize}
\item Retransmissions are triggered by Timeout and Duplicate Acks
\item TCP added pipelined segments, cumulative acks, and single transmission timer
\item Sequence number is the byte stream number of first data byte in segment 
\item Fast Retransmit : If sender receives 3 ACKs for same data, resend unacked segment with smallest sequence number
\end{itemize}

\subsubsection{TCP Flow Control}
\begin{itemize}
\item Receiver controls sender, so sender won't overflow receiver's buffer 
\item Receiver "advertises" free buffer space by including rwnd value in TCP header of receiver 
\item rwnd is typically total buffer size - RcvBuffer 
\item RcvBuffer is typically 4096, but some systems may dynamically change that
\end{itemize}

\subsubsection{Connection Management}
\begin{itemize}
\item Before exchanging data, sender/receeiver need to agree to establish connection and agree on connection parameters
\item Two-way handshake can not work sure to variable delays , retransmitted messages, and message reordering
\item Three-way handshaking is introduced to resolve theses issues
\begin{enumerate}
\item Client sends SYN Request (Synbit = 1, Seq bit = x) 
\item Server chooses seq starting point and sends SYN ACK  (SynBit = 1, seq = y, AckBit =1, AckNum = x+1)
\item Client sends Ack (ACKBit = 1, ACKnum =y+1)
\end{enumerate}
\item Closing a connection 
\begin{enumerate}
\item Client sends packet with FinBit (FinBit = 1, Seq = x)
\item Server Acks packet
\item Server Keeps sending data till its ready to close
\item Server sends FinBit and can no longer send data (FinBit = 1, Seq = y)
\item Client Acks Fin Bit Packet
\item Connection is closed
\end{enumerate}
\end{itemize}

\subsubsection{Congestion Control Principles}
\begin{itemize}
\item informally:  "too many sources sending too much data too fast for network to handle"
\item Different from flow control
\item Can lead to lost packets and long delays
\end{itemize}

\subsubsection{TCP Congestion Control}
\begin{itemize}
\item sender increases transmission rate (window size), probing for usable bandwidth, until loss occurs
\item \textbf{Additive Increase} : increase cwnd by 1 MSS every RTT until loss detected
\item \textbf{Multiplicative Decrease} : cut cwnd in half after loss
\item Sender Limits transmission 
$$ \text{LastByteSent - LastByteAcked} \leq \text{cwnd}$$
$$ \text{send rate} \approx \frac{cwnd}{RTT} \text{Bytes/sec} $$
\item TCP slow start is when cwnd is doubled every RTT. The initial rate is slow, but ramps up exponetially fast
\item Loss is indicated by 3 duplicate acks 
\item TCP RENO cuts window in half and the grows linearly
\item TCP Tahoe resets cwnd to 1
\end{itemize}

\subsubsection{TCP Throught}
\begin{itemize}
\item TCP Throughput = \( \frac{3}{4} \frac{W}{RTT} \) bytes/sec
\item TCP Throughput = \(\frac{1.22 \times MSS}{RTT \sqrt{L}}\)
\end{itemize}

\subsubsection{TCP Fairness}
\begin{itemize}
\item if K TCP sessions share same bottleneck link of bandwidth R, each should have average rate of R/K
\end{itemize}

\subsubsection{Explicit Congestion Notification}
\begin{itemize}
\item two bits in IP header (ToS field) marked by network router to indicate congestion
\item receiver (seeing congestion indication in IP datagram) ) sets ECE bit on receiver-to-sender ACK segment to notify sender of congestion
\end{itemize}



\end{document}





