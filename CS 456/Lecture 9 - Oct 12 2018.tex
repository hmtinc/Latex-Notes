%Notes by Harsh Mistry 
%CS 456/656
%Based on Template From  https://www.cs.cmu.edu/~ggordon/10725-F12/template.tex

\documentclass[twoside]{article}
\setlength{\oddsidemargin}{0.25 in}
\setlength{\evensidemargin}{-0.25 in}
\setlength{\topmargin}{-0.6 in}
\setlength{\textwidth}{6.5 in}
\setlength{\textheight}{8.5 in}
\setlength{\headsep}{0.75 in}
\setlength{\parindent}{0 in}
\setlength{\parskip}{0.1 in}
\usepackage{amsmath,amsfonts,graphicx}
\newcounter{lecnum}
\renewcommand{\thepage}{\thelecnum-\arabic{page}}
\renewcommand{\thesection}{\thelecnum.\arabic{section}}
\renewcommand{\theequation}{\thelecnum.\arabic{equation}}
\renewcommand{\thefigure}{\thelecnum.\arabic{figure}}
\renewcommand{\thetable}{\thelecnum.\arabic{table}}
\newcommand{\lecture}[4]{
   \pagestyle{myheadings}
   \thispagestyle{plain}
   \newpage
   \setcounter{lecnum}{#1}
   \setcounter{page}{1}
   
   
%Info Box 
   \begin{center}
   \framebox{
      \vbox{\vspace{2mm}
    \hbox to 6.28in { {\bf CS 456/656 - Computer Networks
	\hfill Fall 2018} }
       \vspace{4mm}
       \hbox to 6.28in { {\Large \hfill Lecture #1: #2  \hfill} }
       \vspace{2mm}
       \hbox to 6.28in { {\it Lecturer: #3 \hfill Notes By: #4} }
      \vspace{2mm}}
   }
   \end{center}
   
   \markboth{Lecture #1: #2}{Lecture #1: #2}



 
}

\renewcommand{\cite}[1]{[#1]}
\def\beginrefs{\begin{list}%
        {[\arabic{equation}]}{\usecounter{equation}
         \setlength{\leftmargin}{2.0truecm}\setlength{\labelsep}{0.4truecm}%
         \setlength{\labelwidth}{1.6truecm}}}
\def\endrefs{\end{list}}
\def\bibentry#1{\item[\hbox{[#1]}]}

\newcommand{\fig}[3]{
			\vspace{#2}
			\begin{center}
			Figure \thelecnum.#1:~#3
			\end{center}
	}

\newtheorem{theorem}{Theorem}[lecnum]
\newtheorem{lemma}[theorem]{Lemma}
\newtheorem{ex}[theorem]{Example}
\newtheorem{proposition}[theorem]{Proposition}
\newtheorem{claim}[theorem]{Claim}
\newtheorem{corollary}[theorem]{Corollary}
\newtheorem{definition}[theorem]{Definition}
\newenvironment{proof}{{\bf Proof:}}{\hfill\rule{2mm}{2mm}}
\newcommand\E{\mathbb{E}}


%Start of Document 
\begin{document}

\lecture{9}{October 12, 2018}{Kamal Zille Huma}{Harsh Mistry}
\section{Client-Server File Distribution Time}
\begin{itemize}
\item \textbf{Server Transmission}: Server must sequentially send N file copies
\item \textbf{Client}: each client must download a file copy
\item Thus, the time to distribute a file of \(F\) bytes to \(N\) clients using the Client-Server method where server upload capacity is \(u_s\) and client download capacities are \(d_i\)
$$ D_{C-S} \geq \max \left(\frac{NF}{u_s}, \frac{F}{d_{\min}}\right)$$
\end{itemize}
\section{P2P}

\begin{itemize}
\item Intermittently connected clients are referred to as peers
\item A server is often used to maintain a list a peers to connect to 
\item \textbf{Server Transmission} A server must at first transmit atleast one copy of a file
\item Each \textbf{client} must download a copy of the file
\item \textbf{Clients} as aggregate must download \(NF\) bits where N is the number of peers and F is the file size
\item Thus, the time to distribute a file of \(F\) bytes to \(N\) clients using the P2P where server upload capacity is \(u_s\) and peer download capacities are \(d_i\)
$$ D_{P2P} \geq \max \left(\frac{F}{u_s}, \frac{F}{d_{\min}}, \frac{NF}{u_s + \Sigma u_i}\right)$$
\item "Tit-for-Tat" is a common P2P setup where you keep a small list of best performing peers
\end{itemize}

\section{CDN}
\begin{itemize}
\item \textbf{Content Distributed Network} (CDN) is a set of servers distributed globally that physically connect multiple data centres through a dedicated link
\item CDN's make content available throughout a wider geographic area. This raises the question of what content should be distributed and where servers should be placed
\end{itemize}

\end{document}





