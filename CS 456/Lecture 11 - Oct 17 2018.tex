%Notes by Harsh Mistry 
%CS 456/656
%Based on Template From  https://www.cs.cmu.edu/~ggordon/10725-F12/template.tex

\documentclass[twoside]{article}
\setlength{\oddsidemargin}{0.25 in}
\setlength{\evensidemargin}{-0.25 in}
\setlength{\topmargin}{-0.6 in}
\setlength{\textwidth}{6.5 in}
\setlength{\textheight}{8.5 in}
\setlength{\headsep}{0.75 in}
\setlength{\parindent}{0 in}
\setlength{\parskip}{0.1 in}
\usepackage{amsmath,amsfonts,graphicx}
\newcounter{lecnum}
\renewcommand{\thepage}{\thelecnum-\arabic{page}}
\renewcommand{\thesection}{\thelecnum.\arabic{section}}
\renewcommand{\theequation}{\thelecnum.\arabic{equation}}
\renewcommand{\thefigure}{\thelecnum.\arabic{figure}}
\renewcommand{\thetable}{\thelecnum.\arabic{table}}
\newcommand{\lecture}[4]{
   \pagestyle{myheadings}
   \thispagestyle{plain}
   \newpage
   \setcounter{lecnum}{#1}
   \setcounter{page}{1}
   
   
%Info Box 
   \begin{center}
   \framebox{
      \vbox{\vspace{2mm}
    \hbox to 6.28in { {\bf CS 456/656 - Computer Networks
	\hfill Fall 2018} }
       \vspace{4mm}
       \hbox to 6.28in { {\Large \hfill Lecture #1: #2  \hfill} }
       \vspace{2mm}
       \hbox to 6.28in { {\it Lecturer: #3 \hfill Notes By: #4} }
      \vspace{2mm}}
   }
   \end{center}
   
   \markboth{Lecture #1: #2}{Lecture #1: #2}



 
}

\renewcommand{\cite}[1]{[#1]}
\def\beginrefs{\begin{list}%
        {[\arabic{equation}]}{\usecounter{equation}
         \setlength{\leftmargin}{2.0truecm}\setlength{\labelsep}{0.4truecm}%
         \setlength{\labelwidth}{1.6truecm}}}
\def\endrefs{\end{list}}
\def\bibentry#1{\item[\hbox{[#1]}]}

\newcommand{\fig}[3]{
			\vspace{#2}
			\begin{center}
			Figure \thelecnum.#1:~#3
			\end{center}
	}

\newtheorem{theorem}{Theorem}[lecnum]
\newtheorem{lemma}[theorem]{Lemma}
\newtheorem{ex}[theorem]{Example}
\newtheorem{proposition}[theorem]{Proposition}
\newtheorem{claim}[theorem]{Claim}
\newtheorem{corollary}[theorem]{Corollary}
\newtheorem{definition}[theorem]{Definition}
\newenvironment{proof}{{\bf Proof:}}{\hfill\rule{2mm}{2mm}}
\newcommand\E{\mathbb{E}}


%Start of Document 
\begin{document}

\lecture{11}{October 17, 2018}{Kamal Zille Huma}{Harsh Mistry}
\section{Transport Layer Services}
\begin{itemize}
\item Transport Layer typically only offers support for UDP and TCP
\item The Transport layers responsibility is to transfer a packet from an application to an application on the end system
\item Transport layer takes the application layer message and encapsulates it within a Transport Layer Segment
\item Most segments will contain 
\begin{itemize}
\item Source port number
\item Destination port number
\item Source IP address 
\item Destination IP address
\item Checksum
\end{itemize}
\item Integrity and Packet-loss are the primary concerns of the transport layer
\end{itemize}

\subsection{RDT (Reliable Data Transport Protocol)}
\begin{itemize}
\item RDT is a base upon which lessons for TCP were learned
\item To check for corruption, a checksum will be appended to the segment
\item RDT also introduces ACK, NAK, and Sequence Number into the segment
\item RDT requires each packet to be confirmed by the server before it sends another packet. This stop-and-wait approach causes RDT to have poor performance. 
\end{itemize}
\subsection{Go-Back-N}
\begin{itemize}
\item Go-Back-N improves upon RDT by introducing Pipe-lining. This allows for \(N\) packets to be sent out in parallel.
\item Go-Back-N keeps track of sent packets in a window of Size \(N\). Packets are still sent in order as the window can not move till the next sequence number goes beyond the window. 
\item If the timer fires, all packets that have been sent but not acknowledged are resent
\item On the server, a counter is maintained to keep track of the expected packet. If the packet sequence number mismatches the expected packet, the server responds with the last received packet.
\end{itemize}

\end{document}





