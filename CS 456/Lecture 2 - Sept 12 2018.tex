%Notes by Harsh Mistry 
%CS 458/658
%Based on Template From  https://www.cs.cmu.edu/~ggordon/10725-F12/template.tex

\documentclass[twoside]{article}
\setlength{\oddsidemargin}{0.25 in}
\setlength{\evensidemargin}{-0.25 in}
\setlength{\topmargin}{-0.6 in}
\setlength{\textwidth}{6.5 in}
\setlength{\textheight}{8.5 in}
\setlength{\headsep}{0.75 in}
\setlength{\parindent}{0 in}
\setlength{\parskip}{0.1 in}
\usepackage{amsmath,amsfonts,graphicx}
\newcounter{lecnum}
\renewcommand{\thepage}{\thelecnum-\arabic{page}}
\renewcommand{\thesection}{\thelecnum.\arabic{section}}
\renewcommand{\theequation}{\thelecnum.\arabic{equation}}
\renewcommand{\thefigure}{\thelecnum.\arabic{figure}}
\renewcommand{\thetable}{\thelecnum.\arabic{table}}
\newcommand{\lecture}[4]{
   \pagestyle{myheadings}
   \thispagestyle{plain}
   \newpage
   \setcounter{lecnum}{#1}
   \setcounter{page}{1}
   
   
%Info Box 
   \begin{center}
   \framebox{
      \vbox{\vspace{2mm}
    \hbox to 6.28in { {\bf CS 456/656 - Computer Networks
	\hfill Fall 2018} }
       \vspace{4mm}
       \hbox to 6.28in { {\Large \hfill Lecture #1: #2  \hfill} }
       \vspace{2mm}
       \hbox to 6.28in { {\it Lecturer: #3 \hfill Notes By: #4} }
      \vspace{2mm}}
   }
   \end{center}
   
   \markboth{Lecture #1: #2}{Lecture #1: #2}



 
}

\renewcommand{\cite}[1]{[#1]}
\def\beginrefs{\begin{list}%
        {[\arabic{equation}]}{\usecounter{equation}
         \setlength{\leftmargin}{2.0truecm}\setlength{\labelsep}{0.4truecm}%
         \setlength{\labelwidth}{1.6truecm}}}
\def\endrefs{\end{list}}
\def\bibentry#1{\item[\hbox{[#1]}]}

\newcommand{\fig}[3]{
			\vspace{#2}
			\begin{center}
			Figure \thelecnum.#1:~#3
			\end{center}
	}

\newtheorem{theorem}{Theorem}[lecnum]
\newtheorem{lemma}[theorem]{Lemma}
\newtheorem{ex}[theorem]{Example}
\newtheorem{proposition}[theorem]{Proposition}
\newtheorem{claim}[theorem]{Claim}
\newtheorem{corollary}[theorem]{Corollary}
\newtheorem{definition}[theorem]{Definition}
\newenvironment{proof}{{\bf Proof:}}{\hfill\rule{2mm}{2mm}}
\newcommand\E{\mathbb{E}}


%Start of Document 
\begin{document}

\lecture{2}{September 12, 2018}{Kamal Zille Huma}{Harsh Mistry}

\section{What's the Internet}
\begin{itemize}
\item A Network Billions of connected computing devices
\item Hosts are the end systems
\item Communication links are the result of a network established by complex physical infrastructure.
\item Packet switches help forwards packets (chunks of data) between systems on the internet
\end{itemize}

\section{Network Structure}

Network structure consist of multiple components
\begin{itemize}
\item \textbf{Network Edges} : A collection of hosts which are clients and servers
\item \textbf{Access Networks and Physical Media} : Various wired and wireless communication links (Provide Access to the Internet)
\item \textbf{Network Core} : A series of interconnected routers and a network of networks
\end{itemize}

\section{Access Networks}

\begin{definition}
\textbf{Capacity Bandwidth} : The physical limit of amount of data carried by a access network
\end{definition}

\begin{definition}
\textbf{Frequency Division Multiplexing (FDM)} - Splits a access connection into multiple parts to allow multiple connection by modulating the frequency of each connection
\end{definition}

\begin{definition}
\textbf{Time Division Multiplexing (TDM)} - Allocates the full capacity of a access network for one connection for a small period of time before swapping for another connection
\end{definition}

\begin{itemize}
\item Digital Subscriber Line (DSL)
\begin{itemize}
\item DSL Utilizes existing telephone lines to central office DSLAM. 
\item Upstream has a theoretical max of 2.5Mbps, but typically is 1Mbps in practice
\item Downstream has a ideal max of 24Mbps, but typically 10Mbps
\end{itemize}
\item Cable Networks
\begin{itemize}
\item Due to the concentric nature of the cable network, there is less noise which allows for more data to be transferred.
\item \textbf{Frequency Division Multiplexing} allows for different channels transmitted in different frequency band. This can be utilized to data in addition to regular tv signals. 
\item The difference from DSL is that the cable is not dedicated. Coxial Cable are shared by multiple homes along the network. 
\item HFC (Hybrid Fiber Coax) can typically achieve an asymmetric 30Mbps downstream and 2Mbps upstream
\end{itemize}
\end{itemize}

\section{Core Networks}

\begin{itemize}
\item A mesh of interconnected routers
\item packet switching: hosts break application layer messages into packets
\item Access networks connect users to the core and the core forwards connections along to their correct locations
\item Two key functions  is routing and fowarding
\end{itemize}

\begin{definition}
\textbf{Routing } - Determines source destintaion route taken by packets
\end{definition}

\begin{definition}
\textbf{Forwarding } - Move packets from routers input to appropriate routers output
\end{definition}

\begin{itemize}
\item There are two types of core switching, Circuit and Packet Switching
\end{itemize}

\subsection{Circuit Switching}
\begin{itemize}
\item end-end resources are allocated to ensure there is a dedicated link between each user and the end point. 
\item circuit segments are unused if let idle.
\item Circuit switching can use FDM or TDM to divide the singular physical connection
\end{itemize}

\subsection{Packet Switching}



\begin{itemize}
\item Hosts break application-layer messages into packets and forward packets from one router to the next. Each of the packets are transmitted at full link capacity
\item \verb|L| = bits per packet and \verb|R| = bits per second
\item Methods for forwarding
\begin{itemize}
\item Store-And-Forward
\begin{itemize}
\item Entire packet must arrive at router before it can be transmitted on the next link
\item Takes $\frac{L}{R}$ seconds to transmit $L$-bit packet into link at $R$ bps
\item end-end delay is $ 2 \frac{L}{R}$ assuming no propagation delay
\end{itemize}
\item Cut-Through 
\begin{itemize}
\item Switch starts forwarding a frame before the whole frame has been received
\item This technique reduces latency through the switch and relies on teh destination devices for error handling
\item Only possible if the speed of the  outgoing network is equal to /greater than the incoming network
\end{itemize}
\end{itemize}
\item Probability of success of a connection on a packet switch network can be modelled via a binomial distribution
\begin{itemize}
\item $$p(X=r) = {n \choose r } p^r q^r$$
\item $r$ = number of failures
\end{itemize}
\end{itemize}

\begin{definition}
\textbf{Queuing and Loss} - If arrival rate to link exceed transmission rate  of link for a period of time 
\begin{itemize}
\item Packets will queue, wait to be transmitted on link
\item packets can be dropped if memory fills up
\end{itemize}
\end{definition}

\subsection{Packet Delays}

There are four sources of packet delays
\begin{itemize}
\item $d_{proc}$ - Nodal processing delay
\begin{itemize}
\item Occurs on the core and is the result of validating data and determining output link. 
\item Typically is in the milliseconds
\end{itemize}
\item $d_{queue}$ - Queuing delay
\begin{itemize}
\item Time spent waiting at the output link for transmission. 
\item Wait times depend on congestion level of router
\end{itemize}
\item $d_{trans}$ - Transmission Delay
\begin{itemize}
\item  Amount of time required for the router to push out the packet
\item $d_{trans} = \frac{L}{R}$ 
\end{itemize}
\item $d_{prop}$ - Propagation Delay. 
\begin{itemize}
\item Time it takes a bit to propagate from one router to the next.
\item Typically insignificant and irrelevant as the propagation speed is $(\approx 2 \times 10^8  \text{m/sec})$
\item $d_{prop} = \frac{d}{s}$
\end{itemize}
\end{itemize}

\begin{definition}
\textbf{Throughput} : rate (bits/time unit) at which bits transferred between sender/receiver 
\begin{itemize}
\item \textbf{instantaneous: } rate at given point in time
\item \textbf{average: } rate over longer period of time
\end{itemize}
\end{definition}

\subsection{Encapsulation}

On most modern systems there is a series of information added onto a packet to help direct data and abstract network control.
\begin{itemize}
\item $M$ - \textbf{Application} Layer provides the message
\item $H_t$ - \textbf{Transport} Layer adds IP address
\item $H_n$ - \textbf{Network } Layer is a physical hardware layer that appends its own identifiable information (Black box for now) 
\item $H_i$ - \textbf{Link} Layer adds system MAC. address (Media Access Control Address)
\item \textbf{Physical} Layer adds no additional data, but is responsible for physically sending off data across a network. 
\end{itemize}



\end{document}





