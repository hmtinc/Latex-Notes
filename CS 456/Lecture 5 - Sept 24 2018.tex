%Notes by Harsh Mistry 
%CS 456/656
%Based on Template From  https://www.cs.cmu.edu/~ggordon/10725-F12/template.tex

\documentclass[twoside]{article}
\setlength{\oddsidemargin}{0.25 in}
\setlength{\evensidemargin}{-0.25 in}
\setlength{\topmargin}{-0.6 in}
\setlength{\textwidth}{6.5 in}
\setlength{\textheight}{8.5 in}
\setlength{\headsep}{0.75 in}
\setlength{\parindent}{0 in}
\setlength{\parskip}{0.1 in}
\usepackage{amsmath,amsfonts,graphicx}
\newcounter{lecnum}
\renewcommand{\thepage}{\thelecnum-\arabic{page}}
\renewcommand{\thesection}{\thelecnum.\arabic{section}}
\renewcommand{\theequation}{\thelecnum.\arabic{equation}}
\renewcommand{\thefigure}{\thelecnum.\arabic{figure}}
\renewcommand{\thetable}{\thelecnum.\arabic{table}}
\newcommand{\lecture}[4]{
   \pagestyle{myheadings}
   \thispagestyle{plain}
   \newpage
   \setcounter{lecnum}{#1}
   \setcounter{page}{1}
   
   
%Info Box 
   \begin{center}
   \framebox{
      \vbox{\vspace{2mm}
    \hbox to 6.28in { {\bf CS 456/656 - Computer Networks
	\hfill Fall 2018} }
       \vspace{4mm}
       \hbox to 6.28in { {\Large \hfill Lecture #1: #2  \hfill} }
       \vspace{2mm}
       \hbox to 6.28in { {\it Lecturer: #3 \hfill Notes By: #4} }
      \vspace{2mm}}
   }
   \end{center}
   
   \markboth{Lecture #1: #2}{Lecture #1: #2}



 
}

\renewcommand{\cite}[1]{[#1]}
\def\beginrefs{\begin{list}%
        {[\arabic{equation}]}{\usecounter{equation}
         \setlength{\leftmargin}{2.0truecm}\setlength{\labelsep}{0.4truecm}%
         \setlength{\labelwidth}{1.6truecm}}}
\def\endrefs{\end{list}}
\def\bibentry#1{\item[\hbox{[#1]}]}

\newcommand{\fig}[3]{
			\vspace{#2}
			\begin{center}
			Figure \thelecnum.#1:~#3
			\end{center}
	}

\newtheorem{theorem}{Theorem}[lecnum]
\newtheorem{lemma}[theorem]{Lemma}
\newtheorem{ex}[theorem]{Example}
\newtheorem{proposition}[theorem]{Proposition}
\newtheorem{claim}[theorem]{Claim}
\newtheorem{corollary}[theorem]{Corollary}
\newtheorem{definition}[theorem]{Definition}
\newenvironment{proof}{{\bf Proof:}}{\hfill\rule{2mm}{2mm}}
\newcommand\E{\mathbb{E}}


%Start of Document 
\begin{document}

\lecture{5}{September 24, 2018}{Kamal Zille Huma}{Harsh Mistry}

\section{HTTP}
\begin{itemize}
\item HTTP (Hyper Text Transfer Protocol) provides the ability to point elsewhere from text. It is an application later protocol and it how two applications communicate. 
\item HTTP involves requests and responses
\item HTTP/1.0 required connections to be recreated consistently  for each request
\item A request message is in ASCII format
\begin{itemize}
\item Req Line [GET/POST/PUT]
\item Header
\begin{itemize}
\item HOST
\item User-Agent
\item Accept
\item Accept-Language
\item Accept-Encoding
\item etc \(\ldots\)
\end{itemize}
\end{itemize}
\item A response is also in ASCII format
\begin{itemize}
\item Status Line
\item Date
\item Server 
\item Last Modified
\item Content Length 
\item Keep Active 
\item etc \(\ldots\)
\end{itemize}
\item Status Codes 
\begin{itemize}
\item 1xx = not used
\item 2xx = successful 
\item 3xx = redirections
\item 4xx = client error 
\item 5xx = server error
\end{itemize}

\end{itemize}

\section{HTTP 1.1}
\begin{itemize}
\item HTTP 1.1 introduces persistent requests to avoid constant handshaking if the HTTP header "keep-alive" is  set under connection
\item If client does not request close, it will timeout after a certain limit defined in the headers
\item Server must implement keep-alive
\item Time-out duration must be negotiated
\item HTTP has no way to keep track of repeated requests, as its stateless. This leads to the classic DDOS actions.
\item HTTP 1.1 uses \textbf{Pipe-lining} which is when multiple requests are issued from the client. This causes the server to queue requests. This does not lead to performance advantages, as head-of-line blocking occurs.
\end{itemize}

\begin{definition}
RTT \(\rightarrow\) Round Trip Time, How long it takes for one cycle of information. 
\end{definition}


\end{document}





