%Notes by Harsh Mistry 
%CS 456/656
%Based on Template From  https://www.cs.cmu.edu/~ggordon/10725-F12/template.tex

\documentclass[twoside]{article}
\setlength{\oddsidemargin}{0.25 in}
\setlength{\evensidemargin}{-0.25 in}
\setlength{\topmargin}{-0.6 in}
\setlength{\textwidth}{6.5 in}
\setlength{\textheight}{8.5 in}
\setlength{\headsep}{0.75 in}
\setlength{\parindent}{0 in}
\setlength{\parskip}{0.1 in}
\usepackage{amsmath,amsfonts,graphicx}
\newcounter{lecnum}
\renewcommand{\thepage}{\thelecnum-\arabic{page}}
\renewcommand{\thesection}{\thelecnum.\arabic{section}}
\renewcommand{\theequation}{\thelecnum.\arabic{equation}}
\renewcommand{\thefigure}{\thelecnum.\arabic{figure}}
\renewcommand{\thetable}{\thelecnum.\arabic{table}}
\newcommand{\lecture}[4]{
   \pagestyle{myheadings}
   \thispagestyle{plain}
   \newpage
   \setcounter{lecnum}{#1}
   \setcounter{page}{1}
   
   
%Info Box 
   \begin{center}
   \framebox{
      \vbox{\vspace{2mm}
    \hbox to 6.28in { {\bf CS 456/656 - Computer Networks
	\hfill Fall 2018} }
       \vspace{4mm}
       \hbox to 6.28in { {\Large \hfill Lecture #1: #2  \hfill} }
       \vspace{2mm}
       \hbox to 6.28in { {\it Lecturer: #3 \hfill Notes By: #4} }
      \vspace{2mm}}
   }
   \end{center}
   
   \markboth{Lecture #1: #2}{Lecture #1: #2}



 
}

\renewcommand{\cite}[1]{[#1]}
\def\beginrefs{\begin{list}%
        {[\arabic{equation}]}{\usecounter{equation}
         \setlength{\leftmargin}{2.0truecm}\setlength{\labelsep}{0.4truecm}%
         \setlength{\labelwidth}{1.6truecm}}}
\def\endrefs{\end{list}}
\def\bibentry#1{\item[\hbox{[#1]}]}

\newcommand{\fig}[3]{
			\vspace{#2}
			\begin{center}
			Figure \thelecnum.#1:~#3
			\end{center}
	}

\newtheorem{theorem}{Theorem}[lecnum]
\newtheorem{lemma}[theorem]{Lemma}
\newtheorem{ex}[theorem]{Example}
\newtheorem{proposition}[theorem]{Proposition}
\newtheorem{claim}[theorem]{Claim}
\newtheorem{corollary}[theorem]{Corollary}
\newtheorem{definition}[theorem]{Definition}
\newenvironment{proof}{{\bf Proof:}}{\hfill\rule{2mm}{2mm}}
\newcommand\E{\mathbb{E}}


%Start of Document 
\begin{document}

\lecture{4}{September 19, 2018}{Kamal Zille Huma}{Harsh Mistry}

\section{TCP v.s. UDP}
\begin{itemize}
\item UDP involves simply sending packets with no handshaking. This requires that the socket be provided with server information each time.
\item TCP involves handshaking and confirms the client has received packets
\item TCP requires 2 sockets, one to enable handshaking and another to send data.
\item When using TCP, the secondary socket does not require any additional server information to send data, as its a dedicated connection upon which the application can push data
\item TCP port for handshaking is typically standardized 
\item Once a connection has been negotiated, a new port is opened for each connection. As a result, TCP requires a lot of overhead.
\end{itemize} 

\section{Application architecture}
There are typically two different types of architectures 
\begin{itemize}
\item Client-Sever
\begin{itemize}
\item Requires a dedicated server which handles all requests and is solely responsible for directing data to the client
\item This raises issues when attempting to scale up, as more data centers with more servers will be required to support more users. This is very cost prohibitive
\end{itemize}
\item Peer-to-Peer (P2P)
\begin{itemize}
\item Peer-to-Peer attempts to eliminate the central server by linking different clients who have files stored locally
\item Unfortunately, as we scale P2P across large number of users and files, a smaller server may be required to keep track of users and ip addresses
\item P2P raises security issues because you must have an open port on your system to send and receive data. 
\item Some smaller P2P systems use \textbf{Service Discovery}, which involves users broadcasting their presence. Once service is discovered, a client-server relationship is established between the different users. 
\end{itemize}
\end{itemize}

\subsection{Needs of the Transport layer}
A network application requires the transport layer to provide :
\begin{itemize}
\item \textbf{Data Integrity} : Applications require \(100\%\) reliable data transfer
\item \textbf{Timing} : Some apps require low delay
\item \textbf{Throughput} : Some applications require a minimum amount of throughput
\item \textbf{Security} : Applications may require the data to secure (i.e encryption, etc)
\end{itemize}




\end{document}





