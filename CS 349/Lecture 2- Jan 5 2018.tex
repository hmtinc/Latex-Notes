%Notes by Harsh Mistry 
%CS 349
%Based on Template From  https://www.cs.cmu.edu/~ggordon/10725-F12/template.tex

\documentclass[twoside]{article}
\setlength{\oddsidemargin}{0.25 in}
\setlength{\evensidemargin}{-0.25 in}
\setlength{\topmargin}{-0.6 in}
\setlength{\textwidth}{6.5 in}
\setlength{\textheight}{8.5 in}
\setlength{\headsep}{0.75 in}
\setlength{\parindent}{0 in}
\setlength{\parskip}{0.1 in}
\usepackage{amsmath,amsfonts,graphicx}
\newcounter{lecnum}
\renewcommand{\thepage}{\thelecnum-\arabic{page}}
\renewcommand{\thesection}{\thelecnum.\arabic{section}}
\renewcommand{\theequation}{\thelecnum.\arabic{equation}}
\renewcommand{\thefigure}{\thelecnum.\arabic{figure}}
\renewcommand{\thetable}{\thelecnum.\arabic{table}}
\newcommand{\lecture}[4]{
   \pagestyle{myheadings}
   \thispagestyle{plain}
   \newpage
   \setcounter{lecnum}{#1}
   \setcounter{page}{1}
   
   
%Info Box 
   \begin{center}
   \framebox{
      \vbox{\vspace{2mm}
    \hbox to 6.28in { {\bf CS 349 - User Interfaces
	\hfill Winter 2018} }
       \vspace{4mm}
       \hbox to 6.28in { {\Large \hfill Lecture #1: #2  \hfill} }
       \vspace{2mm}
       \hbox to 6.28in { {\it Lecturer: #3 \hfill Notes By: #4} }
      \vspace{2mm}}
   }
   \end{center}
   
   \markboth{Lecture #1: #2}{Lecture #1: #2}



 
}

\renewcommand{\cite}[1]{[#1]}
\def\beginrefs{\begin{list}%
        {[\arabic{equation}]}{\usecounter{equation}
         \setlength{\leftmargin}{2.0truecm}\setlength{\labelsep}{0.4truecm}%
         \setlength{\labelwidth}{1.6truecm}}}
\def\endrefs{\end{list}}
\def\bibentry#1{\item[\hbox{[#1]}]}

\newcommand{\fig}[3]{
			\vspace{#2}
			\begin{center}
			Figure \thelecnum.#1:~#3
			\end{center}
	}

\newtheorem{theorem}{Theorem}[lecnum]
\newtheorem{lemma}[theorem]{Lemma}
\newtheorem{ex}[theorem]{Example}
\newtheorem{proposition}[theorem]{Proposition}
\newtheorem{claim}[theorem]{Claim}
\newtheorem{corollary}[theorem]{Corollary}
\newtheorem{definition}[theorem]{Definition}
\newenvironment{proof}{{\bf Proof:}}{\hfill\rule{2mm}{2mm}}
\newcommand\E{\mathbb{E}}


%Start of Document 
\begin{document}

\lecture{2}{January 5, 2018}{Keiko Katsuragawa}{Harsh Mistry}

\section{History}
\subsection{Interface v.s Interaction}
\begin{itemize}
\item Interface refers to what the system presents to the user
\begin{itemize}
\item Its what you can manipulate and what the system uses to present feedback
\end{itemize}
\item Interaction refers to the sequence of actions a person expresses and the corresponding system response
\begin{itemize}
\item it unfolds over time
\end{itemize}
\item Interaction requires an interface to occur
\item To use an interface there must be interaction
\end{itemize}

\subsection{Major paradigms of interactions}
\begin{itemize}
\item Batch Interfaces (1945-1965)
\begin{itemize}
\item Interaction style
\begin{itemize}
\item Set of instructions prepared a priori, fed to computer via ounch cards, paper tape, and magnetic tape
\item Response typically received via paper printout
\item No real interaction possible as system executes instructions
\item Responses received in hours/days
\end{itemize}
\item Users
\begin{itemize}
\item only uses by highly trained individuals
\end{itemize}
\end{itemize}
\item Conversational Interfaces (1965 - 1985+) 
\begin{itemize}
\item Interaction style
\begin{itemize}
\item Users issues commands
\item Feedback can be given during execution
\item Commands need to be learned 
\item Commands are hard to discover
\end{itemize}
\item Users
\begin{itemize}
\item Trained experts
\end{itemize}
\item Advantages 
\begin{itemize}
\item Highly flexible : Can combine commands to create sophisticated sets
\end{itemize}
\item Disadvantages
\begin{itemize}
\item Users need to the understand the computer 
\item I/O is in the system language, not task language
\item Requires \textbf{Recall} rather than \textbf{Recognition}
\end{itemize}
\item Consequences
\begin{itemize}
\item System in control during execution : user cannot refine execution/ make modifications during program execution
\end{itemize}
\end{itemize}

\item Graphical interfaces 
\begin{itemize}
\item Interaction Style 
\begin{itemize}
\item User in control 
\begin{itemize}
\item system waits for input and then responds
\end{itemize}
\item Recognition over Recall
\begin{itemize}
\item enabled discovery of options and experimentation
\end{itemize}
\item Metaphors 
\begin{itemize}
\item make interaction language closer to users own language and the task domain
\end{itemize}
\item GUI Interaction opens interface up to broader audience
\end{itemize}
\end{itemize}
\end{itemize}

\subsection{Visionaries who inspired advances}
\begin{itemize}
\item Vannevar Bush
\begin{itemize}
\item Headed Office of Scientific Research and Development
\item Goal was to augment human intellect 
\item Known for the memex, a device in which stores all forms of information and recall it on demand 
\end{itemize}
\item Douglas Engelbart
\begin{itemize}
\item Lead team at Stanford Research Institute 
\item Invented the mouse 
\item Implemented hypertext
\item Introduced copy/paste
\item His vision included computer supported collaborative work
\end{itemize}
\item Ivan Sutherland
\begin{itemize}
\item Known for the SketchPad (1963)
\item Expanded computer domain to include artists, draftsmen, etc
\item Helped language of interface move substantially closer to task domains
\end{itemize}
\item Alan Kay
\begin{itemize}
\item Pioneering work on 
\begin{itemize}
\item Object-oriented programming
\item Xerox Star : Graphical user interfaces 
\item Dynabook : conceptual basis for laptops and tablet computers
\end{itemize}
\item "The best way to predict the future is to invent it"
\end{itemize}
\end{itemize}

\subsection{Modern and Future Interaction}
\begin{itemize}
\item Gesture interface
\item Voice interface
\item Augmented Reality 
\item Virtual Reality 
\end{itemize}

\end{document}





