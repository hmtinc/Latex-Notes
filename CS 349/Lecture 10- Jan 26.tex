%Notes by Harsh Mistry 
%CS 349
%Based on Template From  https://www.cs.cmu.edu/~ggordon/10725-F12/template.tex

\documentclass[twoside]{article}
\setlength{\oddsidemargin}{0.25 in}
\setlength{\evensidemargin}{-0.25 in}
\setlength{\topmargin}{-0.6 in}
\setlength{\textwidth}{6.5 in}
\setlength{\textheight}{8.5 in}
\setlength{\headsep}{0.75 in}
\setlength{\parindent}{0 in}
\setlength{\parskip}{0.1 in}
\usepackage{amsmath,amsfonts,graphicx}
\newcounter{lecnum}
\renewcommand{\thepage}{\thelecnum-\arabic{page}}
\renewcommand{\thesection}{\thelecnum.\arabic{section}}
\renewcommand{\theequation}{\thelecnum.\arabic{equation}}
\renewcommand{\thefigure}{\thelecnum.\arabic{figure}}
\renewcommand{\thetable}{\thelecnum.\arabic{table}}
\newcommand{\lecture}[4]{
   \pagestyle{myheadings}
   \thispagestyle{plain}
   \newpage
   \setcounter{lecnum}{#1}
   \setcounter{page}{1}
   
   
%Info Box 
   \begin{center}
   \framebox{
      \vbox{\vspace{2mm}
    \hbox to 6.28in { {\bf CS 349 - User Interfaces
	\hfill Winter 2018} }
       \vspace{4mm}
       \hbox to 6.28in { {\Large \hfill Lecture #1: #2  \hfill} }
       \vspace{2mm}
       \hbox to 6.28in { {\it Lecturer: #3 \hfill Notes By: #4} }
      \vspace{2mm}}
   }
   \end{center}
   
   \markboth{Lecture #1: #2}{Lecture #1: #2}



 
}

\renewcommand{\cite}[1]{[#1]}
\def\beginrefs{\begin{list}%
        {[\arabic{equation}]}{\usecounter{equation}
         \setlength{\leftmargin}{2.0truecm}\setlength{\labelsep}{0.4truecm}%
         \setlength{\labelwidth}{1.6truecm}}}
\def\endrefs{\end{list}}
\def\bibentry#1{\item[\hbox{[#1]}]}

\newcommand{\fig}[3]{
			\vspace{#2}
			\begin{center}
			Figure \thelecnum.#1:~#3
			\end{center}
	}
	
\graphicspath{ {images/} }

\newtheorem{theorem}{Theorem}[lecnum]
\newtheorem{lemma}[theorem]{Lemma}
\newtheorem{ex}[theorem]{Example}
\newtheorem{proposition}[theorem]{Proposition}
\newtheorem{claim}[theorem]{Claim}
\newtheorem{corollary}[theorem]{Corollary}
\newtheorem{definition}[theorem]{Definition}
\newenvironment{proof}{{\bf Proof:}}{\hfill\rule{2mm}{2mm}}
\newcommand\E{\mathbb{E}}


%Start of Document 
\begin{document}

\lecture{9}{January 24, 2018}{Keiko Katsuragawa}{Harsh Mistry}

\section{Layout}
\subsection{Responsive v.s. Adaptive }
\begin{itemize}
\item Responsive : universal design reflows spatial layout to fit width
\item Adaptive: switch between optimized spatial layouts to fit devices
\item In practice, the approaches can be combined 
\end{itemize}

\subsection{Dynamic Layout}
\begin{itemize}
\item If a window is resized, we want to 
\begin{enumerate}
\item maximize use of available space for displaying widgets
but we want to do this such that:
\item maintain consistency with spatial layout
\item preserve visual quality of spatial layout
\end{enumerate}
\item Need to dynamically modify the layout:
\begin{itemize}
\item re-allocate space for widgets
\item adjust location and size of widgets
\item perhaps even change visibility, look, and/or feel of widgets
\end{itemize}
\end{itemize}

\subsection{Intrinsic Size Layout}
\begin{itemize}
\item Query each item for its preferred size
\item Grow the widget to perfectly contain each item
\item A bottom-up approach where top-level widget’s size completely dependent on its contained widgets
\end{itemize}

\subsection{Variable Intrinsic Size Layout}
\begin{itemize}
\item Layout determined in two-passes 
\begin{enumerate}
\item Get each child widget’s preferred size (includes recursively
asking all of its children for their preferred size...)
\item Decide on a layout that satisfies everyone’s preferences, then iterate through each child, and set it’s layout (size/position)
\end{enumerate}
\end{itemize}



\end{document}





