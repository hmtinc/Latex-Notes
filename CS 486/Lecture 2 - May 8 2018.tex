%Notes by Harsh Mistry 
%CS 486
%based on Template from : https://www.cs.cmu.edu/~ggordon/10725-F12/template.tex

\documentclass{article}
\setlength{\oddsidemargin}{0.25 in}
\setlength{\evensidemargin}{-0.25 in}
\setlength{\topmargin}{-0.6 in}
\setlength{\textwidth}{6.5 in}
\setlength{\textheight}{8.5 in}
\setlength{\headsep}{0.75 in}
\setlength{\parindent}{0 in}
\setlength{\parskip}{0.1 in}
\usepackage{amsfonts,graphicx, amssymb}
\usepackage[fleqn]{amsmath}
\usepackage{fixltx2e}
\usepackage{color}
\usepackage{hyperref}
\usepackage{tcolorbox}
\usepackage{lipsum}
\usepackage{listings}
\usepackage{scrextend}
\tcbuselibrary{skins,breakable}
\usetikzlibrary{shadings,shadows}
\newcounter{lecnum}
\renewcommand{\thepage}{\thelecnum-\arabic{page}}
\renewcommand{\thesection}{\thelecnum.\arabic{section}}
\renewcommand{\theequation}{\thelecnum.\arabic{equation}}
\renewcommand{\thefigure}{\thelecnum.\arabic{figure}}
\renewcommand{\thetable}{\thelecnum.\arabic{table}}
\newcommand{\lecture}[4]{
   \pagestyle{myheadings}
   \thispagestyle{plain}
   \newpage
   \setcounter{lecnum}{#1}
   \setcounter{page}{1}
   
   
%Info Box 
   \begin{center}
   \framebox{
      \vbox{\vspace{2mm}
    \hbox to 6.28in { {\bf CS 486 - Introduction to Artificial Intelligence
	\hfill Summer 2018} }
       \vspace{4mm}
       \hbox to 6.28in { {\Large \hfill Lecture #1: #2  \hfill} }
       \vspace{2mm}
       \hbox to 6.28in { {\it Lecturer: #3 \hfill Notes By: #4} }
      \vspace{2mm}}
   }
   \end{center}
   
   \markboth{Lecture #1: #2}{Lecture #1: #2}



 
}

\renewcommand{\cite}[1]{[#1]}
\def\beginrefs{\begin{list}%
        {[\arabic{equation}]}{\usecounter{equation}
         \setlength{\leftmargin}{2.0truecm}\setlength{\labelsep}{0.4truecm}%
         \setlength{\labelwidth}{1.6truecm}}}
\def\endrefs{\end{list}}
\def\bibentry#1{\item[\hbox{[#1]}]}

\newcommand{\fig}[3]{
			\vspace{#2}
			\begin{center}
			Figure \thelecnum.#1:~#3
			\end{center}
	}
	
\newcommand{\pipe}{\(\mid\)}
\newcommand{\ctr}{\(\wedge\)}

\newtheorem{theorem}{Theorem}[lecnum]
\newtheorem{lemma}[theorem]{Lemma}
\newtheorem{ex}[theorem]{Example}
\newtheorem{proposition}[theorem]{Proposition}
\newtheorem{claim}[theorem]{Claim}
\newtheorem{corollary}[theorem]{Corollary}
\newtheorem{definition}[theorem]{Definition}
\newenvironment{proof}{{\bf Proof:}}{\hfill\rule{2mm}{2mm}}
\newcommand\E{\mathbb{E}}

%color definitions :
\definecolor{darkred}{rgb}{0.55, 0.0, 0.0}
\definecolor{lightcoral}{rgb}{0.94, 0.5, 0.5}
\definecolor{tomato}{rgb}{1.0, 0.39, 0.28}
\definecolor{lightgray}{rgb}{.9,.9,.9}
\definecolor{darkgray}{rgb}{.4,.4,.4}
\definecolor{purple}{rgb}{0.65, 0.12, 0.82}
\definecolor{lightgreen}{rgb}{0.56, 0.93, 0.56}
\definecolor{darkgreen}{rgb}{0.0, 0.2, 0.13}
\definecolor{limegreen}{rgb}{0.2, 0.8, 0.2}
\definecolor{lightblue}{rgb}{0.68, 0.85, 0.9}
\definecolor{darkblue}{rgb}{0.0, 0.0, 0.55}


%Environments
\newenvironment{exblock}[1]{%
    \tcolorbox[beamer,%
    noparskip,breakable,
    colback=lightgreen,colframe=darkgreen,%
    colbacklower=limegreen!75!lightgreen,%
    title=#1]}%
    {\endtcolorbox}

\newenvironment{ablock}[1]{%
    \tcolorbox[beamer,%
    noparskip,breakable,
    colback=lightcoral,colframe=darkred,%
    colbacklower=tomato!75!lightcoral,%
    title=#1]}%
    {\endtcolorbox}

\newenvironment{cblock}[1]{%
    \tcolorbox[beamer,%
    noparskip,breakable,
    colback=lightblue,colframe=darkblue,%
    colbacklower=darkblue!75!lightblue,%
    title=#1]}%
    {\endtcolorbox}


%Languages
\lstdefinelanguage{JavaScript}{
  keywords={typeof, new, true, false, catch, function, return, null, catch, switch, var, if, in,  fi, while, do, else, case, break, const},
  keywordstyle=\color{blue}\bfseries,
  ndkeywords={class, export, boolean, throw, implements, import, this, struct},
  ndkeywordstyle=\color{darkgray}\bfseries,
  identifierstyle=\color{black},
  sensitive=false,
  comment=[l]{//},
  morecomment=[s]{/*}{*/},
  commentstyle=\color{purple}\ttfamily,
  stringstyle=\color{red}\ttfamily,
  morestring=[b]',
  morestring=[b]"
}

%Listings
\lstset{
   language=JavaScript,
   backgroundcolor=\color{lightgray},
   extendedchars=true,
   basicstyle=\footnotesize\ttfamily,
   showstringspaces=false,
   showspaces=false,
   numbers=left,
   numberstyle=\footnotesize,
   numbersep=9pt,
   tabsize=2,
   breaklines=true,
   showtabs=false,
   captionpos=b
}


%Start of Document 
\begin{document}

\lecture{2}{May 8th, 2018}{Alice Gao}{Harsh Mistry}

\section{What is Artificial Intelligence?}
Artificial intelligence is 
\begin{itemize}
\item Systems that think like humans
\item Systems that think rationally
\item Systems that act like humans
\item Systems that act rationally
\end{itemize}

\subsection{Thinking humanly or The Cognitive Modelling Approach}
\begin{definition}
Thinking humanly is referred to as The Cognitive Modelling Approach
\end{definition}
\begin{definition}
Cognitive science is a field which combines computer science and psychology to develop a tested model of the human brain
\end{definition}


\subsection{Acting Humanly or The Turing Test Approach}
\begin{definition}
The Turing test is a test that determines if a system is able to act humanly
\end{definition}

The Turing Test and the Total Turing Test has given rise to  6 area of AI
\begin{itemize}
\item Understand natural language
\item Store knowledge
\item Able to reason
\item Able to learn and adapt
\end{itemize}

\subsection{Thinking Rationally or The Laws of Thought Approach }
\begin{definition}
Rationality is an abstract "idea" of intelligence, rather than "whatever humans do"
\end{definition}
\newpage
The law of thought approach :
\begin{itemize}
\item Convert everything to logic and derive conclusions from logic 
\item Difficult to take day to day ideas and translate into logic
\item There will be way to many logical statements which would overcomplicate the system and reduce efficiency 
\end{itemize}

\subsection{Acting Rationally or The Rational Agent Approach }

\begin{itemize}
\item System acts to achieve the best expected outcome
\item A rational agent acts to achieve the best (expected) outcome - learn from experience
\end{itemize}

\subsection{Definition for 486}
\begin{itemize}
\item System is intelligent if and only if it acts rationally 
\item Rationality is well defined, thus easier to scientifically study
\end{itemize}



\section{Sensors, Actuators, and Rational Agents}

\subsection{Agents}
\begin{itemize}
\item Agents are entitles that :
\begin{itemize}
\item Interact with the environment.
\item Perceive the environment using sensors. 
\item Act on the environment using actuators.
\end{itemize} 
\end{itemize}

\begin{definition}\textbf{Rational Agent} :
For each possible percept sequence, a rational agent should select an action that is expected to maximize its performance measure, given the evidence provided by the percept sequence and whatever prior knowledge the agent has.
\end{definition}

\subsection{Properties of Task Environments  *}
\begin{itemize}
\item Problems are Task Environments
\item Solutions are Rational Agents
\item Properties of the task environment 
\begin{itemize}
\item Fully observable vs. partially observable
\item Deterministic vs. stochastic
\item Static vs. dynamic
\item Episodic vs. sequential
\item Known vs. unknown
\item Single agent vs. multi-agent
\end{itemize}
\end{itemize}

\subsubsection{Uncertainty}
\begin{itemize}
\item Fully Observable- The agent knows the state of the world
\item Partial Observable - Many states are possible given an observation 
\end{itemize}

\subsubsection{Uncertain dynamics}
\begin{itemize}
\item Deterministic: the next state is completely determined given the current state and the action
\item Stochastic: the current state and an action can lead to multiple possible next states
\end{itemize}

\subsubsection{Changing environment}
\begin{itemize}
\item Static: the environment does not change
\item Dynamic: the environment changes while the agent interacts with it
Ex. autonomous cars, medical diagnosis
\end{itemize}

\subsubsection{Long-term consequences of actions}
\begin{itemize}
\item Episodic: current action does not affect future actions
\item Sequential: current action could affect all future actions
\end{itemize}

\subsubsection{Learning the rules of the environment}
\begin{itemize}
\item Known: the agent knows all the rules of the environment
\item Unknown: the agent does not know all the rules of the environment
\item Known/Unknown are different from Fully/Partially observables
\end{itemize}

\subsubsection{Number of agents}
\begin{itemize}
\item Single agent: the agent assumes that any other agents are part of the environment
\item Multi-agent: the agent explicitly models other agents and reasons strategically about the other agents
\end{itemize}

\end{document}





