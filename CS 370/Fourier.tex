%Notes by Harsh Mistry 
%CS 370
%Based on Template From  https://www.cs.cmu.edu/~ggordon/10725-F12/template.tex

\documentclass[twoside]{article}
\setlength{\oddsidemargin}{0.25 in}
\setlength{\evensidemargin}{-0.25 in}
\setlength{\topmargin}{-0.6 in}
\setlength{\textwidth}{6.5 in}
\setlength{\textheight}{8.5 in}
\setlength{\headsep}{0.75 in}
\setlength{\parindent}{0 in}
\setlength{\parskip}{0.1 in}
\usepackage{amsfonts,graphicx, amssymb}
\usepackage[fleqn]{amsmath}
\usepackage{fixltx2e}
\usepackage{color}
\usepackage{tcolorbox}
\usepackage{lipsum}
\usepackage{listings}
\usepackage{scrextend}
\tcbuselibrary{skins,breakable}
\usetikzlibrary{shadings,shadows}
\newcounter{lecnum}
\renewcommand{\thepage}{\thelecnum-\arabic{page}}
\renewcommand{\thesection}{\thelecnum.\arabic{section}}
\renewcommand{\theequation}{\thelecnum.\arabic{equation}}
\renewcommand{\thefigure}{\thelecnum.\arabic{figure}}
\renewcommand{\thetable}{\thelecnum.\arabic{table}}
\newcommand{\lecture}[4]{
   \pagestyle{myheadings}
   \thispagestyle{plain}
   \newpage
   \setcounter{lecnum}{#1}
   \setcounter{page}{1}
   
   \graphicspath{ {images/} }
   
%Info Box 
   \begin{center}
   \framebox{
      \vbox{\vspace{2mm}
    \hbox to 6.28in { {\bf CS 370 - Numerical Computation
	\hfill Fall 2018} }
       \vspace{4mm}
       \hbox to 6.28in { {\Large \hfill  #2  \hfill} }
       \vspace{2mm}
       \hbox to 6.28in { {\it Lecturer: #3 \hfill Notes By: #4} }
      \vspace{2mm}}
   }
   \end{center}
   
   \markboth{#1: #2}{#1: #2}



 
}

\renewcommand{\cite}[1]{[#1]}
\def\beginrefs{\begin{list}%
        {[\arabic{equation}]}{\usecounter{equation}
         \setlength{\leftmargin}{2.0truecm}\setlength{\labelsep}{0.4truecm}%
         \setlength{\labelwidth}{1.6truecm}}}
\def\endrefs{\end{list}}
\def\bibentry#1{\item[\hbox{[#1]}]}

\newcommand{\fig}[3]{
			\vspace{#2}
			\begin{center}
			Figure \thelecnum.#1:~#3
			\end{center}
	}

\newtheorem{theorem}{Theorem}[lecnum]
\newtheorem{lemma}[theorem]{Lemma}
\newtheorem{ex}[theorem]{Example}
\newtheorem{proposition}[theorem]{Proposition}
\newtheorem{claim}[theorem]{Claim}
\newtheorem{corollary}[theorem]{Corollary}
\newtheorem{definition}[theorem]{Definition}
\newenvironment{proof}{{\bf Proof:}}{\hfill\rule{2mm}{2mm}}
\newcommand\E{\mathbb{E}}

%color definitions :
\definecolor{darkred}{rgb}{0.55, 0.0, 0.0}
\definecolor{lightcoral}{rgb}{0.94, 0.5, 0.5}
\definecolor{tomato}{rgb}{1.0, 0.39, 0.28}
\definecolor{lightgray}{rgb}{.9,.9,.9}
\definecolor{darkgray}{rgb}{.4,.4,.4}
\definecolor{purple}{rgb}{0.65, 0.12, 0.82}
\definecolor{lightgreen}{rgb}{0.56, 0.93, 0.56}
\definecolor{darkgreen}{rgb}{0.0, 0.2, 0.13}
\definecolor{limegreen}{rgb}{0.2, 0.8, 0.2}
\definecolor{lightblue}{rgb}{0.68, 0.85, 0.9}
\definecolor{darkblue}{rgb}{0.0, 0.0, 0.55}


%Environments
\newenvironment{exblock}[1]{%
    \tcolorbox[beamer,%
    noparskip,breakable,
    colback=lightgreen,colframe=darkgreen,%
    colbacklower=limegreen!75!lightgreen,%
    title=#1]}%
    {\endtcolorbox}

\newenvironment{ablock}[1]{%
    \tcolorbox[beamer,%
    noparskip,breakable,
    colback=lightcoral,colframe=darkred,%
    colbacklower=tomato!75!lightcoral,%
    title=#1]}%
    {\endtcolorbox}

\newenvironment{cblock}[1]{%
    \tcolorbox[beamer,%
    noparskip,breakable,
    colback=lightblue,colframe=darkblue,%
    colbacklower=darkblue!75!lightblue,%
    title=#1]}%
    {\endtcolorbox}



%Start of Document 
\begin{document}

\lecture{5}{Fourier Analysis}{Christopher Batty}{Harsh Mistry}
\begin{itemize}
\item Basic Idea
\begin{itemize}
\item Transform some function/data into a form that reveals frequency of information in data
\item Process/analyze it in this frequency domain form, which makes some tasks easier
\item Transform back 
\end{itemize}
\item Original data/function is in "time domain" if \(f\) is a function of time \(f(t)\)
\item Original data/function is in "Spatial domain" if \(f\) is a function of space/position \(f(x)\)
\end{itemize}

\section{Continuous Fourier Series}
\begin{itemize}
\item Consider some periodic function f(t) with period T so, \(f(t  \pm T) = f(t)\)
\item The goal is to represent any \(f(t)\) as an infinite sum of trig functions 
$$ f(t) = a_0 + \sum_{k=1}^\infty a_k \cos (\frac{2 \pi k t}{T}) + \sum_{k=1}^\infty b_k  \sin  (\frac{2 \pi k t}{T})  $$
\(a_k, b_k\) indicate the amplitude for each sinusoid of a specific period \(\frac{T}{k}\) or frequency \(\frac{k}{T}\)\\
Higher Integer k indicates shorter period \& higher wave frequency
\end{itemize}

\subsection{Determine the Coefficients}
\begin{itemize}
\item Assume that the range of t is \(t \in [0, 2 \pi]\) and period T= 2n 
$$ f(t) = a_0 + \sum_{k=1}^\infty a_k \cos (kt) + \sum_{k=1}^\infty b_k  \sin  (kt)  $$
\end{itemize}

\begin{ablock}{Orthogonality Relations}
\begin{itemize}
\item \(\int_0^{2\pi} \cos(kt) \sin (jt) dt = 0  \) for any integers k,j
\item \(\int_0^{2\pi} \cos(kt) \cos (jt) dt = 0  \) for \(k \neq j\)
\item \(\int_0^{2\pi} \sin(kt) \sin (jt) dt = 0  \) for \(k \neq j\)
\item \(\int_0^{2\pi} \cos(kt) dt = 0  \)
\item \(\int_0^{2\pi} \sin(kt) dt = 0  \)
\end{itemize}
\end{ablock}

Using this we can find the coefficients by solving integrals
$$ a_0 = \frac{\int_0^{2\pi} f(t) dt }{2 \pi 2}$$
$$ a_k = \frac{\int_0^{2\pi} f(t) \cos(kt) dt}{\int_0^{2\pi}  \cos^2 (kt) dt}
$$
$$b_k = \frac{\int_0^{2\pi} f(t) \sin(kt) dt}{\int_0^{2\pi}  \sin^2 (kt) dt}$$

\subsection{Fourier Series with Complex Exponentials}
\begin{itemize}
\item the sinusoidal expression can be represented as 
$$ f(t) = \sum_{k = -\infty}^{+\infty} c_k e^{ikt} $$
Where \(c_k\) are complex numbers
\item We can derive a simple conversion 
$$ a_0 = c_0 $$
$$ c_k = \frac{a_k}{2} - \frac{i b_k}{2} $$
$$ c_{-k} = \frac{a_k}{2} + \frac{ib_k}{2} $$
\item Which gives us the relationships 
$$ | a_0 | = |c_0| $$
$$ | c_k | =  |c_{-k}| = \frac{1}{2} \sqrt{a^2_k + b^2_k}$$
\item The orthogonality property is 
$$ \int_0^{2\pi}  e^{ikt} e^{-ilt} dt = \begin{cases}
0; k \neq 1\\ 2 \pi;  k = 1\end{cases} $$
\item Which gives us 
$$ c_k = \frac{1}{2\pi} \int_0^{2\pi} e^{ikt} f(t) dt$$
\end{itemize}

\section{Discrete Fourier Transform}
\begin{itemize}
\item Given N point and period T, assuming N is even the series can be approximated as
$$ f(t) \approx \sum^{N/2}_{-\frac{N}{2} + 1}  c_k e^{\frac{(2 \pi i ) k t}{T}} $$
By plugging in N data points will give N equations with unknown \\
This leads towards the (inverse) Discrete Fourier Transform 
\end{itemize}

\subsection{Inverse Discrete Fourier Transform}
\begin{itemize}
\item Discrete data \(f_n\) is expressed as sum of coefficients \(F_k\) times complex exponentials
$$ f_n = \sum_{k=0}^{N-1} F_k e^{i (\frac{2\pi n k )}{N}} = \sum_{k=0}^{N-1} F_k W^{nk} $$
\item With \(W = e^{\frac{2\pi i }{N}} \) , \(W^k\) = \( e^{\frac{2\pi i k}{N}} \)  which represents the Nth roots of unit
\item Using this we find that the inverse is (DFT) 
$$ F_k = \frac{1}{N} \sum_{n=0}^{N-1} f_n W^{-nk} $$
\end{itemize}

\section{DFT Properties}
\begin{itemize}
\item The sequence \(\{F_k\}\) is doubly infinite and periodic 
\item Conjugate symmetry : if data \(f_n\) is real, \(F_k = \bar{F_{N-k}}\)\\
Hence the \(|F_k|\) are symmetric about \(k = \frac{N}{2}\)
\end{itemize}

\section{Fast Fourier Transform}
If \(N\neq 2^m \), pad initial data with 0's
\begin{enumerate}
\item Split The full DFT into two DFT's of half the length
\item Repeat recursively
\item Finish at the base case : the DFT of individual pairs of numbers
\end{enumerate}

\subsection{Dividing Up}
$$ g_n = \frac{1}{2} (f_n  + f_{n+\frac{N}{2}})$$
$$ h_n = \frac{1}{2}(f_n - f_{n+\frac{N}{2}}) W^{-n}$$
Then \(F_{even} = \) DFT(g) and \(F_{odd}\) = DFT(h)

\end{document}




