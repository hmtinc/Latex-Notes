%Notes by Harsh Mistry 
%CS 246
%based on Template from : https://www.cs.cmu.edu/~ggordon/10725-F12/template.tex

\documentclass{article}
\setlength{\oddsidemargin}{0.25 in}
\setlength{\evensidemargin}{-0.25 in}
\setlength{\topmargin}{-0.6 in}
\setlength{\textwidth}{6.5 in}
\setlength{\textheight}{8.5 in}
\setlength{\headsep}{0.75 in}
\setlength{\parindent}{0 in}
\setlength{\parskip}{0.1 in}
\usepackage{amsfonts,graphicx, amssymb}
\usepackage[fleqn]{amsmath}
\usepackage{fixltx2e}
\usepackage{color}
\usepackage{tcolorbox}
\usepackage{lipsum}
\usepackage{listings}
\usepackage{scrextend}
\tcbuselibrary{skins,breakable}
\usetikzlibrary{shadings,shadows}
\newcounter{lecnum}
\renewcommand{\thepage}{\thelecnum-\arabic{page}}
\renewcommand{\thesection}{\thelecnum.\arabic{section}}
\renewcommand{\theequation}{\thelecnum.\arabic{equation}}
\renewcommand{\thefigure}{\thelecnum.\arabic{figure}}
\renewcommand{\thetable}{\thelecnum.\arabic{table}}
\newcommand{\lecture}[4]{
   \pagestyle{myheadings}
   \thispagestyle{plain}
   \newpage
   \setcounter{lecnum}{#1}
   \setcounter{page}{1}
   
   
%Info Box 
   \begin{center}
   \framebox{
      \vbox{\vspace{2mm}
    \hbox to 6.28in { {\bf CS 246 - Object Oriented Programming  
	\hfill Fall 2016} }
       \vspace{4mm}
       \hbox to 6.28in { {\Large \hfill Lecture #1: #2  \hfill} }
       \vspace{2mm}
       \hbox to 6.28in { {\it Lecturer: #3 \hfill Notes By: #4} }
      \vspace{2mm}}
   }
   \end{center}
   
   \markboth{Lecture #1: #2}{Lecture #1: #2}



 
}

\renewcommand{\cite}[1]{[#1]}
\def\beginrefs{\begin{list}%
        {[\arabic{equation}]}{\usecounter{equation}
         \setlength{\leftmargin}{2.0truecm}\setlength{\labelsep}{0.4truecm}%
         \setlength{\labelwidth}{1.6truecm}}}
\def\endrefs{\end{list}}
\def\bibentry#1{\item[\hbox{[#1]}]}

\newcommand{\fig}[3]{
			\vspace{#2}
			\begin{center}
			Figure \thelecnum.#1:~#3
			\end{center}
	}
	
\newcommand{\pipe}{\(\mid\)}
\newcommand{\ctr}{\(\wedge\)}

\newtheorem{theorem}{Theorem}[lecnum]
\newtheorem{lemma}[theorem]{Lemma}
\newtheorem{ex}[theorem]{Example}
\newtheorem{proposition}[theorem]{Proposition}
\newtheorem{claim}[theorem]{Claim}
\newtheorem{corollary}[theorem]{Corollary}
\newtheorem{definition}[theorem]{Definition}
\newenvironment{proof}{{\bf Proof:}}{\hfill\rule{2mm}{2mm}}
\newcommand\E{\mathbb{E}}

%color definitions :
\definecolor{darkred}{rgb}{0.55, 0.0, 0.0}
\definecolor{lightcoral}{rgb}{0.94, 0.5, 0.5}
\definecolor{tomato}{rgb}{1.0, 0.39, 0.28}
\definecolor{lightgray}{rgb}{.9,.9,.9}
\definecolor{darkgray}{rgb}{.4,.4,.4}
\definecolor{purple}{rgb}{0.65, 0.12, 0.82}
\definecolor{lightgreen}{rgb}{0.56, 0.93, 0.56}
\definecolor{darkgreen}{rgb}{0.0, 0.2, 0.13}
\definecolor{limegreen}{rgb}{0.2, 0.8, 0.2}
\definecolor{lightblue}{rgb}{0.68, 0.85, 0.9}
\definecolor{darkblue}{rgb}{0.0, 0.0, 0.55}


%Environments
\newenvironment{exblock}[1]{%
    \tcolorbox[beamer,%
    noparskip,breakable,
    colback=lightgreen,colframe=darkgreen,%
    colbacklower=limegreen!75!lightgreen,%
    title=#1]}%
    {\endtcolorbox}

\newenvironment{ablock}[1]{%
    \tcolorbox[beamer,%
    noparskip,breakable,
    colback=lightcoral,colframe=darkred,%
    colbacklower=tomato!75!lightcoral,%
    title=#1]}%
    {\endtcolorbox}

\newenvironment{cblock}[1]{%
    \tcolorbox[beamer,%
    noparskip,breakable,
    colback=lightblue,colframe=darkblue,%
    colbacklower=darkblue!75!lightblue,%
    title=#1]}%
    {\endtcolorbox}


%Languages
\lstdefinelanguage{JavaScript}{
  keywords={typeof, new, true, false, catch, function, return, null, catch, switch, var, if, in, while, do, else, case, break},
  keywordstyle=\color{blue}\bfseries,
  ndkeywords={class, export, boolean, throw, implements, import, this},
  ndkeywordstyle=\color{darkgray}\bfseries,
  identifierstyle=\color{black},
  sensitive=false,
  comment=[l]{//},
  morecomment=[s]{/*}{*/},
  commentstyle=\color{purple}\ttfamily,
  stringstyle=\color{red}\ttfamily,
  morestring=[b]',
  morestring=[b]"
}

%Listings
\lstset{
   language=JavaScript,
   backgroundcolor=\color{lightgray},
   extendedchars=true,
   basicstyle=\footnotesize\ttfamily,
   showstringspaces=false,
   showspaces=false,
   numbers=left,
   numberstyle=\footnotesize,
   numbersep=9pt,
   tabsize=2,
   breaklines=true,
   showtabs=false,
   captionpos=b
}


%Start of Document 
\begin{document}

\lecture{1}{September 8, 2016}{Brad Lushman}{Harsh Mistry}


\section{Administrative Information}
\begin{center}
Lecturer : \textbf{Brad Lushman} \\
Office : \textbf{DC 3110} \\
Note : \textbf{Linux is required in this course} 
\end{center}

\subsection{Grading}
\begin{itemize}
\item Assignments : 40 \% (7\% , 7\% , 7\% , 7\% , 12\%)
\item Midterm : 20 \%
\item Final : 40 \% 
\end{itemize}

\subsection{Options}
\begin{itemize}
\item Work in the Lab 
\item Install linux on yoru personal computer 
\item SSH into the school machines 
\item Cygwin - linux environment for windows
\end{itemize}

\textbf{Also : } Install a XWindows Server. 

\subsection{Modules}
\begin{enumerate}
\item Linux Shell (2 Weeks)
\item C++ (10 Weeks)
\item Tools (Throughout Term)
\item Software Engineering (Throughout Term)
\end{enumerate}

\section{Module 1 : Linux Shell}

\begin{definition}
Shell - An interface that allows the user to interact with the operating system. 
\end{definition}

\begin{ablock}{Note}
Make sure you are running the correct shell. Confirm by typing \verb|echo $0|. Result should be Bash
\end{ablock}

\subsection{Linux File System Commands}

\begin{itemize}
\item Cat - Displays the contents of a file
\begin{itemize}
\item Example Usage : Cat \verb|File Path|
\end{itemize} 
\end{itemize}

\begin{exblock}{Additional Information}
A directory is a special type of file.
\end{exblock}

\begin{itemize}
\item \(\wedge\) C (Control + C) : Stop application 
\item ls : List the files in the current directory 
\item ls -a : List all files in the current directory 
\item pwd : Prints the current directory 
\item \(\wedge\) D (Control + D) : Sends EOF signal 
\end{itemize}

\begin{exblock}{What happens if you just type cat?}
If you just type Cat, the shell will print everything you type
\end{exblock}

\subsection{Output Reduction}
\begin{itemize}
\item In General : args \(>\) file
\end{itemize}

Executes command args and captures the output in file instead of sending output to screen

\subsection{Input Redirection}

\begin{itemize}
\item In General : args \(<\) file
\end{itemize}

Takes contents from file and uses it as keyboard input for args

\subsection{Differences}
\begin{itemize}
\item cat infile.txt : Parses the fie name infile.text as an argument to cat. Cat opens the file and displays the contents
\item cat \(<\) infile.txt : The shell opens the file and parses contents to cat in place of keyboard input. 
\end{itemize}

\begin{ablock}{You can do both}
\verb|Cat < in.txt > out.txt|
\end{ablock}

\subsection{Basic globbing patterns}
* is a basic globbing pattern.Shell finds all files in the current directory that match the pattern and substitutes them onto the command line

\section{Streams}
Every Process is attached to three streams 
\begin{enumerate}
\item stdin
\item stderr
\item stdout
\end{enumerate}

\subsubsection{By default}
\begin{itemize}
\item stdin = keyboard
\item stderr and stddout = Screen
\end{itemize}

\subsubsection{Redirect}
\begin{itemize}
\item stdin \(<\)
\item stdout \(>\)
\item stderr \(2>\)
\end{itemize}

\subsection{Why is stderr seperate?}
\begin{itemize}
\item So that output and error messages can go to different places
\item So that error messages don't corrupt the formatting 
\item Also, stdout may be buffered 
\end{itemize}

\section{Pipes}

Pipes allow for the output of one program to be used as input (stdin) of another program 

\begin{cblock}{Examples}
How many words occur in the first 20 lines of myfile.txt
\begin{itemize}
\item head -20 file.txt \(\mid\) wc -w
\end{itemize}
Suppose wonder*.txt contains a lisr of words. Output all unique words
\begin{itemize}
\item cat wonder*.txt \(\mid\) sort \(\mid\) uniq
\end{itemize}
\end{cblock}

\subsection{Variables}

\begin{cblock}{Example}
echo "Today is \$ (date) "
\end{cblock}

Shell executes the function within the variable and subsitues he result into the command line. 

\begin{itemize}
\item Shell doesn't parse variables in single quotes
\item Both quotes will suppress globbing
\end{itemize}

\section{Pattern Matching}

\textbf{Syntax}  : \verb| egrep pattern file| \\
\textbf{Action} : Prints every line in file that contains pattern

\begin{ablock}{Patterns}
\begin{itemize}
\item "CS246\pipe \pipe cs246"  : One or the other 
\item \textbf{"(cs\pipe CS)246" : Alternate method}
\item "(c\pipe C)(s\pipe S)246" : Combination of options
\item \textbf{"[cC][sS]246" : Alternate Method}
\item "[...]" : Any character between
\item "[\ctr ...]" : Any character except what is inside
\item " ?" : 0 or 1 occurrence of previous character 
\item " *" : 0 or more occurrence of previous character
\item "." : Any one character 
\item ".*" : Anything
\item "+" : 1 or more occurrence
\item "\ctr \$" : Beginning and End of the line   
\end{itemize}
\end{ablock} 

\section{Permissions}

\begin{exblock}{Additional Information}
\verb|ls -l| gives a long form listing of files.
\end{exblock}

\subsection{Groups and Types}
\subsubsection{Groups}
\begin{itemize}
\item A user can belong to one or more groups
\item A file can be associated with one group
\end{itemize}

\subsubsection{Types}
A file can be a directory or ordinary file

\subsection{Changing Permissions}
\textbf{Syntax} : \verb|chmod mode file|

\subsection{Mode}

The mode consists of the user type, operator, and permissions.

\subsubsection{User Types}
\begin{itemize}
\item u : User
\item g : Group
\item o : Other
\item a : All
\end{itemize}

\subsubsection{Operator}
\begin{itemize}
\item + : Add
\item - : Subtract
\item = : Set Exactly
\end{itemize}

\subsubsection{Permissions Values}
\begin{itemize}
\item r : Read
\item w : Write 
\item x : Execute
\end{itemize}


\end{document}
