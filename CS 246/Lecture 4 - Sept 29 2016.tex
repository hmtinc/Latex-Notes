%Notes by Harsh Mistry 
%CS 246
%based on Template from : https://www.cs.cmu.edu/~ggordon/10725-F12/template.tex

\documentclass{article}
\setlength{\oddsidemargin}{0.25 in}
\setlength{\evensidemargin}{-0.25 in}
\setlength{\topmargin}{-0.6 in}
\setlength{\textwidth}{6.5 in}
\setlength{\textheight}{8.5 in}
\setlength{\headsep}{0.75 in}
\setlength{\parindent}{0 in}
\setlength{\parskip}{0.1 in}
\usepackage{amsfonts,graphicx, amssymb}
\usepackage[fleqn]{amsmath}
\usepackage{fixltx2e}
\usepackage{color}
\usepackage{tcolorbox}
\usepackage{lipsum}
\usepackage{listings}
\usepackage{scrextend}
\tcbuselibrary{skins,breakable}
\usetikzlibrary{shadings,shadows}
\newcounter{lecnum}
\renewcommand{\thepage}{\thelecnum-\arabic{page}}
\renewcommand{\thesection}{\thelecnum.\arabic{section}}
\renewcommand{\theequation}{\thelecnum.\arabic{equation}}
\renewcommand{\thefigure}{\thelecnum.\arabic{figure}}
\renewcommand{\thetable}{\thelecnum.\arabic{table}}
\newcommand{\lecture}[4]{
   \pagestyle{myheadings}
   \thispagestyle{plain}
   \newpage
   \setcounter{lecnum}{#1}
   \setcounter{page}{1}
   
   
%Info Box 
   \begin{center}
   \framebox{
      \vbox{\vspace{2mm}
    \hbox to 6.28in { {\bf CS 246 - Object Oriented Programming  
	\hfill Fall 2016} }
       \vspace{4mm}
       \hbox to 6.28in { {\Large \hfill Lecture #1: #2  \hfill} }
       \vspace{2mm}
       \hbox to 6.28in { {\it Lecturer: #3 \hfill Notes By: #4} }
      \vspace{2mm}}
   }
   \end{center}
   
   \markboth{Lecture #1: #2}{Lecture #1: #2}



 
}

\renewcommand{\cite}[1]{[#1]}
\def\beginrefs{\begin{list}%
        {[\arabic{equation}]}{\usecounter{equation}
         \setlength{\leftmargin}{2.0truecm}\setlength{\labelsep}{0.4truecm}%
         \setlength{\labelwidth}{1.6truecm}}}
\def\endrefs{\end{list}}
\def\bibentry#1{\item[\hbox{[#1]}]}

\newcommand{\fig}[3]{
			\vspace{#2}
			\begin{center}
			Figure \thelecnum.#1:~#3
			\end{center}
	}
	
\newcommand{\pipe}{\(\mid\)}
\newcommand{\ctr}{\(\wedge\)}

\newtheorem{theorem}{Theorem}[lecnum]
\newtheorem{lemma}[theorem]{Lemma}
\newtheorem{ex}[theorem]{Example}
\newtheorem{proposition}[theorem]{Proposition}
\newtheorem{claim}[theorem]{Claim}
\newtheorem{corollary}[theorem]{Corollary}
\newtheorem{definition}[theorem]{Definition}
\newenvironment{proof}{{\bf Proof:}}{\hfill\rule{2mm}{2mm}}
\newcommand\E{\mathbb{E}}

%color definitions :
\definecolor{darkred}{rgb}{0.55, 0.0, 0.0}
\definecolor{lightcoral}{rgb}{0.94, 0.5, 0.5}
\definecolor{tomato}{rgb}{1.0, 0.39, 0.28}
\definecolor{lightgray}{rgb}{.9,.9,.9}
\definecolor{darkgray}{rgb}{.4,.4,.4}
\definecolor{purple}{rgb}{0.65, 0.12, 0.82}
\definecolor{lightgreen}{rgb}{0.56, 0.93, 0.56}
\definecolor{darkgreen}{rgb}{0.0, 0.2, 0.13}
\definecolor{limegreen}{rgb}{0.2, 0.8, 0.2}
\definecolor{lightblue}{rgb}{0.68, 0.85, 0.9}
\definecolor{darkblue}{rgb}{0.0, 0.0, 0.55}


%Environments
\newenvironment{exblock}[1]{%
    \tcolorbox[beamer,%
    noparskip,breakable,
    colback=lightgreen,colframe=darkgreen,%
    colbacklower=limegreen!75!lightgreen,%
    title=#1]}%
    {\endtcolorbox}

\newenvironment{ablock}[1]{%
    \tcolorbox[beamer,%
    noparskip,breakable,
    colback=lightcoral,colframe=darkred,%
    colbacklower=tomato!75!lightcoral,%
    title=#1]}%
    {\endtcolorbox}

\newenvironment{cblock}[1]{%
    \tcolorbox[beamer,%
    noparskip,breakable,
    colback=lightblue,colframe=darkblue,%
    colbacklower=darkblue!75!lightblue,%
    title=#1]}%
    {\endtcolorbox}


%Languages
\lstdefinelanguage{JavaScript}{
  keywords={typeof, new, true, false, catch, function, return, null, catch, switch, var, if, in,  fi, while, do, else, case, break, const},
  keywordstyle=\color{blue}\bfseries,
  ndkeywords={class, export, boolean, throw, implements, import, this, struct},
  ndkeywordstyle=\color{darkgray}\bfseries,
  identifierstyle=\color{black},
  sensitive=false,
  comment=[l]{//},
  morecomment=[s]{/*}{*/},
  commentstyle=\color{purple}\ttfamily,
  stringstyle=\color{red}\ttfamily,
  morestring=[b]',
  morestring=[b]"
}

%Listings
\lstset{
   language=JavaScript,
   backgroundcolor=\color{lightgray},
   extendedchars=true,
   basicstyle=\footnotesize\ttfamily,
   showstringspaces=false,
   showspaces=false,
   numbers=left,
   numberstyle=\footnotesize,
   numbersep=9pt,
   tabsize=2,
   breaklines=true,
   showtabs=false,
   captionpos=b
}


%Start of Document 
\begin{document}

\lecture{4}{September 29, 2016}{Brad Lushman}{Harsh Mistry}

\section{Overloading \(<<\) and \(>>\)}

\begin{ex} - Formatting Output Using \(>> \text{and} <<\) Operators
\begin{lstlisting}
#include <iostream>
using namespace std;

struct Grade {
  int theGrade;
};

ostream &operator<<(ostream &out, const Grade &g) {
  out << g.theGrade << "%";
  return out;
}

istream &operator>>(istream &in, Grade &g) {
  in >> g.theGrade;
  if (g.theGrade < 0) g.theGrade = 0;
  if (g.theGrade > 100) g.theGrade = 100;
  return in;
}

int main () {
  Grade g;
  while (cin >> g) cout << g << endl;
}
\end{lstlisting}
\end{ex}

\section{The Preprocessor}
Transforms the code before the compiler sees it. \# \underline{\hspace{2cm}} is the prepocessor directive. e.g \verb|#include|

\subsection{Define}

Defined constants are useful for conditional compilation

\begin{ex} Using define to make programming from multiple operating systems simpler
\begin{lstlisting}
#define IOS 1
#define BBOS 2
#define OS IOS (or BBOS)

#if OS == IOS 
		short int publickey 
#elif OS == BBOS 
		long long int public key 
#end if
\end{lstlisting}
\end{ex}

\textbf{Special Case : }
\begin{lstlisting}
#if 0 
		...
#endif
\end{lstlisting}

\begin{itemize}
\item Never true and all inner text is suppressed before it gets to teh compiler
\item Useful as a heavy-duty "comment out" to comment out anything including other comments
\end{itemize}

\begin{ex} Define symbols via compiler arguments 

\textbf{Bash : }
\begin{lstlisting}
g++14 -DX=15 define.cc -0 define 
./define
\end{lstlisting}
\textbf{Define.cc : }
\begin{lstlisting}
int main(){
		cout << x << endl
}
\end{lstlisting}
\end{ex}


\begin{itemize}
\item \# ifdef NAME : True is Name has been defined
\item \# ifndef Name : True if Name has not been defined 
\end{itemize}

\begin{ex} Using ifdef and ifndef to define a debug mode

\textbf{debug.cc :}
\begin{lstlisting}
int main(){
		#ifdef DEBUG
				cout << "setting x=1" << endl;
		#endif 
		x=1;
		while (x < 10) {
				++x;
				#ifdef DEBUG 
						cout << "x = " << x << endl;
				#endif 
		}
		cout << x << endl;
}
\end{lstlisting}

\textbf{Enabling Debugging :}
\begin{lstlisting}
g++14 -DDEBUG debug.cc -o debig 
\end{lstlisting}
\end{ex}

\section{Separate Compilation}

Split programs into composable modules, with : 
\begin{itemize}
\item Interface (.h) : type definitions. function prototypes 
\item implementation - full definitions  for all provided functions 
\end{itemize}

\begin{cblock} {Recall}
\begin{itemize}
\item declaration - asserts existence
\item definition - full details and allocates space 
\end{itemize}
\end{cblock}

\begin{ex} Simple example of compilation 

\textbf{vec.h :}
\begin{lstlisting}
struct vec{
		int x, y;
}

Vec operator+(const vec &v1, const vec &v2);
\end{lstlisting}

\textbf{main.cc :}
\begin{lstlisting}
#include "vec.h"

int main(){
		Vec v{1,2}'
		v = v + v:
}
\end{lstlisting}

\textbf{vec.cc}
\begin{lstlisting}

#include "vec.h"

Vec operator+(const vec &v1, const vec &v2){
		// definition goes here 
}

\end{lstlisting}
\end{ex}

\begin{cblock}{Recall :}
An entity can be declared many times, but defined at most once.
\end{cblock}

\subsection{Compiling a binary from multiple source files}
Adding a \verb| -c | to \verb|g++14| compiles the files only and does not generate a executable. By doing this, the compiler will produce an object file(.o). 

Once object files are generated for each source file, you can link the object files to generate a executable. 

\begin{itemize}
\item Compile only : \verb| g++14 -c filename.cc|
\item Link Object Files : \verb| g++14 filename.o filename.o -o binary name|
\end{itemize}

\newpage

\begin{ex} Including Header Files

\textbf{linalg.h :}
\begin{lstlisting}
#include "vec.h"
\end{lstlisting}

\textbf{linalg.cc :}
\begin{lstlisting}
#include "linalg.h"
#include "vec.h"
\end{lstlisting}

\textbf{main.cc :}
\begin{lstlisting}
#include "linalg.h"
#include "vec.h"
\end{lstlisting}

\textbf{\textcolor{red}{This Example will not compile}}
\begin{itemize}
\item main.cc and linalg.cc include vec.h, linalg.h. In addition, linalg.h includes vec.h 
\item This causes 2 copies vec.h to be included, thus struct Vec gets defined twice 
\end{itemize}
\end{ex}

\subsection{Include Guard}
To prevent issues when a header file is included multiple times, format header files to check for previous definition.
\begin{lstlisting}
#ifndef VEC.H
#define VEC.H
		// File Contents
#endif 
\end{lstlisting}

The example above will only define vec.h if it has not previously been defined. After the first definition, the rest of the file is suppressed. 
\textbf{Always ensure your header files have include guards.}

\subsection{Stuff to Never Do !!!!}
\begin{enumerate}
\item \textbf{Never} ever \textbf{\textcolor{red}{EVER}} compile header files \textbf{\textcolor{red}{EVER}}
\item \textbf{Never} include .cc files 
\item \textbf{Never} put \verb|using namespace std;|  in header files 
\begin{itemize}
\item the using directive will be forced upon any client that includes the file
\item Always say \verb|std::cin, std::string, std::istream,| etc in headers
\end{itemize}

\end{enumerate}

\section{Classes}
A big idea of OOP is that you can put objects instead structures
\newpage 
\begin{ex} Function instead of a Structure 
\begin{lstlisting}
struct Student {
		int a, b , c;
		float grade(){
				return a * 0.4 + b * 0.2 + c * 0.4
		}
};

Student s {60,70,80}

cou << s.grade() << endl;
\end{lstlisting}
\end{ex}

\begin{itemize}
\item Class - essentially a structure type that can contain functions. 
\begin{itemize}
\item C++ has a class keyword. \textit{Since this is CS246, we will use it later}
\end{itemize}
\item Object - an instance of a class
\item Function grade - called a member function 
\item Methods take a hidden extra parameter called \textbf{this}
\begin{itemize}
\item pointer to the object on which the method was involved
\item \verb|billy.grade()| is a pointer to the object billy 
\end{itemize}
\end{itemize}

\section{Initializing Objects}

\textbf{In C : } \verb| Student billy {60,70,80};|

The c way is OK, but a better way would be to write a method that does initialization. This method is called a constructor. To create a constructor, simply create a method within the object that shares the same name.

\begin{ex} Basic Constructor Example 
\begin{lstlisting}
struct Student{
		int a, b, c;
		float grade (){ ... }
		Student (int a, int b, int c){
				this->a = a
				this->b = b
				this->c = c
		}
}

Student billy {60, 70, 80}; //better 
Student billy = Student {60, 70, 80} // Also works
\end{lstlisting}

\begin{itemize}
\item If a constructor is defined the values are passed as arguments 
\item If no constructor is defined, the values are used to initialize the individual fields of student.
\end{itemize}
\end{ex}

\textbf{Heap allocation :} \verb|Student *Billy = new student {60,70,80}|

\textbf{Advantages of Constructors :} Default parameters, overloading, and sanity checks

\begin{ex} Constructor Example With Default Values
\begin{lstlisting}
struct Student{
		int a, b, c;
		float grade (){ ... }
		Student (int a = 0, int b = 0, int c = 0){
				this->a = a
				this->b = b
				this->c = c
		}
}

\end{lstlisting}
\end{ex}

\begin{ablock}{Note}
Every class comes with a default constructor, which just default constructs any fields that are objects
\end{ablock}

\begin{ex} Default Constructor Example
\begin{lstlisting}
Vec v; //default constructor (Does nothing in this case)
\end{lstlisting}
\begin{itemize}
\item The default constructor goes away if you define your own
\end{itemize}
\end{ex}

\begin{ex} Proof of Default Constructor 
\begin{lstlisting}
struct v {
		int x, y;
		Vec (int x, int y) { 
				this->x = x;
				this->y = y;
		}
}

Vec v {1,2}; // Works
Vec v; .. // Error , thus there must have been a default constructor in ex 4.9
\end{lstlisting}
\end{ex}

\begin{ex} Constants and References in Structures
\begin{lstlisting}
int z;
Struct MyStruct(){
		const int myConst = 6;
		int &myref = z;
}
\end{lstlisting}
\end{ex}

\begin{exblock}{What happens when a object is created?}
\begin{enumerate}
\item Space is allocated 
\item Fields are constructed 
\item Constructor Body Runs 
\end{enumerate}
\end{exblock}
\newpage
\begin{ex} Initializing Constants using the Member Initialization List (MIL)
\begin{lstlisting}
int z;
Struct MyStruct(){
		const int myConst;
		int a, b, c;
		MyStruct(int a, int b, int c):
			myConst{myConst}, a{a}, b{b}. c{c} {}
}
\end{lstlisting}
\begin{itemize}
\item You can initialize any field using MIL, not just constants and references
\item Fields are initialized in the order in which they are declared in the class, even if the MIL orders them differently. 
\item MIL is sometimes more efficient then setting fields in the body 
\item \textbf{\textcolor{red}{Embrace MIL, or else!!. It is your friend.}}
\end{itemize}
\end{ex}

\begin{ex} Field Initialized Inline and MIL
\begin{lstlisting}
struct Vec{
		int x=0, y=0;
		vec(int x):x{x} {}
};
\end{lstlisting}
\begin{itemize}
\item In the example, MIL always takes precedence 
\end{itemize}
\end{ex}

\subsection{Every Class Comes With}
\begin{enumerate}
\item Default Constructor : Default constructs all fields, but is lost if you define your own
\item Copy Constructor : Just copies all fields
\item Copy Assignment Operator 
\item Destructor 
\item Move Constructor 
\item Move Assignment Operator 
\end{enumerate}

\begin{ex} Writing your own copy constructor 
\begin{lstlisting}
struct Student{
		student (const Student &char):
			a{other, a}, b{other, b}, c{other, c}{}	// Same as Default
};
\end{lstlisting}
\end{ex}

\begin{ex} Example case where default copy constructor does not work
\begin{lstlisting}
struct Node{
		int data
		Node *next;
		Node (int data, Node *next) : data {data}, next{}{}
};
Node *n = node Node{1, New Node {2, new Node {3, nullptr}}};
Node m = *n;
Node *p = new Node(*n)
\end{lstlisting}
\begin{itemize}
\item Since pointers are used, the copy constructor only copies the first node, this is called a shallow copy.
\end{itemize}
\end{ex}

\begin{ex} Deepcopy Copy Constructor 
\begin{lstlisting}
struct Student{
		int data
		Node *next;
		Node (const Node &other) : data {other, data}, next{other.next?new Node(*other.next):nullptr}{}
};
\end{lstlisting}

\end{ex}







\end{document}









