%Notes by Harsh Mistry 
%Stat231
%based on Template from : https://www.cs.cmu.edu/~ggordon/10725-F12/template.tex

\documentclass{article}
\setlength{\oddsidemargin}{0.25 in}
\setlength{\evensidemargin}{-0.25 in}
\setlength{\topmargin}{-0.6 in}
\setlength{\textwidth}{6.5 in}
\setlength{\textheight}{8.5 in}
\setlength{\headsep}{0.75 in}
\setlength{\parindent}{0 in}
\setlength{\parskip}{0.1 in}
\usepackage{amsfonts,graphicx, amssymb}
\usepackage[fleqn]{amsmath}
\usepackage{fixltx2e}
\usepackage{tikz}
\usepackage{color}
\usepackage{tcolorbox}
\usepackage{lipsum}
\usepackage{listings}
\usepackage{scrextend}
\tcbuselibrary{skins,breakable}
\usetikzlibrary{shadings,shadows}
\newcounter{lecnum}
\renewcommand{\thepage}{\thelecnum-\arabic{page}}
\renewcommand{\thesection}{\thelecnum.\arabic{section}}
\renewcommand{\theequation}{\thelecnum.\arabic{equation}}
\renewcommand{\thefigure}{\thelecnum.\arabic{figure}}
\renewcommand{\thetable}{\thelecnum.\arabic{table}}
\newcommand{\lecture}[4]{
   \pagestyle{myheadings}
   \thispagestyle{plain}
   \newpage
   \setcounter{lecnum}{#1}
   \setcounter{page}{1}
   
   
%Info Box 
   \begin{center}
   \framebox{
      \vbox{\vspace{2mm}
    \hbox to 6.28in { {\bf CS 240 - Data Structures and Data Management
	\hfill Spring 2017} }
       \vspace{4mm}
       \hbox to 6.28in { {\Large \hfill Lecture #1: #2  \hfill} }
       \vspace{2mm}
       \hbox to 6.28in { {\it Lecturer: #3 \hfill Notes By: #4} }
      \vspace{2mm}}
   }
   \end{center}
   
   \markboth{Lecture #1: #2}{Lecture #1: #2}



 
}

\renewcommand{\cite}[1]{[#1]}
\def\beginrefs{\begin{list}%
        {[\arabic{equation}]}{\usecounter{equation}
         \setlength{\leftmargin}{2.0truecm}\setlength{\labelsep}{0.4truecm}%
         \setlength{\labelwidth}{1.6truecm}}}
\def\endrefs{\end{list}}
\def\bibentry#1{\item[\hbox{[#1]}]}

\newcommand{\fig}[3]{
			\vspace{#2}
			\begin{center}
			Figure \thelecnum.#1:~#3
			\end{center}
	}
	
\newcommand{\pipe}{\(\mid\)}
\newcommand{\ctr}{\(\wedge\)}

\newtheorem{theorem}{Theorem}[lecnum]
\newtheorem{lemma}[theorem]{Lemma}
\newtheorem{ex}[theorem]{Example}
\newtheorem{proposition}[theorem]{Proposition}
\newtheorem{claim}[theorem]{Claim}
\newtheorem{corollary}[theorem]{Corollary}
\newtheorem{definition}[theorem]{Definition}
\newenvironment{proof}{{\bf Proof:}}{\hfill\rule{2mm}{2mm}}
\newcommand\E{\mathbb{E}}

%color definitions :
\definecolor{darkred}{rgb}{0.55, 0.0, 0.0}
\definecolor{lightcoral}{rgb}{0.94, 0.5, 0.5}
\definecolor{tomato}{rgb}{1.0, 0.39, 0.28}
\definecolor{lightgray}{rgb}{.9,.9,.9}
\definecolor{darkgray}{rgb}{.4,.4,.4}
\definecolor{purple}{rgb}{0.65, 0.12, 0.82}
\definecolor{lightgreen}{rgb}{0.56, 0.93, 0.56}
\definecolor{darkgreen}{rgb}{0.0, 0.2, 0.13}
\definecolor{limegreen}{rgb}{0.2, 0.8, 0.2}
\definecolor{lightblue}{rgb}{0.68, 0.85, 0.9}
\definecolor{darkblue}{rgb}{0.0, 0.0, 0.55}


%Environments
\newenvironment{exblock}[1]{%
    \tcolorbox[beamer,%
    noparskip,breakable,
    colback=lightgreen,colframe=darkgreen,%
    colbacklower=limegreen!75!lightgreen,%
    title=#1]}%
    {\endtcolorbox}

\newenvironment{ablock}[1]{%
    \tcolorbox[beamer,%
    noparskip,breakable,
    colback=lightcoral,colframe=darkred,%
    colbacklower=tomato!75!lightcoral,%
    title=#1]}%
    {\endtcolorbox}

\newenvironment{cblock}[1]{%
    \tcolorbox[beamer,%
    noparskip,breakable,
    colback=lightblue,colframe=darkblue,%
    colbacklower=darkblue!75!lightblue,%
    title=#1]}%
    {\endtcolorbox}


%Languages
\lstdefinelanguage{JavaScript}{
  keywords={typeof, new, true, false, catch, function, return, null, catch, switch, var, if, in, while, do, else, case, break},
  keywordstyle=\color{blue}\bfseries,
  ndkeywords={class, export, boolean, throw, implements, import, this},
  ndkeywordstyle=\color{darkgray}\bfseries,
  identifierstyle=\color{black},
  sensitive=false,
  comment=[l]{//},
  morecomment=[s]{/*}{*/},
  commentstyle=\color{purple}\ttfamily,
  stringstyle=\color{red}\ttfamily,
  morestring=[b]',
  morestring=[b]"
}

%Listings
\lstset{
   language=JavaScript,
   backgroundcolor=\color{lightgray},
   extendedchars=true,
   basicstyle=\footnotesize\ttfamily,
   showstringspaces=false,
   showspaces=false,
   numbers=left,
   numberstyle=\footnotesize,
   numbersep=9pt,
   tabsize=2,
   breaklines=true,
   showtabs=false,
   captionpos=b
}


%Start
\begin{document}

\lecture{2}{May 4th, 2017}{Taylor Smith}{Harsh Mistry}
 
\section{Algorithms and Programs}

\subsection{Terminology}

\begin{definition}
Problem : Given a problem instance, carry out a particular computational task 
\end{definition}

\begin{definition}
Problem Instance : Input for the specified problem 
\end{definition}

\begin{definition}
Problem Solution : Output for the specified problem 
\end{definition}

\begin{definition}
Size of a problem instance : Size(I) is a positive integer which is measure of the size of the instance I
\end{definition}

\begin{definition}
Algorithm : An algorithm is a step-by step process for carrying out a series of computations, given an arbitrary problem instance I. 
\end{definition}

\begin{definition}
Algorithm solving a problem : An algorithm A solves a problem (x) is for every instance I of x, A finds a valid solution for the instance I in finite time
\end{definition}

\begin{definition}
Program : A program is an implementation of an algorithm using a specified computer language 
\end{definition}

\subsection{Order Notation}
\begin{enumerate}
\item O-notation (\(\leq\)) : \(f(n) \in O(g(n))\) if there exists constants \(c > 0\) and \(n_0 > 0\) such that \(0 \leq f(n) \leq c g(n)\) for all \(n \geq n_0\)
\item \(\Omega\)-notation (\(\geq\)) : \(f(n) \Omega(g(n))\) if there exists constants \(c > 0\) and \(n_0 > 0\) such that \(0 \leq cg(n) \leq f(n)\) for all \(n \geq n_0\)
\item \(\theta\)-notation (\(=\)) : \(f(n) \in \theta(g(n))\) if there exists constants \(c_1, c_2 > 0\) and \(n_0 > 0\) such that \(0 \leq c_1 g(n) \leq f(n) \leq c_2 g(n)\) for all \(n \geq n_0\)
\item o-notation  (\(<\)):  \(f(n) \in o(g(n))\) if for all  constants \(c > 0\) and \(n_0 > 0\) such that \(0 \leq f(n) \leq cg(n)\) for all \(n \geq n_0\)
\item \(\omega\)-notation (\(>\)) :  \(f(n) \in \omega(g(n))\) if for all  constants \(c > 0\) and \(n_0 > 0\) such that \(0 \leq cg(n) \leq f(n)\) for all \(n \geq n_0\)
\end{enumerate}

\subsection{Complexity of algorithms}
\begin{enumerate}
\item Average-case complexity of an algorithm : 
\[T^{avg}_A(n)= \frac{1}{\mid \{I : Size(I) = n\}\mid} \sum_{\{I:Size(I)=n\}}T_A(I)\]
\item Worst-case complexity of an algorithm : 
\[T_A(n) = max\{T_A(I) : Size (I) = n\} \]
\end{enumerate}

\subsection{Growth Rates}
\begin{itemize}
\item If \(f(n) \in \theta(g(n))\), then growth rates are the same 
\item If \(f(n) \in o(g(n)),\) then growth rate of f(n) is less than g(n)
\item If \(f(n) \in \omega(g(n))\), then the growth rate for f(n) is greater than the growth rate of g(n).
\item Typically f(n) may be "complicated" and g(n) is chosen to be a very simple function.
\end{itemize}

\subsection{Common Growth Rates}
\begin{itemize}
\item \(\theta(1)\) (Constant Complexity)
\item \(\theta(log n)\) (Logarithmic Complexity)
\item \(\theta(n)\) (Linear Complexity)
\item \(\theta(n log n)\) (linearithmic)
\item \(\theta(n log^k n)\) , for some constant k (quasi-linear) 
\item \(\theta(n^2)\) (Quadratic Complexity)
\item \(\theta(n^3)\) (Cubic Complexity) 
\item \(\theta(2^n)\) (Exponential Complexity)
\end{itemize}
\end{document}
