%Notes by Harsh Mistry 
%CS240
%based on Template from : https://www.cs.cmu.edu/~ggordon/10725-F12/template.tex

\documentclass{article}
\setlength{\oddsidemargin}{0.25 in}
\setlength{\evensidemargin}{-0.25 in}
\setlength{\topmargin}{-0.6 in}
\setlength{\textwidth}{6.5 in}
\setlength{\textheight}{8.5 in}
\setlength{\headsep}{0.75 in}
\setlength{\parindent}{0 in}
\setlength{\parskip}{0.1 in}
\usepackage{amsfonts,graphicx, amssymb}
\usepackage[fleqn]{amsmath}
\usepackage{fixltx2e}
\usepackage{tikz}
\usepackage{color}
\usepackage{tcolorbox}
\usepackage{lipsum}
\usepackage{listings}
\usepackage{scrextend}
\tcbuselibrary{skins,breakable}
\usetikzlibrary{shadings,shadows}
\newcounter{lecnum}
\renewcommand{\thepage}{\thelecnum-\arabic{page}}
\renewcommand{\thesection}{\thelecnum.\arabic{section}}
\renewcommand{\theequation}{\thelecnum.\arabic{equation}}
\renewcommand{\thefigure}{\thelecnum.\arabic{figure}}
\renewcommand{\thetable}{\thelecnum.\arabic{table}}
\newcommand{\lecture}[4]{
   \pagestyle{myheadings}
   \thispagestyle{plain}
   \newpage
   \setcounter{lecnum}{#1}
   \setcounter{page}{1}
   
   
%Info Box 
   \begin{center}
   \framebox{
      \vbox{\vspace{2mm}
    \hbox to 6.28in { {\bf CS 240 - Data Structures and Data Management
	\hfill Spring 2017} }
       \vspace{4mm}
       \hbox to 6.28in { {\Large \hfill Lecture #1: #2  \hfill} }
       \vspace{2mm}
       \hbox to 6.28in { {\it Lecturer: #3 \hfill Notes By: #4} }
      \vspace{2mm}}
   }
   \end{center}
   
   \markboth{Lecture #1: #2}{Lecture #1: #2}



 
}

\renewcommand{\cite}[1]{[#1]}
\def\beginrefs{\begin{list}%
        {[\arabic{equation}]}{\usecounter{equation}
         \setlength{\leftmargin}{2.0truecm}\setlength{\labelsep}{0.4truecm}%
         \setlength{\labelwidth}{1.6truecm}}}
\def\endrefs{\end{list}}
\def\bibentry#1{\item[\hbox{[#1]}]}

\newcommand{\fig}[3]{
			\vspace{#2}
			\begin{center}
			Figure \thelecnum.#1:~#3
			\end{center}
	}
	
\newcommand{\pipe}{\(\mid\)}
\newcommand{\ctr}{\(\wedge\)}

\newtheorem{theorem}{Theorem}[lecnum]
\newtheorem{lemma}[theorem]{Lemma}
\newtheorem{ex}[theorem]{Example}
\newtheorem{proposition}[theorem]{Proposition}
\newtheorem{claim}[theorem]{Claim}
\newtheorem{corollary}[theorem]{Corollary}
\newtheorem{definition}[theorem]{Definition}
\newenvironment{proof}{{\bf Proof:}}{\hfill\rule{2mm}{2mm}}
\newcommand\E{\mathbb{E}}

%color definitions :
\definecolor{darkred}{rgb}{0.55, 0.0, 0.0}
\definecolor{lightcoral}{rgb}{0.94, 0.5, 0.5}
\definecolor{tomato}{rgb}{1.0, 0.39, 0.28}
\definecolor{lightgray}{rgb}{.9,.9,.9}
\definecolor{darkgray}{rgb}{.4,.4,.4}
\definecolor{purple}{rgb}{0.65, 0.12, 0.82}
\definecolor{lightgreen}{rgb}{0.56, 0.93, 0.56}
\definecolor{darkgreen}{rgb}{0.0, 0.2, 0.13}
\definecolor{limegreen}{rgb}{0.2, 0.8, 0.2}
\definecolor{lightblue}{rgb}{0.68, 0.85, 0.9}
\definecolor{darkblue}{rgb}{0.0, 0.0, 0.55}


%Environments
\newenvironment{exblock}[1]{%
    \tcolorbox[beamer,%
    noparskip,breakable,
    colback=lightgreen,colframe=darkgreen,%
    colbacklower=limegreen!75!lightgreen,%
    title=#1]}%
    {\endtcolorbox}

\newenvironment{ablock}[1]{%
    \tcolorbox[beamer,%
    noparskip,breakable,
    colback=lightcoral,colframe=darkred,%
    colbacklower=tomato!75!lightcoral,%
    title=#1]}%
    {\endtcolorbox}

\newenvironment{cblock}[1]{%
    \tcolorbox[beamer,%
    noparskip,breakable,
    colback=lightblue,colframe=darkblue,%
    colbacklower=darkblue!75!lightblue,%
    title=#1]}%
    {\endtcolorbox}


%Languages
\lstdefinelanguage{JavaScript}{
  keywords={typeof, new, true, false, catch, function, return, null, catch, switch, var, if, in, while, do, else, case, break},
  keywordstyle=\color{blue}\bfseries,
  ndkeywords={class, export, boolean, throw, implements, import, this},
  ndkeywordstyle=\color{darkgray}\bfseries,
  identifierstyle=\color{black},
  sensitive=false,
  comment=[l]{//},
  morecomment=[s]{/*}{*/},
  commentstyle=\color{purple}\ttfamily,
  stringstyle=\color{red}\ttfamily,
  morestring=[b]',
  morestring=[b]"
}

%Listings
\lstset{
   language=JavaScript,
   backgroundcolor=\color{lightgray},
   extendedchars=true,
   basicstyle=\footnotesize\ttfamily,
   showstringspaces=false,
   showspaces=false,
   numbers=left,
   numberstyle=\footnotesize,
   numbersep=9pt,
   tabsize=2,
   breaklines=true,
   showtabs=false,
   captionpos=b
}


%Start
\begin{document}

\lecture{12, 13, 14, 15, 16,}{June 13 - June 29, 2017}{Taylor Smith}{Harsh Mistry}

\section{Tries}
\begin{definition}
A Trie (Radix Tree) is a dictionary for binary strings
\begin{itemize}
\item Insert : search for x and suppose we finish at a node v
\item Delete : search for x 
\begin{itemize}
\item If x found an internal flagged node, then unflag node
\item if x found at a leaf, delete leaf and all ancestors until we reach an ancestor that has two children or is flagged. 
\end{itemize}
\end{itemize}
Time complexity of all operations is \(\theta(\mid x \mid) \)
\end{definition}

\subsection*{Compressed Tries (Patricia Tries)}
\begin{itemize}
\item Reduces storage requirement : eliminate unflagged nodes with only one child 
\item Every path of one-child unflagged nodes is compressed to a single edge 
\item Each node stores an index indicating the next bit to be tested during a search 
\item A compressed trie storing n keys always has at most n-1 internal nodes 
\end{itemize}

\begin{itemize}
\item Search : follow the proper path from the root down in the tree to a leaf. If search ends in an unflagged node, its unsuccessful. Otherwise check if the key stored is indeed x
\item Delete :
\begin{itemize}
\item Perform search for x
\item if search ends in a internal node, then unflag with there is 2 children or delete the node and promote the only child 
\item if search ends in a leaf then delete the leaf
\item if its parent is unflagged, then delete the parent, 
\end{itemize}
\item Insert : 
\begin{itemize}
\item Perform Search(x)
\item If the search ends at a leaf L with key y, compare x against y.
\item If y is a prex of x, add a child to y containing x.
\item Else, determine the first index i where they disagree and create a new node N with index i .
\item Insert N along the path from the root to L so that the parent of N has index \(<\) i and one child of N is either L or an existing node on the path
from the root to L that has index \(>\) i .
The other child of N will be a new leaf node containing x.
\item If the search ends at an internal node, we and the key corresponding to
that internal node and proceed in a similar way to the previous case.
\end{itemize}
\end{itemize}

\subsection*{Multiway Tries}
\begin{itemize}
\item To represent Strings over any fixed alphabet \(\sum\)
\item Any node will have at most \(\mid \sum \mid \) children 
\item Special characters are used to represent the end of a path 
\end{itemize}

\section{Hashing}
Direct addressing isn't possible if keys are not integers and storage is very wasteful. 

\begin{definition}
Array T of size M (the hash table). An item with key k is stored in T(h(k)]. search, insert, and delete should all cost 0(1). 
\end{definition}

\subsection*{Collision Resolution}
Even the best hash function may have collisions: 
When we want to insert (k,v) into the table, but T[h(k)] is already occupied. \\
Two basic  strategies to resolve this are : 
\begin{itemize}
\item Allow multiple items at each table location (buckets)
\item Allow each item to go into multiple locations. 
\end{itemize}
\subsection*{Chaining}
Each table entry is a bucket containing 0 or more KVPs. This could be implemented by any dictionary (even another has table). \\
The simplest approach is to use an unsorted linked list in each bucket. This is called collision resolution by chaining. 
\begin{itemize}
\item Search : Look for for key k in the list T[h(k)] : \(\theta(1 + \alpha)\)
\item Insert (k,v) : Add (k,v) to the front of the list at T[h(k)] \(O(1)\)
\item Delete (k)) : Perform a search, then delete from the linked list \(\theta (1  + \alpha)\)
\end{itemize} 

\subsection*{Double Hashing}
Say we have two hash functions \(h_1\) that are independent. 
So, under uniform hashing, we assume the probability that a key k has \(h_1(k) = a\) and \(h_2 (k) = b\) for any particular a and b is 
$$ \frac{1}{M^2}$$
For double hashing, define \(h(k,i) = h_1 (k) + i \cdot h_2(k) mod M \)

\subsection*{Basic Hash Functions}
If all keys are integer (or can be mapped to integers) the following two approaches tend to work well 

\begin{itemize}
\item Division Method : h(k) = k mod M, Given M is prime 
\item Multiplication method : h(k) = \(floor(M(kA - floor(kA)))\)
\end{itemize}

\subsection*{Advantages}
\begin{itemize}
\item O(1) cost, but only on average
\item Flexible load factor parameters
\item Cuckoo hashing achieves O(1) worst case for Search and Delete 
\end{itemize}


\end{document}
