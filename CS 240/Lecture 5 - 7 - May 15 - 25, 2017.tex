%Notes by Harsh Mistry 
%CS240
%based on Template from : https://www.cs.cmu.edu/~ggordon/10725-F12/template.tex

\documentclass{article}
\setlength{\oddsidemargin}{0.25 in}
\setlength{\evensidemargin}{-0.25 in}
\setlength{\topmargin}{-0.6 in}
\setlength{\textwidth}{6.5 in}
\setlength{\textheight}{8.5 in}
\setlength{\headsep}{0.75 in}
\setlength{\parindent}{0 in}
\setlength{\parskip}{0.1 in}
\usepackage{amsfonts,graphicx, amssymb}
\usepackage[fleqn]{amsmath}
\usepackage{fixltx2e}
\usepackage{tikz}
\usepackage{color}
\usepackage{tcolorbox}
\usepackage{lipsum}
\usepackage{listings}
\usepackage{scrextend}
\tcbuselibrary{skins,breakable}
\usetikzlibrary{shadings,shadows}
\newcounter{lecnum}
\renewcommand{\thepage}{\thelecnum-\arabic{page}}
\renewcommand{\thesection}{\thelecnum.\arabic{section}}
\renewcommand{\theequation}{\thelecnum.\arabic{equation}}
\renewcommand{\thefigure}{\thelecnum.\arabic{figure}}
\renewcommand{\thetable}{\thelecnum.\arabic{table}}
\newcommand{\lecture}[4]{
   \pagestyle{myheadings}
   \thispagestyle{plain}
   \newpage
   \setcounter{lecnum}{#1}
   \setcounter{page}{1}
   
   
%Info Box 
   \begin{center}
   \framebox{
      \vbox{\vspace{2mm}
    \hbox to 6.28in { {\bf CS 240 - Data Structures and Data Management
	\hfill Spring 2017} }
       \vspace{4mm}
       \hbox to 6.28in { {\Large \hfill Lecture #1: #2  \hfill} }
       \vspace{2mm}
       \hbox to 6.28in { {\it Lecturer: #3 \hfill Notes By: #4} }
      \vspace{2mm}}
   }
   \end{center}
   
   \markboth{Lecture #1: #2}{Lecture #1: #2}



 
}

\renewcommand{\cite}[1]{[#1]}
\def\beginrefs{\begin{list}%
        {[\arabic{equation}]}{\usecounter{equation}
         \setlength{\leftmargin}{2.0truecm}\setlength{\labelsep}{0.4truecm}%
         \setlength{\labelwidth}{1.6truecm}}}
\def\endrefs{\end{list}}
\def\bibentry#1{\item[\hbox{[#1]}]}

\newcommand{\fig}[3]{
			\vspace{#2}
			\begin{center}
			Figure \thelecnum.#1:~#3
			\end{center}
	}
	
\newcommand{\pipe}{\(\mid\)}
\newcommand{\ctr}{\(\wedge\)}

\newtheorem{theorem}{Theorem}[lecnum]
\newtheorem{lemma}[theorem]{Lemma}
\newtheorem{ex}[theorem]{Example}
\newtheorem{proposition}[theorem]{Proposition}
\newtheorem{claim}[theorem]{Claim}
\newtheorem{corollary}[theorem]{Corollary}
\newtheorem{definition}[theorem]{Definition}
\newenvironment{proof}{{\bf Proof:}}{\hfill\rule{2mm}{2mm}}
\newcommand\E{\mathbb{E}}

%color definitions :
\definecolor{darkred}{rgb}{0.55, 0.0, 0.0}
\definecolor{lightcoral}{rgb}{0.94, 0.5, 0.5}
\definecolor{tomato}{rgb}{1.0, 0.39, 0.28}
\definecolor{lightgray}{rgb}{.9,.9,.9}
\definecolor{darkgray}{rgb}{.4,.4,.4}
\definecolor{purple}{rgb}{0.65, 0.12, 0.82}
\definecolor{lightgreen}{rgb}{0.56, 0.93, 0.56}
\definecolor{darkgreen}{rgb}{0.0, 0.2, 0.13}
\definecolor{limegreen}{rgb}{0.2, 0.8, 0.2}
\definecolor{lightblue}{rgb}{0.68, 0.85, 0.9}
\definecolor{darkblue}{rgb}{0.0, 0.0, 0.55}


%Environments
\newenvironment{exblock}[1]{%
    \tcolorbox[beamer,%
    noparskip,breakable,
    colback=lightgreen,colframe=darkgreen,%
    colbacklower=limegreen!75!lightgreen,%
    title=#1]}%
    {\endtcolorbox}

\newenvironment{ablock}[1]{%
    \tcolorbox[beamer,%
    noparskip,breakable,
    colback=lightcoral,colframe=darkred,%
    colbacklower=tomato!75!lightcoral,%
    title=#1]}%
    {\endtcolorbox}

\newenvironment{cblock}[1]{%
    \tcolorbox[beamer,%
    noparskip,breakable,
    colback=lightblue,colframe=darkblue,%
    colbacklower=darkblue!75!lightblue,%
    title=#1]}%
    {\endtcolorbox}


%Languages
\lstdefinelanguage{JavaScript}{
  keywords={typeof, new, true, false, catch, function, return, null, catch, switch, var, if, in, while, do, else, case, break},
  keywordstyle=\color{blue}\bfseries,
  ndkeywords={class, export, boolean, throw, implements, import, this},
  ndkeywordstyle=\color{darkgray}\bfseries,
  identifierstyle=\color{black},
  sensitive=false,
  comment=[l]{//},
  morecomment=[s]{/*}{*/},
  commentstyle=\color{purple}\ttfamily,
  stringstyle=\color{red}\ttfamily,
  morestring=[b]',
  morestring=[b]"
}

%Listings
\lstset{
   language=JavaScript,
   backgroundcolor=\color{lightgray},
   extendedchars=true,
   basicstyle=\footnotesize\ttfamily,
   showstringspaces=false,
   showspaces=false,
   numbers=left,
   numberstyle=\footnotesize,
   numbersep=9pt,
   tabsize=2,
   breaklines=true,
   showtabs=false,
   captionpos=b
}


%Start
\begin{document}

\lecture{5-7}{May 15 - 25, 2017}{Taylor Smith}{Harsh Mistry}

\section{Sorting and Randomized Algorithms}
 
 \subsection*{Selection vs. Sorting}
The selection problem is: Given an array A of n numbers, and 0 ≤ k < n, find the element in position k of the sorted array.

\textbf{Observation} : the kth largest element is the element at position n − k. Best heap-based algorithm had time cost \(\theta(n _ k \log n)\)\\
For median selection, k =   n  , giving cost \(\theta(n \log n)\). 
This is the same cost as our best sorting algorithms.
 
\subsubsection*{Crucial Subroutines}
quick-select and the related algorithm quick-sort rely on two subroutines:
\begin{itemize}
\item choose-pivot(A): Choose an index i such that A[i] will make a
good pivot (hopefully near the middle of the order).
\begin{lstlisting}
choose-pivot1(A) 
return 0
\end{lstlisting}
\item partition(A, p): Using pivot A[p], rearrange A so that all items \(\leq\) the pivot come first,
followed by the pivot,
followed by all items greater than the pivot.
\begin{lstlisting}
partition(A, p)
A: array of size n, p: integer s.t. 0 <= p <= n
	swap(A[0], A[p])
	i <- 1, j < -n−1 
	 loop
		while i < n and A[i] <= A[0] do
			i <- 1 + 1
		while j >= and A[j] > A[0] do
			j <- j -1
		if j < i then break
		else swap(A[i], A[j])
	end loop
	swap(A[0], A[j])
    return j
\end{lstlisting}
\end{itemize}
\newpage
\subsubsection*{Quick Select Algorithm}
\begin{lstlisting}
quick-select1(A, k )
A: array of size n, k: integers.t. 0 < = k< n
	p <-  choose-pivot1(A) 
	i <- partition(A, p) 
	if i=k then
		return A[i]
	 else if i >k then
		return quick-select1(A[0, 1, . . . , i − 1], k) 
	else if i <k then
		return quick-select1(A[i + 1, i + 2, . . . , n -1 ], k - i -1)
\end{lstlisting}

\subsubsection*{Analysis}
\textbf{Worst-case analysis:} Recursive call could always have size n-1
$$ T(n) = cn + c(n -1) + c(n -2) + \ldots + c \cdot 2 + d \in \theta(n^2) $$
\textbf{Best-case analysis:} First chosen pivot could be the kth element. No recursive calls; total cost is \(\theta(n)\) \\
\newline
\textbf{Average case analysis : } Assume all n! permutations are likely. Average cost of sum for all permutations, divided by n!\\
Define T(n, k) as average cost for selecting kth item from size-n array : 
$$T(n) = {max}_{0 \leq k \leq n} T(n,k) $$
The cost is determined by i, the position of the pivot A[0].\\
For more than half of the n! permutations, \(\frac{n}{4} \leq i \leq \frac{3n}{4}\) . \\
In this case, the recursive call will have length at most \(\frac{3n}{4}\) , for any k.
 The average cost is then given by:
 $$ T(n) \leq \begin{cases} cn + \frac{1}{2} (T(n) + T(3n/4)) \text{ } & n \geq 2 \\
 d, &  n = 1\end{cases}$$
 Rearranging displays that \(T(n) \in O(n)\)

\subsection{Randomized Algorithms}
A randomized algorithm is one which relies on some random numbers in addition to the input.
The cost will depend on the input and the random numbers used.

\subsubsection*{Expected Running Time}
Define T(I,R) as the running time of the randomized algorithm for a particular input I and the sequence of random numbers R.\\
The expected running \(T^{(exp)}(I)\) of a randomized algorithm for a particular input I is the “expected” value for T(I,R):
$$ T^{(exp)}(I) = E[T(I, R)] = \sum_R T(I,R) \cdot Pr[R] $$
The worst case running time is then 
$$ T^{(exp)}(n) = \text{max}_{size(i) = n} T^{(exp)}(I) $$

\subsubsection*{Randomized Quick Select }
\begin{lstlisting}
shuffle(A)
A: array of size n
	for i <- 0 to n − 2 do
		swap ( A[i], A[i + random(n-i)] )
\end{lstlisting}
Expected cost becomes the same as the average cost, which is \(\theta(n)\)

\end{document}
