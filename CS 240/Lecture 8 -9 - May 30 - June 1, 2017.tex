%Notes by Harsh Mistry 
%CS240
%based on Template from : https://www.cs.cmu.edu/~ggordon/10725-F12/template.tex

\documentclass{article}
\setlength{\oddsidemargin}{0.25 in}
\setlength{\evensidemargin}{-0.25 in}
\setlength{\topmargin}{-0.6 in}
\setlength{\textwidth}{6.5 in}
\setlength{\textheight}{8.5 in}
\setlength{\headsep}{0.75 in}
\setlength{\parindent}{0 in}
\setlength{\parskip}{0.1 in}
\usepackage{amsfonts,graphicx, amssymb}
\usepackage[fleqn]{amsmath}
\usepackage{fixltx2e}
\usepackage{tikz}
\usepackage{color}
\usepackage{tcolorbox}
\usepackage{lipsum}
\usepackage{listings}
\usepackage{scrextend}
\tcbuselibrary{skins,breakable}
\usetikzlibrary{shadings,shadows}
\newcounter{lecnum}
\renewcommand{\thepage}{\thelecnum-\arabic{page}}
\renewcommand{\thesection}{\thelecnum.\arabic{section}}
\renewcommand{\theequation}{\thelecnum.\arabic{equation}}
\renewcommand{\thefigure}{\thelecnum.\arabic{figure}}
\renewcommand{\thetable}{\thelecnum.\arabic{table}}
\newcommand{\lecture}[4]{
   \pagestyle{myheadings}
   \thispagestyle{plain}
   \newpage
   \setcounter{lecnum}{#1}
   \setcounter{page}{1}
   
   
%Info Box 
   \begin{center}
   \framebox{
      \vbox{\vspace{2mm}
    \hbox to 6.28in { {\bf CS 240 - Data Structures and Data Management
	\hfill Spring 2017} }
       \vspace{4mm}
       \hbox to 6.28in { {\Large \hfill Lecture #1: #2  \hfill} }
       \vspace{2mm}
       \hbox to 6.28in { {\it Lecturer: #3 \hfill Notes By: #4} }
      \vspace{2mm}}
   }
   \end{center}
   
   \markboth{Lecture #1: #2}{Lecture #1: #2}



 
}

\renewcommand{\cite}[1]{[#1]}
\def\beginrefs{\begin{list}%
        {[\arabic{equation}]}{\usecounter{equation}
         \setlength{\leftmargin}{2.0truecm}\setlength{\labelsep}{0.4truecm}%
         \setlength{\labelwidth}{1.6truecm}}}
\def\endrefs{\end{list}}
\def\bibentry#1{\item[\hbox{[#1]}]}

\newcommand{\fig}[3]{
			\vspace{#2}
			\begin{center}
			Figure \thelecnum.#1:~#3
			\end{center}
	}
	
\newcommand{\pipe}{\(\mid\)}
\newcommand{\ctr}{\(\wedge\)}

\newtheorem{theorem}{Theorem}[lecnum]
\newtheorem{lemma}[theorem]{Lemma}
\newtheorem{ex}[theorem]{Example}
\newtheorem{proposition}[theorem]{Proposition}
\newtheorem{claim}[theorem]{Claim}
\newtheorem{corollary}[theorem]{Corollary}
\newtheorem{definition}[theorem]{Definition}
\newenvironment{proof}{{\bf Proof:}}{\hfill\rule{2mm}{2mm}}
\newcommand\E{\mathbb{E}}

%color definitions :
\definecolor{darkred}{rgb}{0.55, 0.0, 0.0}
\definecolor{lightcoral}{rgb}{0.94, 0.5, 0.5}
\definecolor{tomato}{rgb}{1.0, 0.39, 0.28}
\definecolor{lightgray}{rgb}{.9,.9,.9}
\definecolor{darkgray}{rgb}{.4,.4,.4}
\definecolor{purple}{rgb}{0.65, 0.12, 0.82}
\definecolor{lightgreen}{rgb}{0.56, 0.93, 0.56}
\definecolor{darkgreen}{rgb}{0.0, 0.2, 0.13}
\definecolor{limegreen}{rgb}{0.2, 0.8, 0.2}
\definecolor{lightblue}{rgb}{0.68, 0.85, 0.9}
\definecolor{darkblue}{rgb}{0.0, 0.0, 0.55}


%Environments
\newenvironment{exblock}[1]{%
    \tcolorbox[beamer,%
    noparskip,breakable,
    colback=lightgreen,colframe=darkgreen,%
    colbacklower=limegreen!75!lightgreen,%
    title=#1]}%
    {\endtcolorbox}

\newenvironment{ablock}[1]{%
    \tcolorbox[beamer,%
    noparskip,breakable,
    colback=lightcoral,colframe=darkred,%
    colbacklower=tomato!75!lightcoral,%
    title=#1]}%
    {\endtcolorbox}

\newenvironment{cblock}[1]{%
    \tcolorbox[beamer,%
    noparskip,breakable,
    colback=lightblue,colframe=darkblue,%
    colbacklower=darkblue!75!lightblue,%
    title=#1]}%
    {\endtcolorbox}


%Languages
\lstdefinelanguage{JavaScript}{
  keywords={typeof, new, true, false, catch, function, return, null, catch, switch, var, if, in, while, do, else, case, break},
  keywordstyle=\color{blue}\bfseries,
  ndkeywords={class, export, boolean, throw, implements, import, this},
  ndkeywordstyle=\color{darkgray}\bfseries,
  identifierstyle=\color{black},
  sensitive=false,
  comment=[l]{//},
  morecomment=[s]{/*}{*/},
  commentstyle=\color{purple}\ttfamily,
  stringstyle=\color{red}\ttfamily,
  morestring=[b]',
  morestring=[b]"
}

%Listings
\lstset{
   language=JavaScript,
   backgroundcolor=\color{lightgray},
   extendedchars=true,
   basicstyle=\footnotesize\ttfamily,
   showstringspaces=false,
   showspaces=false,
   numbers=left,
   numberstyle=\footnotesize,
   numbersep=9pt,
   tabsize=2,
   breaklines=true,
   showtabs=false,
   captionpos=b
}


%Start
\begin{document}

\lecture{8,9 }{May 30 - June 1, 2017}{Taylor Smith}{Harsh Mistry}


\section{Dictionary ADT}
A dictionary is a collection of items, each of which contains 
\begin{itemize}
\item A key
\item Some data
\end{itemize}
and is called a key-value pair (KVP). Keys can be compared and are typically unique

\subsection{Operations}
\begin{itemize}
\item Search(k)
\item Insert(k,v)
\item Delete(k)
\item optional : join, isEmpty, size, etc,
\end{itemize}

\subsection{Common Assumptions:}
\begin{itemize}
\item Dictionary has n KVPs
\item Each KVP uses constant space
\item Comparing keys takes constant space
\end{itemize}

\textbf{Unordered array of linked list}
\begin{itemize}
\item Search \(\theta(n)\)
\item Insert \(\theta(1)\)
\item Delete \(\theta(n)\) (need to search)
\end{itemize}

\textbf{Ordered array}
\begin{itemize}
\item Search \(\theta(log n)\)
\item Insert \(\theta(n)\)
\item Delete \(\theta(n)\) 
\end{itemize}

\section{AVL Trees}
Introduced by Adel’son-Vel’ski ̆ı and Landis in 1962, an AVL Tree is a BST with additional structural property : The heights of the left and right subtree differ by at most 1 and the height of an empty tree is defined to be -1.

At each non-empty node we store \(height(R) - height(L) \in \{-1, 0, 1\} \) : 
\begin{itemize}
\item -1 means the tree is left heavy 
\item  0 means the tree is balanced
\item 1 means the tree is right heavy 
\end{itemize}

\section{AVL Insertion}
To perform insert(T,k,v) 
\begin{itemize}
\item First, insert (k,v) into T using usual BST insertion
\item Then, move up the tree from the new leaf, updating balance factors.
\item If the balance factor is −1, 0, or 1, then keep going.
\item If the balance factor is  +2 or -2, then call the fix algorithm to “rebalance” at that node. We are done.
\end{itemize}

\subsection{Rotations}
\begin{lstlisting}
rotate-right(T)
	T: AVL tree
	newroot <- T.left
	T.left <- newroot.right
	newroot.right <- T
	return newroot
\end{lstlisting}

\begin{lstlisting}
rotate-left(T)
	T: AVL tree
	newroot <- T.right
	T.right <- newroot.left
	newroot.left <- T
	return newroot
\end{lstlisting}

\subsection{Fixing a Slightly-Unbalanced AVL Tree}
\textbf{Idea : } Identify one of the previous 4 situations apply rotations
\begin{lstlisting}
Fix(T):
	T: AVL tree with T.balance = 2 || T.balance = -2
	if T.balance = -2 then 
		if T.left.balance = 1 then 
			T.left <- rotate-left(T.left)
		return rotate-right(T)
	else if T.balance = 2 then 
		if T.right.balance = -1 then 
			T.right <- rotate-right(T.right)
		return rotate-left(T)
\end{lstlisting}

\subsection{AVL Tree Operations}
\begin{itemize}
\item Search : costs \(\theta(height)\)
\item Insert : Shown already, total cost \(\theta(height)\) 
\begin{itemize}
\item fix restores the height of the tree  it fixes to what it was
\item so fix will be called at most once.
\end{itemize}
\item Delete : First search, then swap with successor, then move the tree and apply fix (as with insert)
\begin{itemize}
\item fix may be called \(\theta(height)\) times 
\end{itemize}
Total cost is \(\theta(height)\)
\end{itemize}

\subsection{Height of an AVL tree}
Define N(h) to be the least number of nodes in a height-h AVL tree.

One subtree must have height at least \(h -1\), the other at least \(h -2\) : 
$$N(h) = \begin{cases}
1 + N(h -1) + N(h-2),  & h \geq 1\\
1, & h = 0\\
0, & h = -1 \\
\end{cases}$$

\subsection{AVL Tree Analysis}
Easier lower bound on N(h);
$$N(h) > 2N(h-2) > 4N(h-4) > 8N(h-6) > \ldots > 2^i N(h-2i) \geq 2^{[h/2]}$$
Since \(n > 2^{h/2}, h \leq 2 \log n\), 
and thus an AVL tree with n nodes has height \(O(\log n)\). 
Also, \(n \leq 2^{h+1} - 1\), so the hight height is \(\theta(\log n)\)

\(\implies\) search, insert, delete all cost \(\theta(\log n)\)

\end{document}
