%Notes by Harsh Mistry 
%CS240
%based on Template from : https://www.cs.cmu.edu/~ggordon/10725-F12/template.tex

\documentclass{article}
\setlength{\oddsidemargin}{0.25 in}
\setlength{\evensidemargin}{-0.25 in}
\setlength{\topmargin}{-0.6 in}
\setlength{\textwidth}{6.5 in}
\setlength{\textheight}{8.5 in}
\setlength{\headsep}{0.75 in}
\setlength{\parindent}{0 in}
\setlength{\parskip}{0.1 in}
\usepackage{amsfonts,graphicx, amssymb}
\usepackage[fleqn]{amsmath}
\usepackage{fixltx2e}
\usepackage{tikz}
\usepackage{color}
\usepackage{tcolorbox}
\usepackage{lipsum}
\usepackage{listings}
\usepackage{scrextend}
\tcbuselibrary{skins,breakable}
\usetikzlibrary{shadings,shadows}
\newcounter{lecnum}
\renewcommand{\thepage}{\thelecnum-\arabic{page}}
\renewcommand{\thesection}{\thelecnum.\arabic{section}}
\renewcommand{\theequation}{\thelecnum.\arabic{equation}}
\renewcommand{\thefigure}{\thelecnum.\arabic{figure}}
\renewcommand{\thetable}{\thelecnum.\arabic{table}}
\newcommand{\lecture}[4]{
   \pagestyle{myheadings}
   \thispagestyle{plain}
   \newpage
   \setcounter{lecnum}{#1}
   \setcounter{page}{1}
   
   
%Info Box 
   \begin{center}
   \framebox{
      \vbox{\vspace{2mm}
    \hbox to 6.28in { {\bf CS 240 - Data Structures and Data Management
	\hfill Spring 2017} }
       \vspace{4mm}
       \hbox to 6.28in { {\Large \hfill Lecture #1: #2  \hfill} }
       \vspace{2mm}
       \hbox to 6.28in { {\it Lecturer: #3 \hfill Notes By: #4} }
      \vspace{2mm}}
   }
   \end{center}
   
   \markboth{Lecture #1: #2}{Lecture #1: #2}



 
}

\renewcommand{\cite}[1]{[#1]}
\def\beginrefs{\begin{list}%
        {[\arabic{equation}]}{\usecounter{equation}
         \setlength{\leftmargin}{2.0truecm}\setlength{\labelsep}{0.4truecm}%
         \setlength{\labelwidth}{1.6truecm}}}
\def\endrefs{\end{list}}
\def\bibentry#1{\item[\hbox{[#1]}]}

\newcommand{\fig}[3]{
			\vspace{#2}
			\begin{center}
			Figure \thelecnum.#1:~#3
			\end{center}
	}
	
\newcommand{\pipe}{\(\mid\)}
\newcommand{\ctr}{\(\wedge\)}

\newtheorem{theorem}{Theorem}[lecnum]
\newtheorem{lemma}[theorem]{Lemma}
\newtheorem{ex}[theorem]{Example}
\newtheorem{proposition}[theorem]{Proposition}
\newtheorem{claim}[theorem]{Claim}
\newtheorem{corollary}[theorem]{Corollary}
\newtheorem{definition}[theorem]{Definition}
\newenvironment{proof}{{\bf Proof:}}{\hfill\rule{2mm}{2mm}}
\newcommand\E{\mathbb{E}}

%color definitions :
\definecolor{darkred}{rgb}{0.55, 0.0, 0.0}
\definecolor{lightcoral}{rgb}{0.94, 0.5, 0.5}
\definecolor{tomato}{rgb}{1.0, 0.39, 0.28}
\definecolor{lightgray}{rgb}{.9,.9,.9}
\definecolor{darkgray}{rgb}{.4,.4,.4}
\definecolor{purple}{rgb}{0.65, 0.12, 0.82}
\definecolor{lightgreen}{rgb}{0.56, 0.93, 0.56}
\definecolor{darkgreen}{rgb}{0.0, 0.2, 0.13}
\definecolor{limegreen}{rgb}{0.2, 0.8, 0.2}
\definecolor{lightblue}{rgb}{0.68, 0.85, 0.9}
\definecolor{darkblue}{rgb}{0.0, 0.0, 0.55}


%Environments
\newenvironment{exblock}[1]{%
    \tcolorbox[beamer,%
    noparskip,breakable,
    colback=lightgreen,colframe=darkgreen,%
    colbacklower=limegreen!75!lightgreen,%
    title=#1]}%
    {\endtcolorbox}

\newenvironment{ablock}[1]{%
    \tcolorbox[beamer,%
    noparskip,breakable,
    colback=lightcoral,colframe=darkred,%
    colbacklower=tomato!75!lightcoral,%
    title=#1]}%
    {\endtcolorbox}

\newenvironment{cblock}[1]{%
    \tcolorbox[beamer,%
    noparskip,breakable,
    colback=lightblue,colframe=darkblue,%
    colbacklower=darkblue!75!lightblue,%
    title=#1]}%
    {\endtcolorbox}


%Languages
\lstdefinelanguage{JavaScript}{
  keywords={typeof, new, true, false, catch, function, return, null, catch, switch, var, if, in, while, do, else, case, break},
  keywordstyle=\color{blue}\bfseries,
  ndkeywords={class, export, boolean, throw, implements, import, this},
  ndkeywordstyle=\color{darkgray}\bfseries,
  identifierstyle=\color{black},
  sensitive=false,
  comment=[l]{//},
  morecomment=[s]{/*}{*/},
  commentstyle=\color{purple}\ttfamily,
  stringstyle=\color{red}\ttfamily,
  morestring=[b]',
  morestring=[b]"
}

%Listings
\lstset{
   language=JavaScript,
   backgroundcolor=\color{lightgray},
   extendedchars=true,
   basicstyle=\footnotesize\ttfamily,
   showstringspaces=false,
   showspaces=false,
   numbers=left,
   numberstyle=\footnotesize,
   numbersep=9pt,
   tabsize=2,
   breaklines=true,
   showtabs=false,
   captionpos=b
}


%Start
\begin{document}

\lecture{3}{May 9th, 2017}{Taylor Smith}{Harsh Mistry}
 
\section{Growth Rate Affect on Running Time}
\begin{itemize}
\item Constant Complexity : \(T(n) = c , T(2n) = c\)
\item Logarithmic Complexity : \(T(n) = c log n, T(2n) = c \)
\item Linear complexity : \(T(n) = cn, T(2n) = 2T(n) \)
\item \(\theta(n log n)\) : \(T(n) = c n log n, T(2n) = 2T(n) + 2cn\)
\item Quadratic Complexity : \(T(n) = cn^2, T(2n) = 4T(n)\)
\item Cubic Complexity : \(T(n) = cn^3, T(2n) = 8T(n)\)
\item Exponential Complexity : \(T(n) = c2^n , T(2n) = (T(n))^2 / c \)
\end{itemize}

\section{Complexity v.s Running Time}
\begin{itemize}
\item Suppose that algorithms \(A_1\) and \(A_2\) both solve some specified problem
\item Suppose that the complexity of algorithm \(A_1\) is lower than the complexity of algorithm \(A_2\). Then for sufficiently large problem instances, \(A_1\) will run faster \(A_2\). However, for small problem instances, \(A_1\) could be slower than \(A_2\)
\item Now suppose that \(A_1\) and \(A_2\) have the same complexity. Then we cannot determine from this information which of \(A_1\) or \(A_2\) is faster; a more delicate analysis of the algorithm \(A_1\) and \(A_2\) is required.
\end{itemize}

\textbf{Note : } It is important not to try and make comparisons between algorithms using O-notation

\section{Techniques for Order Notation}
Suppose that \(f(n) >  0 \) and \(g(n) > 0\) for all \(n \geq n_0\). Suppose that 
$$L = \lim_{n \rightarrow \infty} \frac{f(n)}{g(n)}$$
Then 
$$f(n) \in \begin{cases}o(g(n)) \\ \theta(g(n))  \\ \omega(g(n))\end{cases}  \begin{matrix}
& \text{ if } L = 0 \\ & \text{ if } 0 < L < \infty \\ & \text{ if } L = \infty 
\end{matrix}$$

\section{Relationships between Order Notations}
\begin{itemize}
\item \(f(n) \in \theta(g(n)) \iff g(n) \in \theta(f(n))\)
\item \(f(n) \in  O(g(n)) \iff g(n) \in \Omega(f(n))\)
\item \(f(n) \in o(g(n)) \iff g(n) \in \omega(f(n))\)
\end{itemize}

\begin{itemize}
\item \(f(n) \in \theta(g(n)) \iff f(n) \in O(g(n)) \text{ and } f(n) \in \Omega(g(n))\)
\item \(f(n) \in o(g(n)) \iff f(n) \in O(g(n)) \)
\item \(f(n) \in o(g(n)) \iff f(n) \notin \Omega(g(n)) \)
\item \(f(n) \in \omega(g(n)) \iff f(n) \in \Omega(g(n)) \)
\item \(f(n) \in \omega(g(n)) \iff f(n) \notin O(g(n)) \)\\
\end{itemize}

\section{Algebra of Order Notations}
\textbf{"Maximum" Rules : } Suppose that \(f(n) > 0 \) and \(g(n) > 0 \) for all \(n \geq n_0\). Then : 
\begin{itemize}
\item \(O(f(n) + g(n)) = O(max\{f(n), g(n)\})\)
\item \(\theta(f(n) + g(n)) = \theta(max\{f(n), g(n)\})\)
\item \(\Omega(f(n) + g(n)) = \Omega(max\{f(n), g(n)\})\)
\end{itemize}

\textbf{Transitivity : } If \(f(n) \in O(g(n)) \text{ and } g(n) \in O(h(n))  \text{ then } f(n) \in O(h(n))\).

\section{Summation Formulae}
\begin{itemize}
\item Arithmetic Sequence 
\[\sum_{i=0}^{n-1} (a + di) = na + \frac{dn(n-1)}{2} \in \theta (n^2) \text{ for } d \neq 0\]
\item Geometric Sequence 
\[\sum_{i=0}^{n-1} a r^i = \begin{cases}a \frac{r^n -1}{r -1} \in \theta(r^n) \\ na \in \theta(n) \\ a \frac{1 - r^n}{1 -r} \in \theta(1)\end{cases} \begin{matrix}
\text { if } r > 1 \\ \text{ if } r = 1 \\ \text{ if } 0 < r < 1 \end{matrix}\]
\item Harmonic Sequence 
\[H_n = \sum_{i = 1}^{n} \frac{1}{i} \in \theta (log n)\]
\end{itemize}
\section{Techniques for Algorithm Analysis }
\begin{itemize}
\item Use \(\theta\)-bounds throughout the analysis and obtain a \(\theta\)-bound for the complexity of the algorithm
\item Prove a O-bound and a matching \(\omega\)-bound separately to get a \(\theta\)-bound.   
\end{itemize}

\section{Techniques for Loop Analysis}
\begin{itemize}
\item Identify elementary operations that require constant time. Denoted \( \theta (1) \) time
\item The complexity of a loop is expressed as the sum of complexities of each iteration of the loop.
\item Analyse independent loops separately, and then add the results (use "maximum rules" and simplify whenever possible).
\item If loops are nested, start with the inner most loop and proceed outwards. In general, this kind of analysis requires evaluation of nested summations.
\end{itemize}

\end{document}
