%Notes by Harsh Mistry 
%CS240
%based on Template from : https://www.cs.cmu.edu/~ggordon/10725-F12/template.tex

\documentclass{article}
\setlength{\oddsidemargin}{0.25 in}
\setlength{\evensidemargin}{-0.25 in}
\setlength{\topmargin}{-0.6 in}
\setlength{\textwidth}{6.5 in}
\setlength{\textheight}{8.5 in}
\setlength{\headsep}{0.75 in}
\setlength{\parindent}{0 in}
\setlength{\parskip}{0.1 in}
\usepackage{amsfonts,graphicx, amssymb}
\usepackage[fleqn]{amsmath}
\usepackage{fixltx2e}
\usepackage{tikz}
\usepackage{color}
\usepackage{tcolorbox}
\usepackage{lipsum}
\usepackage{listings}
\usepackage{scrextend}
\tcbuselibrary{skins,breakable}
\usetikzlibrary{shadings,shadows}
\newcounter{lecnum}
\renewcommand{\thepage}{\thelecnum-\arabic{page}}
\renewcommand{\thesection}{\thelecnum.\arabic{section}}
\renewcommand{\theequation}{\thelecnum.\arabic{equation}}
\renewcommand{\thefigure}{\thelecnum.\arabic{figure}}
\renewcommand{\thetable}{\thelecnum.\arabic{table}}
\newcommand{\lecture}[4]{
   \pagestyle{myheadings}
   \thispagestyle{plain}
   \newpage
   \setcounter{lecnum}{#1}
   \setcounter{page}{1}
   
   
%Info Box 
   \begin{center}
   \framebox{
      \vbox{\vspace{2mm}
    \hbox to 6.28in { {\bf CS 240 - Data Structures and Data Management
	\hfill Spring 2017} }
       \vspace{4mm}
       \hbox to 6.28in { {\Large \hfill Lecture #1: #2  \hfill} }
       \vspace{2mm}
       \hbox to 6.28in { {\it Lecturer: #3 \hfill Notes By: #4} }
      \vspace{2mm}}
   }
   \end{center}
   
   \markboth{Lecture #1: #2}{Lecture #1: #2}



 
}

\renewcommand{\cite}[1]{[#1]}
\def\beginrefs{\begin{list}%
        {[\arabic{equation}]}{\usecounter{equation}
         \setlength{\leftmargin}{2.0truecm}\setlength{\labelsep}{0.4truecm}%
         \setlength{\labelwidth}{1.6truecm}}}
\def\endrefs{\end{list}}
\def\bibentry#1{\item[\hbox{[#1]}]}

\newcommand{\fig}[3]{
			\vspace{#2}
			\begin{center}
			Figure \thelecnum.#1:~#3
			\end{center}
	}
	
\newcommand{\pipe}{\(\mid\)}
\newcommand{\ctr}{\(\wedge\)}

\newtheorem{theorem}{Theorem}[lecnum]
\newtheorem{lemma}[theorem]{Lemma}
\newtheorem{ex}[theorem]{Example}
\newtheorem{proposition}[theorem]{Proposition}
\newtheorem{claim}[theorem]{Claim}
\newtheorem{corollary}[theorem]{Corollary}
\newtheorem{definition}[theorem]{Definition}
\newenvironment{proof}{{\bf Proof:}}{\hfill\rule{2mm}{2mm}}
\newcommand\E{\mathbb{E}}

%color definitions :
\definecolor{darkred}{rgb}{0.55, 0.0, 0.0}
\definecolor{lightcoral}{rgb}{0.94, 0.5, 0.5}
\definecolor{tomato}{rgb}{1.0, 0.39, 0.28}
\definecolor{lightgray}{rgb}{.9,.9,.9}
\definecolor{darkgray}{rgb}{.4,.4,.4}
\definecolor{purple}{rgb}{0.65, 0.12, 0.82}
\definecolor{lightgreen}{rgb}{0.56, 0.93, 0.56}
\definecolor{darkgreen}{rgb}{0.0, 0.2, 0.13}
\definecolor{limegreen}{rgb}{0.2, 0.8, 0.2}
\definecolor{lightblue}{rgb}{0.68, 0.85, 0.9}
\definecolor{darkblue}{rgb}{0.0, 0.0, 0.55}


%Environments
\newenvironment{exblock}[1]{%
    \tcolorbox[beamer,%
    noparskip,breakable,
    colback=lightgreen,colframe=darkgreen,%
    colbacklower=limegreen!75!lightgreen,%
    title=#1]}%
    {\endtcolorbox}

\newenvironment{ablock}[1]{%
    \tcolorbox[beamer,%
    noparskip,breakable,
    colback=lightcoral,colframe=darkred,%
    colbacklower=tomato!75!lightcoral,%
    title=#1]}%
    {\endtcolorbox}

\newenvironment{cblock}[1]{%
    \tcolorbox[beamer,%
    noparskip,breakable,
    colback=lightblue,colframe=darkblue,%
    colbacklower=darkblue!75!lightblue,%
    title=#1]}%
    {\endtcolorbox}


%Languages
\lstdefinelanguage{JavaScript}{
  keywords={typeof, new, true, false, catch, function, return, null, catch, switch, var, if, in, while, do, else, case, break},
  keywordstyle=\color{blue}\bfseries,
  ndkeywords={class, export, boolean, throw, implements, import, this},
  ndkeywordstyle=\color{darkgray}\bfseries,
  identifierstyle=\color{black},
  sensitive=false,
  comment=[l]{//},
  morecomment=[s]{/*}{*/},
  commentstyle=\color{purple}\ttfamily,
  stringstyle=\color{red}\ttfamily,
  morestring=[b]',
  morestring=[b]"
}

%Listings
\lstset{
   language=JavaScript,
   backgroundcolor=\color{lightgray},
   extendedchars=true,
   basicstyle=\footnotesize\ttfamily,
   showstringspaces=false,
   showspaces=false,
   numbers=left,
   numberstyle=\footnotesize,
   numbersep=9pt,
   tabsize=2,
   breaklines=true,
   showtabs=false,
   captionpos=b
}


%Start
\begin{document}

\lecture{10, 11}{June 2 - June 9, 2017}{Taylor Smith}{Harsh Mistry}

\section{Optimal Static Ordering}
A list of elements ordered by non-increasing probability of access has minimum expected access cost.
\begin{itemize}
\item \(L = \langle x_1, \ldots, x_n\rangle\)\\
Expected access cost in L is \\
\(E(L) = \sum_{i =1}^n P(x_i) T(x_i) = \sum_{i =1}^n P(x_i) \cdot i \)
\(P(X_i)\) - access probability for \(x_i\)\\
\(T(x_i)\) - position of \(x_i\) in L
\item Example :
P(a) = 0.3 , P(b) = 0.5, P(c) = 0.2 \\
\( L = \langle a, b, c \rangle \implies E(L) = 0.3 + 0.5 \times 2 + 0.2 \times 3 = 1.9\)
\end{itemize} 

\begin{proof}
Proof by contradiction
\begin{itemize}
\item \(L = \langle x_1, \ldots, x_k , x_{k+1}, \ldots, x_n \rangle \)\\
Suppose the access cost of L is optimal and there is k such that \(P(x_k) < P(x_{k+1})\) 
$$ E(L) = P(x_k) \cdot k  + P(x_{k+1}) \cdot (k +1) + \sum_{i \neq k, k+1} P(x_i) \cdot i$$
\item Create another list \(L^\prime\) by swapping \(x_k\) and \(x_{k+1}\)\\
\(L^\prime = \langle x_1, \ldots, x_{k+1}, x_k , \ldots, x_n \rangle \) 
$$ E(L^\prime) = P(x_{k+1}) \cdot k + P(x_k) \cdot (k+1) + \sum_{i \neq k, k+1} P(x_i) \cdot i $$ 
\item \(E(L^\prime) - E(L) = P(x_k) - P(x_{k+1}) < 0 \implies E(L^\prime) < E(L) \) 
\end{itemize} 
\end{proof}

\section{Dynamic Ordering}
If you don't know the probabilities ahead of time, you can 
\begin{itemize}
\item Move-To-Front(MTF): Upon a successful search, move the accessed item to the front of the list
\item Transpose: Upon a successful search, swap the accessed item with the item immediately preceding it
\end{itemize} 

\subsection{Performance of dynamic ordering}
\begin{itemize}
\item Both can be implemented in arrays or linked lists.
\item Transpose does not adapt quickly to changing access patterns.
\item MTF Works well in practice.
\item Theoretically MTF is “competitive”:
\item No more than twice as bad as the optimal “offline” ordering.
\end{itemize}

\section{Skip Lists}
\begin{itemize}
\item Randomized data structure for dictionary ADT 
\item A hierarchy of ordered linked lists
\item A skip list for a set S of items is a series of lists \(S_0,S_1, \ldots ,S_h\) such that:
\begin{itemize}
\item Each list \(S_i\) contains the special keys \(-\infty\) and \(+\infty\)
\item List \(S_0\) contains the keys of S in non-decreasing order 
\item Each list is a subsequence of the previous one, basically each list is a subset of previous list.
\item List \(S_h\) contains only two special keys 
\end{itemize}
\item A two-dimensional collection of positions: levels and towers
\end{itemize}

\subsection*{Search In Skip Lists}
\begin{lstlisting}
skip-search(L,k)
L : A skip list, k : a key 
	p = topmost left position of L
	S = stack of positions initially containing p 
	while below(p) != null do 
		p = below(p)
		while key(after(p)) < k do
			p = after(p)
		end while 
		S.push(p)
	end while
	return S 
\end{lstlisting}
\begin{itemize}
\item S contains positions of the largest key less than k at each level 
\item after(top(s)) will have key k, iff k is in L
\end{itemize}

\subsection*{Sumamry of Skip List}
\begin{itemize}
\item Expected space usage \(O(n)\)
\item Expected height : \(O(\log n)\)\\
A skip list with n items has height at most 3 log n with probability at least \(1 = 1 / n^2\)
\item Skip-Search \(O(\log n)\)
\item Skip-Insert \(O(\log n)\)
\item Skip-Delete \(O(\log n)\)
\item Skip lists are fast and simple to implement in practice
\end{itemize}

\end{document}
