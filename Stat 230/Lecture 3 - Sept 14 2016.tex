%Notes by Harsh Mistry 
%Stat230
%based on Template from : https://www.cs.cmu.edu/~ggordon/10725-F12/template.tex

\documentclass{article}
\setlength{\oddsidemargin}{0.25 in}
\setlength{\evensidemargin}{-0.25 in}
\setlength{\topmargin}{-0.6 in}
\setlength{\textwidth}{6.5 in}
\setlength{\textheight}{8.5 in}
\setlength{\headsep}{0.75 in}
\setlength{\parindent}{0 in}
\setlength{\parskip}{0.1 in}
\usepackage{amsfonts,graphicx, amssymb}
\usepackage[fleqn]{amsmath}
\usepackage{fixltx2e}
\usepackage{color}
\usepackage{tcolorbox}
\usepackage{lipsum}
\usepackage{listings}
\usepackage{scrextend}
\tcbuselibrary{skins,breakable}
\usetikzlibrary{shadings,shadows}
\newcounter{lecnum}
\renewcommand{\thepage}{\thelecnum-\arabic{page}}
\renewcommand{\thesection}{\thelecnum.\arabic{section}}
\renewcommand{\theequation}{\thelecnum.\arabic{equation}}
\renewcommand{\thefigure}{\thelecnum.\arabic{figure}}
\renewcommand{\thetable}{\thelecnum.\arabic{table}}
\newcommand{\lecture}[4]{
   \pagestyle{myheadings}
   \thispagestyle{plain}
   \newpage
   \setcounter{lecnum}{#1}
   \setcounter{page}{1}
   
   
%Info Box 
   \begin{center}
   \framebox{
      \vbox{\vspace{2mm}
    \hbox to 6.28in { {\bf Stat 230 - Probability  
	\hfill Fall 2016} }
       \vspace{4mm}
       \hbox to 6.28in { {\Large \hfill Lecture #1: #2  \hfill} }
       \vspace{2mm}
       \hbox to 6.28in { {\it Lecturer: #3 \hfill Notes By: #4} }
      \vspace{2mm}}
   }
   \end{center}
   
   \markboth{Lecture #1: #2}{Lecture #1: #2}



 
}

\renewcommand{\cite}[1]{[#1]}
\def\beginrefs{\begin{list}%
        {[\arabic{equation}]}{\usecounter{equation}
         \setlength{\leftmargin}{2.0truecm}\setlength{\labelsep}{0.4truecm}%
         \setlength{\labelwidth}{1.6truecm}}}
\def\endrefs{\end{list}}
\def\bibentry#1{\item[\hbox{[#1]}]}

\newcommand{\fig}[3]{
			\vspace{#2}
			\begin{center}
			Figure \thelecnum.#1:~#3
			\end{center}
	}
	
\newcommand{\pipe}{\(\mid\)}
\newcommand{\ctr}{\(\wedge\)}

\newtheorem{theorem}{Theorem}[lecnum]
\newtheorem{lemma}[theorem]{Lemma}
\newtheorem{ex}[theorem]{Example}
\newtheorem{proposition}[theorem]{Proposition}
\newtheorem{claim}[theorem]{Claim}
\newtheorem{corollary}[theorem]{Corollary}
\newtheorem{definition}[theorem]{Definition}
\newenvironment{proof}{{\bf Proof:}}{\hfill\rule{2mm}{2mm}}
\newcommand\E{\mathbb{E}}

%color definitions :
\definecolor{darkred}{rgb}{0.55, 0.0, 0.0}
\definecolor{lightcoral}{rgb}{0.94, 0.5, 0.5}
\definecolor{tomato}{rgb}{1.0, 0.39, 0.28}
\definecolor{lightgray}{rgb}{.9,.9,.9}
\definecolor{darkgray}{rgb}{.4,.4,.4}
\definecolor{purple}{rgb}{0.65, 0.12, 0.82}
\definecolor{lightgreen}{rgb}{0.56, 0.93, 0.56}
\definecolor{darkgreen}{rgb}{0.0, 0.2, 0.13}
\definecolor{limegreen}{rgb}{0.2, 0.8, 0.2}
\definecolor{lightblue}{rgb}{0.68, 0.85, 0.9}
\definecolor{darkblue}{rgb}{0.0, 0.0, 0.55}


%Environments
\newenvironment{exblock}[1]{%
    \tcolorbox[beamer,%
    noparskip,breakable,
    colback=lightgreen,colframe=darkgreen,%
    colbacklower=limegreen!75!lightgreen,%
    title=#1]}%
    {\endtcolorbox}

\newenvironment{ablock}[1]{%
    \tcolorbox[beamer,%
    noparskip,breakable,
    colback=lightcoral,colframe=darkred,%
    colbacklower=tomato!75!lightcoral,%
    title=#1]}%
    {\endtcolorbox}

\newenvironment{cblock}[1]{%
    \tcolorbox[beamer,%
    noparskip,breakable,
    colback=lightblue,colframe=darkblue,%
    colbacklower=darkblue!75!lightblue,%
    title=#1]}%
    {\endtcolorbox}


%Languages
\lstdefinelanguage{JavaScript}{
  keywords={typeof, new, true, false, catch, function, return, null, catch, switch, var, if, in, while, do, else, case, break},
  keywordstyle=\color{blue}\bfseries,
  ndkeywords={class, export, boolean, throw, implements, import, this},
  ndkeywordstyle=\color{darkgray}\bfseries,
  identifierstyle=\color{black},
  sensitive=false,
  comment=[l]{//},
  morecomment=[s]{/*}{*/},
  commentstyle=\color{purple}\ttfamily,
  stringstyle=\color{red}\ttfamily,
  morestring=[b]',
  morestring=[b]"
}

%Listings
\lstset{
   language=JavaScript,
   backgroundcolor=\color{lightgray},
   extendedchars=true,
   basicstyle=\footnotesize\ttfamily,
   showstringspaces=false,
   showspaces=false,
   numbers=left,
   numberstyle=\footnotesize,
   numbersep=9pt,
   tabsize=2,
   breaklines=true,
   showtabs=false,
   captionpos=b
}


%Start of Document 
\begin{document}

\lecture{3}{September 14, 2016}{Nagham Mohammad}{Harsh Mistry}

\section{Sample Spaces Continued}

\begin{exblock}{From Last Lecture}
When calculating probability, a sample space is a collection of all possible possibilities where each possibility has an equal chance of occurring and each possbility only appears at most once. In addition, sample spaces can be discrete or continuous.  
\begin{center}
\( S = \{ \ldots\}\) 
\end{center}
\end{exblock}

\subsection{Events}
\textbf{An event} is any collection (subset) of outcomes contained in the sample space S.
\begin{itemize}
\item \textbf{Simple} if it consists of exactly one outcome (point).
\item  \textbf{Compound} if it consists of more than one outcome
\end{itemize}

\section{Probability Notation}
The notation P(event) = p to denote the probability of an event occurring and (1-p) it will not occur.


\section{Odds}
\begin{definition}
The odds in \textbf{in favour} of an event A is defined by :
 \(\frac{P(A)}{P(A^c)} = \frac{P(A)}{1 - P(A)}\)
\end{definition}

\begin{definition}
The odds \textbf{against} event A is defined by : \(\frac{P(A^c)}{P(A)} = \frac{1 - P(A)}{P(A)}\)
\end{definition}

\section{Rule of Probability}
\begin{enumerate}
\item Probability distribution on S : \(\sum_{\text{all } i} p(a_i) = 1, P(S) = 1 \)
\begin{itemize}
\item Proof : \(P(S) = \sum_{a \in S} P(a) = \sum_{\text all a} P (a) = 1\)
\end{itemize}

\item For any event A, \(0 \leq P(A) \leq 1\) \\
Probabilities are always between 0 and 1 where, 0 = a event never happens and 1 = event always happens.
\begin{itemize}
\item Proof : \(P(A) = \sum_{a \in A} P(a) \leq \sum_{a \in S} P(a) = 1\) and since each  \(P(a) \geq 0\) , We have \(0 \leq P(A) \leq 1\)
\end{itemize}

\item If A and B are two events with \(A \subset B\), Then \(P(A) \leq P(B)\)
\begin{itemize}
\item Proof : \(P(A) = \sum_{a \in A} P(a) \leq \sum_{a \in B} P(a) =  P(B)\) , So \(P(A) \leq P(B)\)
\end{itemize}
\end{enumerate}

\section{Mutually Exclusive or Disjoint Events} 
\begin{itemize}
\item Two events are events are \textbf{mutually exclusive}
\begin{itemize}
\item If they cannot happen simultaneously
\item If they cannot occur at the same time 
\item When they have no outcomes in common. 
\item \(A \cap B = \varnothing \)
\end{itemize}
\item Another word that means \textbf{mutually exclusive} is \textcolor{red}{\textbf{disjoint}}
\end{itemize}




\end{document}
