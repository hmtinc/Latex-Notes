%Notes by Harsh Mistry 
%Stat230
%based on Template from : https://www.cs.cmu.edu/~ggordon/10725-F12/template.tex

\documentclass{article}
\setlength{\oddsidemargin}{0.25 in}
\setlength{\evensidemargin}{-0.25 in}
\setlength{\topmargin}{-0.6 in}
\setlength{\textwidth}{6.5 in}
\setlength{\textheight}{8.5 in}
\setlength{\headsep}{0.75 in}
\setlength{\parindent}{0 in}
\setlength{\parskip}{0.1 in}
\usepackage{amsfonts,graphicx, amssymb}
\usepackage[fleqn]{amsmath}
\usepackage{fixltx2e}
\usepackage{color}
\usepackage{tcolorbox}
\usepackage{lipsum}
\usepackage{listings}
\usepackage{scrextend}
\tcbuselibrary{skins,breakable}
\usetikzlibrary{shadings,shadows}
\newcounter{lecnum}
\renewcommand{\thepage}{\thelecnum-\arabic{page}}
\renewcommand{\thesection}{\thelecnum.\arabic{section}}
\renewcommand{\theequation}{\thelecnum.\arabic{equation}}
\renewcommand{\thefigure}{\thelecnum.\arabic{figure}}
\renewcommand{\thetable}{\thelecnum.\arabic{table}}
\newcommand{\lecture}[4]{
   \pagestyle{myheadings}
   \thispagestyle{plain}
   \newpage
   \setcounter{lecnum}{#1}
   \setcounter{page}{1}
   
   
%Info Box 
   \begin{center}
   \framebox{
      \vbox{\vspace{2mm}
    \hbox to 6.28in { {\bf Stat 230 - Probability  
	\hfill Fall 2016} }
       \vspace{4mm}
       \hbox to 6.28in { {\Large \hfill Lecture #1: #2  \hfill} }
       \vspace{2mm}
       \hbox to 6.28in { {\it Lecturer: #3 \hfill Notes By: #4} }
      \vspace{2mm}}
   }
   \end{center}
   
   \markboth{Lecture #1: #2}{Lecture #1: #2}



 
}

\renewcommand{\cite}[1]{[#1]}
\def\beginrefs{\begin{list}%
        {[\arabic{equation}]}{\usecounter{equation}
         \setlength{\leftmargin}{2.0truecm}\setlength{\labelsep}{0.4truecm}%
         \setlength{\labelwidth}{1.6truecm}}}
\def\endrefs{\end{list}}
\def\bibentry#1{\item[\hbox{[#1]}]}

\newcommand{\fig}[3]{
			\vspace{#2}
			\begin{center}
			Figure \thelecnum.#1:~#3
			\end{center}
	}
	
\newcommand{\pipe}{\(\mid\)}
\newcommand{\ctr}{\(\wedge\)}

\newtheorem{theorem}{Theorem}[lecnum]
\newtheorem{lemma}[theorem]{Lemma}
\newtheorem{ex}[theorem]{Example}
\newtheorem{proposition}[theorem]{Proposition}
\newtheorem{claim}[theorem]{Claim}
\newtheorem{corollary}[theorem]{Corollary}
\newtheorem{definition}[theorem]{Definition}
\newenvironment{proof}{{\bf Proof:}}{\hfill\rule{2mm}{2mm}}
\newcommand\E{\mathbb{E}}

%color definitions :
\definecolor{darkred}{rgb}{0.55, 0.0, 0.0}
\definecolor{lightcoral}{rgb}{0.94, 0.5, 0.5}
\definecolor{tomato}{rgb}{1.0, 0.39, 0.28}
\definecolor{lightgray}{rgb}{.9,.9,.9}
\definecolor{darkgray}{rgb}{.4,.4,.4}
\definecolor{purple}{rgb}{0.65, 0.12, 0.82}
\definecolor{lightgreen}{rgb}{0.56, 0.93, 0.56}
\definecolor{darkgreen}{rgb}{0.0, 0.2, 0.13}
\definecolor{limegreen}{rgb}{0.2, 0.8, 0.2}
\definecolor{lightblue}{rgb}{0.68, 0.85, 0.9}
\definecolor{darkblue}{rgb}{0.0, 0.0, 0.55}


%Environments
\newenvironment{exblock}[1]{%
    \tcolorbox[beamer,%
    noparskip,breakable,
    colback=lightgreen,colframe=darkgreen,%
    colbacklower=limegreen!75!lightgreen,%
    title=#1]}%
    {\endtcolorbox}

\newenvironment{ablock}[1]{%
    \tcolorbox[beamer,%
    noparskip,breakable,
    colback=lightcoral,colframe=darkred,%
    colbacklower=tomato!75!lightcoral,%
    title=#1]}%
    {\endtcolorbox}

\newenvironment{cblock}[1]{%
    \tcolorbox[beamer,%
    noparskip,breakable,
    colback=lightblue,colframe=darkblue,%
    colbacklower=darkblue!75!lightblue,%
    title=#1]}%
    {\endtcolorbox}


%Languages
\lstdefinelanguage{JavaScript}{
  keywords={typeof, new, true, false, catch, function, return, null, catch, switch, var, if, in, while, do, else, case, break},
  keywordstyle=\color{blue}\bfseries,
  ndkeywords={class, export, boolean, throw, implements, import, this},
  ndkeywordstyle=\color{darkgray}\bfseries,
  identifierstyle=\color{black},
  sensitive=false,
  comment=[l]{//},
  morecomment=[s]{/*}{*/},
  commentstyle=\color{purple}\ttfamily,
  stringstyle=\color{red}\ttfamily,
  morestring=[b]',
  morestring=[b]"
}

%Listings
\lstset{
   language=JavaScript,
   backgroundcolor=\color{lightgray},
   extendedchars=true,
   basicstyle=\footnotesize\ttfamily,
   showstringspaces=false,
   showspaces=false,
   numbers=left,
   numberstyle=\footnotesize,
   numbersep=9pt,
   tabsize=2,
   breaklines=true,
   showtabs=false,
   captionpos=b
}


%Start
\begin{document}

\lecture{24 - 28}{November 7 - 16, 2016}{Nagham Mohammad}{Harsh Mistry}


\section{Rectangular Distribution }
A continuous random variable, X is said to have a Uniform distribution from a to b on the interval [a.b], U(a,b), if all subintervals of fixed length are equally likely

$$ f(x) = \begin{cases} \frac{1}{b-a} \\ 0  \end{cases} \begin{matrix}
a \leq x \leq b \\ \text{otherwise} 
\end{matrix} $$
$$ \mu = E(X) = \frac{(a+b)}{2}  \hspace{2cm} \sigma^2 = V(X) = \frac{(b-a)^2}{12} $$

\subsection{The Cumulative Distribution function}
$$ F(x) = P[X \leq x ] = \begin{cases} 0 \\ \frac{x-a}{b-a} \\ 1  \end{cases} \begin{matrix}
x \leq a \\ a \leq x \leq b \\ c > b 
\end{matrix} $$

\section{Exponential Distribution}
\begin{itemize}
\item Widely used in engineering and science disciplines
\item Exponential distribution is used to describe the time or distance until some event happens
\item In Poisson process for events in time, let X be the length of time we wait for the first event occurrence
\item X is said to have Exponential distribution if p.d.f of C is 
$$ f(x ; \lambda) = \begin{cases} \lambda e^{-\lambda x} \\ 0 \end{cases} \begin{matrix}
x > 0 \\ x \leq 0 
\end{matrix} $$
\item Where \(\lambda > 0\)
\end{itemize}

\subsection{The Cumulative Distribution function}
$$ f(x ; \lambda) = \begin{cases} 1- e^{-\lambda x} \\ 0 \end{cases} \begin{matrix}
x > 0 \\ x \leq 0 
\end{matrix} $$
Where : \(\lambda = \frac{1}{\mu}\)

\subsection{Mean and Variance}
$$ E(X) = \mu = \frac{1}{\lambda}$$
$$ \sigma^2 = \frac{1}{\lambda^2}$$
$$ \sigma = \frac{1}{\lambda}$$

\subsection{Alternate Form}
It is common to use the parameter \(\theta =  \frac{1}{\lambda}\) in the Exponential distribution.

$$E(x) = \mu = \theta $$
$$ f_X(x) = \begin{cases} \frac{1}{\theta} e^{-x/\theta} \\ 0 \end{cases} \begin{matrix}
x > 0 \\ x \leq 0 
\end{matrix} $$
$$ F_X(x) = \begin{cases} 1-  e^{-x/\theta} \\ 0 \end{cases} \begin{matrix}
x > 0 \\ x \leq 0 
\end{matrix} $$

\begin{cblock}{Note}
\begin{itemize}
\item Average rate of occurrence = \(\lambda\)
\item Average waiting time for an occurrence = \(\theta\)
\end{itemize}
\end{cblock}

\section{Lack of Memory Property}
$$ P(X > c + b \mid X > b ) = P(X > c) $$
$$ P(X \leq c + b \mid X \geq b ) = P(X \leq c) $$
For a  Poisson process, gievn that you have waited b units of time for the next event, the probability you wait an additional c time does not depend on b, but only depends on c.

\section{Mean and Variance}
\begin{itemize}
\item Finding \(\mu\) and \(\sigma^2\) directly involves integration by parts
\item An easier solution uses properties of Gamma functions
\end{itemize}

\subsection{Gamma Function}
\(\Gamma(\alpha)\), is called a gamma function of \(\alpha\) where \(\alpha > 0\), is defined as : 
$$ \Gamma(\alpha) = \int_{0}^{\infty} e^{-x} x^{\alpha -1 } dx $$
Integration for \(\Gamma(\alpha)\) by parts yields 
\begin{enumerate}
\item \(\Gamma(\alpha) = (\alpha - 1) \Gamma ( \alpha - 1)\)
\item \(\Gamma(\alpha) = (\alpha - 1)! \) , If \(\alpha\) is a positive integer 
\item \(\Gamma(x+1) = x \Gamma(x) \)
\item \(\Gamma(1/2) = \sqrt{\pi}\)
\end{enumerate}

\subsection{Gamma Distribution}
$$ f(x) = \begin{cases} \frac{1}{\Gamma(\alpha) \beta^\alpha} x^{\alpha - 1} e^{\frac{-x}{\beta}} \\ 0 \end{cases} \begin{matrix}
x \geq 0 \\ \text{otherwise} 
\end{matrix}$$
With a shape parameter \(\alpha > 0 \) and a scale parameter \(\beta > 0\)

\begin{itemize}
\item The gamma distribution is a two-parameter family of continuous probability distributions.
\item The common Exponential distribution and Chi-squared distribution are special cases of the Gamma distribution.
\end{itemize}

\subsection{Gamma Mean/Variance}
$$ E(X) = \alpha \beta$$
$$ Var(X) = \alpha \beta^2$$

\section{The Normal Distribution}
A random variable, X, is said to have a Normal distribution with
mean μ and variance \(\sigma^2\), if X is a continuous random variable with probability density function f (x):
$$ f(x) = \frac{1}{\sqrt{2\pi} \sigma} e^{\frac{-(x-\mu)^2}{2\sigma^2}} \hspace{2cm} \begin{matrix}
\sigma > 0 \\ - \infty < x < +\infty \\ -\infty < \mu < +\infty
\end{matrix}$$
The Normal distribution is often denoted by : X \verb|~| \(N (\mu, \sigma^2) \)

\section{The Standard Normal Distribution}
\begin{itemize}
\item The Normal distribution with parameter values μ = 0 and σ=1 is called a standard Normal distribution
\item A r.v. has a standard Normal distribution is called a standard Normal random variable and denoted by Z.
\item Z - N(0,1) where \(\mu = 0\) and \(\sigma = 1\) has the probability density function 
$$ \phi(z) = \frac{1}{\sqrt{2pi}} e^{-\frac{1}{2}z^2} \text{ for } z \in R $$
\end{itemize}


\subsection{Standardized Score}
\begin{itemize}
\item Also known as "standard score" or "z-score"
\item The standardized score is a number that represents the number of standard deviations a data point is from the mean
$$ \text{z-score } = \frac{\text{ observed value - mean }}{\text{standard deviation}}$$
$$z = \frac{x- \mu}{\sigma}$$
\end{itemize}

\section{Percentiles and Standardized Scores}
To find percentiles for normal curves, you need : 
\begin{enumerate}
\item Your own value 
\item The mean of the population 
\item The standard deviation for the population(s.d)
\item Find the standardized score
\end{enumerate}

Then you can use the table to find percentiles. 

\subsection{Useful Results}
\begin{enumerate}
\item \(P(Z \leq -a) = 1 - P(Z \leq a) \)
\item \(P(Z > -a) = P(Z \leq a)\)
\item \(P(\mid Z \mid \leq a) = 2P(Z \leq a) - 1\)
\end{enumerate}

\end{document}
