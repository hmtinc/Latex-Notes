%Notes by Harsh Mistry 
%Stat230
%based on Template from : https://www.cs.cmu.edu/~ggordon/10725-F12/template.tex

\documentclass{article}
\setlength{\oddsidemargin}{0.25 in}
\setlength{\evensidemargin}{-0.25 in}
\setlength{\topmargin}{-0.6 in}
\setlength{\textwidth}{6.5 in}
\setlength{\textheight}{8.5 in}
\setlength{\headsep}{0.75 in}
\setlength{\parindent}{0 in}
\setlength{\parskip}{0.1 in}
\usepackage{amsfonts,graphicx, amssymb}
\usepackage[fleqn]{amsmath}
\usepackage{fixltx2e}
\usepackage{color}
\usepackage{tcolorbox}
\usepackage{lipsum}
\usepackage{listings}
\usepackage{scrextend}
\tcbuselibrary{skins,breakable}
\usetikzlibrary{shadings,shadows}
\newcounter{lecnum}
\renewcommand{\thepage}{\thelecnum-\arabic{page}}
\renewcommand{\thesection}{\thelecnum.\arabic{section}}
\renewcommand{\theequation}{\thelecnum.\arabic{equation}}
\renewcommand{\thefigure}{\thelecnum.\arabic{figure}}
\renewcommand{\thetable}{\thelecnum.\arabic{table}}
\newcommand{\lecture}[4]{
   \pagestyle{myheadings}
   \thispagestyle{plain}
   \newpage
   \setcounter{lecnum}{#1}
   \setcounter{page}{1}
   
   
%Info Box 
   \begin{center}
   \framebox{
      \vbox{\vspace{2mm}
    \hbox to 6.28in { {\bf Stat 230 - Probability  
	\hfill Fall 2016} }
       \vspace{4mm}
       \hbox to 6.28in { {\Large \hfill Lecture #1: #2  \hfill} }
       \vspace{2mm}
       \hbox to 6.28in { {\it Lecturer: #3 \hfill Notes By: #4} }
      \vspace{2mm}}
   }
   \end{center}
   
   \markboth{Lecture #1: #2}{Lecture #1: #2}



 
}

\renewcommand{\cite}[1]{[#1]}
\def\beginrefs{\begin{list}%
        {[\arabic{equation}]}{\usecounter{equation}
         \setlength{\leftmargin}{2.0truecm}\setlength{\labelsep}{0.4truecm}%
         \setlength{\labelwidth}{1.6truecm}}}
\def\endrefs{\end{list}}
\def\bibentry#1{\item[\hbox{[#1]}]}

\newcommand{\fig}[3]{
			\vspace{#2}
			\begin{center}
			Figure \thelecnum.#1:~#3
			\end{center}
	}
	
\newcommand{\pipe}{\(\mid\)}
\newcommand{\ctr}{\(\wedge\)}

\newtheorem{theorem}{Theorem}[lecnum]
\newtheorem{lemma}[theorem]{Lemma}
\newtheorem{ex}[theorem]{Example}
\newtheorem{proposition}[theorem]{Proposition}
\newtheorem{claim}[theorem]{Claim}
\newtheorem{corollary}[theorem]{Corollary}
\newtheorem{definition}[theorem]{Definition}
\newenvironment{proof}{{\bf Proof:}}{\hfill\rule{2mm}{2mm}}
\newcommand\E{\mathbb{E}}

%color definitions :
\definecolor{darkred}{rgb}{0.55, 0.0, 0.0}
\definecolor{lightcoral}{rgb}{0.94, 0.5, 0.5}
\definecolor{tomato}{rgb}{1.0, 0.39, 0.28}
\definecolor{lightgray}{rgb}{.9,.9,.9}
\definecolor{darkgray}{rgb}{.4,.4,.4}
\definecolor{purple}{rgb}{0.65, 0.12, 0.82}
\definecolor{lightgreen}{rgb}{0.56, 0.93, 0.56}
\definecolor{darkgreen}{rgb}{0.0, 0.2, 0.13}
\definecolor{limegreen}{rgb}{0.2, 0.8, 0.2}
\definecolor{lightblue}{rgb}{0.68, 0.85, 0.9}
\definecolor{darkblue}{rgb}{0.0, 0.0, 0.55}


%Environments
\newenvironment{exblock}[1]{%
    \tcolorbox[beamer,%
    noparskip,breakable,
    colback=lightgreen,colframe=darkgreen,%
    colbacklower=limegreen!75!lightgreen,%
    title=#1]}%
    {\endtcolorbox}

\newenvironment{ablock}[1]{%
    \tcolorbox[beamer,%
    noparskip,breakable,
    colback=lightcoral,colframe=darkred,%
    colbacklower=tomato!75!lightcoral,%
    title=#1]}%
    {\endtcolorbox}

\newenvironment{cblock}[1]{%
    \tcolorbox[beamer,%
    noparskip,breakable,
    colback=lightblue,colframe=darkblue,%
    colbacklower=darkblue!75!lightblue,%
    title=#1]}%
    {\endtcolorbox}


%Languages
\lstdefinelanguage{JavaScript}{
  keywords={typeof, new, true, false, catch, function, return, null, catch, switch, var, if, in, while, do, else, case, break},
  keywordstyle=\color{blue}\bfseries,
  ndkeywords={class, export, boolean, throw, implements, import, this},
  ndkeywordstyle=\color{darkgray}\bfseries,
  identifierstyle=\color{black},
  sensitive=false,
  comment=[l]{//},
  morecomment=[s]{/*}{*/},
  commentstyle=\color{purple}\ttfamily,
  stringstyle=\color{red}\ttfamily,
  morestring=[b]',
  morestring=[b]"
}

%Listings
\lstset{
   language=JavaScript,
   backgroundcolor=\color{lightgray},
   extendedchars=true,
   basicstyle=\footnotesize\ttfamily,
   showstringspaces=false,
   showspaces=false,
   numbers=left,
   numberstyle=\footnotesize,
   numbersep=9pt,
   tabsize=2,
   breaklines=true,
   showtabs=false,
   captionpos=b
}


%Start
\begin{document}

\lecture{14 and 15}{October 14 - 17, 2016}{Nagham Mohammad}{Harsh Mistry}

\section{Discrete Probability Distributions}

\subsection{Examples of Discrete Probability Distributions:}
\begin{itemize}
\item Discrete Uniform Distribution 
\item The Hypergeometric Distribution.
\item The Binomial Distributions 
\item The Negative Binomial Distribution 
\item Poisson Distribution
\end{itemize}

\subsection{Discrete Uniform Distribution}
A random variable X has a discrete uniform distribution if each of the n values in its range, say, a, a+1, a+2, …, b has equal probability . Then,
$$ f(x) = P(X = x) = \begin{cases} \frac{1 }{b - a + 1} \textbf{  for  } x = a, a+1, \ldots, b \textbf{ and } n = b - a + 1 \\
0 \end{cases}$$

\subsection{The Hypergeometric Distribution}
\begin{itemize}
\item “n” objects in a sample taken from a finite population of size N.
\item Sample taken without replacement.
\item Objects can be classified into two distinct types: success(S) or
failure(F).
\item There are r successes in the population 
\item Let X be the number of successes obtained
\end{itemize}
\begin{center}
\begin{tabular}{|c|c|}
\hline 
r & The number of “S” in the population \\ 
\hline 
N-r  & The number of “F” in the population \\ 
\hline 
n & The number of items in the sample \\ 
\hline 
x & The number of “S” in the sample \\ 
\hline 
n-x & The number of “F” in the sample \\ 
\hline 
\end{tabular} 
\end{center}

$$ f(x) = P(X = x) = \frac{{r \choose x} {N-r \choose n-x }}{{N 
\choose n}} $$
$$x = 0, 1 \ldots , min(n,r) \\ \text{ Where  } max(0, n-N +r) \leq x \leq min(n, r) $$

\subsection{Binomial Distribution}
 There are many experiments for which the results of each trial
can be reduced to two outcomes: success and failure.

\subsubsection{Physical Setup}
\begin{enumerate}
\item The experiment is repeated for a fixed number of trials, where
n is fixed in advance of the experiment.
\item 2. There are only two possible outcomes of interest for each
trial. The outcomes can be classified as a success S or as a
failure F.
\item The trials are independent of the other trials.
\item The probability of successful P(S) is constant from trial to
trial denoted by p.
\end{enumerate}

\subsubsection{Notation}
\begin{center}
\begin{tabular}{|c|c|}
\hline 
p = P(S) & The probability of success in a single trial. \\ 
\hline 
q = P(F) & The probability of failure in a single trial
(q = 1 – p) \\ 
\hline 
X & The number of successes in n trials:
x = 0, 1, 2, 3, . .n. \\ 
\hline 
\end{tabular} 
\end{center}

\subsubsection{Function}
$$ f(x) = P(X = x) = b (x; n, p) = \begin{cases} {n \choose x} p^x (1-p)^{n-x} \\ 0 \end{cases} \begin{matrix}
&, x = 0, 1, 2,\ldots, n \\ &, \textit{otherwise} \end{matrix} $$


\subsection{Binomial vs. Hyper geometric Distribution}
The main difference is that the binomial distribution requires INDEPENDENT trials, and the probability of success(p) is the same in each trial.  Hypergerometric, the draws are made from a finite number of objects(N) without replacement and the trials are not independent. 
\end{document}
