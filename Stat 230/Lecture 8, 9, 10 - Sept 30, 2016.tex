%Notes by Harsh Mistry 
%Stat230
%based on Template from : https://www.cs.cmu.edu/~ggordon/10725-F12/template.tex

\documentclass{article}
\setlength{\oddsidemargin}{0.25 in}
\setlength{\evensidemargin}{-0.25 in}
\setlength{\topmargin}{-0.6 in}
\setlength{\textwidth}{6.5 in}
\setlength{\textheight}{8.5 in}
\setlength{\headsep}{0.75 in}
\setlength{\parindent}{0 in}
\setlength{\parskip}{0.1 in}
\usepackage{amsfonts,graphicx, amssymb}
\usepackage[fleqn]{amsmath}
\usepackage{fixltx2e}
\usepackage{color}
\usepackage{tcolorbox}
\usepackage{lipsum}
\usepackage{listings}
\usepackage{scrextend}
\tcbuselibrary{skins,breakable}
\usetikzlibrary{shadings,shadows}
\newcounter{lecnum}
\renewcommand{\thepage}{\thelecnum-\arabic{page}}
\renewcommand{\thesection}{\thelecnum.\arabic{section}}
\renewcommand{\theequation}{\thelecnum.\arabic{equation}}
\renewcommand{\thefigure}{\thelecnum.\arabic{figure}}
\renewcommand{\thetable}{\thelecnum.\arabic{table}}
\newcommand{\lecture}[4]{
   \pagestyle{myheadings}
   \thispagestyle{plain}
   \newpage
   \setcounter{lecnum}{#1}
   \setcounter{page}{1}
   
   
%Info Box 
   \begin{center}
   \framebox{
      \vbox{\vspace{2mm}
    \hbox to 6.28in { {\bf Stat 230 - Probability  
	\hfill Fall 2016} }
       \vspace{4mm}
       \hbox to 6.28in { {\Large \hfill Lecture #1: #2  \hfill} }
       \vspace{2mm}
       \hbox to 6.28in { {\it Lecturer: #3 \hfill Notes By: #4} }
      \vspace{2mm}}
   }
   \end{center}
   
   \markboth{Lecture #1: #2}{Lecture #1: #2}



 
}

\renewcommand{\cite}[1]{[#1]}
\def\beginrefs{\begin{list}%
        {[\arabic{equation}]}{\usecounter{equation}
         \setlength{\leftmargin}{2.0truecm}\setlength{\labelsep}{0.4truecm}%
         \setlength{\labelwidth}{1.6truecm}}}
\def\endrefs{\end{list}}
\def\bibentry#1{\item[\hbox{[#1]}]}

\newcommand{\fig}[3]{
			\vspace{#2}
			\begin{center}
			Figure \thelecnum.#1:~#3
			\end{center}
	}
	
\newcommand{\pipe}{\(\mid\)}
\newcommand{\ctr}{\(\wedge\)}

\newtheorem{theorem}{Theorem}[lecnum]
\newtheorem{lemma}[theorem]{Lemma}
\newtheorem{ex}[theorem]{Example}
\newtheorem{proposition}[theorem]{Proposition}
\newtheorem{claim}[theorem]{Claim}
\newtheorem{corollary}[theorem]{Corollary}
\newtheorem{definition}[theorem]{Definition}
\newenvironment{proof}{{\bf Proof:}}{\hfill\rule{2mm}{2mm}}
\newcommand\E{\mathbb{E}}

%color definitions :
\definecolor{darkred}{rgb}{0.55, 0.0, 0.0}
\definecolor{lightcoral}{rgb}{0.94, 0.5, 0.5}
\definecolor{tomato}{rgb}{1.0, 0.39, 0.28}
\definecolor{lightgray}{rgb}{.9,.9,.9}
\definecolor{darkgray}{rgb}{.4,.4,.4}
\definecolor{purple}{rgb}{0.65, 0.12, 0.82}
\definecolor{lightgreen}{rgb}{0.56, 0.93, 0.56}
\definecolor{darkgreen}{rgb}{0.0, 0.2, 0.13}
\definecolor{limegreen}{rgb}{0.2, 0.8, 0.2}
\definecolor{lightblue}{rgb}{0.68, 0.85, 0.9}
\definecolor{darkblue}{rgb}{0.0, 0.0, 0.55}


%Environments
\newenvironment{exblock}[1]{%
    \tcolorbox[beamer,%
    noparskip,breakable,
    colback=lightgreen,colframe=darkgreen,%
    colbacklower=limegreen!75!lightgreen,%
    title=#1]}%
    {\endtcolorbox}

\newenvironment{ablock}[1]{%
    \tcolorbox[beamer,%
    noparskip,breakable,
    colback=lightcoral,colframe=darkred,%
    colbacklower=tomato!75!lightcoral,%
    title=#1]}%
    {\endtcolorbox}

\newenvironment{cblock}[1]{%
    \tcolorbox[beamer,%
    noparskip,breakable,
    colback=lightblue,colframe=darkblue,%
    colbacklower=darkblue!75!lightblue,%
    title=#1]}%
    {\endtcolorbox}


%Languages
\lstdefinelanguage{JavaScript}{
  keywords={typeof, new, true, false, catch, function, return, null, catch, switch, var, if, in, while, do, else, case, break},
  keywordstyle=\color{blue}\bfseries,
  ndkeywords={class, export, boolean, throw, implements, import, this},
  ndkeywordstyle=\color{darkgray}\bfseries,
  identifierstyle=\color{black},
  sensitive=false,
  comment=[l]{//},
  morecomment=[s]{/*}{*/},
  commentstyle=\color{purple}\ttfamily,
  stringstyle=\color{red}\ttfamily,
  morestring=[b]',
  morestring=[b]"
}

%Listings
\lstset{
   language=JavaScript,
   backgroundcolor=\color{lightgray},
   extendedchars=true,
   basicstyle=\footnotesize\ttfamily,
   showstringspaces=false,
   showspaces=false,
   numbers=left,
   numberstyle=\footnotesize,
   numbersep=9pt,
   tabsize=2,
   breaklines=true,
   showtabs=false,
   captionpos=b
}


%Start
\begin{document}

\lecture{8, 9, 10}{September 26, 28, 30, 2016}{Nagham Mohammad}{Harsh Mistry}

\section{Conditional Probability}
The importance of this concept is:

We are often interested in calculating probabilities when some
partial information concerning the result of an experiment is
available.

For any two events A and B with \(P(B)  > 0\), the conditional probability of A given B has occurred is defined by 
$$ P(A \mid B) = \frac{P(A \cap B)}{P(B)} $$

Note that the conditional probability given \(P(A) > 0\) is 
$$ P(B \mid A) = \frac{P(A \cap B)}{P(A)}$$

\begin{exblock}{Note}
$$ P(A \mid B) + P(A^c \mid B) = 1 $$
$$ P(B \mid A) + P(B^c \mid A) = 1 $$
\end{exblock}

\section{The Multiplication Rule for \(P(A \cap B)\)}

$$ P(A \cap B) = P(A \mid B) \cdot P(B) \text{ or } P(A \cap B) = P(A) \cdot P(B \mid A) $$

Let A, B, C, D, ... Be arbitrary events in a sample space. Assume that \(P(A) > 0, P(AB) > 0, P(ABC) > 0\). Then
$$ P(ABC) = P(A)P(B \mid A) P(C \mid AB) $$
$$ P(ABCD) = P(A)P(B \mid A) P(C \mid AB) P(D \mid ABC)$$

\section{Independence}
\begin{itemize}
\item For any two events A and B defined on S with \(P(B) > 0, P(A) > 0 \)
\item Then A and B are independent if and only if either of the statements is true.
\end{itemize}
$$ P(A) = P(A \mid B) $$
$$ P(B) = P(B \mid A) $$

\section{Tree Diagrams}

A tree diagram is useful for displaying all outcomes for a “multistage” experiment and determining their probabilities.

\section{The Law of Total Probability}
\begin{itemize}
\item Let \(A_1, A_2, \ldots, A_k\) be mutually exclusive and exhaustive events. Then for any other event B 
$$ P(B) = P(B\mid A_1) P(A_1) + \ldots + P(B\mid A_k) P(A_k) = \sum_{i=1}^{k} P(B\mid A_i)P(A_i)$$
\item The events are exhaustive if one \(A_i\) must occur, so that \(A_i \cup \ldots \cup A_k = S\)
\item A set of event is said to be exhaustive when at least one of the events compulsorily occurs.
\end{itemize}

\subsection{Terminology}
\begin{itemize}
\item A false positive results when a test indicates a positive status when the true status is negative \((T\mid D^c)\)
\item A false negative results when a test indicates a negative status when the true status is positive \((T^C \mid D)\)
\item The Sensitivity (true positive rate) of a test is a probability of a positive test result given the presence of the disease \(P(T\mid D)\).
\item The Specificity(true negative rate)of a test is a probability of a negative test result given the absence of the
disease \(P(T^c \mid  D^c)\)
\end{itemize}

\begin{exblock}{Note}
\begin{itemize}
\item Sensitivity is complementary to the false negative rate.
$$ P(T \mid D) + (T^c \mid D) = 1 $$
\item Specificity is complementary to the false positive rate.
$$ P(T^c \mid D^c) + (T \mid D^c) = 1 $$
\end{itemize}
\end{exblock}

\section{Bayes' Theorem}

Let \(A_1, A_2, \ldots, A_k\) be mutually exclusive and exhaustive events with prior probabilities \(P(A_i)\) i = 1, 2 ..., k. then for any other event B for which P(B) > 0 the posterior probability of \(A_j\) given that B has occurred is 
$$ \begin{aligned} P(A_j \mid B) & = \frac{P(A_j \cap B)}{P(B)}\\ & = \frac{P(B \mid A_j) P(A_j)}{\sum_{i=1}^{k} P(B\mid A_i) P(A_i)}\textit{j = 1, 2, ..., k }\end{aligned}$$


\end{document}
