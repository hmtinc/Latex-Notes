%Notes by Harsh Mistry 
%Stat230
%based on Template from : https://www.cs.cmu.edu/~ggordon/10725-F12/template.tex

\documentclass{article}
\setlength{\oddsidemargin}{0.25 in}
\setlength{\evensidemargin}{-0.25 in}
\setlength{\topmargin}{-0.6 in}
\setlength{\textwidth}{6.5 in}
\setlength{\textheight}{8.5 in}
\setlength{\headsep}{0.75 in}
\setlength{\parindent}{0 in}
\setlength{\parskip}{0.1 in}
\usepackage{amsfonts,graphicx, amssymb}
\usepackage[fleqn]{amsmath}
\usepackage{fixltx2e}
\usepackage{color}
\usepackage{tcolorbox}
\usepackage{lipsum}
\usepackage{listings}
\usepackage{scrextend}
\tcbuselibrary{skins,breakable}
\usetikzlibrary{shadings,shadows}
\newcounter{lecnum}
\renewcommand{\thepage}{\thelecnum-\arabic{page}}
\renewcommand{\thesection}{\thelecnum.\arabic{section}}
\renewcommand{\theequation}{\thelecnum.\arabic{equation}}
\renewcommand{\thefigure}{\thelecnum.\arabic{figure}}
\renewcommand{\thetable}{\thelecnum.\arabic{table}}
\newcommand{\lecture}[4]{
   \pagestyle{myheadings}
   \thispagestyle{plain}
   \newpage
   \setcounter{lecnum}{#1}
   \setcounter{page}{1}
   
   
%Info Box 
   \begin{center}
   \framebox{
      \vbox{\vspace{2mm}
    \hbox to 6.28in { {\bf Stat 230 - Probability  
	\hfill Fall 2016} }
       \vspace{4mm}
       \hbox to 6.28in { {\Large \hfill Lecture #1: #2  \hfill} }
       \vspace{2mm}
       \hbox to 6.28in { {\it Lecturer: #3 \hfill Notes By: #4} }
      \vspace{2mm}}
   }
   \end{center}
   
   \markboth{Lecture #1: #2}{Lecture #1: #2}



 
}

\renewcommand{\cite}[1]{[#1]}
\def\beginrefs{\begin{list}%
        {[\arabic{equation}]}{\usecounter{equation}
         \setlength{\leftmargin}{2.0truecm}\setlength{\labelsep}{0.4truecm}%
         \setlength{\labelwidth}{1.6truecm}}}
\def\endrefs{\end{list}}
\def\bibentry#1{\item[\hbox{[#1]}]}

\newcommand{\fig}[3]{
			\vspace{#2}
			\begin{center}
			Figure \thelecnum.#1:~#3
			\end{center}
	}
	
\newcommand{\pipe}{\(\mid\)}
\newcommand{\ctr}{\(\wedge\)}

\newtheorem{theorem}{Theorem}[lecnum]
\newtheorem{lemma}[theorem]{Lemma}
\newtheorem{ex}[theorem]{Example}
\newtheorem{proposition}[theorem]{Proposition}
\newtheorem{claim}[theorem]{Claim}
\newtheorem{corollary}[theorem]{Corollary}
\newtheorem{definition}[theorem]{Definition}
\newenvironment{proof}{{\bf Proof:}}{\hfill\rule{2mm}{2mm}}
\newcommand\E{\mathbb{E}}

%color definitions :
\definecolor{darkred}{rgb}{0.55, 0.0, 0.0}
\definecolor{lightcoral}{rgb}{0.94, 0.5, 0.5}
\definecolor{tomato}{rgb}{1.0, 0.39, 0.28}
\definecolor{lightgray}{rgb}{.9,.9,.9}
\definecolor{darkgray}{rgb}{.4,.4,.4}
\definecolor{purple}{rgb}{0.65, 0.12, 0.82}
\definecolor{lightgreen}{rgb}{0.56, 0.93, 0.56}
\definecolor{darkgreen}{rgb}{0.0, 0.2, 0.13}
\definecolor{limegreen}{rgb}{0.2, 0.8, 0.2}
\definecolor{lightblue}{rgb}{0.68, 0.85, 0.9}
\definecolor{darkblue}{rgb}{0.0, 0.0, 0.55}


%Environments
\newenvironment{exblock}[1]{%
    \tcolorbox[beamer,%
    noparskip,breakable,
    colback=lightgreen,colframe=darkgreen,%
    colbacklower=limegreen!75!lightgreen,%
    title=#1]}%
    {\endtcolorbox}

\newenvironment{ablock}[1]{%
    \tcolorbox[beamer,%
    noparskip,breakable,
    colback=lightcoral,colframe=darkred,%
    colbacklower=tomato!75!lightcoral,%
    title=#1]}%
    {\endtcolorbox}

\newenvironment{cblock}[1]{%
    \tcolorbox[beamer,%
    noparskip,breakable,
    colback=lightblue,colframe=darkblue,%
    colbacklower=darkblue!75!lightblue,%
    title=#1]}%
    {\endtcolorbox}


%Languages
\lstdefinelanguage{JavaScript}{
  keywords={typeof, new, true, false, catch, function, return, null, catch, switch, var, if, in, while, do, else, case, break},
  keywordstyle=\color{blue}\bfseries,
  ndkeywords={class, export, boolean, throw, implements, import, this},
  ndkeywordstyle=\color{darkgray}\bfseries,
  identifierstyle=\color{black},
  sensitive=false,
  comment=[l]{//},
  morecomment=[s]{/*}{*/},
  commentstyle=\color{purple}\ttfamily,
  stringstyle=\color{red}\ttfamily,
  morestring=[b]',
  morestring=[b]"
}

%Listings
\lstset{
   language=JavaScript,
   backgroundcolor=\color{lightgray},
   extendedchars=true,
   basicstyle=\footnotesize\ttfamily,
   showstringspaces=false,
   showspaces=false,
   numbers=left,
   numberstyle=\footnotesize,
   numbersep=9pt,
   tabsize=2,
   breaklines=true,
   showtabs=false,
   captionpos=b
}


%Start
\begin{document}

\lecture{11, 12, 13}{October 3, 5, 7, 2016}{Nagham Mohammad}{Harsh Mistry}

\section{Conditional Independence}
Two events A and B are \textbf{Conditionally independent} given a third event Y if the occurrence or non-occurrence of A and the occurrence or non-occurrence of B are independent in their conditional probability given Y.

$$ P(A \cap B \mid Y ) = P(A \mid Y) P (B \mid Y) $$

\section{Discrete Random Variables and Probability Models}

\subsection{Random Variable}
\begin{itemize}
\item When an experiment is performed we are interested mainly in some function of the outcome as opposed to the actual outcome itself 
\item Random Variable : is a function that assigns a real number to each point in a sample space \(S\).
\item In essence, a random variable is a function whose domain is the sample space and whose range is the set of possible values of the variable. 
\end{itemize}

\textbf{Two Types of Random Variables}
\begin{itemize}
\item Discrete random variable is a r.v whose possible values either integer or countable set om which there is a first element, second element, and so on. 
\item Continuous random variable is a r.v whose set of possible values consists of an entire interval on the number line. 
\end{itemize}

\subsection{Probability Function}
\begin{itemize}
\item Probability function of a random variable, X, is a function
$$ f(x) = P(X = x) \text{  Defined for all } x \in A $$
\end{itemize}

\subsection{Probability Distributions for Discrete Random Variables}
\begin{itemize}
\item A probability distributions of a random variable X is a description of the probabilities associated with the possible values of X
\item The set of pairs \(\{ (x, f(x)) : x \in A \}\) is called the probability distribution of X
\item A probability distribution says how the total probability of 1 is distributed among the various value of X
\end{itemize}

\textbf{Properties}
\begin{itemize}
\item \(f(x) \geq 0 \text{ for all } x \in A \)
\item \(\sum_{x \in A} f(x) = 1 \)
\item The p.f can be presented nicely in tabular form
\item The p.f can also be displayed in a line graph
\item The p.f can also be displayed using histogram which is called a probability histogram 
\begin{itemize}
\item The height of each rectangle is proportional to \(f(x)\) 
\item The base is the same for all rectangle 
\end{itemize}
\item A p.f may be specified as a formula
\end{itemize}

\textbf{Note : Its not always possible to find a simple formula}

\subsection{The Cumulative Distributive Function. }

F(x) for discrete r.v for variable X with p.f. f(x) is defined for every number \(x \in R\) by 

$$ F(x) = P(X \leq x) = \sum_{u : u \leq x} f(u) $$

\textbf{Properties :} Let F(X) denote the probability that the random variable X takes on a value that is less than or equal to x
\begin{itemize}
\item F is a non decreasing function; that is, if \(a < b\), then \(F(a) \leq F(b)\).
\item \( 0 \leq F(X) \leq 1 \), for all \(x \in R\) 
\item \(\lim_{x \to \infty} F(x) = 1, \lim_{x \to -\infty} = F(x) = 0 \)
\end{itemize}
\textbf{Proposition:} For any numbers a and b with \(a \leq b\)
$$ P ( a \leq X \leq b) = F(b) = F(a -) $$
Where \((a -)\) is the largest x value that is strictly less than a


\end{document}
