%Notes by Harsh Mistry 
%Math 239
%based on Template from : https://www.cs.cmu.edu/~ggordon/10725-F12/template.tex

\documentclass{article}
\setlength{\oddsidemargin}{0.25 in}
\setlength{\evensidemargin}{-0.25 in}
\setlength{\topmargin}{-0.6 in}
\setlength{\textwidth}{6.5 in}
\setlength{\textheight}{8.5 in}
\setlength{\headsep}{0.75 in}
\setlength{\parindent}{0 in}
\setlength{\parskip}{0.1 in}
\usepackage{amsfonts,graphicx, amssymb}
\usepackage[fleqn]{amsmath}
\usepackage{fixltx2e}
\usepackage{color}
\usepackage{tcolorbox}
\usepackage{lipsum}
\usepackage{listings}
\usepackage{scrextend}
\tcbuselibrary{skins,breakable}
\usetikzlibrary{shadings,shadows}
\newcounter{lecnum}
\renewcommand{\thepage}{\thelecnum-\arabic{page}}
\renewcommand{\thesection}{\thelecnum.\arabic{section}}
\renewcommand{\theequation}{\thelecnum.\arabic{equation}}
\renewcommand{\thefigure}{\thelecnum.\arabic{figure}}
\renewcommand{\thetable}{\thelecnum.\arabic{table}}
\newcommand{\lecture}[4]{
   \pagestyle{myheadings}
   \thispagestyle{plain}
   \newpage
   \setcounter{lecnum}{#1}
   \setcounter{page}{1}
   
   
%Info Box 
   \begin{center}
   \framebox{
      \vbox{\vspace{2mm}
    \hbox to 6.28in { {\bf Math 239 - Introduction to Combinatorics
	\hfill Spring 2017} }
       \vspace{4mm}
       \hbox to 6.28in { {\Large \hfill Lecture #1: #2  \hfill} }
       \vspace{2mm}
       \hbox to 6.28in { {\it Lecturer: #3 \hfill Notes By: #4} }
      \vspace{2mm}}
   }
   \end{center}
   
   \markboth{Lecture #1: #2}{Lecture #1: #2}



 
}

\renewcommand{\cite}[1]{[#1]}
\def\beginrefs{\begin{list}%
        {[\arabic{equation}]}{\usecounter{equation}
         \setlength{\leftmargin}{2.0truecm}\setlength{\labelsep}{0.4truecm}%
         \setlength{\labelwidth}{1.6truecm}}}
\def\endrefs{\end{list}}
\def\bibentry#1{\item[\hbox{[#1]}]}

\newcommand{\fig}[3]{
			\vspace{#2}
			\begin{center}
			Figure \thelecnum.#1:~#3
			\end{center}
	}
	
	\newcommand{\bs}{
		\texttt{\char`\\ \hspace{0.1cm}} 
	}
	

	
\newcommand{\pipe}{\(\mid\)}
\newcommand{\ctr}{\(\wedge\)}

\newtheorem{theorem}{Theorem}[lecnum]
\newtheorem{lemma}[theorem]{Lemma}
\newtheorem{ex}[theorem]{Example}
\newtheorem{expx}[theorem]{Problem}
\newtheorem{prop}[theorem]{Proposition}
\newtheorem{claim}[theorem]{Claim}
\newtheorem{corollary}[theorem]{Corollary}
\newtheorem{definition}[theorem]{Definition}
\newenvironment{proof}{{\bf Proof:}}{\hfill\rule{2mm}{2mm}}
\newcommand\E{\mathbb{E}}

%color definitions :
\definecolor{darkred}{rgb}{0.55, 0.0, 0.0}
\definecolor{lightcoral}{rgb}{0.94, 0.5, 0.5}
\definecolor{tomato}{rgb}{1.0, 0.39, 0.28}
\definecolor{lightgray}{rgb}{.9,.9,.9}
\definecolor{darkgray}{rgb}{.4,.4,.4}
\definecolor{purple}{rgb}{0.65, 0.12, 0.82}
\definecolor{lightgreen}{rgb}{0.56, 0.93, 0.56}
\definecolor{darkgreen}{rgb}{0.0, 0.2, 0.13}
\definecolor{limegreen}{rgb}{0.2, 0.8, 0.2}
\definecolor{lightblue}{rgb}{0.68, 0.85, 0.9}
\definecolor{darkblue}{rgb}{0.0, 0.0, 0.55}


%Environments
\newenvironment{exblock}[1]{%
    \tcolorbox[beamer,%
    noparskip,breakable,
    colback=lightgreen,colframe=darkgreen,%
    colbacklower=limegreen!75!lightgreen,%
    title=#1]}%
    {\endtcolorbox}

\newenvironment{ablock}[1]{%
    \tcolorbox[beamer,%
    noparskip,breakable,
    colback=lightcoral,colframe=darkred,%
    colbacklower=tomato!75!lightcoral,%
    title=#1]}%
    {\endtcolorbox}

\newenvironment{cblock}[1]{%
    \tcolorbox[beamer,%
    noparskip,breakable,
    colback=lightblue,colframe=darkblue,%
    colbacklower=darkblue!75!lightblue,%
    title=#1]}%
    {\endtcolorbox}


%Languages
\lstdefinelanguage{JavaScript}{
  keywords={typeof, new, true, false, catch, function, return, null, catch, switch, var, if, in, while, do, else, case, break},
  keywordstyle=\color{blue}\bfseries,
  ndkeywords={class, export, boolean, throw, implements, import, this},
  ndkeywordstyle=\color{darkgray}\bfseries,
  identifierstyle=\color{black},
  sensitive=false,
  comment=[l]{//},
  morecomment=[s]{/*}{*/},
  commentstyle=\color{purple}\ttfamily,
  stringstyle=\color{red}\ttfamily,
  morestring=[b]',
  morestring=[b]"
}

%Listings
\lstset{
   language=JavaScript,
   backgroundcolor=\color{lightgray},
   extendedchars=true,
   basicstyle=\footnotesize\ttfamily,
   showstringspaces=false,
   showspaces=false,
   numbers=left,
   numberstyle=\footnotesize,
   numbersep=9pt,
   tabsize=2,
   breaklines=true,
   showtabs=false,
   captionpos=b
}


%Start
\begin{document}

\lecture{11}{May 24th, 2017}{Alan Arroyo Guevara}{Harsh Mistry}
\section{Sum, Product, and Star Lemma for Binary Strings} 
In the context of strings, the weight function is \underline{always} defined to be the length of each string.
$$ w(\epsilon) = 0, w(0) = 1, w(0101) = 4 $$
\begin{lemma} \textbf{Sum Lemma} (For strings) If A, B are sets of strings and \(A \cup B\) is unambiguous then, 
$$\phi_{A \cup B} (x) = \phi_A(x) + \phi_B(x)$$
\end{lemma}
 
\begin{lemma}\textbf{Product Lemma} (For strings) If A, B are sets of strings and AB is unambiguous, then 
$$\phi_{AB} (x) = \phi_A(x) \phi_B(x) $$
\end{lemma} 

\begin{proof} Refer to course notes for the full proof, but the key idea is : \\
If AB is unambiguous then there is a bijection between \(AB = A \times B\)
\end{proof}
 
\begin{lemma}\textbf{Star Lemma} - If A is a set of strings and \(A^*\) is unambiguous , then 
$$\phi_{A^*}(x) = \frac{1}{1 - \phi_A(x)} $$ 
\end{lemma} 

\begin{proof}
\[\begin{aligned}
\phi_A(x) & = \phi_{\cup_{k \geq 0} A^k} (x) \\
& = \sum_{k \geq 0} \phi_{A^k} (x) \\
& = \sum_{k \geq 0} (\phi_A(x))^k \\
& = \frac{1}{1-\phi_{A}(x)} \text{ By product lemma}  
\end{aligned}\]
\end{proof}

\begin{expx}
Let S = \(\{1\}^*(\{0\}\{11\}\{11\}^*)^*\), now find \(\phi_S(x)\)\\
\textbf{Solution :}
\[\begin{aligned}
\phi_S(X) & = \phi_{\{1\}^*} (x) \cdot \phi_{(\{0\}\{11\}\{11\}^*)^*}(x)  \\
& = \frac{1}{1 -\phi_{\{1\}}(x) } \frac{1}{1-\phi_{(\{0\}\{11\}\{11\}^*)}(x)} \\
& = \frac{1}{1-x} \frac{1}{1-\frac{x^3}{1-x^2}} \\
& = \ldots \\
& = \frac{1+x}{1-x^3-x^3}
\end{aligned}\]
\end{expx}

\section{Decomposition Rules}
The goal of decomposition is to describe sets of strings in unambiguous ways.\\
\textbf{Consider : } \(s_1\) = all binary strings and \(s_2\) = binary strings where each 0 is a preceded by an odd number of 1's 

\begin{itemize}
\item \textbf{Decompose after each occurrence of 0's} : Describe your strings as "concatenations" of strings 111 10. 
\item \textbf{Decompose after each block of 0's} :  Describe your strings as "concatenation" of strings of the form 111 100 0 
\end{itemize}
\end{document}


