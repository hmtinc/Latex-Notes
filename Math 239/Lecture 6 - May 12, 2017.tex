%Notes by Harsh Mistry 
%Math 239
%based on Template from : https://www.cs.cmu.edu/~ggordon/10725-F12/template.tex

\documentclass{article}
\setlength{\oddsidemargin}{0.25 in}
\setlength{\evensidemargin}{-0.25 in}
\setlength{\topmargin}{-0.6 in}
\setlength{\textwidth}{6.5 in}
\setlength{\textheight}{8.5 in}
\setlength{\headsep}{0.75 in}
\setlength{\parindent}{0 in}
\setlength{\parskip}{0.1 in}
\usepackage{amsfonts,graphicx, amssymb}
\usepackage[fleqn]{amsmath}
\usepackage{fixltx2e}
\usepackage{color}
\usepackage{tcolorbox}
\usepackage{lipsum}
\usepackage{listings}
\usepackage{scrextend}
\tcbuselibrary{skins,breakable}
\usetikzlibrary{shadings,shadows}
\newcounter{lecnum}
\renewcommand{\thepage}{\thelecnum-\arabic{page}}
\renewcommand{\thesection}{\thelecnum.\arabic{section}}
\renewcommand{\theequation}{\thelecnum.\arabic{equation}}
\renewcommand{\thefigure}{\thelecnum.\arabic{figure}}
\renewcommand{\thetable}{\thelecnum.\arabic{table}}
\newcommand{\lecture}[4]{
   \pagestyle{myheadings}
   \thispagestyle{plain}
   \newpage
   \setcounter{lecnum}{#1}
   \setcounter{page}{1}
   
   
%Info Box 
   \begin{center}
   \framebox{
      \vbox{\vspace{2mm}
    \hbox to 6.28in { {\bf Math 239 - Introduction to Combinatorics
	\hfill Spring 2017} }
       \vspace{4mm}
       \hbox to 6.28in { {\Large \hfill Lecture #1: #2  \hfill} }
       \vspace{2mm}
       \hbox to 6.28in { {\it Lecturer: #3 \hfill Notes By: #4} }
      \vspace{2mm}}
   }
   \end{center}
   
   \markboth{Lecture #1: #2}{Lecture #1: #2}



 
}

\renewcommand{\cite}[1]{[#1]}
\def\beginrefs{\begin{list}%
        {[\arabic{equation}]}{\usecounter{equation}
         \setlength{\leftmargin}{2.0truecm}\setlength{\labelsep}{0.4truecm}%
         \setlength{\labelwidth}{1.6truecm}}}
\def\endrefs{\end{list}}
\def\bibentry#1{\item[\hbox{[#1]}]}

\newcommand{\fig}[3]{
			\vspace{#2}
			\begin{center}
			Figure \thelecnum.#1:~#3
			\end{center}
	}
	
	\newcommand{\bs}{
		\texttt{\char`\\ \hspace{0.1cm}} 
	}
	

	
\newcommand{\pipe}{\(\mid\)}
\newcommand{\ctr}{\(\wedge\)}

\newtheorem{theorem}{Theorem}[lecnum]
\newtheorem{lemma}[theorem]{Lemma}
\newtheorem{ex}[theorem]{Example}
\newtheorem{expx}[theorem]{Problem}
\newtheorem{prop}[theorem]{Proposition}
\newtheorem{claim}[theorem]{Claim}
\newtheorem{corollary}[theorem]{Corollary}
\newtheorem{definition}[theorem]{Definition}
\newenvironment{proof}{{\bf Proof:}}{\hfill\rule{2mm}{2mm}}
\newcommand\E{\mathbb{E}}

%color definitions :
\definecolor{darkred}{rgb}{0.55, 0.0, 0.0}
\definecolor{lightcoral}{rgb}{0.94, 0.5, 0.5}
\definecolor{tomato}{rgb}{1.0, 0.39, 0.28}
\definecolor{lightgray}{rgb}{.9,.9,.9}
\definecolor{darkgray}{rgb}{.4,.4,.4}
\definecolor{purple}{rgb}{0.65, 0.12, 0.82}
\definecolor{lightgreen}{rgb}{0.56, 0.93, 0.56}
\definecolor{darkgreen}{rgb}{0.0, 0.2, 0.13}
\definecolor{limegreen}{rgb}{0.2, 0.8, 0.2}
\definecolor{lightblue}{rgb}{0.68, 0.85, 0.9}
\definecolor{darkblue}{rgb}{0.0, 0.0, 0.55}


%Environments
\newenvironment{exblock}[1]{%
    \tcolorbox[beamer,%
    noparskip,breakable,
    colback=lightgreen,colframe=darkgreen,%
    colbacklower=limegreen!75!lightgreen,%
    title=#1]}%
    {\endtcolorbox}

\newenvironment{ablock}[1]{%
    \tcolorbox[beamer,%
    noparskip,breakable,
    colback=lightcoral,colframe=darkred,%
    colbacklower=tomato!75!lightcoral,%
    title=#1]}%
    {\endtcolorbox}

\newenvironment{cblock}[1]{%
    \tcolorbox[beamer,%
    noparskip,breakable,
    colback=lightblue,colframe=darkblue,%
    colbacklower=darkblue!75!lightblue,%
    title=#1]}%
    {\endtcolorbox}


%Languages
\lstdefinelanguage{JavaScript}{
  keywords={typeof, new, true, false, catch, function, return, null, catch, switch, var, if, in, while, do, else, case, break},
  keywordstyle=\color{blue}\bfseries,
  ndkeywords={class, export, boolean, throw, implements, import, this},
  ndkeywordstyle=\color{darkgray}\bfseries,
  identifierstyle=\color{black},
  sensitive=false,
  comment=[l]{//},
  morecomment=[s]{/*}{*/},
  commentstyle=\color{purple}\ttfamily,
  stringstyle=\color{red}\ttfamily,
  morestring=[b]',
  morestring=[b]"
}

%Listings
\lstset{
   language=JavaScript,
   backgroundcolor=\color{lightgray},
   extendedchars=true,
   basicstyle=\footnotesize\ttfamily,
   showstringspaces=false,
   showspaces=false,
   numbers=left,
   numberstyle=\footnotesize,
   numbersep=9pt,
   tabsize=2,
   breaklines=true,
   showtabs=false,
   captionpos=b
}


%Start
\begin{document}

\lecture{6}{May 12th, 2017}{Alan Arroyo Guevara}{Harsh Mistry}
 

\[(1-x)^{-k} = \sum_{n \geq 0} {{n + k - 1} \choose {k -1}} x^n \]
\textbf{Notation : } Let A(x) = \(\sum_{n \geq 0} a_n x^n\) 
\[\begin{bmatrix} x^i  \end{bmatrix} A(x) = a_i\]

\begin{expx}
Find \(\begin{bmatrix} x^4 \end{bmatrix}  (1 - 2x)^{-2} (1 - x^2) ^{-6}\)\\
\textbf{Solution :} 
\[(1 - 2x) ^{-2} = (1 - (2x))^{-2} = \sum_{n \geq 0}{{n + 2 - 1} \choose {2 -1}} (2 x) ^ n = \sum_{n \geq 0} {{n + 1 } \choose 1} 2^n x^n\]
\[(1 - x^2)^{-6} = \sum_{n \geq 0} {{n + 6 - 1} \choose {6 - 1}} (x^2)^n = \sum_{n \geq 0} {{n + 5} \choose 5} x^{2n} \]
\[(1-2x)^{-2} (1 - x^2)^{-6} = \left(\sum_{n \geq 0} (n + 1) 2^n x^n \right) \left(\sum_{n \geq 0} {{n + 5} \choose 5} x^{2n} \right)\]

\begin{exblock}{Tip}
Redefine coefficients in order to apply the definition of multiplication. 
\end{exblock}

\begin{itemize}
\item For \(i \geq 0\), \(a_i = (i + 1) 2^i \)
\item For \(j \geq 0\),
\[b_j = \begin{cases} 0 \text{ , j is odd} \\ {{\frac{j}{2} + 5} \choose 5} \text{ , j is even} \end{cases}\]
\end{itemize}

\[\begin{aligned} (1-2x)^{-2} (1 - x^2)^{-6}  & = \left( \sum_{i \geq 0} a_i x^i \right) \left(\sum_{j \geq 0} b_j x^j \right) \\
& = \sum_{n \geq 0} \left(\sum_{l = 0}^n a_l b_{n - l}\right) x^n  \end{aligned}\]

\[\begin{aligned}\begin{bmatrix} x^4 \end{bmatrix} (1 - 2X)^{-2} (1-x^2)^{-6} & = 
\sum_{l = 0} a_l b_{4 - l} \\ & = a_0 b_4 + a_1 b_3 + a_2 b_2 + a_3 b_1  + a_4 b \\ & = (0 + 1) 2^0 {{4/2 + 5} \choose 5} + (2 + 1) 2^2 {{2/2 + 5} \choose 5} + (4 + 1) 2^4 {{0/2 + 5} \choose 5} \end{aligned}\]
\end{expx}

\begin{prop}
\(1 + x + \ldots + x^k = \frac{1 = x^{k + 1}}{1 - x}\)
\end{prop}

\begin{proof}
\[\begin{aligned}
(1 - x^{k + 1}) ( 1 - x)^{-1} & = ( 1 - x ^{k + 1}) (1 + x + x^2 + \ldots) \\
& = 1 + x + x^2 + x^3 + \ldots + x^k + x^ k + x^{k + 1} + \ldots - x^{k - 1} - x^{k + 2} - \ldots \\
& = 1 + x + x^2 + \ldots + x^k 
\end{aligned}\]
\end{proof}

\section*{Compositions of Formal Power Series}
\begin{ablock}{Fact}
A(B(x)) is only defined when B(x) has 0 as a constant coefficient. \\
\textit{For more information, read 1.7.10 in the course notes}
\end{ablock}



\end{document}
