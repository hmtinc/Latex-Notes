%Notes by Harsh Mistry 
%Math 239
%based on Template from : https://www.cs.cmu.edu/~ggordon/10725-F12/template.tex

\documentclass{article}
\setlength{\oddsidemargin}{0.25 in}
\setlength{\evensidemargin}{-0.25 in}
\setlength{\topmargin}{-0.6 in}
\setlength{\textwidth}{6.5 in}
\setlength{\textheight}{8.5 in}
\setlength{\headsep}{0.75 in}
\setlength{\parindent}{0 in}
\setlength{\parskip}{0.1 in}
\usepackage{amsfonts,graphicx, amssymb}
\usepackage[fleqn]{amsmath}
\usepackage{fixltx2e}
\usepackage{color}
\usepackage{tcolorbox}
\usepackage{lipsum}
\usepackage{listings}
\usepackage{scrextend}
\tcbuselibrary{skins,breakable}
\usetikzlibrary{shadings,shadows}
\newcounter{lecnum}
\renewcommand{\thepage}{\thelecnum-\arabic{page}}
\renewcommand{\thesection}{\thelecnum.\arabic{section}}
\renewcommand{\theequation}{\thelecnum.\arabic{equation}}
\renewcommand{\thefigure}{\thelecnum.\arabic{figure}}
\renewcommand{\thetable}{\thelecnum.\arabic{table}}
\newcommand{\lecture}[4]{
   \pagestyle{myheadings}
   \thispagestyle{plain}
   \newpage
   \setcounter{lecnum}{#1}
   \setcounter{page}{1}
   
   
%Info Box 
   \begin{center}
   \framebox{
      \vbox{\vspace{2mm}
    \hbox to 6.28in { {\bf Math 239 - Introduction to Combinatorics
	\hfill Spring 2017} }
       \vspace{4mm}
       \hbox to 6.28in { {\Large \hfill Lecture #1: #2  \hfill} }
       \vspace{2mm}
       \hbox to 6.28in { {\it Lecturer: #3 \hfill Notes By: #4} }
      \vspace{2mm}}
   }
   \end{center}
   
   \markboth{Lecture #1: #2}{Lecture #1: #2}



 
}

\renewcommand{\cite}[1]{[#1]}
\def\beginrefs{\begin{list}%
        {[\arabic{equation}]}{\usecounter{equation}
         \setlength{\leftmargin}{2.0truecm}\setlength{\labelsep}{0.4truecm}%
         \setlength{\labelwidth}{1.6truecm}}}
\def\endrefs{\end{list}}
\def\bibentry#1{\item[\hbox{[#1]}]}

\newcommand{\fig}[3]{
			\vspace{#2}
			\begin{center}
			Figure \thelecnum.#1:~#3
			\end{center}
	}
	
	\newcommand{\bs}{
		\texttt{\char`\\ \hspace{0.1cm}} 
	}
	

	
\newcommand{\pipe}{\(\mid\)}
\newcommand{\ctr}{\(\wedge\)}

\newtheorem{theorem}{Theorem}[lecnum]
\newtheorem{lemma}[theorem]{Lemma}
\newtheorem{ex}[theorem]{Example}
\newtheorem{expx}[theorem]{Problem}
\newtheorem{prop}[theorem]{Proposition}
\newtheorem{claim}[theorem]{Claim}
\newtheorem{corollary}[theorem]{Corollary}
\newtheorem{definition}[theorem]{Definition}
\newenvironment{proof}{{\bf Proof:}}{\hfill\rule{2mm}{2mm}}
\newcommand\E{\mathbb{E}}

%color definitions :
\definecolor{darkred}{rgb}{0.55, 0.0, 0.0}
\definecolor{lightcoral}{rgb}{0.94, 0.5, 0.5}
\definecolor{tomato}{rgb}{1.0, 0.39, 0.28}
\definecolor{lightgray}{rgb}{.9,.9,.9}
\definecolor{darkgray}{rgb}{.4,.4,.4}
\definecolor{purple}{rgb}{0.65, 0.12, 0.82}
\definecolor{lightgreen}{rgb}{0.56, 0.93, 0.56}
\definecolor{darkgreen}{rgb}{0.0, 0.2, 0.13}
\definecolor{limegreen}{rgb}{0.2, 0.8, 0.2}
\definecolor{lightblue}{rgb}{0.68, 0.85, 0.9}
\definecolor{darkblue}{rgb}{0.0, 0.0, 0.55}


%Environments
\newenvironment{exblock}[1]{%
    \tcolorbox[beamer,%
    noparskip,breakable,
    colback=lightgreen,colframe=darkgreen,%
    colbacklower=limegreen!75!lightgreen,%
    title=#1]}%
    {\endtcolorbox}

\newenvironment{ablock}[1]{%
    \tcolorbox[beamer,%
    noparskip,breakable,
    colback=lightcoral,colframe=darkred,%
    colbacklower=tomato!75!lightcoral,%
    title=#1]}%
    {\endtcolorbox}

\newenvironment{cblock}[1]{%
    \tcolorbox[beamer,%
    noparskip,breakable,
    colback=lightblue,colframe=darkblue,%
    colbacklower=darkblue!75!lightblue,%
    title=#1]}%
    {\endtcolorbox}


%Languages
\lstdefinelanguage{JavaScript}{
  keywords={typeof, new, true, false, catch, function, return, null, catch, switch, var, if, in, while, do, else, case, break},
  keywordstyle=\color{blue}\bfseries,
  ndkeywords={class, export, boolean, throw, implements, import, this},
  ndkeywordstyle=\color{darkgray}\bfseries,
  identifierstyle=\color{black},
  sensitive=false,
  comment=[l]{//},
  morecomment=[s]{/*}{*/},
  commentstyle=\color{purple}\ttfamily,
  stringstyle=\color{red}\ttfamily,
  morestring=[b]',
  morestring=[b]"
}

%Listings
\lstset{
   language=JavaScript,
   backgroundcolor=\color{lightgray},
   extendedchars=true,
   basicstyle=\footnotesize\ttfamily,
   showstringspaces=false,
   showspaces=false,
   numbers=left,
   numberstyle=\footnotesize,
   numbersep=9pt,
   tabsize=2,
   breaklines=true,
   showtabs=false,
   captionpos=b
}


%Start
\begin{document}

\lecture{9}{May 19th, 2017}{Alan Arroyo Guevara}{Harsh Mistry}
 

\begin{expx} How many compositions of n are there? \\
\textbf{Solution : } 
\begin{enumerate}
\item \(\mathbb{N} = \{ 1, 2, \ldots\}\)\\
For k = \(0, 1, 2, \ldots\) , Let \(S_k = \mathbb{N}^k\) and \(S = \cup_{k \geq 0} S_k\)
\[\begin{aligned}
w(C_1, \ldots, C_k) & = C_1 + \ldots + c_k, \hspace{0.2cm} \forall (C_1, \ldots, C_k) \in S \\
w(c) & = c , \forall C \in \mathbb{N}
]
\end{aligned}\]
\item 
\[\begin{aligned}
\phi_{\mathbb{R}} (x) & = x^1 + x^2 + \ldots \\
& = x (x^0 + x^1 + \ldots) \\ 
& = \frac{x}{1-x}\\
\phi_{S_k} (x) & = \phi_{N^k} (x) \\&  = (\phi_{N}(x))^k \\  & = \left(\frac{x}{1-x}\right)^k \\
\phi_{S}(x) & = \phi_{\cup s_k} (x) = \sum \phi_{S_k}(x) = \sum (\frac{x}{1-x})^k = \frac{1-x}{1-2x}
\end{aligned}\]
\item \[\begin{aligned}
[x^n] \phi_S(x) & = [x^n] \frac{1-x}{1-2x} \\
& = [x^n] \frac{1-x-x+x}{1-2x}\left[\frac{1-2x}{1-2x} + \frac{x}{1-2x}\right] \\
& = [x^n] \left(1 + \frac{x}{1-2x} \right) \\
& = [x^n] \left(1 + x\sum_{l \geq 0 } (2x)^l \right)\\
& = [x^n] \left(1 + \sum_{l \geq 0} 2^l x^{l + 1} \right) 
& = \begin{cases} 1 \text{ , } n = 0\\ 2^{n - 1} \text{ , } n \geq 1  
\end{cases}
\end{aligned}\]
\end{enumerate}
\end{expx}

\newpage 
\begin{expx} How many compositions of n are there with k parts, each part being at most 5. \\
\textbf{Solution:}
\begin{enumerate}
\item \(A = \{1, 2, 3, 4, 5\} \), \(S = A^k\) 
\[w(C_1, \ldots, C_k) = C_1 + \ldots + C_k\] 
\[w(c) = c \forall c \in A \] 
\item \[\begin{aligned}
\phi_A(x) & = x^1 + x^2 + x^3 + x^4 + x^5 \\
& = x(1 + x^2 + x^3 + x^4) \\
& = x\left(\frac{1-x^5}{1-x}\right) \\
\phi_S(x)& = \phi_{A^k} (x) = (\phi_A(x))^k = x^k (1-x^5)^k (1-x)^{-k}
\end{aligned}\]
\item 
\[\begin{aligned}
[x^n] x^k (1-5)^k (1-x)^{-k} & = [x^{n-k}] (1-x^5)^k - (1-x)^{-k} \\
& = [x^{n-k}] \left(\sum_{l = 0} {k \choose l} (-x^5)^l\right)\left(\sum_{m \geq 0} {{m+k-1} \choose {k-1}} x^m \right) \\
[x^{n-k}] \sum_{t \geq 0} \left(\sum_{S=0}^t a_S b_{t - S} \right) x^t & = \sum_{s =c }^{n -k} a_S b_{n-k-S}
\end{aligned}
\]
\end{enumerate}
\end{expx}

\end{document}
