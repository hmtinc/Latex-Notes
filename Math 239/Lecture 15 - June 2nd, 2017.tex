%Notes by Harsh Mistry 
%Math 239
%based on Template from : https://www.cs.cmu.edu/~ggordon/10725-F12/template.tex

\documentclass{article}
\setlength{\oddsidemargin}{0.25 in}
\setlength{\evensidemargin}{-0.25 in}
\setlength{\topmargin}{-0.6 in}
\setlength{\textwidth}{6.5 in}
\setlength{\textheight}{8.5 in}
\setlength{\headsep}{0.75 in}
\setlength{\parindent}{0 in}
\setlength{\parskip}{0.1 in}
\usepackage{amsfonts,graphicx, amssymb}
\usepackage[fleqn]{amsmath}
\usepackage{fixltx2e}
\usepackage{color}
\usepackage{tcolorbox}
\usepackage{lipsum}
\usepackage{listings}
\usepackage{scrextend}
\tcbuselibrary{skins,breakable}
\usetikzlibrary{shadings,shadows}
\newcounter{lecnum}
\renewcommand{\thepage}{\thelecnum-\arabic{page}}
\renewcommand{\thesection}{\thelecnum.\arabic{section}}
\renewcommand{\theequation}{\thelecnum.\arabic{equation}}
\renewcommand{\thefigure}{\thelecnum.\arabic{figure}}
\renewcommand{\thetable}{\thelecnum.\arabic{table}}
\newcommand{\lecture}[4]{
   \pagestyle{myheadings}
   \thispagestyle{plain}
   \newpage
   \setcounter{lecnum}{#1}
   \setcounter{page}{1}
   
   
%Info Box 
   \begin{center}
   \framebox{
      \vbox{\vspace{2mm}
    \hbox to 6.28in { {\bf Math 239 - Introduction to Combinatorics
	\hfill Spring 2017} }
       \vspace{4mm}
       \hbox to 6.28in { {\Large \hfill Lecture #1: #2  \hfill} }
       \vspace{2mm}
       \hbox to 6.28in { {\it Lecturer: #3 \hfill Notes By: #4} }
      \vspace{2mm}}
   }
   \end{center}
   
   \markboth{Lecture #1: #2}{Lecture #1: #2}



 
}

\renewcommand{\cite}[1]{[#1]}
\def\beginrefs{\begin{list}%
        {[\arabic{equation}]}{\usecounter{equation}
         \setlength{\leftmargin}{2.0truecm}\setlength{\labelsep}{0.4truecm}%
         \setlength{\labelwidth}{1.6truecm}}}
\def\endrefs{\end{list}}
\def\bibentry#1{\item[\hbox{[#1]}]}

\newcommand{\fig}[3]{
			\vspace{#2}
			\begin{center}
			Figure \thelecnum.#1:~#3
			\end{center}
	}
	
	\newcommand{\bs}{
		\texttt{\char`\\ \hspace{0.1cm}} 
	}
	

	
\newcommand{\pipe}{\(\mid\)}
\newcommand{\ctr}{\(\wedge\)}

\newtheorem{theorem}{Theorem}[lecnum]
\newtheorem{lemma}[theorem]{Lemma}
\newtheorem{ex}[theorem]{Example}
\newtheorem{expx}[theorem]{Problem}
\newtheorem{prop}[theorem]{Proposition}
\newtheorem{claim}[theorem]{Claim}
\newtheorem{corollary}[theorem]{Corollary}
\newtheorem{definition}[theorem]{Definition}
\newenvironment{proof}{{\bf Proof:}}{\hfill\rule{2mm}{2mm}}
\newcommand\E{\mathbb{E}}

%color definitions :
\definecolor{darkred}{rgb}{0.55, 0.0, 0.0}
\definecolor{lightcoral}{rgb}{0.94, 0.5, 0.5}
\definecolor{tomato}{rgb}{1.0, 0.39, 0.28}
\definecolor{lightgray}{rgb}{.9,.9,.9}
\definecolor{darkgray}{rgb}{.4,.4,.4}
\definecolor{purple}{rgb}{0.65, 0.12, 0.82}
\definecolor{lightgreen}{rgb}{0.56, 0.93, 0.56}
\definecolor{darkgreen}{rgb}{0.0, 0.2, 0.13}
\definecolor{limegreen}{rgb}{0.2, 0.8, 0.2}
\definecolor{lightblue}{rgb}{0.68, 0.85, 0.9}
\definecolor{darkblue}{rgb}{0.0, 0.0, 0.55}


%Environments
\newenvironment{exblock}[1]{%
    \tcolorbox[beamer,%
    noparskip,breakable,
    colback=lightgreen,colframe=darkgreen,%
    colbacklower=limegreen!75!lightgreen,%
    title=#1]}%
    {\endtcolorbox}

\newenvironment{ablock}[1]{%
    \tcolorbox[beamer,%
    noparskip,breakable,
    colback=lightcoral,colframe=darkred,%
    colbacklower=tomato!75!lightcoral,%
    title=#1]}%
    {\endtcolorbox}

\newenvironment{cblock}[1]{%
    \tcolorbox[beamer,%
    noparskip,breakable,
    colback=lightblue,colframe=darkblue,%
    colbacklower=darkblue!75!lightblue,%
    title=#1]}%
    {\endtcolorbox}


%Languages
\lstdefinelanguage{JavaScript}{
  keywords={typeof, new, true, false, catch, function, return, null, catch, switch, var, if, in, while, do, else, case, break},
  keywordstyle=\color{blue}\bfseries,
  ndkeywords={class, export, boolean, throw, implements, import, this},
  ndkeywordstyle=\color{darkgray}\bfseries,
  identifierstyle=\color{black},
  sensitive=false,
  comment=[l]{//},
  morecomment=[s]{/*}{*/},
  commentstyle=\color{purple}\ttfamily,
  stringstyle=\color{red}\ttfamily,
  morestring=[b]',
  morestring=[b]"
}

%Listings
\lstset{
   language=JavaScript,
   backgroundcolor=\color{lightgray},
   extendedchars=true,
   basicstyle=\footnotesize\ttfamily,
   showstringspaces=false,
   showspaces=false,
   numbers=left,
   numberstyle=\footnotesize,
   numbersep=9pt,
   tabsize=2,
   breaklines=true,
   showtabs=false,
   captionpos=b
}


%Start
\begin{document}

\lecture{15}{June 2nd, 2017}{Alan Arroyo Guevara}{Harsh Mistry}

\begin{center}
\textcolor{red}{Special Thanks to Sam Kantor and Ashley Melanson for taking notes while I foolishly slept in. }
\end{center}

\section{Non Homogeneous Linear Recurrence}

\begin{definition}
A non-homogeneous linear  recurrence is a sequence that satisfies
$$ a_n +  b_1 a_{n-1} + b_2 a_{n-2} + \ldots +  b_{k} a_{n -k} $$
and initial conditions \(a_0, a_1, \ldots, a_{k -1}\)
\end{definition}

\begin{expx}
Find the explicit \(a_n,   a_n - 3a_{n-1} + 2a_{n-2} = 10 \cdot 3^{n -1}  \text{ , }  (1n \geq 2) \) 

\textbf{Solution : } Guess Method
\begin{enumerate}
\item Goal : Find \(\{b_n\}_{n \geq 0}\) satisfying \(b_n - 3b_{n-1} + 2b_{n-2} = 10 \cdot 3^{n -1} \)
\item Subtract hypothetical \(b_n\) from \(a_n\) because \( \{a_n - b_n\}_{n \geq 0}\)
\[0 = (a_n - b_n)  - 3(a_{n-1} - b_{n-1} ) + 2 (a_{n-2} - b_{n-2} ) = 10 \cdot 3^{n-1} - 10 \cdot 3^{n-1}\]
\[\begin{aligned}
\{a_n - b_n\}_{n \geq 0 } \text{ has a } C(x) & = (1) x^2 + (-3)x + 2\\
& = (x-2) (x-1)  
\end{aligned}\]
\[\begin{aligned}a_n - b_n  & =  2^n + B 1^n \\
& = a_n = b_n + A 2^n + B \end{aligned}\]
\item Guess to find \(\{b_n\}_{n\geq 0} \)\\
Guess that \(b_n \) is similar to \( f(n) \implies \) in this case it would be like \(10 \cdot 3^{n-1}\) 
\[\begin{aligned}
10 \cdot 3^{n-1} & = b_n - 3 b_{n-1} + 2b_{n-2} \\
& = \alpha 3^n - 3(\alpha 3^{n=1}) + 2(\alpha 3^{n-2}) \\
& = \alpha 3^{n-2} (3^2 - 3 _ 2) \\
& = \alpha \cdot 2 \cdot 3^{n-2}\\
\alpha & = \frac{10 \cdot 3^{n-1}}{2 \cdot 3^{n-2}} = 15
\end{aligned}\]
So we let \(b_n = 15 \cdot 3^n\) 
\item Use initial condition 
\[a_0 = 11 = b_0 + A \cdot 2^0 + B  = 15 \cdot 3^0  + A \cdot 2^0 + B = 15 + A + B\]
\[a_1 = 42 = b_1 + A \cdot 2^1 + B = 15 \cdot 3^1 + A \cdot 2^1 + B = 46  + 2A + B\]
\[\textbf{Solve the Syetem to Get : } A = 1, B = -5 \]
\[ a_n = 15 \cdot 3^n + 2^n - 5 \]
\end{enumerate}
\end{expx}

\subsection*{How to Guess \(\{b_n\}_{n\geq 0}\)}
\begin{itemize}
\item If \(f(n) = \beta \cdot C^n\) \\
Try \(b_n = \alpha C^n\) 
\item If \(f(n)\) is a polynomial in n 
Try \(b_n\) = polynomial of degree at most n 
\end{itemize}

\end{document}


