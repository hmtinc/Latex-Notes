%Notes by Harsh Mistry 
%Math 239
%based on Template from : https://www.cs.cmu.edu/~ggordon/10725-F12/template.tex

\documentclass{article}
\setlength{\oddsidemargin}{0.25 in}
\setlength{\evensidemargin}{-0.25 in}
\setlength{\topmargin}{-0.6 in}
\setlength{\textwidth}{6.5 in}
\setlength{\textheight}{8.5 in}
\setlength{\headsep}{0.75 in}
\setlength{\parindent}{0 in}
\setlength{\parskip}{0.1 in}
\usepackage{amsfonts,graphicx, amssymb}
\usepackage[fleqn]{amsmath}
\usepackage{fixltx2e}
\usepackage{color}
\usepackage{tcolorbox}
\usepackage{lipsum}
\usepackage{listings}
\usepackage{scrextend}
\tcbuselibrary{skins,breakable}
\usetikzlibrary{shadings,shadows}
\newcounter{lecnum}
\renewcommand{\thepage}{\thelecnum-\arabic{page}}
\renewcommand{\thesection}{\thelecnum.\arabic{section}}
\renewcommand{\theequation}{\thelecnum.\arabic{equation}}
\renewcommand{\thefigure}{\thelecnum.\arabic{figure}}
\renewcommand{\thetable}{\thelecnum.\arabic{table}}
\newcommand{\lecture}[4]{
   \pagestyle{myheadings}
   \thispagestyle{plain}
   \newpage
   \setcounter{lecnum}{#1}
   \setcounter{page}{1}
   
   
%Info Box 
   \begin{center}
   \framebox{
      \vbox{\vspace{2mm}
    \hbox to 6.28in { {\bf Math 239 - Introduction to Combinatorics
	\hfill Spring 2017} }
       \vspace{4mm}
       \hbox to 6.28in { {\Large \hfill Lecture #1: #2  \hfill} }
       \vspace{2mm}
       \hbox to 6.28in { {\it Lecturer: #3 \hfill Notes By: #4} }
      \vspace{2mm}}
   }
   \end{center}
   
   \markboth{Lecture #1: #2}{Lecture #1: #2}



 
}

\renewcommand{\cite}[1]{[#1]}
\def\beginrefs{\begin{list}%
        {[\arabic{equation}]}{\usecounter{equation}
         \setlength{\leftmargin}{2.0truecm}\setlength{\labelsep}{0.4truecm}%
         \setlength{\labelwidth}{1.6truecm}}}
\def\endrefs{\end{list}}
\def\bibentry#1{\item[\hbox{[#1]}]}

\newcommand{\fig}[3]{
			\vspace{#2}
			\begin{center}
			Figure \thelecnum.#1:~#3
			\end{center}
	}
	
	\newcommand{\bs}{
		\texttt{\char`\\ \hspace{0.1cm}} 
	}
	

	
\newcommand{\pipe}{\(\mid\)}
\newcommand{\ctr}{\(\wedge\)}

\newtheorem{theorem}{Theorem}[lecnum]
\newtheorem{lemma}[theorem]{Lemma}
\newtheorem{ex}[theorem]{Example}
\newtheorem{expx}[theorem]{Problem}
\newtheorem{prop}[theorem]{Proposition}
\newtheorem{claim}[theorem]{Claim}
\newtheorem{corollary}[theorem]{Corollary}
\newtheorem{definition}[theorem]{Definition}
\newenvironment{proof}{{\bf Proof:}}{\hfill\rule{2mm}{2mm}}
\newcommand\E{\mathbb{E}}

%color definitions :
\definecolor{darkred}{rgb}{0.55, 0.0, 0.0}
\definecolor{lightcoral}{rgb}{0.94, 0.5, 0.5}
\definecolor{tomato}{rgb}{1.0, 0.39, 0.28}
\definecolor{lightgray}{rgb}{.9,.9,.9}
\definecolor{darkgray}{rgb}{.4,.4,.4}
\definecolor{purple}{rgb}{0.65, 0.12, 0.82}
\definecolor{lightgreen}{rgb}{0.56, 0.93, 0.56}
\definecolor{darkgreen}{rgb}{0.0, 0.2, 0.13}
\definecolor{limegreen}{rgb}{0.2, 0.8, 0.2}
\definecolor{lightblue}{rgb}{0.68, 0.85, 0.9}
\definecolor{darkblue}{rgb}{0.0, 0.0, 0.55}


%Environments
\newenvironment{exblock}[1]{%
    \tcolorbox[beamer,%
    noparskip,breakable,
    colback=lightgreen,colframe=darkgreen,%
    colbacklower=limegreen!75!lightgreen,%
    title=#1]}%
    {\endtcolorbox}

\newenvironment{ablock}[1]{%
    \tcolorbox[beamer,%
    noparskip,breakable,
    colback=lightcoral,colframe=darkred,%
    colbacklower=tomato!75!lightcoral,%
    title=#1]}%
    {\endtcolorbox}

\newenvironment{cblock}[1]{%
    \tcolorbox[beamer,%
    noparskip,breakable,
    colback=lightblue,colframe=darkblue,%
    colbacklower=darkblue!75!lightblue,%
    title=#1]}%
    {\endtcolorbox}


%Languages
\lstdefinelanguage{JavaScript}{
  keywords={typeof, new, true, false, catch, function, return, null, catch, switch, var, if, in, while, do, else, case, break},
  keywordstyle=\color{blue}\bfseries,
  ndkeywords={class, export, boolean, throw, implements, import, this},
  ndkeywordstyle=\color{darkgray}\bfseries,
  identifierstyle=\color{black},
  sensitive=false,
  comment=[l]{//},
  morecomment=[s]{/*}{*/},
  commentstyle=\color{purple}\ttfamily,
  stringstyle=\color{red}\ttfamily,
  morestring=[b]',
  morestring=[b]"
}

%Listings
\lstset{
   language=JavaScript,
   backgroundcolor=\color{lightgray},
   extendedchars=true,
   basicstyle=\footnotesize\ttfamily,
   showstringspaces=false,
   showspaces=false,
   numbers=left,
   numberstyle=\footnotesize,
   numbersep=9pt,
   tabsize=2,
   breaklines=true,
   showtabs=false,
   captionpos=b
}


%Start
\begin{document}

\lecture{5}{May 10th, 2017}{Alan Arroyo Guevara}{Harsh Mistry}
 
 \begin{definition}
 Given \(A(x) = \sum_{n \geq 0 } a_n x^n\) and \(B(x) = \sum_{n \geq 0 } b_n x^n\)
 \[A(x) + B(x) = \sum_{n \geq 0 } (a_n + b_n)x^n\]
 \[A(x) \cdot B(x) = \sum_{n  \geq 0} \left( \sum_{i = 0}^{n} a_i \cdot b_{n-1} \right) x^n\]
 \[A(x) - B(x) = \sum_{n \geq 0} (a_n - b_n) x^n\]
 \[cA(x) = \sum_{n /geq 0} ca_n x^n, \hspace{0.1cm} c \in Q\]
 \end{definition}
 
 \begin{definition}
 B(x) is the inverse of A(x) if 
 \[A(x) \cdot B(x) = 1 \]
 \textbf{Note :} B(x) is usually denoted as \(A^{-1}(x)\) or \(\frac{1}{A(x)}\)
 \end{definition}
 
 \begin{prop} Geometric Series  \[(1- x)^{-1} = \frac{1}{1-x} = 1 + x + x^2 + x^3 + \ldots \]
 \end{prop}
 
 \begin{proof}
 \[\begin{aligned}
 (1-x)(1 + x + x^2 + x^3 + \ldots) & = 1x^0 + (1 \cdot 1 - 1 \cdot 1) x^1 + (1 \cdot 1 - 1 \cdot 1) x^2 + \ldots \\
 & = 1  
 \end{aligned}\]
 \end{proof}
 \begin{exblock}{Tip}
 Use the \((1-x)^{-1} = 1 + x + x^2 + \ldots\) to find other inverses. 
 \end{exblock}

\begin{ex} Inverse Examples 
\[(1 + x) ^{-1} = (1 - (-x)) = 1 + (-x) + (-x)^2 + \ldots + \ldots = 1 - x + x^2 + \ldots \]
\[(1 - x + 2x^2) ^{-1} = (1 - (x - 2x^2))^{-1} = 1 + (x - 2x^2) + (x - 2x^2)^2 + \ldots \]
\end{ex}

\textbf{Note :} Not every A(x) has an inverse 

\begin{expx}
Show A(x) = x has no inverse \\
\textbf{Solution :} Suppose B(x) = \(\sum_{n \geq 0} b_n x^n\) is the inverse of x.
\[\begin{aligned}
1 & = x \cdot B(x) \\
& = x \cdot (b_0 + b_1 x + b_2 x^2 + \ldots ) \\
& = 0 + b_0 x + b_1x^2 + b_2x^3 + \ldots
\end{aligned}\]
The resulting expression has 0 has a constant coefficient. Therefore, contraction. Thus, A(x) = x has no inverse.
\end{expx}

\begin{theorem}
\[Q(x) = \sum_{n \geq 0} q_n x^n \text{ has an inverse } \iff q_0 \neq 0\]
\end{theorem}

\begin{prop}
(Inverse binomial series) 
\[(1 - x)^{-k} = \sum_{n \geq 0} {{n + k - 1} \choose {k -1}} x^n\]
\end{prop}

\begin{proof} Induction on k \\
\begin{itemize}
\item [\underline{Base}] \[\sum_{n \geq 0} {{n + 1 - 1} \choose 1 - 1} x ^n \sum_{n \geq 0} {n \choose 0}x^n = \sum_{n \geq 0} x^n = (1 -x ) ^{-1} \]
\item [\underline{I.H.}]  For k = m we assume 
\[(1 -x)^{-m} = \sum_{n \geq 0} {{n + m - 1} \choose {m - 1}} x^n\]
\item [\underline{I.S.}] We must prove for k = m + 1
\[\begin{aligned}
(1 - x) ^{- (m + 1)} & = ( 1 - x) ^ {-m} \cdot ( 1- x) ^{-1} \\
& = \left( \sum_{n \geq 0} {{n + m -1} \choose {m - 1}} x^n \right) \left( \sum_{n \geq 0} x^n \right) \\
& = \sum_{n \geq 0} \left(\sum_{i = 0}^{n} a_i \cdot b_{n -i} \right) x^n \\
& = \sum_{n \geq 0} \left( \sum_{i = 0}^{n} {{i + m -1} \choose {m-1}} \right) x^n \\
& = \sum_{n \geq 0} {{n + m} \choose {m}} x^n \\
& = \sum_{n \geq 0} {{n + (m + 1) - 1} \choose {(m+1) - 1}} x^n 
\end{aligned}\]
\end{itemize}


\end{proof}

\end{document}
