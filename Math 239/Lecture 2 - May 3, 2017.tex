%Notes by Harsh Mistry 
%Math 239
%based on Template from : https://www.cs.cmu.edu/~ggordon/10725-F12/template.tex

\documentclass{article}
\setlength{\oddsidemargin}{0.25 in}
\setlength{\evensidemargin}{-0.25 in}
\setlength{\topmargin}{-0.6 in}
\setlength{\textwidth}{6.5 in}
\setlength{\textheight}{8.5 in}
\setlength{\headsep}{0.75 in}
\setlength{\parindent}{0 in}
\setlength{\parskip}{0.1 in}
\usepackage{amsfonts,graphicx, amssymb}
\usepackage[fleqn]{amsmath}
\usepackage{fixltx2e}
\usepackage{color}
\usepackage{tcolorbox}
\usepackage{lipsum}
\usepackage{listings}
\usepackage{scrextend}
\tcbuselibrary{skins,breakable}
\usetikzlibrary{shadings,shadows}
\newcounter{lecnum}
\renewcommand{\thepage}{\thelecnum-\arabic{page}}
\renewcommand{\thesection}{\thelecnum.\arabic{section}}
\renewcommand{\theequation}{\thelecnum.\arabic{equation}}
\renewcommand{\thefigure}{\thelecnum.\arabic{figure}}
\renewcommand{\thetable}{\thelecnum.\arabic{table}}
\newcommand{\lecture}[4]{
   \pagestyle{myheadings}
   \thispagestyle{plain}
   \newpage
   \setcounter{lecnum}{#1}
   \setcounter{page}{1}
   
   
%Info Box 
   \begin{center}
   \framebox{
      \vbox{\vspace{2mm}
    \hbox to 6.28in { {\bf Math 239 - Introduction to Combinatorics
	\hfill Spring 2017} }
       \vspace{4mm}
       \hbox to 6.28in { {\Large \hfill Lecture #1: #2  \hfill} }
       \vspace{2mm}
       \hbox to 6.28in { {\it Lecturer: #3 \hfill Notes By: #4} }
      \vspace{2mm}}
   }
   \end{center}
   
   \markboth{Lecture #1: #2}{Lecture #1: #2}



 
}

\renewcommand{\cite}[1]{[#1]}
\def\beginrefs{\begin{list}%
        {[\arabic{equation}]}{\usecounter{equation}
         \setlength{\leftmargin}{2.0truecm}\setlength{\labelsep}{0.4truecm}%
         \setlength{\labelwidth}{1.6truecm}}}
\def\endrefs{\end{list}}
\def\bibentry#1{\item[\hbox{[#1]}]}

\newcommand{\fig}[3]{
			\vspace{#2}
			\begin{center}
			Figure \thelecnum.#1:~#3
			\end{center}
	}
	
	\newcommand{\bs}{
		\texttt{\char`\\ \hspace{0.1cm}} 
	}
	

	
\newcommand{\pipe}{\(\mid\)}
\newcommand{\ctr}{\(\wedge\)}

\newtheorem{theorem}{Theorem}[lecnum]
\newtheorem{lemma}[theorem]{Lemma}
\newtheorem{ex}[theorem]{Example}
\newtheorem{expx}[theorem]{Problem}
\newtheorem{prop}[theorem]{Proposition}
\newtheorem{claim}[theorem]{Claim}
\newtheorem{corollary}[theorem]{Corollary}
\newtheorem{definition}[theorem]{Definition}
\newenvironment{proof}{{\bf Proof:}}{\hfill\rule{2mm}{2mm}}
\newcommand\E{\mathbb{E}}

%color definitions :
\definecolor{darkred}{rgb}{0.55, 0.0, 0.0}
\definecolor{lightcoral}{rgb}{0.94, 0.5, 0.5}
\definecolor{tomato}{rgb}{1.0, 0.39, 0.28}
\definecolor{lightgray}{rgb}{.9,.9,.9}
\definecolor{darkgray}{rgb}{.4,.4,.4}
\definecolor{purple}{rgb}{0.65, 0.12, 0.82}
\definecolor{lightgreen}{rgb}{0.56, 0.93, 0.56}
\definecolor{darkgreen}{rgb}{0.0, 0.2, 0.13}
\definecolor{limegreen}{rgb}{0.2, 0.8, 0.2}
\definecolor{lightblue}{rgb}{0.68, 0.85, 0.9}
\definecolor{darkblue}{rgb}{0.0, 0.0, 0.55}


%Environments
\newenvironment{exblock}[1]{%
    \tcolorbox[beamer,%
    noparskip,breakable,
    colback=lightgreen,colframe=darkgreen,%
    colbacklower=limegreen!75!lightgreen,%
    title=#1]}%
    {\endtcolorbox}

\newenvironment{ablock}[1]{%
    \tcolorbox[beamer,%
    noparskip,breakable,
    colback=lightcoral,colframe=darkred,%
    colbacklower=tomato!75!lightcoral,%
    title=#1]}%
    {\endtcolorbox}

\newenvironment{cblock}[1]{%
    \tcolorbox[beamer,%
    noparskip,breakable,
    colback=lightblue,colframe=darkblue,%
    colbacklower=darkblue!75!lightblue,%
    title=#1]}%
    {\endtcolorbox}


%Languages
\lstdefinelanguage{JavaScript}{
  keywords={typeof, new, true, false, catch, function, return, null, catch, switch, var, if, in, while, do, else, case, break},
  keywordstyle=\color{blue}\bfseries,
  ndkeywords={class, export, boolean, throw, implements, import, this},
  ndkeywordstyle=\color{darkgray}\bfseries,
  identifierstyle=\color{black},
  sensitive=false,
  comment=[l]{//},
  morecomment=[s]{/*}{*/},
  commentstyle=\color{purple}\ttfamily,
  stringstyle=\color{red}\ttfamily,
  morestring=[b]',
  morestring=[b]"
}

%Listings
\lstset{
   language=JavaScript,
   backgroundcolor=\color{lightgray},
   extendedchars=true,
   basicstyle=\footnotesize\ttfamily,
   showstringspaces=false,
   showspaces=false,
   numbers=left,
   numberstyle=\footnotesize,
   numbersep=9pt,
   tabsize=2,
   breaklines=true,
   showtabs=false,
   captionpos=b
}


%Start
\begin{document}

\lecture{2}{May 3rd, 2017}{Alan Arroyo Guevara}{Harsh Mistry}

\section{Admin Info}
\begin{center}
Alan's office hours \\
Mon, Fri : 12:30 - 13:30 \\
Tue : 11:30 - 12:30
\end{center}

\section{Recap}

\[{{n}\choose{k}} = \frac{n!}{k(n-k)!} = \# \text{ k-subsets of an n-element set = \# of ways we can choose k objects amoung n}\]

\section{•}

\begin{definition}
A \underline{Bijection} is a function \(f : S \rightarrow T\) such that 

\begin{itemize}
\item f is 1-1 : for every \(x_1 , x_2 \in S, x_1 \neq x_2, f(x_1) \neq f(x_2)\)
\item f is onto : for every \(y \in T \) there exists \(x \in S\) such that \(f(x) = y \) 
\end{itemize}
\end{definition} 

\begin{definition}
An inverse for f is a function \(f^{-1} : T \rightarrow S\) such that 
\begin{itemize}
\item \(f^{-1}(f(x)) = x \text{ ,  }\forall x \in S\) 
\item \(f(f^{-1}(y)) = Y \text{ , }\forall y \in T\) 
\end{itemize}
\end{definition}

\begin{lemma}
f is a bijection \(\iff\) f has an inverse 
\end{lemma}


\begin{expx}
Show that \({{n} \choose {k}} = {{n} \choose {n - k}}\) by showing that there is a bijection between \\
\[S = \text{set of all k-subsets of } \{1, ..., n\}\]
\[T = \text{set of all (n-k) subsets of } \{1, \ldots, n\}\]

\textbf{Solution :}
Define \(f : S \rightarrow T \) and \(f ^{-1} : T \rightarrow S\). \\
\(f(A) = \{1, \ldots, n\} \bs A , \forall A \in S\) \\
\(f^{-1}(B) = \{1, \ldots, n\} \bs B , \forall B \in T\)  
\end{expx}

\textbf{Note} : Note that f and \(f^{-1}\) are well defined because \\
\(\{1, \ldots, n\}  \bs A\) has (n-k) elements \(\forall A \in S\)\\
\(\{1, \ldots, n \}  \bs b \) has k elements \(\forall B \in T \)


\(f^{-1}\) is the inverse of f because 
\[\begin{aligned}
f^{-1} (f(A)) & = f^{-1}( \{1, \ldots , n \} \bs A ) \\
& = \{1 \ldots, n\} \ (\{1, \ldots, n\}) \bs  A \\
& = A , \forall A \in S
\end{aligned}\]


\section{Binomial Theorem}
\((1 + x )^2 = 1 + 2x + x^2\)
\begin{theorem}
\[(1 + x)^n = \sum_{k=0}^{n} {{n}\choose{k}}x^k = {n \choose 0} x^0 + {n \choose 1} x^1 + \ldots + {n \choose n } x^n\]
\end{theorem}

\begin{proof}
\[\begin{aligned}
(1 + x)^n   & = (1 + x) + (1 + x) + \ldots + (1 + x) \\
			& = \underline{1} 1 +  \underline(n) x + \underline{{n \choose 2}} x^2 + \ldots + \underline{{n \choose k}}x^k + \ldots + \underline{{n \choose n}} x^n 
\end{aligned}\]

To get \(x^k\) among the n copies of (1 + x) we choose k. Thre are n choose k ways to do this
\end{proof}

\section{Combinatorial Proofs}
\begin{prop}
\(2^n = \sum_{k = 0}^{n} {n \choose k}\)
\end{prop}

\textbf{Algebraic Proof : } Plug x = 1 in to the binomial theorem

\textbf{Combinatorial Proof : } \\
The idea is to count a set in two differing ways such that one way gives you the 
\underline{left hand side} f the identity you want to prove and the another way gives you the the \underline{right hand side}

\begin{proof}Let S be the subset of all subsets \(\{1, \ldots, n\}\)
\begin{itemize}
\item[LHS : ] \(\mid S \mid = s^n \), since every element has two possibilities. Its either within a subset or not included within a subsets 
\item[RHS : ] For every \(k \in \{0, \ldots, n\}\), let \(s_k\) = k-subsets of \(\{1, \ldots, n\}\) 
\[\begin{aligned}
 S & = S_0 \cup S_1 \cup S_2 \cup \ldots \cup S_n \\
S_i \cap S_j  & =  \emptyset \text{\hspace{0.1cm} for  } i \neq j \\
\mid S \mid \hspace{0.1cm} & = \hspace{0.1cm} \mid S_0  \mid + \ldots +  \mid S_n \mid  \\
& = {n \choose 0} + {n \choose 1} + \ldots + {n \choose n}
\end{aligned}\]

\end{itemize}
\end{proof}


\end{document}
