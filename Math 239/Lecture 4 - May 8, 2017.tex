%Notes by Harsh Mistry 
%Math 239
%based on Template from : https://www.cs.cmu.edu/~ggordon/10725-F12/template.tex

\documentclass{article}
\setlength{\oddsidemargin}{0.25 in}
\setlength{\evensidemargin}{-0.25 in}
\setlength{\topmargin}{-0.6 in}
\setlength{\textwidth}{6.5 in}
\setlength{\textheight}{8.5 in}
\setlength{\headsep}{0.75 in}
\setlength{\parindent}{0 in}
\setlength{\parskip}{0.1 in}
\usepackage{amsfonts,graphicx, amssymb}
\usepackage[fleqn]{amsmath}
\usepackage{fixltx2e}
\usepackage{color}
\usepackage{tcolorbox}
\usepackage{lipsum}
\usepackage{listings}
\usepackage{scrextend}
\tcbuselibrary{skins,breakable}
\usetikzlibrary{shadings,shadows}
\newcounter{lecnum}
\renewcommand{\thepage}{\thelecnum-\arabic{page}}
\renewcommand{\thesection}{\thelecnum.\arabic{section}}
\renewcommand{\theequation}{\thelecnum.\arabic{equation}}
\renewcommand{\thefigure}{\thelecnum.\arabic{figure}}
\renewcommand{\thetable}{\thelecnum.\arabic{table}}
\newcommand{\lecture}[4]{
   \pagestyle{myheadings}
   \thispagestyle{plain}
   \newpage
   \setcounter{lecnum}{#1}
   \setcounter{page}{1}
   
   
%Info Box 
   \begin{center}
   \framebox{
      \vbox{\vspace{2mm}
    \hbox to 6.28in { {\bf Math 239 - Introduction to Combinatorics
	\hfill Spring 2017} }
       \vspace{4mm}
       \hbox to 6.28in { {\Large \hfill Lecture #1: #2  \hfill} }
       \vspace{2mm}
       \hbox to 6.28in { {\it Lecturer: #3 \hfill Notes By: #4} }
      \vspace{2mm}}
   }
   \end{center}
   
   \markboth{Lecture #1: #2}{Lecture #1: #2}



 
}

\renewcommand{\cite}[1]{[#1]}
\def\beginrefs{\begin{list}%
        {[\arabic{equation}]}{\usecounter{equation}
         \setlength{\leftmargin}{2.0truecm}\setlength{\labelsep}{0.4truecm}%
         \setlength{\labelwidth}{1.6truecm}}}
\def\endrefs{\end{list}}
\def\bibentry#1{\item[\hbox{[#1]}]}

\newcommand{\fig}[3]{
			\vspace{#2}
			\begin{center}
			Figure \thelecnum.#1:~#3
			\end{center}
	}
	
	\newcommand{\bs}{
		\texttt{\char`\\ \hspace{0.1cm}} 
	}
	

	
\newcommand{\pipe}{\(\mid\)}
\newcommand{\ctr}{\(\wedge\)}

\newtheorem{theorem}{Theorem}[lecnum]
\newtheorem{lemma}[theorem]{Lemma}
\newtheorem{ex}[theorem]{Example}
\newtheorem{expx}[theorem]{Problem}
\newtheorem{prop}[theorem]{Proposition}
\newtheorem{claim}[theorem]{Claim}
\newtheorem{corollary}[theorem]{Corollary}
\newtheorem{definition}[theorem]{Definition}
\newenvironment{proof}{{\bf Proof:}}{\hfill\rule{2mm}{2mm}}
\newcommand\E{\mathbb{E}}

%color definitions :
\definecolor{darkred}{rgb}{0.55, 0.0, 0.0}
\definecolor{lightcoral}{rgb}{0.94, 0.5, 0.5}
\definecolor{tomato}{rgb}{1.0, 0.39, 0.28}
\definecolor{lightgray}{rgb}{.9,.9,.9}
\definecolor{darkgray}{rgb}{.4,.4,.4}
\definecolor{purple}{rgb}{0.65, 0.12, 0.82}
\definecolor{lightgreen}{rgb}{0.56, 0.93, 0.56}
\definecolor{darkgreen}{rgb}{0.0, 0.2, 0.13}
\definecolor{limegreen}{rgb}{0.2, 0.8, 0.2}
\definecolor{lightblue}{rgb}{0.68, 0.85, 0.9}
\definecolor{darkblue}{rgb}{0.0, 0.0, 0.55}


%Environments
\newenvironment{exblock}[1]{%
    \tcolorbox[beamer,%
    noparskip,breakable,
    colback=lightgreen,colframe=darkgreen,%
    colbacklower=limegreen!75!lightgreen,%
    title=#1]}%
    {\endtcolorbox}

\newenvironment{ablock}[1]{%
    \tcolorbox[beamer,%
    noparskip,breakable,
    colback=lightcoral,colframe=darkred,%
    colbacklower=tomato!75!lightcoral,%
    title=#1]}%
    {\endtcolorbox}

\newenvironment{cblock}[1]{%
    \tcolorbox[beamer,%
    noparskip,breakable,
    colback=lightblue,colframe=darkblue,%
    colbacklower=darkblue!75!lightblue,%
    title=#1]}%
    {\endtcolorbox}


%Languages
\lstdefinelanguage{JavaScript}{
  keywords={typeof, new, true, false, catch, function, return, null, catch, switch, var, if, in, while, do, else, case, break},
  keywordstyle=\color{blue}\bfseries,
  ndkeywords={class, export, boolean, throw, implements, import, this},
  ndkeywordstyle=\color{darkgray}\bfseries,
  identifierstyle=\color{black},
  sensitive=false,
  comment=[l]{//},
  morecomment=[s]{/*}{*/},
  commentstyle=\color{purple}\ttfamily,
  stringstyle=\color{red}\ttfamily,
  morestring=[b]',
  morestring=[b]"
}

%Listings
\lstset{
   language=JavaScript,
   backgroundcolor=\color{lightgray},
   extendedchars=true,
   basicstyle=\footnotesize\ttfamily,
   showstringspaces=false,
   showspaces=false,
   numbers=left,
   numberstyle=\footnotesize,
   numbersep=9pt,
   tabsize=2,
   breaklines=true,
   showtabs=false,
   captionpos=b
}


%Start
\begin{document}

\lecture{4}{May 8th, 2017}{Alan Arroyo Guevara}{Harsh Mistry}


\section{Generating Series}
\textbf{\underline{General Counting Problem}} : \\ \newline Suppose S is a set, and for each element \(\sigma \in S\) has associated a non-negative weight \(w(\sigma)\). \\
Given \(k \in \langle 0, 1, 2, \ldots\rangle\), how many elements in s have weight k? 

\begin{ex} General Counting problem with S = all subsets of \(\langle 1, 2, 3, \rangle\).
\[w(\sigma)  = \mid \sigma \mid \]
\[S = \langle \emptyset , \{1\}, \{2\}, \{3\}, \{2, 1\}, \{2, 3\}
, \{1 , 2, 3\} \rangle\]

\begin{center}
\begin{tabular}{|c|c|}
\hline 
k & Number of \(\sigma\) with \(w(\sigma) = k\) \\ 
\hline 
0 & 1 \\ 
\hline 
1 & 3 \\ 
\hline 
2 & 3 \\ 
\hline 
3 & 1 \\ 
\hline 
\end{tabular} 
\end{center}

Instead of using a table we encode this information as a coefficients of a polynomial
\end{ex}

\begin{definition}
Let S be a set, suppose each \(\sigma \in S\) has a non-negative weight \(w(\sigma)\). The \underline{generating series} for S with respect to w is defined as : 
\[\phi_s(x) = \sum_{\sigma in S}x^{w(\sigma)}\]
Equivalently, this is equal to 
\[\phi_s(x) = \sum_{n \geq 0} a_n x^n \]
where \(a_n\) denotes the number of elements in S that have weight k 
\end{definition}


\begin{expx} Find the generating series of the previous example \\
\begin{center}
\begin{tabular}{|c|c|c|}
\hline 
\(\sigma\) & \(w(\sigma)\) & \(x^{w(\sigma)}\) \\ 
\hline 
$\emptyset$ & 0 & $x^0$ \\ 
\hline 
$ \{1\}$ & 1 & $x^1$ \\ 
\hline 
$ \{2\}$  & 1 & $x^1$ \\ 
\hline 
$ \{3\}$  & 1 & $x^1$ \\ 
\hline 
$ \{1, 2\}$  & 2 & $x^2$ \\ 
\hline 
$ \{1, 3\}$  & 2 & $x^2$ \\ 
\hline 
$ \{2, 3\}$  & 2 & $x^2$ \\ 
\hline 
$ \{1, 2, 3\}$  & 3 & $x^3$ \\ 
\hline 
\end{tabular} 
\(\hspace{0.2cm}\phi_S(x) = x^0 + 3x^1 + 3x^2 + x^3\)
\end{center}

In general, if S is the set of all subsets of \(\langle 1, \ldots, n \rangle\), \(w(\sigma) = \hspace{0.1cm} \mid \sigma \mid\) then,
\[\phi_S(x) = \underline{{n \choose 0}} x^0 + \underline{{n \choose 1}} x^1 + \ldots + \underline{{n \choose n}} x^n = (1 + x)^n\]
\end{expx}

\begin{theorem}
Let S be a finite set with a weight function w. 
\begin{enumerate}
\item[(a)] \(\phi_S(1) = \hspace{0.1cm}\mid S \mid\)
\item[(b)] \(\phi^{\prime}_S(1) = \) sum of all weights = \(\sum_{\sigma \in S} w(\sigma)\)
\item[(c)] \(\frac{\phi^{\prime}S(1)}{\phi_S(1)} =\) average weight = \( \frac{1}{\mid S \mid} \sum_{\sigma \in S} w (\sigma) \)
\end{enumerate}
\end{theorem}

\begin{ex} S = all subsets  of \(\langle 1, \ldots, n \rangle\) \\
\[ \begin{aligned}
& w(\sigma) = \hspace{0.1cm} \mid \sigma \mid \\
& \phi_S(x) = (1 + x)^n \\
& \phi_S(1) = (1 + 1)^n = 2^n \\
& \phi^{\prime}_S = ((1 + x)^n) = n ( 1 + x) ^{n -1} \\
& \phi^{\prime}_S = n \cdot (1 + 1)^{n - 1} = n \cdot 2^{n - 1} = \text{sum of all weights}
& \text{average} = \frac{n \cdot 2^{n -1}}{2^n} = \frac{n}{2}
\end{aligned}\]
\end{ex}

\begin{proof}
\begin{itemize}
\item[(a)] \[ \phi_S(1) = \sum_{\sigma \in S} (1)^{w(\sigma)} = \sum 1 = \mid S \mid  \]
\item[(b)] \[\phi^{\prime}_S(x) = (\sum_{\sigma \in S} x^{w(\sigma)})\prime = \sum_{\sigma \in S} w(\sigma) \cdot x^{w(\sigma) - 1} \]
\[\phi^{\prime}_S (1) = \sum_{\sigma \in S} w(\sigma) 1^{w(\sigma) -1 } = \sum_{\sigma \in S} w(\sigma)\]
\item [(c)] Follows from (a) and (b)
\end{itemize}
\end{proof}

\section{Formal Power Series}
\begin{definition}
Given a sequence \(a_n, a_1, a_3, \ldots\) of rational numbers (i.e \(\frac{2}{3}, 4,\) and not \(\pi\))
\[A(x) = a_0 + a_1 x + a_2 x^2 + \ldots = \sum_{n \geq 0} a_n x^n = \sum_{n = 0} a_n X^n\]

\begin{ex}
\[(1, 1, \ldots, ) \rightarrow P(x) = 1 + x + x^2 + x^3 + \ldots = \sum_{n = 0}^{\infty} x^n\] 
\[(0, 1, 2, 3, \ldots) \rightarrow Q(x) = 0 + x + 2x^2 + 3x^3 + \ldots = \sum_{n \geq 0 } n x^n\]
\[P(x) + Q(x) = (1 + 0) + (1 + 1)x + (1 + 2)x^2 + \ldots \]
\[
\end{ex}
\end{definition}



\end{document}
