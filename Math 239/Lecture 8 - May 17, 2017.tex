%Notes by Harsh Mistry 
%Math 239
%based on Template from : https://www.cs.cmu.edu/~ggordon/10725-F12/template.tex

\documentclass{article}
\setlength{\oddsidemargin}{0.25 in}
\setlength{\evensidemargin}{-0.25 in}
\setlength{\topmargin}{-0.6 in}
\setlength{\textwidth}{6.5 in}
\setlength{\textheight}{8.5 in}
\setlength{\headsep}{0.75 in}
\setlength{\parindent}{0 in}
\setlength{\parskip}{0.1 in}
\usepackage{amsfonts,graphicx, amssymb}
\usepackage[fleqn]{amsmath}
\usepackage{fixltx2e}
\usepackage{color}
\usepackage{tcolorbox}
\usepackage{lipsum}
\usepackage{listings}
\usepackage{scrextend}
\tcbuselibrary{skins,breakable}
\usetikzlibrary{shadings,shadows}
\newcounter{lecnum}
\renewcommand{\thepage}{\thelecnum-\arabic{page}}
\renewcommand{\thesection}{\thelecnum.\arabic{section}}
\renewcommand{\theequation}{\thelecnum.\arabic{equation}}
\renewcommand{\thefigure}{\thelecnum.\arabic{figure}}
\renewcommand{\thetable}{\thelecnum.\arabic{table}}
\newcommand{\lecture}[4]{
   \pagestyle{myheadings}
   \thispagestyle{plain}
   \newpage
   \setcounter{lecnum}{#1}
   \setcounter{page}{1}
   
   
%Info Box 
   \begin{center}
   \framebox{
      \vbox{\vspace{2mm}
    \hbox to 6.28in { {\bf Math 239 - Introduction to Combinatorics
	\hfill Spring 2017} }
       \vspace{4mm}
       \hbox to 6.28in { {\Large \hfill Lecture #1: #2  \hfill} }
       \vspace{2mm}
       \hbox to 6.28in { {\it Lecturer: #3 \hfill Notes By: #4} }
      \vspace{2mm}}
   }
   \end{center}
   
   \markboth{Lecture #1: #2}{Lecture #1: #2}



 
}

\renewcommand{\cite}[1]{[#1]}
\def\beginrefs{\begin{list}%
        {[\arabic{equation}]}{\usecounter{equation}
         \setlength{\leftmargin}{2.0truecm}\setlength{\labelsep}{0.4truecm}%
         \setlength{\labelwidth}{1.6truecm}}}
\def\endrefs{\end{list}}
\def\bibentry#1{\item[\hbox{[#1]}]}

\newcommand{\fig}[3]{
			\vspace{#2}
			\begin{center}
			Figure \thelecnum.#1:~#3
			\end{center}
	}
	
	\newcommand{\bs}{
		\texttt{\char`\\ \hspace{0.1cm}} 
	}
	

	
\newcommand{\pipe}{\(\mid\)}
\newcommand{\ctr}{\(\wedge\)}

\newtheorem{theorem}{Theorem}[lecnum]
\newtheorem{lemma}[theorem]{Lemma}
\newtheorem{ex}[theorem]{Example}
\newtheorem{expx}[theorem]{Problem}
\newtheorem{prop}[theorem]{Proposition}
\newtheorem{claim}[theorem]{Claim}
\newtheorem{corollary}[theorem]{Corollary}
\newtheorem{definition}[theorem]{Definition}
\newenvironment{proof}{{\bf Proof:}}{\hfill\rule{2mm}{2mm}}
\newcommand\E{\mathbb{E}}

%color definitions :
\definecolor{darkred}{rgb}{0.55, 0.0, 0.0}
\definecolor{lightcoral}{rgb}{0.94, 0.5, 0.5}
\definecolor{tomato}{rgb}{1.0, 0.39, 0.28}
\definecolor{lightgray}{rgb}{.9,.9,.9}
\definecolor{darkgray}{rgb}{.4,.4,.4}
\definecolor{purple}{rgb}{0.65, 0.12, 0.82}
\definecolor{lightgreen}{rgb}{0.56, 0.93, 0.56}
\definecolor{darkgreen}{rgb}{0.0, 0.2, 0.13}
\definecolor{limegreen}{rgb}{0.2, 0.8, 0.2}
\definecolor{lightblue}{rgb}{0.68, 0.85, 0.9}
\definecolor{darkblue}{rgb}{0.0, 0.0, 0.55}


%Environments
\newenvironment{exblock}[1]{%
    \tcolorbox[beamer,%
    noparskip,breakable,
    colback=lightgreen,colframe=darkgreen,%
    colbacklower=limegreen!75!lightgreen,%
    title=#1]}%
    {\endtcolorbox}

\newenvironment{ablock}[1]{%
    \tcolorbox[beamer,%
    noparskip,breakable,
    colback=lightcoral,colframe=darkred,%
    colbacklower=tomato!75!lightcoral,%
    title=#1]}%
    {\endtcolorbox}

\newenvironment{cblock}[1]{%
    \tcolorbox[beamer,%
    noparskip,breakable,
    colback=lightblue,colframe=darkblue,%
    colbacklower=darkblue!75!lightblue,%
    title=#1]}%
    {\endtcolorbox}


%Languages
\lstdefinelanguage{JavaScript}{
  keywords={typeof, new, true, false, catch, function, return, null, catch, switch, var, if, in, while, do, else, case, break},
  keywordstyle=\color{blue}\bfseries,
  ndkeywords={class, export, boolean, throw, implements, import, this},
  ndkeywordstyle=\color{darkgray}\bfseries,
  identifierstyle=\color{black},
  sensitive=false,
  comment=[l]{//},
  morecomment=[s]{/*}{*/},
  commentstyle=\color{purple}\ttfamily,
  stringstyle=\color{red}\ttfamily,
  morestring=[b]',
  morestring=[b]"
}

%Listings
\lstset{
   language=JavaScript,
   backgroundcolor=\color{lightgray},
   extendedchars=true,
   basicstyle=\footnotesize\ttfamily,
   showstringspaces=false,
   showspaces=false,
   numbers=left,
   numberstyle=\footnotesize,
   numbersep=9pt,
   tabsize=2,
   breaklines=true,
   showtabs=false,
   captionpos=b
}


%Start
\begin{document}

\lecture{8}{May 17th, 2017}{Alan Arroyo Guevara}{Harsh Mistry}
 
\begin{ablock}{Remark}
The sum lemma generalizes when \(S_0, S_1, S_2, \ldots\) is a partition of S. $$\phi_S = \sum_{n \geq 0} \phi_{S_n}(x)$$ 
\end{ablock}


\section{Compositions of an Integer}
\begin{definition}
For integer \(n \geq 0, k \geq 0\), a \underline{composition} of n is a k-tuple of positive integers \(c_1, c_2, \ldots, c_k\) such that \(c_1 + c_2 + \ldots + c_k = n\)
\end{definition}

\begin{ablock}{Remark}
\begin{itemize}
\item The \(c_i\)'s are called the parts of the composition 
\item There is one composition of 0, The empty composition
\end{itemize}
\end{ablock}

\subsection*{Tackling a Composition Problem}
\textbf{Notation :} \(\mathbb{N} = \langle 1, 2, 3, \ldots\rangle , \mathbb{N}_0 = \langle 0 , 1, 2, 3, \ldots \rangle\)

\textbf{Goal :} Craft a set S and weight function w, such that the number of compositions of n we are counting is \([x^n] \phi_S(x) \)

\textbf{Steps :}
\begin{enumerate}
\item Find a set S that fits your problem 
\begin{itemize}
\item If your compositions have k parts, let S be a product of k sets.
\item w usually is the sum of the parts
\item If the number of parts is unrestricted then
\[S = S_0 \cup S_1 \cup S_2 \cup \ldots \]
each \(S_k\) is the number of compositions with k parts that we are considering.
\end{itemize}
\item Find \(\phi_S(x)\) by applying Sum and Product Lemmas 
\item Find \([x^n] \phi_S(x)\), Using the tricks provided by the Inverse Bin. Series) 
\end{enumerate}

\newpage
\begin{expx}
Let K, \(n \in \mathbb{N}, n \geq k\). \
How many compositions of n with k parts are there?\\
\newline
\textbf{Solution : }
\begin{enumerate}
\item \(S= \frac{\mathbb{N} \times \ldots \times \mathbb{N}}{k} = \mathbb{N}^k\) \\
For \((C_1, C_2, \ldots, C_k) \in S\), let w\((C_1, \ldots, C_k)\) = \(C_1 + C_2 + \ldots + C_k\) 
\item By Product Lemma, 
\[\begin{aligned}\phi_S(x) & = \phi_{\mathbb{N} \times \ldots \times \mathbb{N} }(x)\\
 & = \phi_{\mathbb{N}} (x) \cdot \phi_{\mathbb{N}} (x) \cdot \ldots \cdot \phi_{\mathbb{N}} (x) \\
 & = \left(\phi_{\mathbb{N}} (x)\right)^k \\
 & = (x^1 + x^2 + x^3 1 \ldots)^k \\
 & = (x(x^0 + x^1 + x^2 + \ldots))^k
 & = \left(\frac{x}{1-x}\right)^k \end{aligned}\]
 \item \[\begin{aligned}
 [x^n]\left(\frac{1}{1-x}\right)^k & = [x^n](x^k)(1-k)^{-k} \\
 & = [x^{n-k}](1-x)^{-k} \\
 & = [x^{n-k}] \sum_{j = 0} {{j + k - 1} \choose {k-1}} x^j \\
 & = {{n-k+k-1} \choose {k-1}} \\
 & = {{n - 1} \choose {k-1}}
 \end{aligned}\]
 \end{enumerate}
\end{expx}

\begin{expx} How many compositions of n are there? \\
\textbf{Solution : } Next lecture 
\end{expx}

\end{document}
