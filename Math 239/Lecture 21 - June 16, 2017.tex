%Notes by Harsh Mistry 
%Math 239
%based on Template from : https://www.cs.cmu.edu/~ggordon/10725-F12/template.tex

\documentclass{article}
\setlength{\oddsidemargin}{0.25 in}
\setlength{\evensidemargin}{-0.25 in}
\setlength{\topmargin}{-0.6 in}
\setlength{\textwidth}{6.5 in}
\setlength{\textheight}{8.5 in}
\setlength{\headsep}{0.75 in}
\setlength{\parindent}{0 in}
\setlength{\parskip}{0.1 in}
\usepackage{amsfonts,graphicx, amssymb}
\usepackage[fleqn]{amsmath}
\usepackage{fixltx2e}
\usepackage{color}
\usepackage{tcolorbox}
\usepackage{lipsum}
\usepackage{listings}
\usepackage{scrextend}
\tcbuselibrary{skins,breakable}
\usetikzlibrary{shadings,shadows}
\newcounter{lecnum}
\renewcommand{\thepage}{\thelecnum-\arabic{page}}
\renewcommand{\thesection}{\thelecnum.\arabic{section}}
\renewcommand{\theequation}{\thelecnum.\arabic{equation}}
\renewcommand{\thefigure}{\thelecnum.\arabic{figure}}
\renewcommand{\thetable}{\thelecnum.\arabic{table}}
\newcommand{\lecture}[4]{
   \pagestyle{myheadings}
   \thispagestyle{plain}
   \newpage
   \setcounter{lecnum}{#1}
   \setcounter{page}{1}
   
   
%Info Box 
   \begin{center}
   \framebox{
      \vbox{\vspace{2mm}
    \hbox to 6.28in { {\bf Math 239 - Introduction to Combinatorics
	\hfill Spring 2017} }
       \vspace{4mm}
       \hbox to 6.28in { {\Large \hfill Lecture #1: #2  \hfill} }
       \vspace{2mm}
       \hbox to 6.28in { {\it Lecturer: #3 \hfill Notes By: #4} }
      \vspace{2mm}}
   }
   \end{center}
   
   \markboth{Lecture #1: #2}{Lecture #1: #2}



 
}

\renewcommand{\cite}[1]{[#1]}
\def\beginrefs{\begin{list}%
        {[\arabic{equation}]}{\usecounter{equation}
         \setlength{\leftmargin}{2.0truecm}\setlength{\labelsep}{0.4truecm}%
         \setlength{\labelwidth}{1.6truecm}}}
\def\endrefs{\end{list}}
\def\bibentry#1{\item[\hbox{[#1]}]}

\newcommand{\fig}[3]{
			\vspace{#2}
			\begin{center}
			Figure \thelecnum.#1:~#3
			\end{center}
	}
	
	\newcommand{\bs}{
		\texttt{\char`\\ \hspace{0.1cm}} 
	}
	

	
\newcommand{\pipe}{\(\mid\)}
\newcommand{\ctr}{\(\wedge\)}

\newtheorem{theorem}{Theorem}[lecnum]
\newtheorem{lemma}[theorem]{Lemma}
\newtheorem{ex}[theorem]{Example}
\newtheorem{expx}[theorem]{Problem}
\newtheorem{prop}[theorem]{Proposition}
\newtheorem{claim}[theorem]{Claim}
\newtheorem{corollary}[theorem]{Corollary}
\newtheorem{definition}[theorem]{Definition}
\newenvironment{proof}{{\bf Proof:}}{\hfill\rule{2mm}{2mm}}
\newcommand\E{\mathbb{E}}

%color definitions :
\definecolor{darkred}{rgb}{0.55, 0.0, 0.0}
\definecolor{lightcoral}{rgb}{0.94, 0.5, 0.5}
\definecolor{tomato}{rgb}{1.0, 0.39, 0.28}
\definecolor{lightgray}{rgb}{.9,.9,.9}
\definecolor{darkgray}{rgb}{.4,.4,.4}
\definecolor{purple}{rgb}{0.65, 0.12, 0.82}
\definecolor{lightgreen}{rgb}{0.56, 0.93, 0.56}
\definecolor{darkgreen}{rgb}{0.0, 0.2, 0.13}
\definecolor{limegreen}{rgb}{0.2, 0.8, 0.2}
\definecolor{lightblue}{rgb}{0.68, 0.85, 0.9}
\definecolor{darkblue}{rgb}{0.0, 0.0, 0.55}


%Environments
\newenvironment{exblock}[1]{%
    \tcolorbox[beamer,%
    noparskip,breakable,
    colback=lightgreen,colframe=darkgreen,%
    colbacklower=limegreen!75!lightgreen,%
    title=#1]}%
    {\endtcolorbox}

\newenvironment{ablock}[1]{%
    \tcolorbox[beamer,%
    noparskip,breakable,
    colback=lightcoral,colframe=darkred,%
    colbacklower=tomato!75!lightcoral,%
    title=#1]}%
    {\endtcolorbox}

\newenvironment{cblock}[1]{%
    \tcolorbox[beamer,%
    noparskip,breakable,
    colback=lightblue,colframe=darkblue,%
    colbacklower=darkblue!75!lightblue,%
    title=#1]}%
    {\endtcolorbox}


%Languages
\lstdefinelanguage{JavaScript}{
  keywords={typeof, new, true, false, catch, function, return, null, catch, switch, var, if, in, while, do, else, case, break},
  keywordstyle=\color{blue}\bfseries,
  ndkeywords={class, export, boolean, throw, implements, import, this},
  ndkeywordstyle=\color{darkgray}\bfseries,
  identifierstyle=\color{black},
  sensitive=false,
  comment=[l]{//},
  morecomment=[s]{/*}{*/},
  commentstyle=\color{purple}\ttfamily,
  stringstyle=\color{red}\ttfamily,
  morestring=[b]',
  morestring=[b]"
}

%Listings
\lstset{
   language=JavaScript,
   backgroundcolor=\color{lightgray},
   extendedchars=true,
   basicstyle=\footnotesize\ttfamily,
   showstringspaces=false,
   showspaces=false,
   numbers=left,
   numberstyle=\footnotesize,
   numbersep=9pt,
   tabsize=2,
   breaklines=true,
   showtabs=false,
   captionpos=b
}


%Start
\begin{document}

\lecture{21}{June 16th, 2017}{Alan Arroyo Guevara}{Harsh Mistry}

\begin{definition} Let \(X \subseteq V (G), (\emptyset \subset X \subset V(G))\) and let \(y = V(G) \backslash X\)\\
The \underline{cut induced by X} (denoted as \(\delta(x)\))1 is the set of edges that have one end in X and the other end in y.  
\end{definition}

\begin{theorem}
A graph is not connected if and only if there exists \(X \subseteq V(G)\) with \(\emptyset \subset X \subset V(G)\), such that \(\delta(X)\) is not empty.
\end{theorem}

\begin{proof}
\begin{itemize}
\item [\(\leftarrow\)] Suppose G is connected. Let \(X \subseteq V(G)\) and \(\emptyset \subset X \subset V(G)\)\\
Let \(y = V \backslash X\). Pick \(u \in X\) and \(v \in y\). Since G is reconnected there is a path 
$$P : u x_1, x_2, \ldots, x_{n-1} V$$
As \(u \in X\) and \(v \in y\), there is a first \(i \in \langle 1, 2, \ldots, n \rangle\), such that 
$$ x_{i+1} \in y$$
$$ x_i \in X \implies X_i X_{i+1} \in \delta(X)$$
\item[\(\rightarrow\)] Suppose G is disconnected. Let C be a component of G and Let \(X = V(C)\).\\
Since, G is not connected, it has atleast two components \(\emptyset \subset  X 1\subset V(G)\). \\
Now, let \(y = V \backslash X\). Every edge with \(x \in X\),
has \( y \in X\) has \(x\) and \(j\)are in the same  component  
\end{itemize}
\end{proof}

\section{Bridges}

\textbf{Notation :} If \(e \in E(G)\) we denote \(G -e \) (or \(G \backslash e\)) the graph whose vertex if V(G) and whose edge set is \(E(G) \backslash \{e\}\). (So \(G -e\) is the graph obtained from G by deleting the edge \(e\). ) 

\begin{definition}
A \textbf{Bridge} is an edge \(e \in E(G)\) such that \(G -e\) has more components than G. 
\end{definition}

\begin{lemma}
Let G be a connected graph and \(e = xy \in E(G) \) a bride. Then \(G - e\) has two components one including x and the others including y.
\end{lemma}

\begin{proof}
Let \(z \in V(G-e) = V(G)\)
Since G is connected, there is a path \(P : z v_1, v_2, \ldots v_{n-1} X\).
\begin{itemize}
\item [\textbf{Case 1} :] In this case, P is also a path in \(G - e\), so \(z\) and \(x\) are in the same components of \(G, r\)
\item [\textbf{Case 2} :] In this case \(z - V_{n-1} X\), so\(V_{n-1} = y \). As a result, \(P^\prime = Z v_1 v_2 \ldots, V_{n-2} y\) is a path 
\end{itemize}
In any case \(z \in C_x\) or \(z \in C_y\). Where \(C_x\) is the component including \(x\)and \(C_y\) is the component including \(y\). \\
Since \(G-e\) is disconnected \(C_x \neq C_y\) 
\end{proof}


\end{document} 