%Notes by Harsh Mistry 
%Math 239
%based on Template from : https://www.cs.cmu.edu/~ggordon/10725-F12/template.tex

\documentclass{article}
\setlength{\oddsidemargin}{0.25 in}
\setlength{\evensidemargin}{-0.25 in}
\setlength{\topmargin}{-0.6 in}
\setlength{\textwidth}{6.5 in}
\setlength{\textheight}{8.5 in}
\setlength{\headsep}{0.75 in}
\setlength{\parindent}{0 in}
\setlength{\parskip}{0.1 in}
\usepackage{amsfonts,graphicx, amssymb}
\usepackage[fleqn]{amsmath}
\usepackage{fixltx2e}
\usepackage{color}
\usepackage{tcolorbox}
\usepackage{lipsum}
\usepackage{listings}
\usepackage{scrextend}
\tcbuselibrary{skins,breakable}
\usetikzlibrary{shadings,shadows}
\newcounter{lecnum}
\renewcommand{\thepage}{\thelecnum-\arabic{page}}
\renewcommand{\thesection}{\thelecnum.\arabic{section}}
\renewcommand{\theequation}{\thelecnum.\arabic{equation}}
\renewcommand{\thefigure}{\thelecnum.\arabic{figure}}
\renewcommand{\thetable}{\thelecnum.\arabic{table}}
\newcommand{\lecture}[4]{
   \pagestyle{myheadings}
   \thispagestyle{plain}
   \newpage
   \setcounter{lecnum}{#1}
   \setcounter{page}{1}
   
   
%Info Box 
   \begin{center}
   \framebox{
      \vbox{\vspace{2mm}
    \hbox to 6.28in { {\bf Math 239 - Introduction to Combinatorics
	\hfill Spring 2017} }
       \vspace{4mm}
       \hbox to 6.28in { {\Large \hfill Lecture #1: #2  \hfill} }
       \vspace{2mm}
       \hbox to 6.28in { {\it Lecturer: #3 \hfill Notes By: #4} }
      \vspace{2mm}}
   }
   \end{center}
   
   \markboth{Lecture #1: #2}{Lecture #1: #2}



 
}

\renewcommand{\cite}[1]{[#1]}
\def\beginrefs{\begin{list}%
        {[\arabic{equation}]}{\usecounter{equation}
         \setlength{\leftmargin}{2.0truecm}\setlength{\labelsep}{0.4truecm}%
         \setlength{\labelwidth}{1.6truecm}}}
\def\endrefs{\end{list}}
\def\bibentry#1{\item[\hbox{[#1]}]}

\newcommand{\fig}[3]{
			\vspace{#2}
			\begin{center}
			Figure \thelecnum.#1:~#3
			\end{center}
	}
	
	\newcommand{\bs}{
		\texttt{\char`\\ \hspace{0.1cm}} 
	}
	

	
\newcommand{\pipe}{\(\mid\)}
\newcommand{\ctr}{\(\wedge\)}

\newtheorem{theorem}{Theorem}[lecnum]
\newtheorem{lemma}[theorem]{Lemma}
\newtheorem{ex}[theorem]{Example}
\newtheorem{expx}[theorem]{Problem}
\newtheorem{prop}[theorem]{Proposition}
\newtheorem{claim}[theorem]{Claim}
\newtheorem{corollary}[theorem]{Corollary}
\newtheorem{definition}[theorem]{Definition}
\newenvironment{proof}{{\bf Proof:}}{\hfill\rule{2mm}{2mm}}
\newcommand\E{\mathbb{E}}

%color definitions :
\definecolor{darkred}{rgb}{0.55, 0.0, 0.0}
\definecolor{lightcoral}{rgb}{0.94, 0.5, 0.5}
\definecolor{tomato}{rgb}{1.0, 0.39, 0.28}
\definecolor{lightgray}{rgb}{.9,.9,.9}
\definecolor{darkgray}{rgb}{.4,.4,.4}
\definecolor{purple}{rgb}{0.65, 0.12, 0.82}
\definecolor{lightgreen}{rgb}{0.56, 0.93, 0.56}
\definecolor{darkgreen}{rgb}{0.0, 0.2, 0.13}
\definecolor{limegreen}{rgb}{0.2, 0.8, 0.2}
\definecolor{lightblue}{rgb}{0.68, 0.85, 0.9}
\definecolor{darkblue}{rgb}{0.0, 0.0, 0.55}


%Environments
\newenvironment{exblock}[1]{%
    \tcolorbox[beamer,%
    noparskip,breakable,
    colback=lightgreen,colframe=darkgreen,%
    colbacklower=limegreen!75!lightgreen,%
    title=#1]}%
    {\endtcolorbox}

\newenvironment{ablock}[1]{%
    \tcolorbox[beamer,%
    noparskip,breakable,
    colback=lightcoral,colframe=darkred,%
    colbacklower=tomato!75!lightcoral,%
    title=#1]}%
    {\endtcolorbox}

\newenvironment{cblock}[1]{%
    \tcolorbox[beamer,%
    noparskip,breakable,
    colback=lightblue,colframe=darkblue,%
    colbacklower=darkblue!75!lightblue,%
    title=#1]}%
    {\endtcolorbox}


%Languages
\lstdefinelanguage{JavaScript}{
  keywords={typeof, new, true, false, catch, function, return, null, catch, switch, var, if, in, while, do, else, case, break},
  keywordstyle=\color{blue}\bfseries,
  ndkeywords={class, export, boolean, throw, implements, import, this},
  ndkeywordstyle=\color{darkgray}\bfseries,
  identifierstyle=\color{black},
  sensitive=false,
  comment=[l]{//},
  morecomment=[s]{/*}{*/},
  commentstyle=\color{purple}\ttfamily,
  stringstyle=\color{red}\ttfamily,
  morestring=[b]',
  morestring=[b]"
}

%Listings
\lstset{
   language=JavaScript,
   backgroundcolor=\color{lightgray},
   extendedchars=true,
   basicstyle=\footnotesize\ttfamily,
   showstringspaces=false,
   showspaces=false,
   numbers=left,
   numberstyle=\footnotesize,
   numbersep=9pt,
   tabsize=2,
   breaklines=true,
   showtabs=false,
   captionpos=b
}


%Start
\begin{document}

\lecture{12}{May 26th, 2017}{Alan Arroyo Guevara}{Harsh Mistry}

\begin{ablock}{Remark}
In general. an unambiguous expression is one for which every string can be constructed in a unique way by the rules in the expression.\\
\newline 
Sometimes it is better to decompose after each occurrence or block of 1's 
\end{ablock}

\section{Tackling Binary problems}
\begin{enumerate}
\item Decompose your set S using decomposition rules.
\item Find \([x^4] \phi_S(x)\)
\end{enumerate}

\begin{expx}
Show that the number of binary strings of length n, where no block has length exactly two is equal to 
$$ [x^n] \frac{1-x^2+x^3}{1 - 2x + x^2 - x^3} $$
\end{expx}
\textbf{Solution:}
\begin{enumerate}
\item \[S = (\{\epsilon, 0\} \cup \{000\}\{0\}^*)(\{1, 111, 1111,\ldots\}\{0, 000, 0000\})^* (\{\epsilon, 1\} \cup \{111\}\{1\}^*) \]
\item 
\[\begin{aligned}
\phi_{S_1}(x) & = 1 + x + x^3 + x^4 + \ldots \\
& = 1 + x + x^3 (1 + x + x^2 + \ldots)\\
& = 1 + x + \frac{x^3}{1-x} \\
& = \frac{1-x^2 + x^3}{1-x}
\end{aligned}\]
\[\phi_{S_3}(x) = \frac{1-x^2+x^3}{1-x}\]
\[\begin{aligned}\phi_{S_2} (x) & = \phi_{\{1, 111, \ldots\}}(x) \phi_{\{0,000,0000\}} \\
& = (x+x^3+x^4+ \ldots) (x + x^3 + x^4 \ldots)\\
& = (x +x^3 + (1 + x + x^2 + \ldots))^2 \\
& = \left(x + \frac{x^3}{1-x}\right)^2 \\
& = \left( \frac{x-x^2+x^3}{1-x}\right)^2 \\
\end{aligned}\]
\[\begin{aligned}
\phi_S(x)&  = \phi_{S_1}(x) \phi_{S_2}(x) \phi_{S_3}(x) \\
& = \frac{1-x^2+x^3}{1-x}\frac{1}{1-\left(\frac{x-x^2+x^3}{1-x}\right)^2} \frac{1-x^2 + x^3}{1-x} \\
& = \ldots\\
& = \frac{1-x^2+x^3}{1-2x+x^2-x^3}
\end{aligned}\]
\end{enumerate}

\begin{expx}
Find the number of binary strings of length n in which 
\begin{itemize}
\item Every even block of 0's is followed by an odd number of 1's.
\item Every odd block of 0's is followed by  an even number of 1s (possibly followed by zero 1's) 
\end{itemize}

\textbf{Solution : }
\begin{enumerate}
\item \[S = \{1\}^* (\{00\}\{00\}^*\{1\}\{11\}^*\cup \{0\}\{00\}^*\{11\}\{11\}^*)^* (\{\epsilon\} \cup \{0\}\{00\}^*)\]
\item \[\phi_{S_1} = \frac{1}{ 1- \phi_{S_1}(x)} = \frac{1}{1-x}\]
\[\phi_{S_2} (x) = x^2 \frac{1}{1-x^2} x \frac{1}{1-x^2}\]
\[\phi_{S_3} (x) = x \frac{1}{1-x^2} x^2 \frac{1}{1-x^2}\]

\end{enumerate}
\end{expx}

\end{document}


