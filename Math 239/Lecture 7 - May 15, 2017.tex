%Notes by Harsh Mistry 
%Math 239
%based on Template from : https://www.cs.cmu.edu/~ggordon/10725-F12/template.tex

\documentclass{article}
\setlength{\oddsidemargin}{0.25 in}
\setlength{\evensidemargin}{-0.25 in}
\setlength{\topmargin}{-0.6 in}
\setlength{\textwidth}{6.5 in}
\setlength{\textheight}{8.5 in}
\setlength{\headsep}{0.75 in}
\setlength{\parindent}{0 in}
\setlength{\parskip}{0.1 in}
\usepackage{amsfonts,graphicx, amssymb}
\usepackage[fleqn]{amsmath}
\usepackage{fixltx2e}
\usepackage{color}
\usepackage{tcolorbox}
\usepackage{lipsum}
\usepackage{listings}
\usepackage{scrextend}
\tcbuselibrary{skins,breakable}
\usetikzlibrary{shadings,shadows}
\newcounter{lecnum}
\renewcommand{\thepage}{\thelecnum-\arabic{page}}
\renewcommand{\thesection}{\thelecnum.\arabic{section}}
\renewcommand{\theequation}{\thelecnum.\arabic{equation}}
\renewcommand{\thefigure}{\thelecnum.\arabic{figure}}
\renewcommand{\thetable}{\thelecnum.\arabic{table}}
\newcommand{\lecture}[4]{
   \pagestyle{myheadings}
   \thispagestyle{plain}
   \newpage
   \setcounter{lecnum}{#1}
   \setcounter{page}{1}
   
   
%Info Box 
   \begin{center}
   \framebox{
      \vbox{\vspace{2mm}
    \hbox to 6.28in { {\bf Math 239 - Introduction to Combinatorics
	\hfill Spring 2017} }
       \vspace{4mm}
       \hbox to 6.28in { {\Large \hfill Lecture #1: #2  \hfill} }
       \vspace{2mm}
       \hbox to 6.28in { {\it Lecturer: #3 \hfill Notes By: #4} }
      \vspace{2mm}}
   }
   \end{center}
   
   \markboth{Lecture #1: #2}{Lecture #1: #2}



 
}

\renewcommand{\cite}[1]{[#1]}
\def\beginrefs{\begin{list}%
        {[\arabic{equation}]}{\usecounter{equation}
         \setlength{\leftmargin}{2.0truecm}\setlength{\labelsep}{0.4truecm}%
         \setlength{\labelwidth}{1.6truecm}}}
\def\endrefs{\end{list}}
\def\bibentry#1{\item[\hbox{[#1]}]}

\newcommand{\fig}[3]{
			\vspace{#2}
			\begin{center}
			Figure \thelecnum.#1:~#3
			\end{center}
	}
	
	\newcommand{\bs}{
		\texttt{\char`\\ \hspace{0.1cm}} 
	}
	

	
\newcommand{\pipe}{\(\mid\)}
\newcommand{\ctr}{\(\wedge\)}

\newtheorem{theorem}{Theorem}[lecnum]
\newtheorem{lemma}[theorem]{Lemma}
\newtheorem{ex}[theorem]{Example}
\newtheorem{expx}[theorem]{Problem}
\newtheorem{prop}[theorem]{Proposition}
\newtheorem{claim}[theorem]{Claim}
\newtheorem{corollary}[theorem]{Corollary}
\newtheorem{definition}[theorem]{Definition}
\newenvironment{proof}{{\bf Proof:}}{\hfill\rule{2mm}{2mm}}
\newcommand\E{\mathbb{E}}

%color definitions :
\definecolor{darkred}{rgb}{0.55, 0.0, 0.0}
\definecolor{lightcoral}{rgb}{0.94, 0.5, 0.5}
\definecolor{tomato}{rgb}{1.0, 0.39, 0.28}
\definecolor{lightgray}{rgb}{.9,.9,.9}
\definecolor{darkgray}{rgb}{.4,.4,.4}
\definecolor{purple}{rgb}{0.65, 0.12, 0.82}
\definecolor{lightgreen}{rgb}{0.56, 0.93, 0.56}
\definecolor{darkgreen}{rgb}{0.0, 0.2, 0.13}
\definecolor{limegreen}{rgb}{0.2, 0.8, 0.2}
\definecolor{lightblue}{rgb}{0.68, 0.85, 0.9}
\definecolor{darkblue}{rgb}{0.0, 0.0, 0.55}


%Environments
\newenvironment{exblock}[1]{%
    \tcolorbox[beamer,%
    noparskip,breakable,
    colback=lightgreen,colframe=darkgreen,%
    colbacklower=limegreen!75!lightgreen,%
    title=#1]}%
    {\endtcolorbox}

\newenvironment{ablock}[1]{%
    \tcolorbox[beamer,%
    noparskip,breakable,
    colback=lightcoral,colframe=darkred,%
    colbacklower=tomato!75!lightcoral,%
    title=#1]}%
    {\endtcolorbox}

\newenvironment{cblock}[1]{%
    \tcolorbox[beamer,%
    noparskip,breakable,
    colback=lightblue,colframe=darkblue,%
    colbacklower=darkblue!75!lightblue,%
    title=#1]}%
    {\endtcolorbox}


%Languages
\lstdefinelanguage{JavaScript}{
  keywords={typeof, new, true, false, catch, function, return, null, catch, switch, var, if, in, while, do, else, case, break},
  keywordstyle=\color{blue}\bfseries,
  ndkeywords={class, export, boolean, throw, implements, import, this},
  ndkeywordstyle=\color{darkgray}\bfseries,
  identifierstyle=\color{black},
  sensitive=false,
  comment=[l]{//},
  morecomment=[s]{/*}{*/},
  commentstyle=\color{purple}\ttfamily,
  stringstyle=\color{red}\ttfamily,
  morestring=[b]',
  morestring=[b]"
}

%Listings
\lstset{
   language=JavaScript,
   backgroundcolor=\color{lightgray},
   extendedchars=true,
   basicstyle=\footnotesize\ttfamily,
   showstringspaces=false,
   showspaces=false,
   numbers=left,
   numberstyle=\footnotesize,
   numbersep=9pt,
   tabsize=2,
   breaklines=true,
   showtabs=false,
   captionpos=b
}


%Start
\begin{document}

\lecture{7}{May 15th, 2017}{Alan Arroyo Guevara}{Harsh Mistry}
 
\section{Product and Sum Lemmas} 
 
\textbf{Recall : } Given a set S and a weight function w 
\[ \phi_s(x) = \sum^{w(\sigma)} = \sum_{n \geq 0} a_n x^n \] 

\begin{lemma} Sum Lemma \\
Let S be a set with a weight function w. Suppose \(S_0, S_1, \ldots, S_k\) is a partition of S. Then, 
\[ \phi_S(x) = \phi_{S_0}(x) + \phi_{S_1}(x) + \ldots + \phi_{S_k}(x)\]
\end{lemma}

\begin{proof}
\[\begin{aligned}
\phi_S(x) = \sum_{\sigma \in S} X^{w(\sigma)} 
& = \sum_{\sigma \in S_0} X^{w(\sigma)} + 
\sum_{\sigma \in S_1} 
X^{w(\sigma)} + \ldots + \sum_{\sigma \in S_k} X^{w(\sigma)} \\
& = \phi_{S_0} (x) + \phi_{S_1}(x) + \ldots + \phi_{S_k} (x)
\end{aligned}\]
\end{proof}

\begin{ex}
Let \(S = \mathbb{N}_0 = \langle 0, 1, 2, 3, \ldots \rangle \)
\[w(\sigma) = \begin{cases} \frac{\sigma}{2} \hspace{0.3cm} \sigma \text{ is even}  \\ \sigma \hspace{0.3cm} \text{ is odd} \end{cases}\]
\(w(2) = 1, w(3) = 3\), now find \(\phi_S(x)\)

\textbf{Solution :} \(S = E \cup O\)
\[E = \langle 0, 2, 4, \ldots \rangle \hspace{0.4cm} O = \langle 1, 3, 5, 7, \rangle \] 
\[\begin{aligned}
\phi_E(x) & = x^0 + x^1 + x^2 + \ldots = \frac{1}{1-x} \\
\phi_O(x) & = x^1 + x^3 + x^5 + x^7 + \ldots  \\ & = x (x^0 + x^2 + x + x^4+ \ldots) \\ & = x \cdot \left(\frac{1}{1 - x^2}\right)
\end{aligned}\]
By Sum Lemma 
\[\phi_S(x) = \phi_E(x) + \phi_O(x) = \frac{1}{1-x} + \frac{x}{1-x^2}\]
\end{ex}

\begin{lemma}
Product Lemma \\
Let A, B be sets.\\
Suppose A has weight function \(\alpha\) \\
Suppose B has weight function \(\beta\) \\
Let S = A \(\times\) B and suppose that for every (a, b) \(\in S\), 
\[\phi ((a,b)) = \alpha(a) + \beta(b) \implies \phi_S(x) = \phi_A(x) \cdot \phi)_B(x)\] 
\end{lemma}

\begin{proof}
\[ \begin{aligned} \phi_A(x) \cdot \phi_B(x)  & = \left( \sum_{a \in A} X^{\alpha(a)} \right) \left(\sum_{b \in B} X^{\beta(b)}\right) \\ 
& = \sum_{a \in A , b \in B} X^{\alpha(a)} \cdot X^{\beta(b)} \\ & = \sum_{(a, b) \in A \times B} X^{\alpha(a) + \beta(b)} \\& = \sum_{(a,b) \in S} X^{w((a, b))} \\ & = \phi_S (x) \end{aligned}\]
\end{proof}

\begin{expx}
We throw a die (with values 1, 2, 3, 4, 5, 6) and a 4-sided die (With values 1 , 2, 3, 4). Given \(k \in \langle 2, 3, \ldots, 10\rangle\). In how many ways can we get k, as the subset of the dice. \\ \newline 
\textbf{Equivalent Problem :} Let A = \(\langle 1, \ldots, 6 \rangle\), \(S = \langle 1, 2, 3, 4\rangle\), \(S = A \times B\), and w((a,b)) = a + b. \\
Find \([x^k] \phi_s(x)\).  \\ \newline
\textbf{Brute Force Solution : } \\ 
\begin{center}
\begin{tabular}{|c|c|c|c|c|}
\hline 
6 sided / 4 sided & 1 & 2 & 3 & 4 \\ 
\hline 
1 & $ x^2$ & $ x^3 $ & $x^4 $ & $ x^5 $  \\ 
\hline 
2 & $ x^3 $ & $x^4 $ & $ x^5 $  & $ x^6 $  \\ 
\hline 
3 & $x^4 $ & $ x^5 $  & $ x^6 $  & $x^7 $ \\ 
\hline 
4 & $ x^5 $  & $ x^6 $  & $x^7 $ & $x^8$ \\ 
\hline 
5 & $ x^6 $  & $x^7 $ & $x^8$ & $x^9$ \\ 
\hline 
6 & $x^7 $ & $x^8$ & $x^9$ & $ x^{10} $  \\ 
\hline 
\end{tabular} 
$$\phi_S(x) = x^2 + 2x^3 + 3x^4 + 4x^5 + 4x^6 + 4x^7 + 3x^9 + x^10$$
\end{center}
\textbf{Using the Product Lemma : } \\
A = \(\langle 1, 2, \ldots, 6 \rangle\) and 
B = \(\langle 1, 2, 3, 4 \rangle \) \\
\[ \begin{aligned}
\phi_S(x) & = \phi_A(x) \cdot \phi_B(x) \\
& = (x^1 + x^2 + \ldots + x^6 ) (x^1 + x^2 + x^3 + x^4) 
\end{aligned}\]
\end{expx} 




\end{document}
