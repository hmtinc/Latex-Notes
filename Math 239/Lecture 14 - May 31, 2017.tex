%Notes by Harsh Mistry 
%Math 239
%based on Template from : https://www.cs.cmu.edu/~ggordon/10725-F12/template.tex

\documentclass{article}
\setlength{\oddsidemargin}{0.25 in}
\setlength{\evensidemargin}{-0.25 in}
\setlength{\topmargin}{-0.6 in}
\setlength{\textwidth}{6.5 in}
\setlength{\textheight}{8.5 in}
\setlength{\headsep}{0.75 in}
\setlength{\parindent}{0 in}
\setlength{\parskip}{0.1 in}
\usepackage{amsfonts,graphicx, amssymb}
\usepackage[fleqn]{amsmath}
\usepackage{fixltx2e}
\usepackage{color}
\usepackage{tcolorbox}
\usepackage{lipsum}
\usepackage{listings}
\usepackage{scrextend}
\tcbuselibrary{skins,breakable}
\usetikzlibrary{shadings,shadows}
\newcounter{lecnum}
\renewcommand{\thepage}{\thelecnum-\arabic{page}}
\renewcommand{\thesection}{\thelecnum.\arabic{section}}
\renewcommand{\theequation}{\thelecnum.\arabic{equation}}
\renewcommand{\thefigure}{\thelecnum.\arabic{figure}}
\renewcommand{\thetable}{\thelecnum.\arabic{table}}
\newcommand{\lecture}[4]{
   \pagestyle{myheadings}
   \thispagestyle{plain}
   \newpage
   \setcounter{lecnum}{#1}
   \setcounter{page}{1}
   
   
%Info Box 
   \begin{center}
   \framebox{
      \vbox{\vspace{2mm}
    \hbox to 6.28in { {\bf Math 239 - Introduction to Combinatorics
	\hfill Spring 2017} }
       \vspace{4mm}
       \hbox to 6.28in { {\Large \hfill Lecture #1: #2  \hfill} }
       \vspace{2mm}
       \hbox to 6.28in { {\it Lecturer: #3 \hfill Notes By: #4} }
      \vspace{2mm}}
   }
   \end{center}
   
   \markboth{Lecture #1: #2}{Lecture #1: #2}



 
}

\renewcommand{\cite}[1]{[#1]}
\def\beginrefs{\begin{list}%
        {[\arabic{equation}]}{\usecounter{equation}
         \setlength{\leftmargin}{2.0truecm}\setlength{\labelsep}{0.4truecm}%
         \setlength{\labelwidth}{1.6truecm}}}
\def\endrefs{\end{list}}
\def\bibentry#1{\item[\hbox{[#1]}]}

\newcommand{\fig}[3]{
			\vspace{#2}
			\begin{center}
			Figure \thelecnum.#1:~#3
			\end{center}
	}
	
	\newcommand{\bs}{
		\texttt{\char`\\ \hspace{0.1cm}} 
	}
	

	
\newcommand{\pipe}{\(\mid\)}
\newcommand{\ctr}{\(\wedge\)}

\newtheorem{theorem}{Theorem}[lecnum]
\newtheorem{lemma}[theorem]{Lemma}
\newtheorem{ex}[theorem]{Example}
\newtheorem{expx}[theorem]{Problem}
\newtheorem{prop}[theorem]{Proposition}
\newtheorem{claim}[theorem]{Claim}
\newtheorem{corollary}[theorem]{Corollary}
\newtheorem{definition}[theorem]{Definition}
\newenvironment{proof}{{\bf Proof:}}{\hfill\rule{2mm}{2mm}}
\newcommand\E{\mathbb{E}}

%color definitions :
\definecolor{darkred}{rgb}{0.55, 0.0, 0.0}
\definecolor{lightcoral}{rgb}{0.94, 0.5, 0.5}
\definecolor{tomato}{rgb}{1.0, 0.39, 0.28}
\definecolor{lightgray}{rgb}{.9,.9,.9}
\definecolor{darkgray}{rgb}{.4,.4,.4}
\definecolor{purple}{rgb}{0.65, 0.12, 0.82}
\definecolor{lightgreen}{rgb}{0.56, 0.93, 0.56}
\definecolor{darkgreen}{rgb}{0.0, 0.2, 0.13}
\definecolor{limegreen}{rgb}{0.2, 0.8, 0.2}
\definecolor{lightblue}{rgb}{0.68, 0.85, 0.9}
\definecolor{darkblue}{rgb}{0.0, 0.0, 0.55}


%Environments
\newenvironment{exblock}[1]{%
    \tcolorbox[beamer,%
    noparskip,breakable,
    colback=lightgreen,colframe=darkgreen,%
    colbacklower=limegreen!75!lightgreen,%
    title=#1]}%
    {\endtcolorbox}

\newenvironment{ablock}[1]{%
    \tcolorbox[beamer,%
    noparskip,breakable,
    colback=lightcoral,colframe=darkred,%
    colbacklower=tomato!75!lightcoral,%
    title=#1]}%
    {\endtcolorbox}

\newenvironment{cblock}[1]{%
    \tcolorbox[beamer,%
    noparskip,breakable,
    colback=lightblue,colframe=darkblue,%
    colbacklower=darkblue!75!lightblue,%
    title=#1]}%
    {\endtcolorbox}


%Languages
\lstdefinelanguage{JavaScript}{
  keywords={typeof, new, true, false, catch, function, return, null, catch, switch, var, if, in, while, do, else, case, break},
  keywordstyle=\color{blue}\bfseries,
  ndkeywords={class, export, boolean, throw, implements, import, this},
  ndkeywordstyle=\color{darkgray}\bfseries,
  identifierstyle=\color{black},
  sensitive=false,
  comment=[l]{//},
  morecomment=[s]{/*}{*/},
  commentstyle=\color{purple}\ttfamily,
  stringstyle=\color{red}\ttfamily,
  morestring=[b]',
  morestring=[b]"
}

%Listings
\lstset{
   language=JavaScript,
   backgroundcolor=\color{lightgray},
   extendedchars=true,
   basicstyle=\footnotesize\ttfamily,
   showstringspaces=false,
   showspaces=false,
   numbers=left,
   numberstyle=\footnotesize,
   numbersep=9pt,
   tabsize=2,
   breaklines=true,
   showtabs=false,
   captionpos=b
}


%Start
\begin{document}

\lecture{14}{May 31st, 2017}{Alan Arroyo Guevara}{Harsh Mistry}
\section{Homogeneous Linear Recurrences}
\begin{definition}
A sequence \(\{a_n\}_{n \geq 0}\) is defined by a linear homogeneous recurrence, if for some integer \(n \geq k\), 
$$a_n + q_1 a_{n-1} + \ldots + q_k a_{n-k} = 0 $$
and initial conditions \(a_0, a_1, a_2, \ldots, a_{k -1}\) are given. The characteristic polynomial of this recurrence is :
$$C(x) = x^k + q_1 x^{k+1} + q_2 x^{k+2} + \ldots + q_k $$
\end{definition}

\begin{theorem}
Let \(\beta_1, \ldots, \beta_j\) be distinct roots of C(x) and suppose \(m_1, m_2, \ldots, m_j\) are such that 
$$ C(x) = (x - \beta_1)^m (x - \beta_2)^m \ldots (x - \beta_j)^m $$
Then, 
$$a_n = P_1(n) \beta^n_1 + P_2(n) \beta^n_2 + \ldots + P_j(n) \beta_j^n $$
Where for each i, \(P_i(n)\) is a polynomial on n of degree less than \(m_1\), whose coefficients are determined by the initial values \(a_0, a_1, \ldots, a_{k-1}\)
\end{theorem}


\begin{expx}
Find \(a_n\) explicitly where
$$ a_n - 4a_{n-1} - 5a_{n -2} + 2 a_{n -3} \hspace{0.5cm} (n\geq3) $$
$$ a_0 = 4, a_1 = 9, a_2 = 17$$
\textbf{Solution:}
\begin{enumerate}
\item Find C(x) 
\[a_n - 4 a_{n-1}  + 5 a_{n -2} - 2a_{n-3} = 0\]
\[C(x) = x^3 - 4x^2 + 5x - 2\]
\item Factor C(x)
\[ C(1) = 0, \text{ so 1 is a root } \]
\[ \begin{aligned}
C(x) & = (x-1)(x^2 - 3x + 2) \\
& = (x-1) (x-2) (x -1) \\
& = (x-1)^2 (x-2)\\
\text{Roots Are : } & \beta_1 = 1, \beta_2 = 2, m_1 = 2, m_2 = 1 
\end{aligned}\]
\item Apply the Theorem that solves L.H.R 
\[a_n = A+Bn (1)^m + C (2)^n = A+Bn + C (2)^n\]
\item Obtain unknowns by using initial conditions
\begin{itemize}
\item \(4 = a_0 = A + B(0) + C \cdot 2^0 = A + C\)
\item \(9 = a_1 = A + B(1) + C \cdot 2^1 = A + B + 2C\)
\item \(17 = a_2 = A + B(2) + C \cdot 2^2 = A + 2B + 4C\)
\end{itemize}
Solve the linear system A = 1, B = 2, C = 3. So, 
$$ a_n = 1 + 2n + 3(2^n) $$
\end{enumerate}
\end{expx}

\section{From Recurrences to Coefficients}
\begin{expx}
Find the formal power series \(\sum_{n\geq j} a_nx^n\)  so that \(\langle a_n \rangle_{n \geq 0}\) satisfy
$$ a_n - a_{n-1} - a_{n-2} + a_{n-3} = 0 \hspace{0.5cm} (n \geq 4)$$
$$ a_0 = 8, a_1 = 2, a_2 = 10, a_3 = 6 $$
\textbf{ Solution : }
\begin{enumerate}
\item \(q(x) = 1 - x - x^2 + x^3\) \\
Complete q(x) that has same coefficients as C(x) but in reverse direction. 
\item Let \(p(x) = q(x) \sum_{n \geq 0} a_n x^n\)
\[\begin{aligned}
P(x) & = (1 - x - x^2 + x^3) (a_0 + a_1x + a_2x^2 + \ldots) \\
& = a_0 + (a_1 - a_0)x (a_2 - a_1 - a_0) x^2 + (a_3 - a_2 - a_1 + a_0)x^3  + \strikethrough{(a_2 - a_3 - a_2 + a_1)} x^4 + x^5 \\
\end{aligned}\]
\end{enumerate}
\end{expx}
\end{document}


