%Notes by Harsh Mistry 
%Math 239
%based on Template from : https://www.cs.cmu.edu/~ggordon/10725-F12/template.tex

\documentclass{article}
\setlength{\oddsidemargin}{0.25 in}
\setlength{\evensidemargin}{-0.25 in}
\setlength{\topmargin}{-0.6 in}
\setlength{\textwidth}{6.5 in}
\setlength{\textheight}{8.5 in}
\setlength{\headsep}{0.75 in}
\setlength{\parindent}{0 in}
\setlength{\parskip}{0.1 in}
\usepackage{amsfonts,graphicx, amssymb}
\usepackage[fleqn]{amsmath}
\usepackage{fixltx2e}
\usepackage{color}
\usepackage{tcolorbox}
\usepackage{lipsum}
\usepackage{listings}
\usepackage{scrextend}
\tcbuselibrary{skins,breakable}
\usetikzlibrary{shadings,shadows}
\newcounter{lecnum}
\renewcommand{\thepage}{\thelecnum-\arabic{page}}
\renewcommand{\thesection}{\thelecnum.\arabic{section}}
\renewcommand{\theequation}{\thelecnum.\arabic{equation}}
\renewcommand{\thefigure}{\thelecnum.\arabic{figure}}
\renewcommand{\thetable}{\thelecnum.\arabic{table}}
\newcommand{\lecture}[4]{
   \pagestyle{myheadings}
   \thispagestyle{plain}
   \newpage
   \setcounter{lecnum}{#1}
   \setcounter{page}{1}
   
   
%Info Box 
   \begin{center}
   \framebox{
      \vbox{\vspace{2mm}
    \hbox to 6.28in { {\bf Math 239 - Introduction to Combinatorics
	\hfill Spring 2017} }
       \vspace{4mm}
       \hbox to 6.28in { {\Large \hfill Lecture #1: #2  \hfill} }
       \vspace{2mm}
       \hbox to 6.28in { {\it Lecturer: #3 \hfill Notes By: #4} }
      \vspace{2mm}}
   }
   \end{center}
   
   \markboth{Lecture #1: #2}{Lecture #1: #2}



 
}

\renewcommand{\cite}[1]{[#1]}
\def\beginrefs{\begin{list}%
        {[\arabic{equation}]}{\usecounter{equation}
         \setlength{\leftmargin}{2.0truecm}\setlength{\labelsep}{0.4truecm}%
         \setlength{\labelwidth}{1.6truecm}}}
\def\endrefs{\end{list}}
\def\bibentry#1{\item[\hbox{[#1]}]}

\newcommand{\fig}[3]{
			\vspace{#2}
			\begin{center}
			Figure \thelecnum.#1:~#3
			\end{center}
	}
	
	\newcommand{\bs}{
		\texttt{\char`\\ \hspace{0.1cm}} 
	}
	

	
\newcommand{\pipe}{\(\mid\)}
\newcommand{\ctr}{\(\wedge\)}

\newtheorem{theorem}{Theorem}[lecnum]
\newtheorem{lemma}[theorem]{Lemma}
\newtheorem{ex}[theorem]{Example}
\newtheorem{expx}[theorem]{Problem}
\newtheorem{prop}[theorem]{Proposition}
\newtheorem{claim}[theorem]{Claim}
\newtheorem{corollary}[theorem]{Corollary}
\newtheorem{definition}[theorem]{Definition}
\newenvironment{proof}{{\bf Proof:}}{\hfill\rule{2mm}{2mm}}
\newcommand\E{\mathbb{E}}

%color definitions :
\definecolor{darkred}{rgb}{0.55, 0.0, 0.0}
\definecolor{lightcoral}{rgb}{0.94, 0.5, 0.5}
\definecolor{tomato}{rgb}{1.0, 0.39, 0.28}
\definecolor{lightgray}{rgb}{.9,.9,.9}
\definecolor{darkgray}{rgb}{.4,.4,.4}
\definecolor{purple}{rgb}{0.65, 0.12, 0.82}
\definecolor{lightgreen}{rgb}{0.56, 0.93, 0.56}
\definecolor{darkgreen}{rgb}{0.0, 0.2, 0.13}
\definecolor{limegreen}{rgb}{0.2, 0.8, 0.2}
\definecolor{lightblue}{rgb}{0.68, 0.85, 0.9}
\definecolor{darkblue}{rgb}{0.0, 0.0, 0.55}


%Environments
\newenvironment{exblock}[1]{%
    \tcolorbox[beamer,%
    noparskip,breakable,
    colback=lightgreen,colframe=darkgreen,%
    colbacklower=limegreen!75!lightgreen,%
    title=#1]}%
    {\endtcolorbox}

\newenvironment{ablock}[1]{%
    \tcolorbox[beamer,%
    noparskip,breakable,
    colback=lightcoral,colframe=darkred,%
    colbacklower=tomato!75!lightcoral,%
    title=#1]}%
    {\endtcolorbox}

\newenvironment{cblock}[1]{%
    \tcolorbox[beamer,%
    noparskip,breakable,
    colback=lightblue,colframe=darkblue,%
    colbacklower=darkblue!75!lightblue,%
    title=#1]}%
    {\endtcolorbox}


%Languages
\lstdefinelanguage{JavaScript}{
  keywords={typeof, new, true, false, catch, function, return, null, catch, switch, var, if, in, while, do, else, case, break},
  keywordstyle=\color{blue}\bfseries,
  ndkeywords={class, export, boolean, throw, implements, import, this},
  ndkeywordstyle=\color{darkgray}\bfseries,
  identifierstyle=\color{black},
  sensitive=false,
  comment=[l]{//},
  morecomment=[s]{/*}{*/},
  commentstyle=\color{purple}\ttfamily,
  stringstyle=\color{red}\ttfamily,
  morestring=[b]',
  morestring=[b]"
}

%Listings
\lstset{
   language=JavaScript,
   backgroundcolor=\color{lightgray},
   extendedchars=true,
   basicstyle=\footnotesize\ttfamily,
   showstringspaces=false,
   showspaces=false,
   numbers=left,
   numberstyle=\footnotesize,
   numbersep=9pt,
   tabsize=2,
   breaklines=true,
   showtabs=false,
   captionpos=b
}


%Start
\begin{document}

\lecture{10}{May 23rd, 2017}{Alan Arroyo Guevara}{Harsh Mistry}
 
\section{Binary Strings}

\begin{definition}
The empty string \(\epsilon\) is a string of length 0 with the property that for every other string a 
$$\epsilon a  = a = a \epsilon$$
\end{definition}


\begin{definition}(String Product) Let A,B be sets of string, then 
$$ A B = \{ ab \mid a \in A, b \in B\} $$
\end{definition}

\begin{ex}
\(A = \{1, 01\}\), \(B = \{1, 10\}\) 
\[AB = \{11, 110, 011, 0110\}\]
\[BA = \{11, 101, 101, 1001\} = \{11, 101, 1001\}\]
\end{ex}

\begin{definition}
We say AB is unambiguous if for every string \(S \in AB\) there is a unique \(a \in A\) and unique \(b \in B\) such that \(S = ab\)  
\end{definition}

\begin{definition}
\(A \cup B \) is unambiguous \(\iff\) \(A \cap B = \emptyset\)
\end{definition} 

\begin{definition}
(String Power) 
$$ A^k := A A \ldots A $$
$$ A^0 := \{\epsilon\} ^k $$
\end{definition}

\begin{ex}
\(\{0,1\}^k = all \{0,1\} of length k\)\\
\(A^k\) is \underline{Unambiguous} if for every \(s \in A^k\), there are unique \(a_1, a_2, a_3 , \ldots, a_k \in A\) such that \(s = a_1 a_2 \ldots a_k\)
\end{ex}

\begin{ex}
\(A = \{\epsilon, 1\}\), is \(A^3\) ambiguous?\\
\textbf{Yes,} because \(1 = (1)(\epsilon)(\epsilon) = (\epsilon) (1) (\epsilon) = (\epsilon)(\epsilon)(1) \)
\end{ex}

\begin{definition}(Star)
$$ A^* := \{\epsilon\} \cup A \cup A^2 \cup \ldots $$
\(A^k\) is unambiguous if 
\begin{itemize}
\item Each \(A^k\) is unambiguous \((k = 0, 1,\ldots)\)
\item \(A^k  \cap A^j = \emptyset, \hspace{0.1cm} k \neq j\)
\end{itemize}
\end{definition}

\begin{definition}
Given a string s, a block is a maximal non-empty substring of 0's and 1's  
\end{definition}

\begin{expx}
Describe in words the following set of strings
\begin{itemize}
\item \(\{0,1\}^*\) = All binary strings
\item \(\{\epsilon, 0, 1\}^*\) = All binary strings, but this is ambiguous
\item \(\{11\}\) = all even length binary strings with only 1's
\item \(\{0\}\{0\}^*\{11\}^* = \) strings starting with a block of zeros, and followed by an even number of 1's
\item \(\left(\{0\}\{0\}^0\{11\}^*\right)^* =\) strings starting with a block of 0's in which every 0 is followed by an even number of 1s 
\end{itemize} 
\end{expx}
\end{document}


