%Notes by Harsh Mistry 
%Math 239
%based on Template from : https://www.cs.cmu.edu/~ggordon/10725-F12/template.tex

\documentclass{article}
\setlength{\oddsidemargin}{0.25 in}
\setlength{\evensidemargin}{-0.25 in}
\setlength{\topmargin}{-0.6 in}
\setlength{\textwidth}{6.5 in}
\setlength{\textheight}{8.5 in}
\setlength{\headsep}{0.75 in}
\setlength{\parindent}{0 in}
\setlength{\parskip}{0.1 in}
\usepackage{amsfonts,graphicx, amssymb}
\usepackage[fleqn]{amsmath}
\usepackage{fixltx2e}
\usepackage{color}
\usepackage{tcolorbox}
\usepackage{lipsum}
\usepackage{listings}
\usepackage{scrextend}
\tcbuselibrary{skins,breakable}
\usetikzlibrary{shadings,shadows}
\newcounter{lecnum}
\renewcommand{\thepage}{\thelecnum-\arabic{page}}
\renewcommand{\thesection}{\thelecnum.\arabic{section}}
\renewcommand{\theequation}{\thelecnum.\arabic{equation}}
\renewcommand{\thefigure}{\thelecnum.\arabic{figure}}
\renewcommand{\thetable}{\thelecnum.\arabic{table}}
\newcommand{\lecture}[4]{
   \pagestyle{myheadings}
   \thispagestyle{plain}
   \newpage
   \setcounter{lecnum}{#1}
   \setcounter{page}{1}
   
   
%Info Box 
   \begin{center}
   \framebox{
      \vbox{\vspace{2mm}
    \hbox to 6.28in { {\bf Math 239 - Introduction to Combinatorics
	\hfill Spring 2017} }
       \vspace{4mm}
       \hbox to 6.28in { {\Large \hfill Lecture #1: #2  \hfill} }
       \vspace{2mm}
       \hbox to 6.28in { {\it Lecturer: #3 \hfill Notes By: #4} }
      \vspace{2mm}}
   }
   \end{center}
   
   \markboth{Lecture #1: #2}{Lecture #1: #2}



 
}

\renewcommand{\cite}[1]{[#1]}
\def\beginrefs{\begin{list}%
        {[\arabic{equation}]}{\usecounter{equation}
         \setlength{\leftmargin}{2.0truecm}\setlength{\labelsep}{0.4truecm}%
         \setlength{\labelwidth}{1.6truecm}}}
\def\endrefs{\end{list}}
\def\bibentry#1{\item[\hbox{[#1]}]}

\newcommand{\fig}[3]{
			\vspace{#2}
			\begin{center}
			Figure \thelecnum.#1:~#3
			\end{center}
	}
	
	\newcommand{\bs}{
		\texttt{\char`\\ \hspace{0.1cm}} 
	}
	

	
\newcommand{\pipe}{\(\mid\)}
\newcommand{\ctr}{\(\wedge\)}

\newtheorem{theorem}{Theorem}[lecnum]
\newtheorem{lemma}[theorem]{Lemma}
\newtheorem{ex}[theorem]{Example}
\newtheorem{expx}[theorem]{Problem}
\newtheorem{prop}[theorem]{Proposition}
\newtheorem{claim}[theorem]{Claim}
\newtheorem{corollary}[theorem]{Corollary}
\newtheorem{definition}[theorem]{Definition}
\newenvironment{proof}{{\bf Proof:}}{\hfill\rule{2mm}{2mm}}
\newcommand\E{\mathbb{E}}

%color definitions :
\definecolor{darkred}{rgb}{0.55, 0.0, 0.0}
\definecolor{lightcoral}{rgb}{0.94, 0.5, 0.5}
\definecolor{tomato}{rgb}{1.0, 0.39, 0.28}
\definecolor{lightgray}{rgb}{.9,.9,.9}
\definecolor{darkgray}{rgb}{.4,.4,.4}
\definecolor{purple}{rgb}{0.65, 0.12, 0.82}
\definecolor{lightgreen}{rgb}{0.56, 0.93, 0.56}
\definecolor{darkgreen}{rgb}{0.0, 0.2, 0.13}
\definecolor{limegreen}{rgb}{0.2, 0.8, 0.2}
\definecolor{lightblue}{rgb}{0.68, 0.85, 0.9}
\definecolor{darkblue}{rgb}{0.0, 0.0, 0.55}


%Environments
\newenvironment{exblock}[1]{%
    \tcolorbox[beamer,%
    noparskip,breakable,
    colback=lightgreen,colframe=darkgreen,%
    colbacklower=limegreen!75!lightgreen,%
    title=#1]}%
    {\endtcolorbox}

\newenvironment{ablock}[1]{%
    \tcolorbox[beamer,%
    noparskip,breakable,
    colback=lightcoral,colframe=darkred,%
    colbacklower=tomato!75!lightcoral,%
    title=#1]}%
    {\endtcolorbox}

\newenvironment{cblock}[1]{%
    \tcolorbox[beamer,%
    noparskip,breakable,
    colback=lightblue,colframe=darkblue,%
    colbacklower=darkblue!75!lightblue,%
    title=#1]}%
    {\endtcolorbox}


%Languages
\lstdefinelanguage{JavaScript}{
  keywords={typeof, new, true, false, catch, function, return, null, catch, switch, var, if, in, while, do, else, case, break},
  keywordstyle=\color{blue}\bfseries,
  ndkeywords={class, export, boolean, throw, implements, import, this},
  ndkeywordstyle=\color{darkgray}\bfseries,
  identifierstyle=\color{black},
  sensitive=false,
  comment=[l]{//},
  morecomment=[s]{/*}{*/},
  commentstyle=\color{purple}\ttfamily,
  stringstyle=\color{red}\ttfamily,
  morestring=[b]',
  morestring=[b]"
}

%Listings
\lstset{
   language=JavaScript,
   backgroundcolor=\color{lightgray},
   extendedchars=true,
   basicstyle=\footnotesize\ttfamily,
   showstringspaces=false,
   showspaces=false,
   numbers=left,
   numberstyle=\footnotesize,
   numbersep=9pt,
   tabsize=2,
   breaklines=true,
   showtabs=false,
   captionpos=b
}


%Start
\begin{document}

\lecture{3}{May 5th, 2017}{Alan Arroyo Guevara}{Harsh Mistry}

In general, given a set S, a partition of S is a collection \(S_0, S_1, \ldots, S_n\) of subsets of S such that : 

\begin{enumerate}
\item \(S = S_0 \cup S_1 \cup \ldots \cup S_n\)
\item \(S_i \cap S_j = \emptyset\) for \(i \neq j\)
\end{enumerate}

\begin{cblock}{Tip}
In proofs of combinatorial identities, partitions are used to represent sums. 
\end{cblock}

\begin{prop}
\[{n \choose k} = {{n - 1} \choose {k - 1}} + {{n - 1} \choose k } \hspace{0.1cm} , 1 \leq k \leq n \]
\end{prop}

\begin{proof}
Let S be the set of k-subsets of \(\langle 1, \ldots, n \rangle\) \\
Clearly \(\mid S \mid \hspace{0.1cm} = {n \choose k}\), so we want a partition such that 
\[\mid S_0 \mid \hspace{0.1cm} = {{n-1}\choose {k-1}} \hspace{0.1cm} \text{ and } \hspace{0.1cm} 
\mid S_1 \mid \hspace{0.1cm} = {{n-1}\choose k}\]
\begin{ablock}{Define}
\begin{itemize}
\item \(S_0 = \) all k-subsets of \(\langle 1, \ldots, n \rangle\), that contains n.
\item \(S_1 = \) all k subsets of  \(\langle 1, \ldots, n \rangle\), that do not contain n
\end{itemize}

Clearly, \(S_0 , S_1\) is a partition.
\end{ablock}

So, Let T be the set of (k - 1) subsets\(\langle 1, \ldots, n \rangle\). 

Using this we can define \(f : S \rightarrow T,\) and \(f^{-1} : T \rightarrow S \) such that : 

\begin{itemize}
\item \(f(A) = A \bs \{n\} , \forall A \in S_0 \)
\item \(f^{-1}(B) = B \cup \{n\} , \forall B \in T \)
\end{itemize}

\begin{exblock}{Exercise}
Prove \(f^{-1}\) is the inverse of f 
\end{exblock}

Therefore , \(\mid S_n \mid  \hspace{0.1cm} = \mid T \mid \hspace{0.1cm} = {{n - 1} \choose {k - 1}} \) and \(\mid S_1 \mid \hspace{0.1cm} = {{n-1} \choose k}\), since \(S_1\) is the set of k-subsets of \(\langle 1, \ldots, n -1 \rangle\). 

Thus, 
\[{n \choose k} = \hspace{0.1cm} \mid S \mid \hspace{0.1cm} = \hspace{0.1cm} \mid S_0 \mid \hspace{0.1cm} + \mid S_1 \mid  \hspace{0.1cm} = {{n-1} \choose {k-1}} + {{n-1} \choose k}\]
\end{proof}

\begin{prop}
\[{{n+k} \choose n} = \sum_{i=0}^{k} {{n + i - 1} \choose {n - 1}}\]
\end{prop}

\begin{proof}
Let S be the set of n-subsets of \(\langle 1, \ldots, n, 
\ldots , n + k\rangle\). 
\[\mid S_0 \mid \hspace{0.1cm} = {{n + k} \choose n} \]
Consider a partition \(S_0 \cup S_1 \cup \ldots \cup S_k \) of S. For \(i = 0, 1, \ldots, k\) we let 
\[S_ i = \langle A \cup \langle n _ i \rangle : A \subset \langle 1, \ldots, n + i - 1\rangle , \hspace{0.1cm} \mid A \mid \hspace{0.1cm} = n - 1 \rangle\]
\[\mid S_i \mid \hspace{0.1cm} = {{n+i-1} \choose {n-1}}\]
The elements in \(S_i\) text are in correspondence with the elements in \(\{A : A \subset \langle 1, \ldots, n+i-1 \rangle , \hspace{0.1cm} \mid A \mid \hspace{0.1cm} = n - 1\}\) \\
\newline 
\(S_i\) can also be described as the set of n-sets of \(\langle 1, \ldots, n+k \rangle\), that have \(n+i\) as its maximum element.\\
\newline
So, \(S_i \cap S_j = \emptyset \) for \(i \neq j\). As a result 
\[S = S_0 \cup S_1 \cup \ldots \cup S_k\]
because for every \(A \subset\) \(\langle 1, \ldots, n +1 \rangle\), \hspace{0.1cm} \(\mid A \mid \) \hspace{0.1cm} and A have \(n+i\) as the largest element
\end{proof}



\end{document}
