\documentclass[12pt]{article}

\usepackage{fullpage}
\usepackage{verbatim}

\usepackage{listings}
\usepackage{amsmath}
\usepackage{graphicx}
\usepackage{mathtools}
\usepackage{tikz}
\usepackage[colorinlistoftodos]{todonotes}
\usepackage[colorlinks=true, allcolors=blue]{hyperref}

\title{CS 444\\
Lecture 15, March 2nd, 2020 \\
Matthew Walinga\\
20627018}

\begin{document}

\maketitle

\bigskip

\section{Casts and stuff}

\underline{$C,L,\sigma \vdash E : \tau_1 $  $\tau_1 := \tau_2 \vee \tau_2 := \tau_1$}

$C,L,\sigma \vdash (\tau_2) E$ : $\tau_2$
\newline

\underline{$C,L,\sigma \vdash E : \tau $  $\tau := D \vee D := \tau$}

$C,L,\sigma \vdash E \text{ instanceof } D $ : boolean

\section{Methods}

- Java closest match is JLS 15.12.2 (we won't do this, we check exact types only)
\newline

$\forall i:$ \underline{$C,L,\sigma \vdash E_i : \tau_i$  \quad static $\tau \;  m(\tau_1,...,\tau_k) \in $ continue(D)}

$C,L,\sigma \vdash D.m(E_1,...E_k) $ : E
\newline

$\forall i:$  \underline{$C,L,\sigma \vdash E_i : \tau_i \quad C,L,\sigma \vdash E : D \quad \tau \; m(\tau_1,...,\tau_k) \in$ continue(D)}

$C,L,\sigma \vdash E.m(E_1,...E_k) : \tau$ 
\newline

$\forall i:$  \underline{$C,L,\sigma \vdash E_i : \tau_i \quad D(\tau_1,...,\tau_k) \in D$}

$C,L,\sigma \vdash new D(E_1,...E_k) : D$ 
\newline

\subsection{Return types}

\underline{$C,L,\sigma \vdash E_i : \tau_i \quad \sigma := \tau$}

$C,L,\sigma \vdash \text{return } E$ 

$C,L,\sigma \vdash \text{return}$ 

\section{Arrays}

\underline{$C,L,\sigma \vdash E_1 : \tau_1[] \quad C,L,\sigma \vdash E_2 : \tau_2 \quad num(\tau_2)$}

$C,L,\sigma \vdash E_1[E_2] : \tau_1$ 
\newline

\underline{$C,L,\sigma \vdash E_1 : \tau_1[] \quad C,L,\sigma \vdash E_2 : \tau_2 \quad num(\tau_2) \quad C,L,\sigma \vdash E_3 : \tau_3 \quad \tau_1 := \tau_3$}

$C,L,\sigma \vdash E_1[E_2]  = E_3 : \tau_1$ 
\newline

\underline{$C,L,\sigma \vdash E : \tau[]$}

$C,L,\sigma \vdash E.length : int$ 
\newline

\underline{$C,L,\sigma \vdash E : \tau_2 \quad num(\tau_2)$}

$C,L,\sigma \vdash \text{new } \tau_1[E] : \tau_1[]$ 

\subsection{Array Assignability}

- Arrays can be assigned to the following types:

Object := $\sigma[]$

Cloneable := $\sigma[]$

java.io.Serializable := $\sigma[]$
\newline

\subsubsection{An evil example}

\underline{$D \leq C$}

$C[] := D[] \text{ if } C := D$ (covariant)
\newline

Here's some code as an example:
\newline

\begin{verbatim}
B[] bs = new B[1];
Object[] os = bs;
os[0] = new C();
B b = bs[0];
\end{verbatim}

This program is not type correct! It assigns a $C$-type object to a $B$-type object :o

The rule we defined makes the static type system \underline{unsound}
\newline

Solution: To preserve type-correctness, we must "tag" each array element. Java checks the dynamic type at every array store. This may result in a runtime exception: ArrayStoreException (JLS 10.10)
\newline

The covariant rule we defined above was the problem. Covariants should only be used for data structures that are read-only

\subsubsection{Other soundess holes}

- Casts are intentional soundness holes (programmer tells the compiler "trust me")

\subsubsection{A few more rules}

Two-dimensional arrays (not in Joos so who really cares, but here for completeness)

$\sigma[] := \tau[]$

$\sigma[][] := \tau[][]$

\section{Type-checking pseudocode (Joos)}

"Demand-driven approach"
\newline

(A)
\underline{premises}
	
	$C,L,\sigma \vdash E : \tau$

\begin{verbatim}
typecheck(E) {
	1. find a rule of the form (A) above
	2. check premises (with recursive calls to typecheck subexpressions)
	3. return tau
}
\end{verbatim}

\section{Static Analysis}

- analyze a program to prove properties of its runtime behaviour

- Applications: generate efficient optimized code

\newpage

\end{document}