%Notes by Harsh Mistry 
%CS 444
%based on Template from : https://www.cs.cmu.edu/~ggordon/10725-F12/template.tex

\documentclass{article}
\setlength{\oddsidemargin}{0.25 in}
\setlength{\evensidemargin}{-0.25 in}
\setlength{\topmargin}{-0.6 in}
\setlength{\textwidth}{6.5 in}
\setlength{\textheight}{8.5 in}
\setlength{\headsep}{0.75 in}
\setlength{\parindent}{0 in}
\setlength{\parskip}{0.1 in}
\usepackage{amsfonts,graphicx, amssymb}
\usepackage[fleqn]{amsmath}
\usepackage{fixltx2e}
\usepackage{color}
\usepackage{tcolorbox}
\usepackage{lipsum}
\usepackage{listings}
\graphicspath{ {./images/} }
\usepackage{scrextend}
\tcbuselibrary{skins,breakable}
\usetikzlibrary{shadings,shadows}
\newcounter{lecnum}
\renewcommand{\thepage}{\thelecnum-\arabic{page}}
\renewcommand{\thesection}{\thelecnum.\arabic{section}}
\renewcommand{\theequation}{\thelecnum.\arabic{equation}}
\renewcommand{\thefigure}{\thelecnum.\arabic{figure}}
\renewcommand{\thetable}{\thelecnum.\arabic{table}}
\newcommand{\lecture}[4]{
   \pagestyle{myheadings}
   \thispagestyle{plain}
   \newpage
   \setcounter{lecnum}{#1}
   \setcounter{page}{1}
   
   
%Info Box 
   \begin{center}
   \framebox{
      \vbox{\vspace{2mm}
    \hbox to 6.28in { {\bf CS 444 - Compiler Construction 
	\hfill Winter 2020} }
       \vspace{4mm}
       \hbox to 6.28in { {\Large \hfill Lecture #1: #2  \hfill} }
       \vspace{2mm}
       \hbox to 6.28in { {\it Lecturer: #3 \hfill Notes By: #4} }
      \vspace{2mm}}
   }
   \end{center}
   
   \markboth{Lecture #1: #2}{Lecture #1: #2}



 
}

\renewcommand{\cite}[1]{[#1]}
\def\beginrefs{\begin{list}%
        {[\arabic{equation}]}{\usecounter{equation}
         \setlength{\leftmargin}{2.0truecm}\setlength{\labelsep}{0.4truecm}%
         \setlength{\labelwidth}{1.6truecm}}}
\def\endrefs{\end{list}}
\def\bibentry#1{\item[\hbox{[#1]}]}

\newcommand{\fig}[3]{
			\vspace{#2}
			\begin{center}
			Figure \thelecnum.#1:~#3
			\end{center}
	}
	
\newcommand{\pipe}{\(\mid\)}
\newcommand{\ctr}{\(\wedge\)}

\newtheorem{theorem}{Theorem}[lecnum]
\newtheorem{lemma}[theorem]{Lemma}
\newtheorem{ex}[theorem]{Example}
\newtheorem{proposition}[theorem]{Proposition}
\newtheorem{claim}[theorem]{Claim}
\newtheorem{corollary}[theorem]{Corollary}
\newtheorem{definition}[theorem]{Definition}
\newenvironment{proof}{{\bf Proof:}}{\hfill\rule{2mm}{2mm}}
\newcommand\E{\mathbb{E}}

%color definitions :
\definecolor{darkred}{rgb}{0.55, 0.0, 0.0}
\definecolor{lightcoral}{rgb}{0.94, 0.5, 0.5}
\definecolor{tomato}{rgb}{1.0, 0.39, 0.28}
\definecolor{lightgray}{rgb}{.9,.9,.9}
\definecolor{darkgray}{rgb}{.4,.4,.4}
\definecolor{purple}{rgb}{0.65, 0.12, 0.82}
\definecolor{lightgreen}{rgb}{0.56, 0.93, 0.56}
\definecolor{darkgreen}{rgb}{0.0, 0.2, 0.13}
\definecolor{limegreen}{rgb}{0.2, 0.8, 0.2}
\definecolor{lightblue}{rgb}{0.68, 0.85, 0.9}
\definecolor{darkblue}{rgb}{0.0, 0.0, 0.55}


%Environments
\newenvironment{exblock}[1]{%
    \tcolorbox[beamer,%
    noparskip,breakable,
    colback=lightgreen,colframe=darkgreen,%
    colbacklower=limegreen!75!lightgreen,%
    title=#1]}%
    {\endtcolorbox}

\newenvironment{ablock}[1]{%
    \tcolorbox[beamer,%
    noparskip,breakable,
    colback=lightcoral,colframe=darkred,%
    colbacklower=tomato!75!lightcoral,%
    title=#1]}%
    {\endtcolorbox}

\newenvironment{cblock}[1]{%
    \tcolorbox[beamer,%
    noparskip,breakable,
    colback=lightblue,colframe=darkblue,%
    colbacklower=darkblue!75!lightblue,%
    title=#1]}%
    {\endtcolorbox}


%Languages
\lstdefinelanguage{JavaScript}{
  keywords={typeof, new, true, false, catch, function, return, null, catch, switch, var, if, in,  fi, while, do, else, case, break},
  keywordstyle=\color{blue}\bfseries,
  ndkeywords={class, export, boolean, throw, implements, import, this},
  ndkeywordstyle=\color{darkgray}\bfseries,
  identifierstyle=\color{black},
  sensitive=false,
  comment=[l]{//},
  morecomment=[s]{/*}{*/},
  commentstyle=\color{purple}\ttfamily,
  stringstyle=\color{red}\ttfamily,
  morestring=[b]',
  morestring=[b]"
}

%Listings
\lstset{
   language=JavaScript,
   backgroundcolor=\color{lightgray},
   extendedchars=true,
   basicstyle=\footnotesize\ttfamily,
   showstringspaces=false,
   showspaces=false,
   numbers=left,
   numberstyle=\footnotesize,
   numbersep=9pt,
   tabsize=2,
   breaklines=true,
   showtabs=false,
   captionpos=b
}

\newcommand*\circled[1]{\tikz[baseline=(char.base)]{
            \node[shape=circle,draw,inner sep=2pt] (char) {#1};}}

%Start of Document 
\begin{document}

\lecture{12}{February 12th , 2020}{Ond\u{r}ej Lhot\'{a}k}{Harsh Mistry}



\section{Context-sensitive analysis continued}

\subsection{Name Resolution}



\begin{itemize}
\item In Java : 
\textcolor{blue}{A2} and  \textcolor{red}{A3}
\begin{enumerate}
\color{blue}
\item Build global environment
\item Resolve type names
\item Build/check class hierarchy (methods/files)

\color{red}
\item Disambiguate ambiguous namespace 
\begin{itemize}
\item In Java you can't simply determine the namespace based on location of usage
\end{itemize} 
\item Resolve "expressions" (Variables, static fields)
\item Type checking
\item Resolve methods instance (non-static) fields
\end{enumerate}
\item (2) Resolving Name Types Continued : More Rules
\begin{itemize}
\item $$
\frac{\begin{matrix}
S \in super(T) \hspace{0.5cm} m \in contain(S) \hspace{0.5cm}  abstract \not\in mods(m) \hspace{0.5cm} nodecl(T, m)  \\
S^\prime \in super(T) \hspace{0.5cm} m^\prime \in contain(S^\prime) \hspace{0.5cm}  abstract \not\in mods(m^\prime ) \hspace{0.5cm} sig(m) = sig(m^\prime) 
\end{matrix}}{(m, m^\prime) \in replace}
$$\\
$$nodecl(T, m) = \forall m^\prime  \in declare(T) : sig(m) \neq sig(m^\prime)$$
\item An interface without super interfaces implicitly declares an abstract version of every public method in \verb|java.lang.Object|  \textcolor{orange}{(Refer to JLS 9.2 )}
\end{itemize}
\item (3) Checking Hierarchy 
\begin{enumerate}
\item not \(\exists T : T < T\) (Basically there should be no cycles)
\item \(\forall m,  m^\prime \in declare(T) : m \neq m^\prime  \implies sig(m) \neq sig(m^prime)\)
\item  \(\forall m, m^\prime \in contains(T): sig(m) = sign(m^\prime) \implies type(m) = type(m^\prime)\)
\item  \(\forall m \in contains(T): abstract \in mods(m) \implies abstract \in mods(T))\)
\item \(\forall(m, m^\prime) \in replace: static \in mods(m^\prime) \iff  static \in mods(m)\)
\item \(\forall(m, m^\prime) \in replace: type(m) = type(m^\prime))\)
\item \(\forall(m, m^\prime) \in public \in mods(m^\prime) \implies public \in mods(m)\)
\item No 8, because the course is attempting to be consistent with numbering schemes in other offerings. 8 would normally involve exception handling
\item \(\forall(m, m^\prime) \in replace: final \not\in mods(m^\prime) \)
\item \(\forall f, f^\prime \in declare(T) : f^\prime \neq f \implies name(f)\neq name(f^\prime) \)
\end{enumerate}
\item (4) Disambiguating ambiguous name spaces
\begin{itemize}
\item Notation 
\begin{itemize}
\item P = package
\item T  = Type
\item E = Expression 
\item M = Method
\end{itemize}
\item \textcolor{orange}{(Refer to JLS 6.5.2 )}  (\(a_1 \cdot a_2 \cdot a_3 \cdot a_4 \cdot a_k ) \)
\begin{enumerate}
\item If local variables/parameters named \(a_1\) exists , resolve to it
\item If field name \(a_1 \in contain( \text{enclosing class})\), resolve to it
\item For each k in increasing order
\begin{itemize}
\item if \(a_1 \cdot \ldots \cdot a_k\) resolves to a type, resolve to it 
\begin{itemize}
\item  \(a_{k+1}\) is a static field
\item \(a_{k+2 \ldots}\) are non static fields
\end{itemize}
\end{itemize}
\item Resolve instance fields, methods after type declaration
\item General search outwards from inner most scope
\end{enumerate}
\end{itemize}
\end{itemize}

\end{document}







