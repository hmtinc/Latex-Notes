%Notes by Harsh Mistry 
%CS 444
%based on Template from : https://www.cs.cmu.edu/~ggordon/10725-F12/template.tex

\documentclass{article}
\setlength{\oddsidemargin}{0.25 in}
\setlength{\evensidemargin}{-0.25 in}
\setlength{\topmargin}{-0.6 in}
\setlength{\textwidth}{6.5 in}
\setlength{\textheight}{8.5 in}
\setlength{\headsep}{0.75 in}
\setlength{\parindent}{0 in}
\setlength{\parskip}{0.1 in}
\usepackage{amsfonts,graphicx, amssymb}
\usepackage[fleqn]{amsmath}
\usepackage{fixltx2e}
\usepackage{color}
\usepackage{tcolorbox}
\usepackage{lipsum}
\usepackage{listings}
\graphicspath{ {./images/} }
\usepackage{scrextend}
\tcbuselibrary{skins,breakable}
\usetikzlibrary{shadings,shadows}
\newcounter{lecnum}
\renewcommand{\thepage}{\thelecnum-\arabic{page}}
\renewcommand{\thesection}{\thelecnum.\arabic{section}}
\renewcommand{\theequation}{\thelecnum.\arabic{equation}}
\renewcommand{\thefigure}{\thelecnum.\arabic{figure}}
\renewcommand{\thetable}{\thelecnum.\arabic{table}}
\newcommand{\lecture}[4]{
   \pagestyle{myheadings}
   \thispagestyle{plain}
   \newpage
   \setcounter{lecnum}{#1}
   \setcounter{page}{1}
   
   
%Info Box 
   \begin{center}
   \framebox{
      \vbox{\vspace{2mm}
    \hbox to 6.28in { {\bf CS 444 - Compiler Construction 
	\hfill Winter 2020} }
       \vspace{4mm}
       \hbox to 6.28in { {\Large \hfill Lecture #1: #2  \hfill} }
       \vspace{2mm}
       \hbox to 6.28in { {\it Lecturer: #3 \hfill Notes By: #4} }
      \vspace{2mm}}
   }
   \end{center}
   
   \markboth{Lecture #1: #2}{Lecture #1: #2}



 
}

\renewcommand{\cite}[1]{[#1]}
\def\beginrefs{\begin{list}%
        {[\arabic{equation}]}{\usecounter{equation}
         \setlength{\leftmargin}{2.0truecm}\setlength{\labelsep}{0.4truecm}%
         \setlength{\labelwidth}{1.6truecm}}}
\def\endrefs{\end{list}}
\def\bibentry#1{\item[\hbox{[#1]}]}

\newcommand{\fig}[3]{
			\vspace{#2}
			\begin{center}
			Figure \thelecnum.#1:~#3
			\end{center}
	}
	
\newcommand{\pipe}{\(\mid\)}
\newcommand{\ctr}{\(\wedge\)}

\newtheorem{theorem}{Theorem}[lecnum]
\newtheorem{lemma}[theorem]{Lemma}
\newtheorem{ex}[theorem]{Example}
\newtheorem{proposition}[theorem]{Proposition}
\newtheorem{claim}[theorem]{Claim}
\newtheorem{corollary}[theorem]{Corollary}
\newtheorem{definition}[theorem]{Definition}
\newenvironment{proof}{{\bf Proof:}}{\hfill\rule{2mm}{2mm}}
\newcommand\E{\mathbb{E}}

%color definitions :
\definecolor{darkred}{rgb}{0.55, 0.0, 0.0}
\definecolor{lightcoral}{rgb}{0.94, 0.5, 0.5}
\definecolor{tomato}{rgb}{1.0, 0.39, 0.28}
\definecolor{lightgray}{rgb}{.9,.9,.9}
\definecolor{darkgray}{rgb}{.4,.4,.4}
\definecolor{purple}{rgb}{0.65, 0.12, 0.82}
\definecolor{lightgreen}{rgb}{0.56, 0.93, 0.56}
\definecolor{darkgreen}{rgb}{0.0, 0.2, 0.13}
\definecolor{limegreen}{rgb}{0.2, 0.8, 0.2}
\definecolor{lightblue}{rgb}{0.68, 0.85, 0.9}
\definecolor{darkblue}{rgb}{0.0, 0.0, 0.55}


%Environments
\newenvironment{exblock}[1]{%
    \tcolorbox[beamer,%
    noparskip,breakable,
    colback=lightgreen,colframe=darkgreen,%
    colbacklower=limegreen!75!lightgreen,%
    title=#1]}%
    {\endtcolorbox}

\newenvironment{ablock}[1]{%
    \tcolorbox[beamer,%
    noparskip,breakable,
    colback=lightcoral,colframe=darkred,%
    colbacklower=tomato!75!lightcoral,%
    title=#1]}%
    {\endtcolorbox}

\newenvironment{cblock}[1]{%
    \tcolorbox[beamer,%
    noparskip,breakable,
    colback=lightblue,colframe=darkblue,%
    colbacklower=darkblue!75!lightblue,%
    title=#1]}%
    {\endtcolorbox}


%Languages
\lstdefinelanguage{JavaScript}{
  keywords={typeof, new, true, false, catch, function, return, null, catch, switch, var, if, in,  fi, while, do, else, case, break},
  keywordstyle=\color{blue}\bfseries,
  ndkeywords={class, export, boolean, throw, implements, import, this},
  ndkeywordstyle=\color{darkgray}\bfseries,
  identifierstyle=\color{black},
  sensitive=false,
  comment=[l]{//},
  morecomment=[s]{/*}{*/},
  commentstyle=\color{purple}\ttfamily,
  stringstyle=\color{red}\ttfamily,
  morestring=[b]',
  morestring=[b]"
}

%Listings
\lstset{
   language=JavaScript,
   backgroundcolor=\color{lightgray},
   extendedchars=true,
   basicstyle=\footnotesize\ttfamily,
   showstringspaces=false,
   showspaces=false,
   numbers=left,
   numberstyle=\footnotesize,
   numbersep=9pt,
   tabsize=2,
   breaklines=true,
   showtabs=false,
   captionpos=b
}

\newcommand*\circled[1]{\tikz[baseline=(char.base)]{
            \node[shape=circle,draw,inner sep=2pt] (char) {#1};}}

%Start of Document 
\begin{document}

\lecture{13}{February 24th , 2020}{Ond\u{r}ej Lhot\'{a}k}{Harsh Mistry}



\section{Context-sensitive analysis continued}

\subsection{Name Resolution}



\begin{itemize}
\item In Java : 
\textcolor{blue}{A2} and  \textcolor{red}{A3}
\begin{enumerate}
\color{blue}
\item Build global environment
\item Resolve type names
\item Build/check class hierarchy (methods/files)

\color{red}
\item Disambiguate ambiguous namespace 
\begin{itemize}
\item In Java you can't simply determine the namespace based on location of usage
\end{itemize} 
\item Resolve "expressions" (Variables, static fields)
\item Type checking
\item Resolve methods instance (non-static) fields
\end{enumerate}
\item (5) Resolving expressions (Variables)
\begin{itemize}
\item General: search outwards in nested scopes
\item  Shadowing
\begin{lstlisting}
{
	int x;
	{
		int y; 
	}
	{
		int x; // Not allowed
		int y; //Allowed
	}
}
\end{lstlisting}
\item Solution 
\begin{itemize}
\item New scope/environment for each block
\item when inserting variable into environment, check outwards for duplicates
\item Variables may shadow fields 
\end{itemize}
\item Problem
\begin{lstlisting}
{
	int x;
	x = 1;     //Can not access y or z ye
	int y;
	y  = 2;
	int z;
	z = 3;
}	
\end{lstlisting}
\item A Solution: Start a new block for every declaration
\begin{lstlisting}
{
	int x;
	x = 1;     //Can not access y or z 
	 { int y;
	y  = 2;
	{ int z;
	z = 3; 
}}}
\end{lstlisting}
\end{itemize}
\item (6) Type Checking - Will be covered after step 7
\item (7) Resolve method instances (Non-static) fields
\begin{itemize}
\item After type checking, every sub expression has a type
\item \begin{ex} Usages
\begin{itemize}
\item a.b //look for a field named b in a 
\item a.b(c) //Look for method named b in contain(type of a). Use type of c to select among  overloaded methods
\end{itemize}
\end{ex}
\end{itemize}
\item (6) Type Checking 
\begin{itemize}
\item \begin{definition} \underline{Type} is a collection of values or an interpretation of a sequence of bits \textcolor{orange}{(Reference: Luca Cardell: Type Systems)} 
\end{definition}
\item \begin{definition} \underline{Static Type}  of an expression E is a set containing all possible values of E
\end{definition}
\item \begin{definition} \underline{Dynamic Type} of a variable is an assertion that the variable will only contain values of that type
\end{definition}
\item \begin{definition} \underline{Type Checking}: Enforce declared type assertions
\end{definition}
\item \begin{definition} \underline{Static Type Checking}: Prove that every expression evaluates to a value in its type
\end{definition}
\item \begin{definition} \underline{Dynamic Type Checking}: Runtime check that the tag of a value is in the declared type of the variable to which the value is assigned
\end{definition}
\item \begin{definition} A program is \underline{type correct} if type assertions hold in all executions 
\end{definition}
\item \begin{definition} A program is \underline{statically type correct} if the program satisfies  a system of type inference rules (type system)
\end{definition}
\item \begin{definition} A type system  is \underline{sound} if statically typed correct \(\implies\) type correct
\end{definition} 
\end{itemize}
\end{itemize}

\end{document}







