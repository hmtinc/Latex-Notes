%Notes by Harsh Mistry 
%Math 136 
%Template Taken from : https://www.cs.cmu.edu/~ggordon/10725-F12/template.tex

\documentclass{article}
\setlength{\oddsidemargin}{0.25 in}
\setlength{\evensidemargin}{-0.25 in}
\setlength{\topmargin}{-0.6 in}
\setlength{\textwidth}{6.5 in}
\setlength{\textheight}{8.5 in}
\setlength{\headsep}{0.75 in}
\setlength{\parindent}{0 in}
\setlength{\parskip}{0.1 in}
\usepackage{amsfonts,graphicx, amssymb}
\usepackage[fleqn]{amsmath}
\newcounter{lecnum}
\renewcommand{\thepage}{\thelecnum-\arabic{page}}
\renewcommand{\thesection}{\thelecnum.\arabic{section}}
\renewcommand{\theequation}{\thelecnum.\arabic{equation}}
\renewcommand{\thefigure}{\thelecnum.\arabic{figure}}
\renewcommand{\thetable}{\thelecnum.\arabic{table}}
\newcommand{\lecture}[4]{
   \pagestyle{myheadings}
   \thispagestyle{plain}
   \newpage
   \setcounter{lecnum}{#1}
   \setcounter{page}{1}
   
   
%Info Box 
   \begin{center}
   \framebox{
      \vbox{\vspace{2mm}
    \hbox to 6.28in { {\bf Math 136 - Linear Algebra 
	\hfill Winter 2016} }
       \vspace{4mm}
       \hbox to 6.28in { {\Large \hfill Lecture #1: #2  \hfill} }
       \vspace{2mm}
       \hbox to 6.28in { {\it Lecturer: #3 \hfill Notes By: #4} }
      \vspace{2mm}}
   }
   \end{center}
   
   \markboth{Lecture #1: #2}{Lecture #1: #2}



 
}

\renewcommand{\cite}[1]{[#1]}
\def\beginrefs{\begin{list}%
        {[\arabic{equation}]}{\usecounter{equation}
         \setlength{\leftmargin}{2.0truecm}\setlength{\labelsep}{0.4truecm}%
         \setlength{\labelwidth}{1.6truecm}}}
\def\endrefs{\end{list}}
\def\bibentry#1{\item[\hbox{[#1]}]}

\newcommand{\fig}[3]{
			\vspace{#2}
			\begin{center}
			Figure \thelecnum.#1:~#3
			\end{center}
	}

\newtheorem{theorem}{Theorem}[lecnum]
\newtheorem{lemma}[theorem]{Lemma}
\newtheorem{ex}[theorem]{Example}
\newtheorem{proposition}[theorem]{Proposition}
\newtheorem{claim}[theorem]{Claim}
\newtheorem{corollary}[theorem]{Corollary}
\newtheorem{definition}[theorem]{Definition}
\newenvironment{proof}{{\bf Proof:}}{\hfill\rule{2mm}{2mm}}
\newcommand\E{\mathbb{E}}


%Start of Document 
\begin{document}

\lecture{7}{January 18, 2016}{Yongqiang Zhao}{Harsh Mistry}


\section{Scaler Equations Examples}
\begin{ex} Find a normal vector of the plane \(x_1 + x_2 + 3x_3 = 2\) \\
$$\vec{n} = \begin{bmatrix} 1 \\ 1 \\ 3 \end{bmatrix} $$
\end{ex}

\begin{ex} Find a scaler equivalent of the of the plane with ethe normal vector \(\begin{bmatrix} -1 \\ 1 \\ 2 \end{bmatrix} \) and passes through the point (2, 1, 0) \\
\(\vec{b} =  \begin{bmatrix} 2 \\ 1 \\ 0 \end{bmatrix}    \ \ \ ( \vec{x} - \vec{b} ) \cdot \begin{bmatrix} -1 \\ 1 \\ 2 \end{bmatrix} = 0 \implies -x_1+x_2+2x_3 = \begin{bmatrix} 2 \\ 1 \\ 0 \end{bmatrix} \cdot \begin{bmatrix} -1 \\ 1 \\ 2 \end{bmatrix} = 1  \)
\( \ \ \therefore x_1 - x_2 -2x_3 = 1\)
\end{ex}

\begin{ex} Find a scaler equation of the plan that passes through P(1, 2 , 0) , \(P_2\)(2, 1, 1), \(P_3\)(-1, 0, 2) \\
Let \( \vec{u} = \vec{P_1P_2} = \begin{bmatrix} 1 \\ -1 \\ 1 \end{bmatrix} \) and \( \vec{v} = \vec{P_1P_3} = \begin{bmatrix} -2 \\ -2 \\ 2  \end{bmatrix} \)\\
 \( \{ \vec{x} , \vec{v}\} \) is linear independant \( \implies \vec{x} = s\vec{u} + t\vec{v} + \begin{bmatrix} 1 \\ 2 \\ 0 \end{bmatrix}  \ \ \leftarrow \text{ Vector equation}\) \\
 Scaler : \( \vec{n} = \vec{u} \times \vec{v} = \begin{bmatrix} 1 \\ -1 \\ 1 \end{bmatrix} \times \begin{bmatrix} -2 \\ 2 \\ 2 \end{bmatrix} = \begin{bmatrix} 6 \\ -4 \\ 4 \end{bmatrix} \rightarrow \vec{x} \cdot \vec{n} = \begin{bmatrix} 1 \\ 2 \\ 0 \end{bmatrix} \cdot \vec{n} \implies x_2 + x_3 = 2 \) 
\end{ex}

\section{Projections}
Given \( \vec{u} \in \mathbb{R}^n \) , a Line L with \( \vec{x} = t\vec{v}+ \vec{b} , t \in \mathbb{R} \) \\
\( \vec{u} \cdot \vec{v} = ( c\vec{u} + w) \vec{v} = c \| \vec{v} \| ^ 2 \) 
\begin{definition} 
- Given \(\vec{v} \in \mathbb{R} , \vec{u} \in \mathbb{R}^n, \vec{v} \neq \vec{0 } \) \\
We define the projection of \( \vec{u} \text{ on } \vec{v} \) as
 \[ Proj_{\vec{v}} ( \vec{u}) = \frac{\vec{u} \cdot \vec{v}}{\|\vec{v} \| ^2} \vec{v} \]
 Also the perpendicular vector of \( \vec{u} \text{ onto }  \vec{v} \) is defined as \\
 \[ Perp_{\vec{v}} (\vec{u}) = \vec{u} - Proj_{\vec{v}} ( \vec{u}) \]
\end {definition}

\begin{proposition} Given \(\vec{v} \in \mathbb{R}^n , \vec{v} \neq \vec{0} , \text{ and } c\in \mathbb{R} , c \neq 0 \) we have 
\begin{enumerate}
\item  \( Proj_{c\vec{v}} ( \vec{u}) = Proj_{\vec{v}} ( \vec{u}) \)
\item \( Proj_{\vec{v}} (\cdot)\) is the "Linear Map"
\item \( Perp_{\vec{v}} (\cdot)\) is also a linearmap
\end{enumerate}
\end{proposition}

\section{Projection onto Planes}
For a Plane P with \( \vec{n} \) , we define the projection to P as \( Proj_P (\vec{u}) = Perp_{\vec{n}} ( \vec{u}) \)\\

\textbf{To Be Continued Next Lecture }

\begin{center}
\textbf{End of Lecture Notes} \\
\textbf{Notes By : Harsh Mistry}
\end{center}
\end{document}
