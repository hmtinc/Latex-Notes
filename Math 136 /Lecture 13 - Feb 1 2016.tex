%Notes by Harsh Mistry 
%Math 136 
%based on Template from : https://www.cs.cmu.edu/~ggordon/10725-F12/template.tex

\documentclass{article}
\setlength{\oddsidemargin}{0.25 in}
\setlength{\evensidemargin}{-0.25 in}
\setlength{\topmargin}{-0.6 in}
\setlength{\textwidth}{6.5 in}
\setlength{\textheight}{8.5 in}
\setlength{\headsep}{0.75 in}
\setlength{\parindent}{0 in}
\setlength{\parskip}{0.1 in}
\usepackage{amsfonts,graphicx, amssymb}
\usepackage[fleqn]{amsmath}
\newcounter{lecnum}
\renewcommand{\thepage}{\thelecnum-\arabic{page}}
\renewcommand{\thesection}{\thelecnum.\arabic{section}}
\renewcommand{\theequation}{\thelecnum.\arabic{equation}}
\renewcommand{\thefigure}{\thelecnum.\arabic{figure}}
\renewcommand{\thetable}{\thelecnum.\arabic{table}}
\newcommand{\lecture}[4]{
   \pagestyle{myheadings}
   \thispagestyle{plain}
   \newpage
   \setcounter{lecnum}{#1}
   \setcounter{page}{1}
   
   
%Info Box 
   \begin{center}
   \framebox{
      \vbox{\vspace{2mm}
    \hbox to 6.28in { {\bf Math 136 - Linear Algebra 
	\hfill Winter 2016} }
       \vspace{4mm}
       \hbox to 6.28in { {\Large \hfill Lecture #1: #2  \hfill} }
       \vspace{2mm}
       \hbox to 6.28in { {\it Lecturer: #3 \hfill Notes By: #4} }
      \vspace{2mm}}
   }
   \end{center}
   
   \markboth{Lecture #1: #2}{Lecture #1: #2}



 
}

\renewcommand{\cite}[1]{[#1]}
\def\beginrefs{\begin{list}%
        {[\arabic{equation}]}{\usecounter{equation}
         \setlength{\leftmargin}{2.0truecm}\setlength{\labelsep}{0.4truecm}%
         \setlength{\labelwidth}{1.6truecm}}}
\def\endrefs{\end{list}}
\def\bibentry#1{\item[\hbox{[#1]}]}

\newcommand{\fig}[3]{
			\vspace{#2}
			\begin{center}
			Figure \thelecnum.#1:~#3
			\end{center}
	}

\newtheorem{theorem}{Theorem}[lecnum]
\newtheorem{lemma}[theorem]{Lemma}
\newtheorem{ex}[theorem]{Example}
\newtheorem{proposition}[theorem]{Proposition}
\newtheorem{claim}[theorem]{Claim}
\newtheorem{corollary}[theorem]{Corollary}
\newtheorem{definition}[theorem]{Definition}
\newenvironment{proof}{{\bf Proof:}}{\hfill\rule{2mm}{2mm}}
\newcommand\E{\mathbb{E}}


%Start of Document 
\begin{document}

\lecture{13}{February 1, 2016}{Yongqiang Zhao}{Harsh Mistry}


\section{Matrix Multiplication Continued}
Matrix Multiplication can be used to denote dot product \( \vec{x}^T \vec{y} = \vec{x} \cdot \vec{y} \) \\


\textbf{Identity Matrix : } \\
\[ I_1 = \begin{bmatrix} 1 \end{bmatrix}   I_2 = \begin{bmatrix} 1 & 0 \\ 0 & 1\end{bmatrix} \  \ldots  \ I_n - \begin{bmatrix}
1 & 0 & \ldots & 0 \\ 0 & 1 & \ldots & 0 \\ 0 & 0 & \ldots & 1 \end{bmatrix} \]
\[ \forall A = (a_{ij}) \in M_{m \times n} (\mathbb{R}), AI_n = A \  I_mA = A \]

\begin{proposition} For materices A, B, C (assuming the required product makes sense), \( t \in \mathbb{R}\) we have, 
\begin{enumerate}
\item A(B+C) = AB + AC
\item t(AB) = (tA)B = A(tB)
\item A(BC) = (AB) C
\item \((AB)^T) = B^T A^T\)
\end{enumerate}

\begin{ex}
Given \( A_{m\times n} ,  B_{n\times s} ,  C_{n \times s} \) Does the following statement hold : \( AB = AC \implies B = C \) ?\\
No,  To show this, Let A = \( \begin{bmatrix}  0 & 1 \\ 0 & 0  \end{bmatrix} \) B = \( \begin{bmatrix} 1 & 2 \\ 0 & 0  \end{bmatrix} \) c = \( \begin{bmatrix}  0 & 2016 \\ 0 & 0  \end{bmatrix} \)
\[  AB = \begin{bmatrix}  0 & 0 \\ 0 & 0  \end{bmatrix}  \text{ and } AC = \begin{bmatrix}  0 & 0 \\ 0 & 0  \end{bmatrix}\implies AB = AC, 
\text{ but } B \neq C \]
\end{ex}
\end{proposition}

\begin{theorem}
\(\forall A , B \in M_{m\times n } (\mathbb{R}) \) Such That \(A\vec{x} = B \vec{x} \) for any \(\vec{x} \in \mathbb{R}^n , then A = B \) 
\end{theorem}

\begin{proof}
Given A = \( (a_1, a_2, \ldots, a_n) \) and B = \( (b_1, b_2, \ldots, b_n) \) \\
\[ \text{Set } \vec{a_i} = \begin{bmatrix} a_{1i} \\ \vdots \\ a_{mi} \end{bmatrix} , 1 \leq i \leq n \text{ and } \vec{b_i} = \begin{bmatrix} b_{1i} \\ \vdots \\ b_{mi} \end{bmatrix} , 1 \leq i \leq n \]
\[ A\vec{e_1} = \vec{a_1} \  \ldots \ A\vec{e_n} = \vec{a_n} \text{ and } B\vec{e_1} = \vec{b_1} \  \ldots \ B\vec{e_n} = \vec{b_n}  \implies A
 = B\]

\end{proof}

\textbf{Note} : 
\[ AB = A (b_1, b_2, \ldots, b_n) =  (Ab_1, Ab_2, \ldots, Ab_n)\]

\begin{center}
\textbf{End of Lecture Notes} \\
\textbf{Notes By : Harsh Mistry}
\end{center}
\end{document}
