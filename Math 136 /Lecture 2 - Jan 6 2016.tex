%Notes by Harsh Mistry 
%Math 136 
%Template Taken from : https://www.cs.cmu.edu/~ggordon/10725-F12/template.tex

\documentclass[twoside]{article}
\setlength{\oddsidemargin}{0.25 in}
\setlength{\evensidemargin}{-0.25 in}
\setlength{\topmargin}{-0.6 in}
\setlength{\textwidth}{6.5 in}
\setlength{\textheight}{8.5 in}
\setlength{\headsep}{0.75 in}
\setlength{\parindent}{0 in}
\setlength{\parskip}{0.1 in}
\usepackage{amsfonts,graphicx, amssymb}
\usepackage[fleqn]{amsmath}
\newcounter{lecnum}
\renewcommand{\thepage}{\thelecnum-\arabic{page}}
\renewcommand{\thesection}{\thelecnum.\arabic{section}}
\renewcommand{\theequation}{\thelecnum.\arabic{equation}}
\renewcommand{\thefigure}{\thelecnum.\arabic{figure}}
\renewcommand{\thetable}{\thelecnum.\arabic{table}}
\newcommand{\lecture}[4]{
   \pagestyle{myheadings}
   \thispagestyle{plain}
   \newpage
   \setcounter{lecnum}{#1}
   \setcounter{page}{1}
   
   
%Info Box 
   \begin{center}
   \framebox{
      \vbox{\vspace{2mm}
    \hbox to 6.28in { {\bf Math 136 - Linear Algebra 
	\hfill Winter 2016} }
       \vspace{4mm}
       \hbox to 6.28in { {\Large \hfill Lecture #1: #2  \hfill} }
       \vspace{2mm}
       \hbox to 6.28in { {\it Lecturer: #3 \hfill Notes By: #4} }
      \vspace{2mm}}
   }
   \end{center}
   
   \markboth{Lecture #1: #2}{Lecture #1: #2}



 
}

\renewcommand{\cite}[1]{[#1]}
\def\beginrefs{\begin{list}%
        {[\arabic{equation}]}{\usecounter{equation}
         \setlength{\leftmargin}{2.0truecm}\setlength{\labelsep}{0.4truecm}%
         \setlength{\labelwidth}{1.6truecm}}}
\def\endrefs{\end{list}}
\def\bibentry#1{\item[\hbox{[#1]}]}

\newcommand{\fig}[3]{
			\vspace{#2}
			\begin{center}
			Figure \thelecnum.#1:~#3
			\end{center}
	}

\newtheorem{theorem}{Theorem}[lecnum]
\newtheorem{lemma}[theorem]{Lemma}
\newtheorem{ex}[theorem]{Example}
\newtheorem{proposition}[theorem]{Proposition}
\newtheorem{claim}[theorem]{Claim}
\newtheorem{corollary}[theorem]{Corollary}
\newtheorem{definition}[theorem]{Definition}
\newenvironment{proof}{{\bf Proof:}}{\hfill\rule{2mm}{2mm}}
\newcommand\E{\mathbb{E}}


%Start of Document 
\begin{document}

\lecture{2}{January 6, 2016}{Yongqiang Zhao}{Harsh Mistry}


% **** YOUR NOTES GO HERE:



%
\section{Proofs for Fundemental Operations}

\textbf{Commutativity}
$$\text{Let  } \vec{x}=\begin{bmatrix} x_1 \\ \vdots \\ xn \end{bmatrix} \text{  and  }\vec{y}=\begin{bmatrix} y_1 \\ \vdots \\ y_n \end{bmatrix} \text{   be 2 vectors}$$
$$\vec{x}+\vec{y} = \begin{bmatrix} x_1 +  y_1 \\ \vdots \\ x_n + y_n  \end{bmatrix} =  \begin{bmatrix} x_1 \\ \vdots \\ x_n \end{bmatrix} + \begin{bmatrix} y_1 \\ \vdots \\ y_n \end{bmatrix} = \vec{y} + \vec{x} $$


$$\therefore \vec{x}+\vec{y} = \vec{y} + \vec{x}$$


\textbf{Vector Distributivity}
$$ c(\vec{x}+\vec{y}) = c \begin{bmatrix} x_1 +  y_1 \\ \vdots \\ x_n + y_n  \end{bmatrix} = \begin{bmatrix} c(x_1 +  y_1) \\ \vdots \\ c(x_n + y_n)  \end{bmatrix} = \begin{bmatrix} cx_1 + c y_1 \\ \vdots \\ cx_n + cy_n  \end{bmatrix} = \begin{bmatrix} cx_1 \\ \vdots \\ cx_n \end{bmatrix} + \begin{bmatrix} cy_1 \\ \vdots \\ cy_n \end{bmatrix} = c\vec{x} + c\vec{y}$$



$$\therefore c(\vec{x}+\vec{y})= c\vec{x} + c\vec{y}$$

\section{Span}
\begin{definition}
\textbf{Linear Span}
\\ For a given set of vectors \[B = \{ \vec{u} , \vec{u_2} \ldots \vec{u_n} \}  \subset \mathbb{R}^n \]
we define the span set as
\[ Span_b  = \{ t_1\vec{v_1} + t_2\vec{v_2} + t_n\vec{v_n} \mid t_1 \ldots t_n \in \mathbb{R}\} \, \  \ \text{B is a spanning set of } Span_b \]

So, a span is just the set of all linear combinations of the vectors in the set, which can also be written as
\[Span{\vec{v_1}, \vec{v_2} \ldots \vec{v_n}} \] 
\end {definition}

\newpage
\begin{ex}
\[\textbf{Show that  } S = Span\left\{ \begin{bmatrix} 1 \\ 0 \\ 0\end{bmatrix} \begin{bmatrix} 0 \\ 1 \\ 0\end{bmatrix} \begin{bmatrix} 0 \\ 0 \\ 1\end{bmatrix} \right\} = \mathbb{R}^3 \]
\[\text{Clearly } S \subseteq \mathbb{R}^3 \text{, we need to show } \mathbb{R}^3 \subseteq S \]
\[\implies \forall \vec{x} = \begin{bmatrix} x \\ y \\z \end{bmatrix} \in \mathbb{R}^3 \]
\[\implies \vec{x} =\begin{bmatrix} x \\ y \\z \end{bmatrix} =  \vec{x}\begin{bmatrix} 1 \\ 0 \\ 0\end{bmatrix} \vec{y}\begin{bmatrix} 0 \\ 1 \\ 0\end{bmatrix} \vec{z}\begin{bmatrix} 0 \\ 0 \\ 1\end{bmatrix}  \in S \]
\[\implies \mathbb{R}^3 \subseteq S \text{ thus } S = \mathbb{R}^3 \]
\end{ex}

\begin{ex}
\[ \text {Does the spanning set } S = Span \left\{ \begin{bmatrix} 1 \\ 0 \\ 2 \end{bmatrix} \begin{bmatrix} 2 \\ 0 \\ 1 \end{bmatrix} \right\} \text{ contain the vector } \begin{bmatrix} 0 \\ 1 \\ 0 \end{bmatrix} \]
\[ \textbf{Solution : } \text{Suppose that } \begin{bmatrix} 0 \\ 1 \\ 0\end{bmatrix} \in S \]
\begin{equation}
\begin{aligned}
\begin{bmatrix} 0 \\ 1 \\ 0\end{bmatrix} & =  t_1\begin{bmatrix} 1 \\ 0 \\ 2 \end{bmatrix} + t_2 \begin{bmatrix} 2 \\ 0 \\ 1 \end{bmatrix} \\
& = \begin{bmatrix} t_1 + 2t_2 \\ 0 \\ 2t_1 + t_1\end{bmatrix} \\
1 & = 0 \textbf{, Contradiction, 1 can't equal 0}
\end{aligned}
\end{equation}

\[ \therefore  \begin{bmatrix} 0 \\ 1 \\ 0 \end{bmatrix} \text{ is not in the span set } S \]

\end{ex} 

\begin{ex}
\[ S = Span \left\{ \begin{bmatrix} 1 \\ 2 \end{bmatrix} \begin{bmatrix} 2 \\ 4 \end{bmatrix} \right\} \subseteq \mathbb{R}^2 \textbf{ Show } S = Span \left\{\begin{bmatrix} 1 \\ 2 \end{bmatrix} \right\} \]

\begin{equation}
\begin{aligned}
\text{for any } \vec{x} \in S \text{ , } \vec{x} = t_1 \begin{bmatrix} 1 \\ 2\end{bmatrix} + \begin{bmatrix} 2 \\ 4 \end{bmatrix} & = t_1 \begin{bmatrix} 1 \\ 2\end{bmatrix} + 2t_2 \begin{bmatrix} 1 \\ 2\end{bmatrix} \\
& = (t_1 + t_2) \begin{bmatrix} 1 \\ 2\end{bmatrix} \ \text{Let t =  } (t_1 + t_2) \\
& = t\begin{bmatrix} 1 \\ 2\end{bmatrix}  \in Span\begin{bmatrix} 1 \\ 2\end{bmatrix} 
\end{aligned}
\end{equation}
\end{ex}



\begin{center}
\textbf{End of Lecture Notes} \\
\textbf{Notes By : Harsh Mistry}
\end{center}
\end{document}