%Notes by Harsh Mistry 
%Math 136 
%Template Taken from : https://www.cs.cmu.edu/~ggordon/10725-F12/template.tex

\documentclass{article}
\setlength{\oddsidemargin}{0.25 in}
\setlength{\evensidemargin}{-0.25 in}
\setlength{\topmargin}{-0.6 in}
\setlength{\textwidth}{6.5 in}
\setlength{\textheight}{8.5 in}
\setlength{\headsep}{0.75 in}
\setlength{\parindent}{0 in}
\setlength{\parskip}{0.1 in}
\usepackage{amsfonts,graphicx, amssymb}
\usepackage[fleqn]{amsmath}
\newcounter{lecnum}
\renewcommand{\thepage}{\thelecnum-\arabic{page}}
\renewcommand{\thesection}{\thelecnum.\arabic{section}}
\renewcommand{\theequation}{\thelecnum.\arabic{equation}}
\renewcommand{\thefigure}{\thelecnum.\arabic{figure}}
\renewcommand{\thetable}{\thelecnum.\arabic{table}}
\newcommand{\lecture}[4]{
   \pagestyle{myheadings}
   \thispagestyle{plain}
   \newpage
   \setcounter{lecnum}{#1}
   \setcounter{page}{1}
   
   
%Info Box 
   \begin{center}
   \framebox{
      \vbox{\vspace{2mm}
    \hbox to 6.28in { {\bf Math 136 - Linear Algebra 
	\hfill Winter 2016} }
       \vspace{4mm}
       \hbox to 6.28in { {\Large \hfill Lecture #1: #2  \hfill} }
       \vspace{2mm}
       \hbox to 6.28in { {\it Lecturer: #3 \hfill Notes By: #4} }
      \vspace{2mm}}
   }
   \end{center}
   
   \markboth{Lecture #1: #2}{Lecture #1: #2}



 
}

\renewcommand{\cite}[1]{[#1]}
\def\beginrefs{\begin{list}%
        {[\arabic{equation}]}{\usecounter{equation}
         \setlength{\leftmargin}{2.0truecm}\setlength{\labelsep}{0.4truecm}%
         \setlength{\labelwidth}{1.6truecm}}}
\def\endrefs{\end{list}}
\def\bibentry#1{\item[\hbox{[#1]}]}

\newcommand{\fig}[3]{
			\vspace{#2}
			\begin{center}
			Figure \thelecnum.#1:~#3
			\end{center}
	}

\newtheorem{theorem}{Theorem}[lecnum]
\newtheorem{lemma}[theorem]{Lemma}
\newtheorem{ex}[theorem]{Example}
\newtheorem{proposition}[theorem]{Proposition}
\newtheorem{claim}[theorem]{Claim}
\newtheorem{corollary}[theorem]{Corollary}
\newtheorem{definition}[theorem]{Definition}
\newenvironment{proof}{{\bf Proof:}}{\hfill\rule{2mm}{2mm}}
\newcommand\E{\mathbb{E}}


%Start of Document 
\begin{document}

\lecture{8}{January 20, 2016}{Yongqiang Zhao}{Harsh Mistry}


\section{Projection Examples }
\begin{ex}
Find projection of \(\vec{u} = \begin{bmatrix} 1 \\ -1 \\-1 \end{bmatrix} \) onto the line \( \vec{x} = t\begin{bmatrix} 1 \\ 2 \\2 \end{bmatrix} + \begin{bmatrix} 2016 \\ 2017\\ 2018 \end{bmatrix}\)\\
\[ Proj_{\vec{v}} (\vec{u}) = \frac{\vec{u}\cdot\vec{v}}{\|\vec{v}\|^2} \vec{v} = \frac{-3}{9} \begin{bmatrix} 1 \\ 2\\ 2 \end{bmatrix} = \begin{bmatrix} \frac{-1}{3} \\ \frac{-2}{3} \\ \frac{-2}{3}  \end{bmatrix}\]
\end{ex}
\begin{ex}
Find the projection of \( \vec{u}\) onto the plane \( 2x_1 - x_2 + 2x_3 = 2016 \)\\
\[ \begin{aligned} \vec{n} & = \begin{bmatrix} 2 \\ -1 \\ 2 \end{bmatrix} \\
 Proj_{\vec{v}} (\vec{u}) & = \frac{\vec{u}\cdot\vec{v}}{\|\vec{v}\|^2} \vec{v} = \frac{1}{9} \begin{bmatrix} 2\\ -1\\ 2 \end{bmatrix}\\
 Proj_{Plane} (\vec{u}) & = \vec{u} - Proj_{\vec{n}} (\vec{u}) \\
& = \begin{bmatrix} 1 \\ -1\\ -1 \end{bmatrix} - \begin{bmatrix} \frac{2}{9} \\ \frac{3}{9}\\ \frac{2}{9} \end{bmatrix} = \begin{bmatrix} \frac{7}{9} \\ \frac{-8}{9}\\ \frac{-11}{9} \end{bmatrix} \end{aligned} \]
\end{ex}


\section{Chapter 2 : System of Linear Equations}
\begin{definition}
A set of m linear equations with n variables \(x_1 \ldots x_n \) is called a system of m linear equations \\
\[ (\ast) \begin{cases} a_{11} x_1 + \ldots + a_{1n} x_n = b_1 \\ a_{21}x_1 + \ldots + a_{2n} x_n = b_2 \\
\vdots \\ a_{m1}x_1 + \ldots + a_{mn} x_n = b_m \end{cases} \]

A solution for \( (\ast) \) will be written as \(\begin{bmatrix} x_1 \\ \vdots \\ x_n \end{bmatrix} \) solution vector \\
If \( (\ast) \) has atleast 1 solution, then it is consistent, otherwise it is inconsistent
\end{definition}

\begin{ex}
\[\begin{cases} x + y = 1 \\ x - y = 1 \end{cases} \]
\[\begin{cases} x = 1 \\ y = 0 \end{cases}  \text{ A unique solution} \]
\end{ex}

\begin{ex}
\[\begin{cases} 2x - y = 2 \\ -x - \frac{1}{2}y = 1 \end{cases} \iff \begin{cases} 2x -  y = 2 \\ 2x - y = 2 \end{cases} \iff 2x - y = 2 \text{ Let y = 2t - 2 and x = t}\]
\[\begin{bmatrix} x \\ y \end{bmatrix} = \begin{bmatrix} t \\ 2t-2 \end{bmatrix}  = t \begin{bmatrix} 1 \\ 2 \end{bmatrix} + \begin{bmatrix} 0 \\ -2 \end{bmatrix}\]
Solution Set : \( \{ \vec{x} \mid \vec{x} = t \begin{bmatrix} 1 \\ 2 \end{bmatrix} + \begin{bmatrix} 0 \\ 2  \end{bmatrix} , t \in \mathbb{R} \} \),  There is a infinite number of solutions 
\end{ex}

\begin{ex}
\[\begin{cases} x + 3y = 6 \\ 3x + 9y = 10 \end{cases} \iff  \begin{cases} x + 3y = 1 \\ x + 3y = \frac{10}{3} \end{cases} \] 
Not consistent! , So there is no solution
\end{ex}

\textbf{Geometry} : Solving linear system \( \iff \) finding the intersection of set of hypothesis in \( \mathbb{R}^n\) 

\begin{definition} - \\ A linear system that has the form \\
\[ (\ast \ast) \begin{cases} a_{11} x_1 + \ldots + a_{1n} x_n = 0 \\ a_{21}x_1 + \ldots + a_{2n} x_n = 0 \\
\vdots \\ a_{m1}x_1 + \ldots + a_{mn} x_n = 0 \end{cases} \]
is called a homogeneous system
\end{definition}

\textbf{Remark : } A homogeneous systenm is alwasy consistent as \( \vec{0} = \begin{bmatrix}
0 \\ 0  \\ \vdots \\ 0
\end{bmatrix}\)

\begin{theorem} The solution set of \( (\ast \ast) \) is a subspace of \( \mathbb{R} ^n \) 
\end{theorem}
\begin{theorem} Given a linear system that is consistent \\
\[ (A) \begin{cases} a_{11} x_1 + \ldots + a_{1n} x_n = b_1 \\ a_{21}x_1 + \ldots + a_{2n} x_n = b_2 \\
\vdots \\ a_{m1}x_1 + \ldots + a_{mn} x_n = b_m \end{cases} \]
\[ (B) \begin{cases} a_{11} x_1 + \ldots + a_{1n} x_n = 0 \\ a_{21}x_1 + \ldots + a_{2n} x_n = 0 \\
\vdots \\ a_{m1}x_1 + \ldots + a_{mn} x_n = 0 \end{cases} \]
To be Continued Next Lecture
\end{theorem}

\begin{center}
\textbf{End of Lecture Notes} \\
\textbf{Notes By : Harsh Mistry}
\end{center}
\end{document}
