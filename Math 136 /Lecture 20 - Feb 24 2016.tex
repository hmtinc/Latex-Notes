%Notes by Harsh Mistry 
%Math 136 
%based on Template from : https://www.cs.cmu.edu/~ggordon/10725-F12/template.tex

\documentclass{article}
\setlength{\oddsidemargin}{0.25 in}
\setlength{\evensidemargin}{-0.25 in}
\setlength{\topmargin}{-0.6 in}
\setlength{\textwidth}{6.5 in}
\setlength{\textheight}{8.5 in}
\setlength{\headsep}{0.75 in}
\setlength{\parindent}{0 in}
\setlength{\parskip}{0.1 in}
\usepackage{amsfonts,graphicx, amssymb}
\usepackage[fleqn]{amsmath}
\usepackage{fixltx2e}
\newcounter{lecnum}
\renewcommand{\thepage}{\thelecnum-\arabic{page}}
\renewcommand{\thesection}{\thelecnum.\arabic{section}}
\renewcommand{\theequation}{\thelecnum.\arabic{equation}}
\renewcommand{\thefigure}{\thelecnum.\arabic{figure}}
\renewcommand{\thetable}{\thelecnum.\arabic{table}}
\newcommand{\lecture}[4]{
   \pagestyle{myheadings}
   \thispagestyle{plain}
   \newpage
   \setcounter{lecnum}{#1}
   \setcounter{page}{1}
   
   
%Info Box 
   \begin{center}
   \framebox{
      \vbox{\vspace{2mm}
    \hbox to 6.28in { {\bf Math 136 - Linear Algebra 
	\hfill Winter 2016} }
       \vspace{4mm}
       \hbox to 6.28in { {\Large \hfill Lecture #1: #2  \hfill} }
       \vspace{2mm}
       \hbox to 6.28in { {\it Lecturer: #3 \hfill Notes By: #4} }
      \vspace{2mm}}
   }
   \end{center}
   
   \markboth{Lecture #1: #2}{Lecture #1: #2}



 
}

\renewcommand{\cite}[1]{[#1]}
\def\beginrefs{\begin{list}%
        {[\arabic{equation}]}{\usecounter{equation}
         \setlength{\leftmargin}{2.0truecm}\setlength{\labelsep}{0.4truecm}%
         \setlength{\labelwidth}{1.6truecm}}}
\def\endrefs{\end{list}}
\def\bibentry#1{\item[\hbox{[#1]}]}

\newcommand{\fig}[3]{
			\vspace{#2}
			\begin{center}
			Figure \thelecnum.#1:~#3
			\end{center}
	}

\newtheorem{theorem}{Theorem}[lecnum]
\newtheorem{lemma}[theorem]{Lemma}
\newtheorem{ex}[theorem]{Example}
\newtheorem{proposition}[theorem]{Proposition}
\newtheorem{claim}[theorem]{Claim}
\newtheorem{corollary}[theorem]{Corollary}
\newtheorem{definition}[theorem]{Definition}
\newenvironment{proof}{{\bf Proof:}}{\hfill\rule{2mm}{2mm}}
\newcommand\E{\mathbb{E}}


%Start of Document 
\begin{document}

\lecture{20}{February 24, 2016}{Yongqiang Zhao}{Harsh Mistry}

\section{Vector Spaces Continued}
$$C(\mathbb{R}) = \{ f : \mathbb{R} \rightarrow \mathbb{R} \mid \text{ f is continuous } \}$$
\begin{definition}
Let V and W be 2 vector spaces we define the cartesian product of V and W as :
$$ V \times W = \{(\vec{v}, \vec{w}) \mid \vec{v} \in V, \vec{w} \in W \} $$
\end{definition}

If we define addition and scaler multiplication as following \\
$$(\vec{v_1},\vec{w_1}) + (\vec{v_2},\vec{w_2}) = (\vec{v_1} + \vec{v_2} , \vec{w_1}  + \vec{w_2}) $$
$$ c(\vec{v_1},\vec{w_1}) = (c\vec{v_1},c\vec{w_1}) $$
then V x W is a vector space 

\begin{theorem}
In any vector in Span V, \( 0 \vec{v} = \vec{0}, \forall \vec{v} \in V \), Also, -\(\vec{v} = (-1)\vec{v} \)
\end{theorem}

\begin{proof}
\[0\vec{v} = (0  + 0) \vec{v} = 0\vec{v} + 0\vec{v} \]
\[0\vec{v} + (-0\vec{v}) = 0\vec{v} + 0\vec{v} + (-0\vec{v})\]
\[\vec{0} = 0\vec{v} + \vec{0} = \vec{0\vec{v}}\]
\[\therefore \vec{0} = 0\vec{v}\]
\end{proof}

\begin{definition}
A non=empty subset of a vector space V is called a subaspce if :
\begin{itemize}
\item \(\forall \vec{x}, \vec{y} \in S , \vec{x} + \vec{y} \in S \)
\item \(\forall \vec{x} \in S, \forall c \in \mathbb{R}, c\vec{x} \in S \)
\end{itemize}
\end{definition}

\textbf{Trivial Examples : } V and \(\{\vec{0}\}\) are subspaces
\newpage
\begin{ex}
Prove S is subsapce of \(M_{m\times n}\) 
$$ S = \left\{ \begin{bmatrix} a & b \\ c & d \end{bmatrix} \mid a + b - c - d = 0 \right\} $$
Clearly S is non-empty as \( \begin{bmatrix} 0 &  0 \\ 0 & 0 \end{bmatrix} \in S \)
 \[ \begin{aligned} A_1 + A_2 & = \begin{bmatrix} a_1 + a_2 & b_1 + b_2 \\ c_1 + c_2 & d_1 + d_2\end{bmatrix} \\ 
& = (a_1 + a_2) + (b_1 + b_2) - (c_1 + c_2) - (d_1 + d_2) \\
& = (a_1 + b_1 - c_2 - d_1) + (a_2 + b_2 - c_2 - d_2) = 0\\
& \implies A_1 + A_2 \in S \\
 tA & = \begin{bmatrix} ta & tb \\ tc & td \end{bmatrix} \\
 & = ta + tb - tc - td = 0  \\
 & \implies tA \in S \end{aligned} \]
\(\therefore\) S is a subspace of \(M_{m\times n}\)
\end{ex}

\begin{definition}
A square matrix A is called symmetric if \(A = A^T\) 
$$ Sym_n(\mathbb{R}) = \{ A \in M_{m\times n} (\mathbb{R}) \mid A = A^T \} $$
\end{definition}

\textbf{Check to see if its a subspace}
\begin{itemize}
\item \( 0 \in Sym_n(\mathbb{R}) \rightarrow 0T = 0 \)
\item \(\forall AB \in Sym_n(\mathbb{R}), (A + B)^T = A^T + B^T = A + B \implies A + B \in Sym_n(\mathbb{R}) \)
\item \(tA \in Sym_n(\mathbb{R})\)
\end{itemize}

\textbf{Note : }\(Sym_n(\mathbb{R}) = \{ ((a_{ij}) \mid a_{ij} = a_{ji} \} \)

\begin{definition}
A square matrix M is called Skew-symmetric if \(A = -A^T\)
$$ S_n = \{ A \in M_{m\times n} (\mathbb{R}) \mid A = -A^T \} $$
\end{definition}

\begin{center}
\textbf{End of Lecture Notes}\\
\textbf{Notes by : Harsh Mistry}
\end{center}
\end{document}
