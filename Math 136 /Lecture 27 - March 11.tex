%Notes by Harsh Mistry 
%Math 136 
%based on Template from : https://www.cs.cmu.edu/~ggordon/10725-F12/template.tex

\documentclass{article}
\setlength{\oddsidemargin}{0.25 in}
\setlength{\evensidemargin}{-0.25 in}
\setlength{\topmargin}{-0.6 in}
\setlength{\textwidth}{6.5 in}
\setlength{\textheight}{8.5 in}
\setlength{\headsep}{0.75 in}
\setlength{\parindent}{0 in}
\setlength{\parskip}{0.1 in}
\usepackage{amsfonts,graphicx, amssymb}
\usepackage[fleqn]{amsmath}
\usepackage{fixltx2e}
\newcounter{lecnum}
\renewcommand{\thepage}{\thelecnum-\arabic{page}}
\renewcommand{\thesection}{\thelecnum.\arabic{section}}
\renewcommand{\theequation}{\thelecnum.\arabic{equation}}
\renewcommand{\thefigure}{\thelecnum.\arabic{figure}}
\renewcommand{\thetable}{\thelecnum.\arabic{table}}
\newcommand{\lecture}[4]{
   \pagestyle{myheadings}
   \thispagestyle{plain}
   \newpage
   \setcounter{lecnum}{#1}
   \setcounter{page}{1}
   
   
%Info Box 
   \begin{center}
   \framebox{
      \vbox{\vspace{2mm}
    \hbox to 6.28in { {\bf Math 136 - Linear Algebra 
	\hfill Winter 2016} }
       \vspace{4mm}
       \hbox to 6.28in { {\Large \hfill Lecture #1: #2  \hfill} }
       \vspace{2mm}
       \hbox to 6.28in { {\it Lecturer: #3 \hfill Notes By: #4} }
      \vspace{2mm}}
   }
   \end{center}
   
   \markboth{Lecture #1: #2}{Lecture #1: #2}



 
}

\renewcommand{\cite}[1]{[#1]}
\def\beginrefs{\begin{list}%
        {[\arabic{equation}]}{\usecounter{equation}
         \setlength{\leftmargin}{2.0truecm}\setlength{\labelsep}{0.4truecm}%
         \setlength{\labelwidth}{1.6truecm}}}
\def\endrefs{\end{list}}
\def\bibentry#1{\item[\hbox{[#1]}]}

\newcommand{\fig}[3]{
			\vspace{#2}
			\begin{center}
			Figure \thelecnum.#1:~#3
			\end{center}
	}

\newtheorem{theorem}{Theorem}[lecnum]
\newtheorem{lemma}[theorem]{Lemma}
\newtheorem{ex}[theorem]{Example}
\newtheorem{proposition}[theorem]{Proposition}
\newtheorem{claim}[theorem]{Claim}
\newtheorem{corollary}[theorem]{Corollary}
\newtheorem{definition}[theorem]{Definition}
\newenvironment{proof}{{\bf Proof:}}{\hfill\rule{2mm}{2mm}}
\newcommand\E{\mathbb{E}}


%Start of Document 
\begin{document}

\lecture{27}{March 11, 2016}{Yongqiang Zhao}{Harsh Mistry}

\section{Determinants}

\begin{definition}
Let \(A = \begin{bmatrix} a & b \\ c & d\end{bmatrix}\). We define the determinant of \(A\) to be \(ad - bc\) and write 
$$ det A = ad - bc = \begin{vmatrix} a & b \\ c & d\end{vmatrix} $$
\end{definition}

\begin{definition}
Let \(A\) bs an \(n \times n\) matrix with \(n > 1\). Let A(i,j) be the \((n-1) \times (n - 1)\) matrix obtained from A by deleting the i-th row and the j-th column. the cofactor \(a_{ij}\) is 
$$ C_{ij} = (-1)^{i + j} detA(i,j)$$
\end{definition}

\begin{definition}
If  \(A\) is the \(1 \times 1\) matrix \( A = [a]\), then \(detA = a\). If A is an \(n \times n\) matrix with \(n \geqslant 2\), then the \textbf{determinant} of A is defined to be
$$ detA = \sum_{j = 1}^{n} a_{1j}C_{1j} $$
\end{definition}

\textbf{Remarks :}\\
\begin{enumerate}
\item The Determinant of am \(n \times n\) matrix is defined in terms of cofactors which are determinants of \((n-1) \times (n-1)\)
\item We often repersent the determinant of a matrix with vertical straight lines. 
$$det\begin{bmatrix} a_{11} & a_{12} \\ a _{21} & a{22}\end{bmatrix} = \begin{vmatrix} a_{11} & a_{12} \\ a _{21} & a_{22}\end{vmatrix}$$
\end{enumerate}

\begin{theorem}
Let A be an \(n \times n\) matrix, For any \(i\) with \(1 \leqslant i \leqslant n\) \\
$$ detA = \sum_{k = 1}^{n} a_{ik}C_{ik} $$
is called \textbf{the cofactor expansion across the i-th row}, Or for any j with \(1 \leqslant j \leqslant n\) 
$$ detA = \sum_{k = 1}^{n} a_{kj}C_{kj} $$
is called \textbf{the cofactor expansion across the j-th column}
\end{theorem}

\begin{center}
\textbf{End of Lecture Notes}\\
\textbf{Notes by : Harsh Mistry}
\end{center}
\end{document}
