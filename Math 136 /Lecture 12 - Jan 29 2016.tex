%Notes by Harsh Mistry 
%Math 136 
%Template Taken from : https://www.cs.cmu.edu/~ggordon/10725-F12/template.tex

\documentclass{article}
\setlength{\oddsidemargin}{0.25 in}
\setlength{\evensidemargin}{-0.25 in}
\setlength{\topmargin}{-0.6 in}
\setlength{\textwidth}{6.5 in}
\setlength{\textheight}{8.5 in}
\setlength{\headsep}{0.75 in}
\setlength{\parindent}{0 in}
\setlength{\parskip}{0.1 in}
\usepackage{amsfonts,graphicx, amssymb}
\usepackage[fleqn]{amsmath}
\newcounter{lecnum}
\renewcommand{\thepage}{\thelecnum-\arabic{page}}
\renewcommand{\thesection}{\thelecnum.\arabic{section}}
\renewcommand{\theequation}{\thelecnum.\arabic{equation}}
\renewcommand{\thefigure}{\thelecnum.\arabic{figure}}
\renewcommand{\thetable}{\thelecnum.\arabic{table}}
\newcommand{\lecture}[4]{
   \pagestyle{myheadings}
   \thispagestyle{plain}
   \newpage
   \setcounter{lecnum}{#1}
   \setcounter{page}{1}
   
   
%Info Box 
   \begin{center}
   \framebox{
      \vbox{\vspace{2mm}
    \hbox to 6.28in { {\bf Math 136 - Linear Algebra 
	\hfill Winter 2016} }
       \vspace{4mm}
       \hbox to 6.28in { {\Large \hfill Lecture #1: #2  \hfill} }
       \vspace{2mm}
       \hbox to 6.28in { {\it Lecturer: #3 \hfill Notes By: #4} }
      \vspace{2mm}}
   }
   \end{center}
   
   \markboth{Lecture #1: #2}{Lecture #1: #2}



 
}

\renewcommand{\cite}[1]{[#1]}
\def\beginrefs{\begin{list}%
        {[\arabic{equation}]}{\usecounter{equation}
         \setlength{\leftmargin}{2.0truecm}\setlength{\labelsep}{0.4truecm}%
         \setlength{\labelwidth}{1.6truecm}}}
\def\endrefs{\end{list}}
\def\bibentry#1{\item[\hbox{[#1]}]}

\newcommand{\fig}[3]{
			\vspace{#2}
			\begin{center}
			Figure \thelecnum.#1:~#3
			\end{center}
	}

\newtheorem{theorem}{Theorem}[lecnum]
\newtheorem{lemma}[theorem]{Lemma}
\newtheorem{ex}[theorem]{Example}
\newtheorem{proposition}[theorem]{Proposition}
\newtheorem{claim}[theorem]{Claim}
\newtheorem{corollary}[theorem]{Corollary}
\newtheorem{definition}[theorem]{Definition}
\newenvironment{proof}{{\bf Proof:}}{\hfill\rule{2mm}{2mm}}
\newcommand\E{\mathbb{E}}


%Start of Document 
\begin{document}

\lecture{12}{January 29, 2016}{Yongqiang Zhao}{Harsh Mistry}


\section{Matrix Operations}
\((A)_{ij} = a_{ij}\) The ij-th entry 

\textbf{Special Cases}
\begin{itemize}
\item \(M_{m\times 1 } (\mathbb{R}) = \mathbb{R}^m  \ \ \ \ \begin{bmatrix} a_1 \\ \vdots \\ a_n \end{bmatrix} \in M_{n x 1} (R) \)
\item \(M_{1 \times n} (\mathbb{R}) \rightarrow ( a_1, a_2 , \ldots , a_n ) \) Row vector 
\item \(M_{1\times 1} (\mathbb{R}) = \mathbb{R} \)
\end{itemize}

\begin{definition}
\(\forall A, B \in M_{m\times n} (\mathbb{R}) , C \in \mathbb{R} \) we define \(A + B = (a_{ij} + b_{ij}\) \\
If \( A = (a_{ij})  \ B = (b_{ij}) \ CA = (Ca_{ij}) \) 
\end{definition}

The Set \( M_{m\times n} (\mathbb{R})\) is a "vector space"

\begin{theorem} \(\forall A, B \in M_{m\times n} (\mathbb{R}) , c, d \in \mathbb{R} \)
\begin{enumerate}
\item \( A + B \in M_{m\times n} (\mathbb{R})\) 
\item (A + B) + C = A + (B + C) 
\item A + B = B + A 
\item \( \exists O_{m \times n } = \begin{bmatrix} 0 \ldots 0 \\ 0 \ldots 0 \end{bmatrix} \in M_{m\times n} (\mathbb{R}) \) Such That \( \forall A, A + O_{m \times n} = A \)
\item \( \forall A\) , \( -A = (-1) (A)  \implies A + (-A) = 0_{m \times n} \)
\item \( cA \in M_{m\times n} (\mathbb{R})  \)
\item c(dA) = (cd)A = d(cA) 
\item (c + d)A = cA + dA
\item c(A + B) = cA + cB 
\end{enumerate}
\end{theorem}

\begin{definition}
The Transpose of \( A_{m \times n}\) is the the matrix \( A^T\) with \( (A^T)_{ij} = a_{ji} \) then \( A^T \in M_{n \times n} (\mathbb{R}) \)
\end{definition}

\begin{ex} - \\
\( A = \begin{bmatrix} 1 & 2 \\ 3 & 4 \end{bmatrix}    \ \ A^T = \begin{bmatrix} 1 & 3 \\ 2 & 4 \end{bmatrix}\) \\
\( \vec{v} = \begin{bmatrix} 1 \\ 2 \\ 3 \\ 4 \end{bmatrix}  \ \ \vec{v}^T = \begin{bmatrix}1 & 2 & 3 & 4 \end{bmatrix}\)
\end{ex}

\textbf{Properties of transpose } \( \forall A, B \in M_{m \times n} (\mathbb{R}) , \forall x \in \mathbb{R} \) 
\begin{enumerate}
\item \( (A^T)^T = A\)
\item \( (A+B)^T = A^T + B^T\)
\item \( (cA)^T = cA^T\)
\end{enumerate}

\section{Matrix Multiplication}
Given \(A = (a_{ij}) \in M_{m \times n} (\mathbb{R})  \) and  \(B = (b_{ij}) \in M_{n \times s} (\mathbb{R})  \), we define matrix multiplication 
\( AB \in M_{m \times s} (\mathbb{R})  \) as \((AB)_{ij} = \sum_{k=1}^{n} a_{ik} b_{kj} \)
\[\begin{bmatrix} a_{11} & \ldots & a_{1n} \\ \ldots & \ldots & \ldots  \\ a_{m1} & \ldots & a_{mn}  \end{bmatrix} \begin{bmatrix} b_{11} & \ldots & b_{1s} \\ \ldots & \ldots & \ldots  \\ b_{n1} & \ldots & a_{ns}  \end{bmatrix} = \begin{bmatrix} \sum_{k=1}^{n} a_{k1} b_{k1} & \ldots & \sum_{k=1}^{n} a_{k} b_{ks}\\ \ldots & \ldots & \ldots  \\ \sum_{k=1}^{n} a_{km} b_{k1} & \ldots & \sum_{k=1}^{n} a_{mk} b_{ks}  \end{bmatrix}\] 

\begin{ex}
\[\begin{bmatrix} 1 & 2 \\ 0 & -1 \end{bmatrix} \begin{bmatrix} 1 & 2 \\ 3 & 4 \end{bmatrix} = \begin{bmatrix} 1 + 6 & 2+8 \\ 0 - 3 & 0 = 4 \end{bmatrix} = \begin{bmatrix} 7 & 10 \\ -3 & -4 \end{bmatrix} \]
\end{ex}

Note : \( AB \neq BA\)

\begin{ex}
\[\begin{cases} a_{11}x_1 + \ldots + a_{xn} x_n = b_1 \\ \vdots \\ a_m x_1 + \ldots + a_{mn}x_n = b_m \end{cases} \iff A\vec{x} = \vec{b} \] 
\end{ex}

\begin{center}
\textbf{End of Lecture Notes} \\
\textbf{Notes By : Harsh Mistry}
\end{center}
\end{document}
