%Notes by Harsh Mistry 
%Math 136 
%based on Template from : https://www.cs.cmu.edu/~ggordon/10725-F12/template.tex

\documentclass{article}
\setlength{\oddsidemargin}{0.25 in}
\setlength{\evensidemargin}{-0.25 in}
\setlength{\topmargin}{-0.6 in}
\setlength{\textwidth}{6.5 in}
\setlength{\textheight}{8.5 in}
\setlength{\headsep}{0.75 in}
\setlength{\parindent}{0 in}
\setlength{\parskip}{0.1 in}
\usepackage{amsfonts,graphicx, amssymb}
\usepackage[fleqn]{amsmath}
\usepackage{fixltx2e}
\newcounter{lecnum}
\renewcommand{\thepage}{\thelecnum-\arabic{page}}
\renewcommand{\thesection}{\thelecnum.\arabic{section}}
\renewcommand{\theequation}{\thelecnum.\arabic{equation}}
\renewcommand{\thefigure}{\thelecnum.\arabic{figure}}
\renewcommand{\thetable}{\thelecnum.\arabic{table}}
\newcommand{\lecture}[4]{
   \pagestyle{myheadings}
   \thispagestyle{plain}
   \newpage
   \setcounter{lecnum}{#1}
   \setcounter{page}{1}
   
   
%Info Box 
   \begin{center}
   \framebox{
      \vbox{\vspace{2mm}
    \hbox to 6.28in { {\bf Math 136 - Linear Algebra 
	\hfill Winter 2016} }
       \vspace{4mm}
       \hbox to 6.28in { {\Large \hfill Lecture #1: #2  \hfill} }
       \vspace{2mm}
       \hbox to 6.28in { {\it Lecturer: #3 \hfill Notes By: #4} }
      \vspace{2mm}}
   }
   \end{center}
   
   \markboth{Lecture #1: #2}{Lecture #1: #2}



 
}

\renewcommand{\cite}[1]{[#1]}
\def\beginrefs{\begin{list}%
        {[\arabic{equation}]}{\usecounter{equation}
         \setlength{\leftmargin}{2.0truecm}\setlength{\labelsep}{0.4truecm}%
         \setlength{\labelwidth}{1.6truecm}}}
\def\endrefs{\end{list}}
\def\bibentry#1{\item[\hbox{[#1]}]}

\newcommand{\fig}[3]{
			\vspace{#2}
			\begin{center}
			Figure \thelecnum.#1:~#3
			\end{center}
	}

\newtheorem{theorem}{Theorem}[lecnum]
\newtheorem{lemma}[theorem]{Lemma}
\newtheorem{ex}[theorem]{Example}
\newtheorem{proposition}[theorem]{Proposition}
\newtheorem{claim}[theorem]{Claim}
\newtheorem{corollary}[theorem]{Corollary}
\newtheorem{definition}[theorem]{Definition}
\newenvironment{proof}{{\bf Proof:}}{\hfill\rule{2mm}{2mm}}
\newcommand\E{\mathbb{E}}


%Start of Document 
\begin{document}

\lecture{24}{March 4, 2016}{Yongqiang Zhao}{Harsh Mistry}

\section{Matrix Inverses}

\begin{theorem}
If \(\beta\) and \(\zeta\) are bases for an n-dimensional vector space \(\mathbb{V}\), then the change of coordinate matrices \({}_\zeta P_\beta \) and \({}_\beta P_\zeta \) satisfy $${}_\zeta P_\beta {}_\beta P_\zeta  = I = {}_\beta P_\zeta {}_\zeta P_\beta $$
\end{theorem}

This shows that \({}_\zeta P_\beta \) and \({}_\beta P_\zeta \) are multiplicative inverses of each other 

\section{Left and Right Inverse}
\begin{definition}
Let  A be a \(m \times n \) matrix. \\ If B is an \(m \times n \) matrix  such that \(AB = I_m\), then B is a right inverse of A \\
If C is an \(n \times m\) matrix such that \(CA = I_n\), then C is called a left inverse of A
\end{definition}

\textbf{Note (For Right Inverses) : } \([\vec{e_1} \ldots \vec{e_m}] = AB = [A\vec{b_1} \ldots A\vec{b_m}]\)

\begin{theorem}
If A is an \(m \times n\) matrix with \(m > n\), then A cannot have a right inverse
\end{theorem}

\begin{corollary}
If A is an \(m \times n\) matrix with \( m < n \), then A cannot have a left inverse  
\end{corollary}

\section{Matrix Inverse}
\begin{definition}
An \( n \times n \) matrix is called a square matrix 
\end{definition}

\begin{definition}
Let A be a \( n \times n \) matrix. IF B is a matrix such that AB = I = BA, then B is called a inverse of A.\\
We write \(B = A^{-1}\) and we A is said to be invertible 
\end{definition}

\textbf{Remark : } If \( B = A^{-1}\) then \( A = B^{-1} \) 

\begin{theorem}
The inverse of a matrix is unique
\end{theorem} 

\begin{proof}
B = BI = B(AC) = (BA)C = IC = C 
\end{proof}

\begin{theorem}
If A and B are \( n \times n \) matrices such that AB = I, then A and B are invertible and \(rank A = rank B = n \)
\end{theorem}



\begin{center}
\textbf{End of Lecture Notes}\\
\textbf{Notes by : Harsh Mistry}
\end{center}
\end{document}
