%Notes by Harsh Mistry 
%Math 136 
%based on Template from : https://www.cs.cmu.edu/~ggordon/10725-F12/template.tex

\documentclass{article}
\setlength{\oddsidemargin}{0.25 in}
\setlength{\evensidemargin}{-0.25 in}
\setlength{\topmargin}{-0.6 in}
\setlength{\textwidth}{6.5 in}
\setlength{\textheight}{8.5 in}
\setlength{\headsep}{0.75 in}
\setlength{\parindent}{0 in}
\setlength{\parskip}{0.1 in}
\usepackage{amsfonts,graphicx, amssymb}
\usepackage[fleqn]{amsmath}
\usepackage{fixltx2e}
\newcounter{lecnum}
\renewcommand{\thepage}{\thelecnum-\arabic{page}}
\renewcommand{\thesection}{\thelecnum.\arabic{section}}
\renewcommand{\theequation}{\thelecnum.\arabic{equation}}
\renewcommand{\thefigure}{\thelecnum.\arabic{figure}}
\renewcommand{\thetable}{\thelecnum.\arabic{table}}
\newcommand{\lecture}[4]{
   \pagestyle{myheadings}
   \thispagestyle{plain}
   \newpage
   \setcounter{lecnum}{#1}
   \setcounter{page}{1}
   
   
%Info Box 
   \begin{center}
   \framebox{
      \vbox{\vspace{2mm}
    \hbox to 6.28in { {\bf Math 136 - Linear Algebra 
	\hfill Winter 2016} }
       \vspace{4mm}
       \hbox to 6.28in { {\Large \hfill Lecture #1: #2  \hfill} }
       \vspace{2mm}
       \hbox to 6.28in { {\it Lecturer: #3 \hfill Notes By: #4} }
      \vspace{2mm}}
   }
   \end{center}
   
   \markboth{Lecture #1: #2}{Lecture #1: #2}



 
}

\renewcommand{\cite}[1]{[#1]}
\def\beginrefs{\begin{list}%
        {[\arabic{equation}]}{\usecounter{equation}
         \setlength{\leftmargin}{2.0truecm}\setlength{\labelsep}{0.4truecm}%
         \setlength{\labelwidth}{1.6truecm}}}
\def\endrefs{\end{list}}
\def\bibentry#1{\item[\hbox{[#1]}]}

\newcommand{\fig}[3]{
			\vspace{#2}
			\begin{center}
			Figure \thelecnum.#1:~#3
			\end{center}
	}

\newtheorem{theorem}{Theorem}[lecnum]
\newtheorem{lemma}[theorem]{Lemma}
\newtheorem{ex}[theorem]{Example}
\newtheorem{proposition}[theorem]{Proposition}
\newtheorem{claim}[theorem]{Claim}
\newtheorem{corollary}[theorem]{Corollary}
\newtheorem{definition}[theorem]{Definition}
\newenvironment{proof}{{\bf Proof:}}{\hfill\rule{2mm}{2mm}}
\newcommand\E{\mathbb{E}}


%Start of Document 
\begin{document}

\lecture{26}{March 9, 2016}{Yongqiang Zhao}{Harsh Mistry}

\section{Elementary Matrices}

\begin{definition}
An \(n \times m\) matrix \(E\) is called an elementary matrix if it can be obtained from the \(n \times n \) indentiy matrix by performing exactly one matrix operation. 
\end{definition}

\begin{ex}
The following are elementary matricies 
$$\begin{bmatrix} 1 & 0 & 0 \\ 0 & 2 & 0 \\ 0 & 0 & 1\end{bmatrix} \ \ \ \ \  \begin{bmatrix} 1 & 0 & 0 \\ 0 & 0 & 0 \\ 0 & 0 & 1 \end{bmatrix}$$
\end{ex}

\begin{theorem}
If \(A\) is an \(m \times n\) matrix and \(E\) is the \(m \times m \) elementary matrix corresponding to the row operation \(R_i + cR_j\), for \(i \neq j \), then \(EA\) is the matrix obtained from A by performing the row operation \(R_i + cR_j\) on A.
\end{theorem}

\begin{theorem}
If \(A\) is an \(m \times n\) matrix and \(E\) is the \(m \times m \) matrix corresponding to the row operation \(cR_i\), then \(EA\) is the matrix obtained from A by performing the row operation \(cR_i\) on A
\end{theorem}

\begin{theorem}
If \(A\) is an \(m \times n\) matrix and \(E\) is the \(m \times m \) matrix corresponding to the row operation \(cR_i \longleftrightarrow r_j\) for \(i \neq j\), then \(EA\) is the matrix obtained from A by performing the row operation \(cR_i \longleftrightarrow r_j\) on A
\end{theorem}

\textbf{Note :} Multiplying a matrix on the left by an elemtary matrix is the same as performing the corresponding elementary row operation on A

\begin{corollary}
If A is an \(m \times n\) matrix and \(E\) is an \(m \times m \) elementary matrix, then 
$$rank(EA) = rankA$$
\end{corollary}

\begin{theorem}
If \(A\) is an \(m \times n\) matrix with reduced row ecelon form R, then there exists a sequence \(E_1, \ldots , E_k\) of \(m \times m\) elementary matrices such that \(E_k \ldots E_2 E_1 A = E\), particularly 
$$ A = E_1^{-1}E_2^{-1} \ldots \ E_k^{-1}R $$
\end{theorem}

\begin{proof}
The first conclusion follows from the Gauss Elimination. The second part holds since 
$$ A  = (E_1 \ldots E_k)^{-1}R = E_k^{-1} \ldots E_1^{-1}R $$
\end{proof}

\begin{corollary}
If A is a \(n \times n\) invertible matrix, then \(A\) and \(A^{-1}\) can be written as a product of elemntary matricies 
\end{corollary}


\begin{center}
\textbf{End of Lecture Notes}\\
\textbf{Notes by : Harsh Mistry}
\end{center}
\end{document}
