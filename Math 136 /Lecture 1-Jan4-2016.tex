%Notes by Harsh Mistry 
%Math 136 
%Template Taken from : https://www.cs.cmu.edu/~ggordon/10725-F12/template.tex

\documentclass[twoside]{article}
\setlength{\oddsidemargin}{0.25 in}
\setlength{\evensidemargin}{-0.25 in}
\setlength{\topmargin}{-0.6 in}
\setlength{\textwidth}{6.5 in}
\setlength{\textheight}{8.5 in}
\setlength{\headsep}{0.75 in}
\setlength{\parindent}{0 in}
\setlength{\parskip}{0.1 in}
\usepackage{amsmath,amsfonts,graphicx}
\newcounter{lecnum}
\renewcommand{\thepage}{\thelecnum-\arabic{page}}
\renewcommand{\thesection}{\thelecnum.\arabic{section}}
\renewcommand{\theequation}{\thelecnum.\arabic{equation}}
\renewcommand{\thefigure}{\thelecnum.\arabic{figure}}
\renewcommand{\thetable}{\thelecnum.\arabic{table}}
\newcommand{\lecture}[4]{
   \pagestyle{myheadings}
   \thispagestyle{plain}
   \newpage
   \setcounter{lecnum}{#1}
   \setcounter{page}{1}
   
   
%Info Box 
   \begin{center}
   \framebox{
      \vbox{\vspace{2mm}
    \hbox to 6.28in { {\bf Math 136 - Linear Algebra 
	\hfill Winter 2016} }
       \vspace{4mm}
       \hbox to 6.28in { {\Large \hfill Lecture #1: #2  \hfill} }
       \vspace{2mm}
       \hbox to 6.28in { {\it Lecturer: #3 \hfill Notes By: #4} }
      \vspace{2mm}}
   }
   \end{center}
   
   \markboth{Lecture #1: #2}{Lecture #1: #2}



 
}

\renewcommand{\cite}[1]{[#1]}
\def\beginrefs{\begin{list}%
        {[\arabic{equation}]}{\usecounter{equation}
         \setlength{\leftmargin}{2.0truecm}\setlength{\labelsep}{0.4truecm}%
         \setlength{\labelwidth}{1.6truecm}}}
\def\endrefs{\end{list}}
\def\bibentry#1{\item[\hbox{[#1]}]}

\newcommand{\fig}[3]{
			\vspace{#2}
			\begin{center}
			Figure \thelecnum.#1:~#3
			\end{center}
	}

\newtheorem{theorem}{Theorem}[lecnum]
\newtheorem{lemma}[theorem]{Lemma}
\newtheorem{ex}[theorem]{Example}
\newtheorem{proposition}[theorem]{Proposition}
\newtheorem{claim}[theorem]{Claim}
\newtheorem{corollary}[theorem]{Corollary}
\newtheorem{definition}[theorem]{Definition}
\newenvironment{proof}{{\bf Proof:}}{\hfill\rule{2mm}{2mm}}
\newcommand\E{\mathbb{E}}


%Start of Document 
\begin{document}

\lecture{1}{January 4, 2016}{Yongqiang Zhao}{Harsh Mistry}


% **** YOUR NOTES GO HERE:

% Admin Info
\begin{center}
Admin Info \\
Yongqiang Zhao 
Office : MC 6461 \\
Office Hours : 10:30am - 12:30pm \\
Email: y354zhao@uwaterloo.ca \\
Midterm date : Feb 8th, 2016 

\end{center}

%
\section{Vector Addition and Scaler Multiplication}

Recall : \\
2-dimensional Euclidian Space
\[\mathbb{R}^2 = \{(x,y)| x,y \in \mathbb{R}\}\]
3-dimensional Euclidian Space
\[\mathbb{R}^3 = \{(x,y,z)| x,y,z \in \mathbb{R}\}\]


\begin{ex}
\[ \vec{x} = \begin{bmatrix} x \\ y \end{bmatrix} \text{corresponds to the point} (x,y) \]
\end{ex}

\begin{definition}
We represent vectors as column vectors (matrices of size n × 1) to distinguish them from points\[\vec{v} = \begin{bmatrix} v_1 \\ \vdots \\ v_n \end{bmatrix}, v_i \in \mathbb{R}, 1 \le i \le n\]
This is called a n-dimensional Euclidean Space
\end {definition}


\section{Operations}
\textbf{Equality}
\[\vec{x}=\begin{bmatrix} x_1 \\ x_2 \\ x_3 \end{bmatrix} \vec{y}=\begin{bmatrix} y_1 \\ y_2 \\ y_3 \end{bmatrix} \]
\[\vec{x} = \vec{y} \text{ iff } x_i = y_i ,1 \leq i \leq n \]

\newpage 

\textbf{Addition/Subtraction}
\[\text{Let  } \vec{x}=\begin{bmatrix} x_1 \\ \vdots \\ x_n \end{bmatrix} \text{  and  }\vec{y}=\begin{bmatrix} y_1 \\ \vdots \\ y_n \end{bmatrix} \text{   be 2 vectors}\]
\[\vec{x}\pm\vec{y} = \begin{bmatrix} x_1 \pm y_1 \\ \vdots \\ x_n \pm y_n \end{bmatrix} \]


\textbf{Scaler Multiplication}
\[c\vec{x} = \begin{bmatrix} cx_1 \\ \vdots \\ cx_n \end{bmatrix}\]

\section{Linear Combination}
\begin{definition}
\[\text{For  } \vec{V_1} \ldots \vec{V_2} \ldots \vec{V_n} \in \mathbb{R}\]
\[\text{We call the sum   } V_1C_1 + V_2C_2 + \ldots V_nC_i \
\text{   the linear combination of   } \vec{V_1} \ldots \vec{V_2} \ldots \vec{V_n}  \]
\end{definition}

\section{Fundemental Properties of Vector Operations}
\begin{theorem}
\[\text{If  } \ \vec{x}, \vec{y}, \vec{w}  \ \in \mathbb{R}^n \ , c,d \in \mathbb{R}\]
\begin{itemize}
\item Closure Under Addition: \[\vec{x} + \vec{y} \in \mathbb{R}^n\]
\item Associativity : \[(\vec{x} + \vec{y}) + \vec{w} = \vec{x} + (\vec{y} + \vec{w})\]
\item Commutativity : \[\vec{x} + \vec{y} = \vec{y} + \vec{x}\]
\item Additive identity : \[\exists \vec{0} \in \mathbb{R}, \forall \vec{x} \in \mathbb{R}^n, \vec{x} + \vec{0} = \vec{x}\]
\item Additive inverse : \[\forall \vec{x} \in \mathbb{R}^n, \exists -\vec{x} \in \mathbb{R}^n, \vec{x} + (-\vec{x}) = \vec{0}\]
\item Closure under scalar multiplication: \[c\vec{x} \in \mathbb{R}^n\]
\item \[c(d\vec{x}) = (cd)\vec{x}\]
\item  Scalar distributivity: \[(c + d)\vec{x} = c\vec{x} + d\vec{x}\]
\item Vector distributivity: \[c(\vec{x} + \vec{y}) = c\vec{x} + c\vec{y}\]
\item Multiplicative identity: \[1\vec{x} = \vec{x}\]




\end{itemize}

\end{theorem}
\begin{center}
\textbf{End of Lecture Notes} \\
\textbf{Notes By : Harsh Mistry}
\end{center}
\end{document}





