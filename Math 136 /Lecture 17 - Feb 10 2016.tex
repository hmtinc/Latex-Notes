%Notes by Harsh Mistry 
%Math 136 
%based on Template from : https://www.cs.cmu.edu/~ggordon/10725-F12/template.tex

\documentclass{article}
\setlength{\oddsidemargin}{0.25 in}
\setlength{\evensidemargin}{-0.25 in}
\setlength{\topmargin}{-0.6 in}
\setlength{\textwidth}{6.5 in}
\setlength{\textheight}{8.5 in}
\setlength{\headsep}{0.75 in}
\setlength{\parindent}{0 in}
\setlength{\parskip}{0.1 in}
\usepackage{amsfonts,graphicx, amssymb}
\usepackage[fleqn]{amsmath}
\newcounter{lecnum}
\renewcommand{\thepage}{\thelecnum-\arabic{page}}
\renewcommand{\thesection}{\thelecnum.\arabic{section}}
\renewcommand{\theequation}{\thelecnum.\arabic{equation}}
\renewcommand{\thefigure}{\thelecnum.\arabic{figure}}
\renewcommand{\thetable}{\thelecnum.\arabic{table}}
\newcommand{\lecture}[4]{
   \pagestyle{myheadings}
   \thispagestyle{plain}
   \newpage
   \setcounter{lecnum}{#1}
   \setcounter{page}{1}
   
   
%Info Box 
   \begin{center}
   \framebox{
      \vbox{\vspace{2mm}
    \hbox to 6.28in { {\bf Math 136 - Linear Algebra 
	\hfill Winter 2016} }
       \vspace{4mm}
       \hbox to 6.28in { {\Large \hfill Lecture #1: #2  \hfill} }
       \vspace{2mm}
       \hbox to 6.28in { {\it Lecturer: #3 \hfill Notes By: #4} }
      \vspace{2mm}}
   }
   \end{center}
   
   \markboth{Lecture #1: #2}{Lecture #1: #2}



 
}

\renewcommand{\cite}[1]{[#1]}
\def\beginrefs{\begin{list}%
        {[\arabic{equation}]}{\usecounter{equation}
         \setlength{\leftmargin}{2.0truecm}\setlength{\labelsep}{0.4truecm}%
         \setlength{\labelwidth}{1.6truecm}}}
\def\endrefs{\end{list}}
\def\bibentry#1{\item[\hbox{[#1]}]}

\newcommand{\fig}[3]{
			\vspace{#2}
			\begin{center}
			Figure \thelecnum.#1:~#3
			\end{center}
	}

\newtheorem{theorem}{Theorem}[lecnum]
\newtheorem{lemma}[theorem]{Lemma}
\newtheorem{ex}[theorem]{Example}
\newtheorem{proposition}[theorem]{Proposition}
\newtheorem{claim}[theorem]{Claim}
\newtheorem{corollary}[theorem]{Corollary}
\newtheorem{definition}[theorem]{Definition}
\newenvironment{proof}{{\bf Proof:}}{\hfill\rule{2mm}{2mm}}
\newcommand\E{\mathbb{E}}


%Start of Document 
\begin{document}

\lecture{17}{February 10, 2016}{Yongqiang Zhao}{Harsh Mistry}


\section{Special Subspaces Continued}
\begin{ex}
For \(\vec{a} \neq 0 \in \mathbb{R}^n  \) Find a basis for the range of \(Proj_{\vec{n}}\) \\
\[Range (Proj_{\vec{a}}) = \{Proj_{\vec{a}}(\vec{x})\mid \vec{x} \in \mathbb{R}^n \} \]
\[\forall \vec{x} \in \mathbb{R}^n  Proj_{\vec{n}} (\vec{x}) = \frac{\vec{x} \cdot \vec{n}}{\|\vec{a}\|^2}\vec{a} \in Span \{\vec{a}\} \implies Range(Proj_{\vec{a}}) \subset Span\{\vec{a}\}n \]
On the other hand, \( Proj_{\vec{a}}(t\vec{a}) = \frac{t \vec{a} \cdot \vec{a}}{\|\vec{a}\|^2} \vec{a} = t\vec{a} \  \forall  t \in \mathbb{R}\)
\[ \implies Span\{\vec{a}\} \subset Range(Proj_{\vec{a}}) \implies Range(proj_{\vec{a}}) = Span\{\vec{a}\} \]
\end{ex}

\begin{theorem}
Suppose L : \(\mathbb{R}^n \rightarrow \mathbb{R}^m \) is linear with the standard matrix 
\( \begin{bmatrix} L \end{bmatrix} = (L(\vec{e_1}, \ldots, L(\vec{e_n})\)) Then, \\
\[ Range(L) = Span\{L(\vec{e_1}, \ldots, L(\vec{e_n})\}\]
\end{theorem}

\begin{proof}
\[L(\vec{x}) = \text{[L]}\vec{x} = \begin{bmatrix}L(\vec{e_1}, \ldots, L(\vec{e_n})\end{bmatrix} \begin{bmatrix} x_1 \\ \vdots \\ x_n \end{bmatrix} = x_1L(\vec{e_1} +  \ldots + x_n L(\vec{e_n})\]
\[\therefore  L(\vec{x}) \in Range(L) \iff  L(\vec{x})  \in Span\{L(\vec{e_1}, \ldots, L(\vec{e_n})\}\]
\end{proof}

\begin{definition}
Let L : \(\mathbb{R}^n \rightarrow \mathbb{R}^m\) be a linear mapping, we define its kernel as 
\[Ker(L) := \{ \vec{x} \in \mathbb{R}^n \mid L(\vec{x}) = 0 \}\]
 \end{definition}
 For L : \(\mathbb{R}^n \rightarrow \mathbb{R}^m\)  and Standard matrix [L]
 \[L(\vec{x}) = \vec{0} \text{ iff } \begin{bmatrix} L \end{bmatrix} \vec{x} = \vec{0} \]
 \[\vec{x} \in Ker(L) \text{ iff } \vec{x} \text{ is a soution of } (\begin{bmatrix}L\end{bmatrix} \mid \vec{0})\]
 
 \begin{theorem}
 If L : \(\mathbb{R}^n \rightarrow \mathbb{R}^m\) is linear then \(Ker(L)\) is a subspace of \(\mathbb{R}^n\) 
 \end{theorem}
 
 \newpage
 
 \begin{ex}
 Fix \(\vec{a} \neq 0, \vec{a} \in \mathbb{R}\) , What is the \(Ker(Proj_{\vec{a}})\)? For \( \vec{a} = \begin{bmatrix}
 2 \\ -1 \\ 5 \end{bmatrix} \) write down a basis for \(Ker(Proj_{\vec{a}})\) \\
 \[\vec{x} \in Ker(Proj_{\vec{a}}) \iff Proj_{\vec{a}}(\vec{x}) = 0 \iff \frac{x \cdot a }{\|\vec{a}\|^2}\vec{a} = \vec{0} \iff \vec{x} \cdot \vec{a} = 0\]
 \[\implies Ker(Proj_{\vec{a}}) = \{ \vec{x} \in \mathbb{R}^n \mid \vec{x} \cdot \vec{n}  = 0 \} \]
 Since \( \vec{a} = \begin{bmatrix} 2 \\ -1 \\5 \end{bmatrix} \) 
 \[\implies Ker(Proj_{\vec{a}}) = \{ \vec{x} \in \mathbb{R}^3 \mid 2x_1-x_2+5x_3 =0 \} \]
 \[\therefore \text{ The basis is } \left\{\begin{bmatrix}1 \\ 2 \\ 0 \end{bmatrix} \begin{bmatrix} 0 \\ 5 \\ 1\end{bmatrix} \right\} \]
 \end{ex}
 
\begin{definition}
Given A \( \in M_{m\times n} (\mathbb{R})\) we define the nullspace of A as, \\
\[Null(A) = \{\vec{x} \in \mathbb{R}^n \mid A\vec{x} = 0 \} \]\\
and if A = (\(\vec{a_1} , \ldots , \vec{a_n} \) ) we define its column space as \\
\[Col (A) = Span \{\vec{a_1} , \ldots , \vec{a_n}\} \subset \mathbb{R}^m \]
 \end{definition} 
 
 for given A \( \in M_{m\times n } (\mathbb{R}) \) we define L : \( \mathbb{R}^n \rightarrow \mathbb{R}^m\) as, 
 \[L(\vec{x}) = A \vec{x} = \begin{bmatrix} L\end{bmatrix} \vec{x} \]
 then, [L] = A \\
 \[Ker(L) = \{ \vec{x} \in \mathbb{R}^n \mid L(\vec{x}) = 0 \} = \{ \vec{x} \in \mathbb{R}^n \mid A\vec{x} = 0 \} = Null(A) \]
\[Range(L) = Span\{\vec{a_1} , \ldots , \vec{a_n}\} = Col(A) \]

\begin{center}
\textbf{End of Lecture Notes} \\
\textbf{Notes By : Harsh Mistry}
\end{center}
\end{document}
