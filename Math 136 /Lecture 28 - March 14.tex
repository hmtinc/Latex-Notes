%Notes by Harsh Mistry 
%Math 136 
%based on Template from : https://www.cs.cmu.edu/~ggordon/10725-F12/template.tex

\documentclass{article}
\setlength{\oddsidemargin}{0.25 in}
\setlength{\evensidemargin}{-0.25 in}
\setlength{\topmargin}{-0.6 in}
\setlength{\textwidth}{6.5 in}
\setlength{\textheight}{8.5 in}
\setlength{\headsep}{0.75 in}
\setlength{\parindent}{0 in}
\setlength{\parskip}{0.1 in}
\usepackage{amsfonts,graphicx, amssymb}
\usepackage[fleqn]{amsmath}
\usepackage{fixltx2e}
\newcounter{lecnum}
\renewcommand{\thepage}{\thelecnum-\arabic{page}}
\renewcommand{\thesection}{\thelecnum.\arabic{section}}
\renewcommand{\theequation}{\thelecnum.\arabic{equation}}
\renewcommand{\thefigure}{\thelecnum.\arabic{figure}}
\renewcommand{\thetable}{\thelecnum.\arabic{table}}
\newcommand{\lecture}[4]{
   \pagestyle{myheadings}
   \thispagestyle{plain}
   \newpage
   \setcounter{lecnum}{#1}
   \setcounter{page}{1}
   
   
%Info Box 
   \begin{center}
   \framebox{
      \vbox{\vspace{2mm}
    \hbox to 6.28in { {\bf Math 136 - Linear Algebra 
	\hfill Winter 2016} }
       \vspace{4mm}
       \hbox to 6.28in { {\Large \hfill Lecture #1: #2  \hfill} }
       \vspace{2mm}
       \hbox to 6.28in { {\it Lecturer: #3 \hfill Notes By: #4} }
      \vspace{2mm}}
   }
   \end{center}
   
   \markboth{Lecture #1: #2}{Lecture #1: #2}



 
}

\renewcommand{\cite}[1]{[#1]}
\def\beginrefs{\begin{list}%
        {[\arabic{equation}]}{\usecounter{equation}
         \setlength{\leftmargin}{2.0truecm}\setlength{\labelsep}{0.4truecm}%
         \setlength{\labelwidth}{1.6truecm}}}
\def\endrefs{\end{list}}
\def\bibentry#1{\item[\hbox{[#1]}]}

\newcommand{\fig}[3]{
			\vspace{#2}
			\begin{center}
			Figure \thelecnum.#1:~#3
			\end{center}
	}

\newtheorem{theorem}{Theorem}[lecnum]
\newtheorem{lemma}[theorem]{Lemma}
\newtheorem{ex}[theorem]{Example}
\newtheorem{proposition}[theorem]{Proposition}
\newtheorem{claim}[theorem]{Claim}
\newtheorem{corollary}[theorem]{Corollary}
\newtheorem{definition}[theorem]{Definition}
\newenvironment{proof}{{\bf Proof:}}{\hfill\rule{2mm}{2mm}}
\newcommand\E{\mathbb{E}}


%Start of Document 
\begin{document}

\lecture{28}{March 14, 2016}{Yongqiang Zhao}{Harsh Mistry}

\section{Determinants Continued}

\begin{definition}
An \(m \times n\) matrix \(U\) is said to be upper triangular if \(u_{ij} = 0\) whenever \( i > j\). An \(m \times n \)  matrix L is said to be lower triangular if \(l_{ij} = 0\) whenever \( i < j \) \\
\end{definition}

\begin{ex}-\\
Upper Matrix Example \\
$$ \begin{bmatrix} 3 & 1 & 4 \\ 0 & 2 & 1 \\ 0 & 0 & 5 \end{bmatrix}$$
Lower Matrix Example \\
$$ \begin{bmatrix} 1 & 0 & 0 \\ 0 & 2 & 0 \\ 0 & 0 & 3 \end{bmatrix}$$
\end{ex}

\begin{theorem}
If an \(n \times m \) matrix A is upper triangular or lower triangular, then 
$$ det A = a_{11}a_{22} \ldots a_{nn} $$
\end{theorem}

\begin{theorem}
If B is an \(n \times n\) matrix obtained from A by swapping two rows of A, then \( detB = - detA\)
\end{theorem}

\begin{corollary}
If \(n \times n \) matrix A has two identical rows, then \(detA = 0\)
\end{corollary}

\begin{theorem}
If B is the matrix obtained from S by multiplying one row of A by a non-zero constant c, then \(det B  = c der A\)
\end{theorem}

\begin{theorem}
If B is the matrix obtained from A by adding r times the k-th row of a to the j-th row, then \(det B = det A\)
\end{theorem}

\begin{theorem}
If A is an \(n \times n\) matrix, then \(det A = det A^T\)
\end{theorem}

Combining Theorms 28.8, 28.7, and 28.6, we find that we can do column operations 
\begin{itemize}
\item Adding a multiple of one column to another does not change the determinant 
\item Multiplying a column by a non-zero scaler c multiplies the determinant by c
\item Swapping two columns multiplies the determinant by (-1)
\end{itemize}

\textbf{Note :} Column Operations can be only used when simplifying a determinant

\begin{center}
\textbf{End of Lecture Notes}\\
\textbf{Notes by : Harsh Mistry}
\end{center}
\end{document}
