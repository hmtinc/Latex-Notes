%Notes by Harsh Mistry 
%Math 136 
%Template Taken from : https://www.cs.cmu.edu/~ggordon/10725-F12/template.tex

\documentclass{article}
\setlength{\oddsidemargin}{0.25 in}
\setlength{\evensidemargin}{-0.25 in}
\setlength{\topmargin}{-0.6 in}
\setlength{\textwidth}{6.5 in}
\setlength{\textheight}{8.5 in}
\setlength{\headsep}{0.75 in}
\setlength{\parindent}{0 in}
\setlength{\parskip}{0.1 in}
\usepackage{amsfonts,graphicx, amssymb}
\usepackage[fleqn]{amsmath}
\newcounter{lecnum}
\renewcommand{\thepage}{\thelecnum-\arabic{page}}
\renewcommand{\thesection}{\thelecnum.\arabic{section}}
\renewcommand{\theequation}{\thelecnum.\arabic{equation}}
\renewcommand{\thefigure}{\thelecnum.\arabic{figure}}
\renewcommand{\thetable}{\thelecnum.\arabic{table}}
\newcommand{\lecture}[4]{
   \pagestyle{myheadings}
   \thispagestyle{plain}
   \newpage
   \setcounter{lecnum}{#1}
   \setcounter{page}{1}
   
   
%Info Box 
   \begin{center}
   \framebox{
      \vbox{\vspace{2mm}
    \hbox to 6.28in { {\bf Math 136 - Linear Algebra 
	\hfill Winter 2016} }
       \vspace{4mm}
       \hbox to 6.28in { {\Large \hfill Lecture #1: #2  \hfill} }
       \vspace{2mm}
       \hbox to 6.28in { {\it Lecturer: #3 \hfill Notes By: #4} }
      \vspace{2mm}}
   }
   \end{center}
   
   \markboth{Lecture #1: #2}{Lecture #1: #2}



 
}

\renewcommand{\cite}[1]{[#1]}
\def\beginrefs{\begin{list}%
        {[\arabic{equation}]}{\usecounter{equation}
         \setlength{\leftmargin}{2.0truecm}\setlength{\labelsep}{0.4truecm}%
         \setlength{\labelwidth}{1.6truecm}}}
\def\endrefs{\end{list}}
\def\bibentry#1{\item[\hbox{[#1]}]}

\newcommand{\fig}[3]{
			\vspace{#2}
			\begin{center}
			Figure \thelecnum.#1:~#3
			\end{center}
	}

\newtheorem{theorem}{Theorem}[lecnum]
\newtheorem{lemma}[theorem]{Lemma}
\newtheorem{ex}[theorem]{Example}
\newtheorem{proposition}[theorem]{Proposition}
\newtheorem{claim}[theorem]{Claim}
\newtheorem{corollary}[theorem]{Corollary}
\newtheorem{definition}[theorem]{Definition}
\newenvironment{proof}{{\bf Proof:}}{\hfill\rule{2mm}{2mm}}
\newcommand\E{\mathbb{E}}


%Start of Document 
\begin{document}

\lecture{9}{January 22, 2016}{Yongqiang Zhao}{Harsh Mistry}


\section{Linear Systems Continued }

\begin{theorem} Given a linear system 
\[ (A) \begin{cases} a_{11} x_1 + \ldots + a_{1n} x_n = b_1 \\ a_{21}x_1 + \ldots + a_{2n} x_n = b_2 \\
\vdots \\ a_{m1}x_1 + \ldots + a_{mn} x_n = b_m \end{cases} \]
with the solution set \(S_1\) and corresponding  homogeneous system 
\[ (B) \begin{cases} a_{11} x_1 + \ldots + a_{1n} x_n = 0 \\ a_{21}x_1 + \ldots + a_{2n} x_n = 0 \\
\vdots \\ a_{m1}x_1 + \ldots + a_{mn} x_n = 0 \end{cases} \]
with the solution set \(S_2\) then, \begin{itemize}
\item \(\forall \vec{a} , \vec{c} \in S_1 , \vec{a} - \vec{c} \in S_2 \)
\item \(\forall \vec{x} \in S_1 ,  \vec{y} \in S_2 , \vec{x} + \vec{y} \in S_1 \)
\item For a fixed \( \vec{x_0} \in S_1, \forall \vec{x} \in S_1 , \vec{x} = \vec{y} + \vec{x_0} \text{ with } \vec{y} \in S_2\)
\end{itemize}
\end{theorem}

\begin{proof} \( \forall \vec{a} , \vec{c} \in S_1, \) Let \( \vec{a} = \begin{bmatrix} a_1 \\ \vdots \\ a_n \end{bmatrix} \) and \( \vec{c} = \begin{bmatrix} c_1 \\ \vdots \\ c_n \end{bmatrix} \) \\
\[ \begin{aligned} & (1 ) \ a_{i1} a_1 + \ldots + a_{in} a_n  = b_i  \  \ 1 \leq i \leq m \\
& (2 ) \ a_{i1} c_1 + \ldots + a_{in} c_n  = b_i  \  \ 1 \leq i \leq m \\
&  (3 ) \ a_{i1} (a_1 - c_1 ) + \ldots + a_{in} (a_n - c_n)  = 0  \  \ 1 \leq i \leq m \ \textbf( (1) - (2))  \\
& \implies \vec{a} - \vec{c} \in S_1 \end{aligned} \]
\end{proof}

\begin{proof} \( \forall \vec{x} = \begin{bmatrix} x_1 \\ \vdots \\ x_n \end{bmatrix} , \vec{y} = \begin{bmatrix} y_1 \\ \vdots \\ y_n \end{bmatrix} \in S_2 \) \\
\[ \begin{aligned} & (1 ) \ a_{i1} a_1 + \ldots + a_{in} a_n  = b_i \  \ 1 \leq i \leq m \\
& (2 ) \ a_{i1} x_1 + \ldots + a_{in} y_n  = 0 \  \ 1 \leq i \leq m \\
&  (3 ) \ a_{i1} (x_1 + y_1 ) + \ldots + a_{in} (x_n + y_n)  = b_i  \  \ 1 \leq i \leq m \ \textbf( (1) + (2))  \\
& \implies \vec{x} + \vec{y} \in S_1 \end{aligned} \]
\end{proof}

\begin{proof}
\( \forall \vec{x} \in S_1, \vec{x} - \vec{x_0} \in S_2, if \vec{y} = \vec{x} - \vec{x_0} \implies \vec{x} = \vec{x_0} + \vec{y}, \vec{y} \in S_2\)
\end{proof}

\textbf{Solution Sets }

Let \( \alpha_1 = \begin{bmatrix} a_{11} \\ \vdots \\ a_{1n}  \end{bmatrix} \ldots  \alpha_m = \begin{bmatrix} a_{m1} \\ \vdots \\ a_{mn} \end{bmatrix}\) \\
\( \begin{cases} a_{11} x_1 + \ldots + a_{1n} x_n = 0 \\ a_{21}x_1 + \ldots + a_{2n} x_n = 0 \\
\vdots \\ a_{m1}x_1 + \ldots + a_{mn} x_n = 0 \end{cases} \implies \begin{cases} \alpha_1 \vec{x} = 0\\ \alpha_2 \vec{x} = 0  \\ \vdots \\ \alpha_n \vec{x} = 0 \end{cases} \implies \) Solution Set = \( \{ \vec{x} \in \mathbb{R} \mid \vec{x} \bot d_1 1 \leq i \leq m \}\) \\

solution set  can also be written as \( \{ \alpha_1 \ldots \alpha_n \} \)

\section{Solving Linear Systems}

\begin{ex} - \\
\[ \begin{cases} 2x + y = 2 \ (1) \\ x-2y = 4 \ (2) \end{cases} \rightarrow \begin{cases} 2x+y = 2 \ (1) \\ 5y = 10 \ (3) \end{cases} \rightarrow \begin{cases} 2x+y = 2 \ (1) \\ y = 2 \ (4) \end{cases} \rightarrow \begin{cases} 2x = 0 \ (5) \  \\ y = 2 \ (6)  \end{cases} \rightarrow \begin{cases} x = 0  \\ y = 0  \end{cases} \]
\end{ex}
\begin{definition} -\\
We Call  \(\begin{bmatrix} a_{11}  \ a_{12} \ \ldots \ a_{1n} \\ a_{21}  \ a_{22} \ \ldots \ a_{2n}  \\ \ldots \\ a_{m1}  \ a_{m2} \ \ldots \ a_{mn} \end{bmatrix} \) the coefficient Matrix \\
\\
and \(\begin{bmatrix} a_{11}  \ a_{12} \ \ldots \ a_{1n} \mid b_1 \\ a_{21}  \ a_{22} \ \ldots \ a_{2n} \mid b_2  \\ \ldots \\ a_{m1}  \ a_{m2} \ \ldots \ a_{mn} \mid b_m\end{bmatrix} \) A augmented matrix \( (A \mid \vec{b} )\)
\end{definition}

\textbf{Elementary Row Operations (ERO) : } As a previous example we perform teh following 3 operations \\
\begin{enumerate}
\item Multiply a row with a constant (c\(R_i\)) 
\item Add a multiple of one row to another \( R_i + cR_j\)
\item Swap one row for another \( (R_i \longleftrightarrow R_j )\)
\end{enumerate}



\begin{theorem}
If the augmented matrix \( (A_2 \mid \vec{b_2} )\) can be obtained from \( (A_1 \mid \vec{b_1} )\) by performing ERO's, then the linear systems are equivalent
\end{theorem}



\begin{center}
\textbf{End of Lecture Notes} \\
\textbf{Notes By : Harsh Mistry}
\end{center}
\end{document}
