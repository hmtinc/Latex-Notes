%Notes by Harsh Mistry 
%Math 136 
%based on Template from : https://www.cs.cmu.edu/~ggordon/10725-F12/template.tex

\documentclass{article}
\setlength{\oddsidemargin}{0.25 in}
\setlength{\evensidemargin}{-0.25 in}
\setlength{\topmargin}{-0.6 in}
\setlength{\textwidth}{6.5 in}
\setlength{\textheight}{8.5 in}
\setlength{\headsep}{0.75 in}
\setlength{\parindent}{0 in}
\setlength{\parskip}{0.1 in}
\usepackage{amsfonts,graphicx, amssymb}
\usepackage[fleqn]{amsmath}
\usepackage{fixltx2e}
\newcounter{lecnum}
\renewcommand{\thepage}{\thelecnum-\arabic{page}}
\renewcommand{\thesection}{\thelecnum.\arabic{section}}
\renewcommand{\theequation}{\thelecnum.\arabic{equation}}
\renewcommand{\thefigure}{\thelecnum.\arabic{figure}}
\renewcommand{\thetable}{\thelecnum.\arabic{table}}
\newcommand{\lecture}[4]{
   \pagestyle{myheadings}
   \thispagestyle{plain}
   \newpage
   \setcounter{lecnum}{#1}
   \setcounter{page}{1}
   
   
%Info Box 
   \begin{center}
   \framebox{
      \vbox{\vspace{2mm}
    \hbox to 6.28in { {\bf Math 136 - Linear Algebra 
	\hfill Winter 2016} }
       \vspace{4mm}
       \hbox to 6.28in { {\Large \hfill Lecture #1: #2  \hfill} }
       \vspace{2mm}
       \hbox to 6.28in { {\it Lecturer: #3 \hfill Notes By: #4} }
      \vspace{2mm}}
   }
   \end{center}
   
   \markboth{Lecture #1: #2}{Lecture #1: #2}



 
}

\renewcommand{\cite}[1]{[#1]}
\def\beginrefs{\begin{list}%
        {[\arabic{equation}]}{\usecounter{equation}
         \setlength{\leftmargin}{2.0truecm}\setlength{\labelsep}{0.4truecm}%
         \setlength{\labelwidth}{1.6truecm}}}
\def\endrefs{\end{list}}
\def\bibentry#1{\item[\hbox{[#1]}]}

\newcommand{\fig}[3]{
			\vspace{#2}
			\begin{center}
			Figure \thelecnum.#1:~#3
			\end{center}
	}

\newtheorem{theorem}{Theorem}[lecnum]
\newtheorem{lemma}[theorem]{Lemma}
\newtheorem{ex}[theorem]{Example}
\newtheorem{proposition}[theorem]{Proposition}
\newtheorem{claim}[theorem]{Claim}
\newtheorem{corollary}[theorem]{Corollary}
\newtheorem{definition}[theorem]{Definition}
\newenvironment{proof}{{\bf Proof:}}{\hfill\rule{2mm}{2mm}}
\newcommand\E{\mathbb{E}}


%Start of Document 
\begin{document}

\lecture{18}{February 12, 2016}{Yongqiang Zhao}{Harsh Mistry}


\section{Operations on Linear Mappings}

\begin{definition}
Let L : \(\mathbb{R}^n \rightarrow \mathbb{R}^m\) and  M : \(\mathbb{R}^n \rightarrow \mathbb{R}^m\) be linear mappings and let \( c \in \mathbb{R}\), \\ we define L + M : \(\mathbb{R}^n \rightarrow \mathbb{R}^m\) and cL : \(\mathbb{R}^n \rightarrow \mathbb{R}^m\) by 
\[(L+M)(\vec{x}) - L(\vec{x}) + M(\vec{x})\]
\[(cL)(\vec{x}) = cL(\vec{x})\]
 \end{definition}
 
 \begin{theorem}
 If L,M : \(\mathbb{R}^n \rightarrow \mathbb{R}^m\) are liner mappings and c \(\in \mathbb{R}\), then L + M  : \(\mathbb{R}^n \rightarrow \mathbb{R}^m\) and cL : \(\mathbb{R}^n \rightarrow \mathbb{R}^m\) are linear mappings. \\
 Furthermore, 
 \[\begin{bmatrix} L + M \end{bmatrix} = \begin{bmatrix} L \end{bmatrix} + \begin{bmatrix} M \end{bmatrix}\]
 \[\begin{bmatrix} cL \end{bmatrix} = c\begin{bmatrix} L \end{bmatrix} \]
 \end{theorem}
 
 \begin{definition}
We denote the set of all possible linear mappings with domain \(\mathbb{R}^n\) \\ and co-domain \(\mathbb{R}^m\) as \(\mathbb{L}\)
 \end{definition}

\begin{theorem} If L, M \( \in \mathbb{L}\) and c, d \( \in \mathbb{R}\) then, 
\begin{enumerate}
\item \( L + M \in \mathbb{L}\)
\item \( (L + M) + N = L + (M + N) \)
\item \( L + M = M + L \)
\item There exists a linear mapping O : \(\mathbb{R}^n \rightarrow \mathbb{R}^m\), such that L + O = L for all L 
\item There exists a linear mapping (-L): \(\mathbb{R}^n \rightarrow \mathbb{R}^m\) with the property that L + (-L) = O
\item \( cL \in \mathbb{L}\)
\item \(c(dL) = (cd)L \)
\item \( (c+d)L = cL + cM \) 
\item \( c(L+M) = cL + cM\)
\item 1L = L
\end{enumerate}
\end{theorem}

\newpage 

 \begin{definition}
Let L : \(\mathbb{R}^n \rightarrow \mathbb{R}^m\) and M : \(\mathbb{R}^n \rightarrow \mathbb{R}^m\) be linear mappings. \\
The composition of M and L is the function M \(\circ\) L : \(\mathbb{R}^n \rightarrow \mathbb{R}^m\) defined by 
\[(M \circ L ) (\vec{x}) = M (L(\vec{x}))\]
 \end{definition}
 
 \textbf{Remarks : } The range of L must be a subset of the domain of M for M \(\circ\) L to be defined 
 
 \begin{theorem}
 If L : \(\mathbb{R}^n \rightarrow \mathbb{R}^m\)  and M : \(\mathbb{R}^m \rightarrow \mathbb{R}^p\) are linear mappings, then M \( \circ \) L : \(\mathbb{R}^n \rightarrow \mathbb{R}^m\) is a linear mapping and
 \[ (M \circ L) = (M)(L) \]
 \end{theorem}
 
 \begin{definition}
The Linear Mapping ld : \(\mathbb{R}^n \rightarrow \mathbb{R}^m\) defined by \(ld(\vec{x}) = \vec{x}\) is called the identity mapping 
 \end{definition}
 
 \section{Vector Spaces}
 \begin{definition}
A set \(\mathbb{V}\) with an operation of addition, denoted \(\vec{x} + \vec{y}\) and an operation of scaler multiplication denoted \( c\vec{x} \) is called a vector space over \( \mathbb{R}\) if for every \(\vec{v}, \vec{x}, \vec{y} \in \mathbb{V} \)and  c, d \( \in \mathbb{R}\) we have : 
\begin{enumerate}
\item \( \vec{x} + \vec{y} \in \mathbb{V} \)
\item \( (\vec{x} + \vec{y}) + \vec{v} = \vec{x} + (\vec{y} + \vec{v}) \)
\item \( \vec{x} + \vec{y} = \vec{y} + \vec{x}\)
\item There is a vector \(\vec{0} \in \mathbb{V} \) called the zero vector, such that \( \vec{x} + \vec{0} = \vec{x} \ \forall \vec{x} \in \mathbb{V}\) 
\item There exists an element \( -\vec{x} \in \mathbb{V}\) called the additive inverse of \( \vec{x} \), such that \(\vec{x} + \vec{-x} = \vec{0}\)
\item \(c\vec{x} \in \mathbb{V}\)
\item \(c(d\vec{x}) = (cd) \vec{x}\) 
\item \( (c+d)\vec{x} = c\vec{x} + d\vec{x}\)
\item \(c(\vec{x} + \vec{y}) = c\vec{x} + c\vec{y}\)
\item 1x = x
\end{enumerate}

Elements of \(\mathbb{V}\) are called vectors 
 \end{definition}

\textbf{Remarks : }  some times \(\oplus\) and \(\odot\) are used to differentiate these from normal scaler multiplication and scaler addition 

\begin{center}
\textbf{End of Lecture Notes} \\
\textbf{Notes By : Harsh Mistry}
\end{center}
\end{document}
