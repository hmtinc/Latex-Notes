%Notes by Harsh Mistry 
%Math 136 
%Template Taken from : https://www.cs.cmu.edu/~ggordon/10725-F12/template.tex

\documentclass{article}
\setlength{\oddsidemargin}{0.25 in}
\setlength{\evensidemargin}{-0.25 in}
\setlength{\topmargin}{-0.6 in}
\setlength{\textwidth}{6.5 in}
\setlength{\textheight}{8.5 in}
\setlength{\headsep}{0.75 in}
\setlength{\parindent}{0 in}
\setlength{\parskip}{0.1 in}
\usepackage{amsfonts,graphicx, amssymb}
\usepackage[fleqn]{amsmath}
\newcounter{lecnum}
\renewcommand{\thepage}{\thelecnum-\arabic{page}}
\renewcommand{\thesection}{\thelecnum.\arabic{section}}
\renewcommand{\theequation}{\thelecnum.\arabic{equation}}
\renewcommand{\thefigure}{\thelecnum.\arabic{figure}}
\renewcommand{\thetable}{\thelecnum.\arabic{table}}
\newcommand{\lecture}[4]{
   \pagestyle{myheadings}
   \thispagestyle{plain}
   \newpage
   \setcounter{lecnum}{#1}
   \setcounter{page}{1}
   
   
%Info Box 
   \begin{center}
   \framebox{
      \vbox{\vspace{2mm}
    \hbox to 6.28in { {\bf Math 136 - Linear Algebra 
	\hfill Winter 2016} }
       \vspace{4mm}
       \hbox to 6.28in { {\Large \hfill Lecture #1: #2  \hfill} }
       \vspace{2mm}
       \hbox to 6.28in { {\it Lecturer: #3 \hfill Notes By: #4} }
      \vspace{2mm}}
   }
   \end{center}
   
   \markboth{Lecture #1: #2}{Lecture #1: #2}



 
}

\renewcommand{\cite}[1]{[#1]}
\def\beginrefs{\begin{list}%
        {[\arabic{equation}]}{\usecounter{equation}
         \setlength{\leftmargin}{2.0truecm}\setlength{\labelsep}{0.4truecm}%
         \setlength{\labelwidth}{1.6truecm}}}
\def\endrefs{\end{list}}
\def\bibentry#1{\item[\hbox{[#1]}]}

\newcommand{\fig}[3]{
			\vspace{#2}
			\begin{center}
			Figure \thelecnum.#1:~#3
			\end{center}
	}

\newtheorem{theorem}{Theorem}[lecnum]
\newtheorem{lemma}[theorem]{Lemma}
\newtheorem{ex}[theorem]{Example}
\newtheorem{proposition}[theorem]{Proposition}
\newtheorem{claim}[theorem]{Claim}
\newtheorem{corollary}[theorem]{Corollary}
\newtheorem{definition}[theorem]{Definition}
\newenvironment{proof}{{\bf Proof:}}{\hfill\rule{2mm}{2mm}}
\newcommand\E{\mathbb{E}}


%Start of Document 
\begin{document}

\lecture{5}{January 13, 2016}{Yongqiang Zhao}{Harsh Mistry}


\section{Subspace Basis Examples}
\textbf{Find the basis for a given subspace}
\[\begin{aligned} S_2 = \left\{ \begin{bmatrix} a-b \\ b-c \\ c-a \end{bmatrix} \mid a , b , c \in \mathbb{R}^3 \right\} \end{aligned}\]
\[\forall \vec{x} \in S_2 \ \vec{x} = \begin{bmatrix} a-b \\ b-c \\ c-a \end{bmatrix} = a \begin{bmatrix} 1 \\ 0 \\ -1 \end{bmatrix} + b \begin{bmatrix} -1 \\ 1 \\ 1 \end{bmatrix} + c\begin{bmatrix} 0 \\ -1 \\ 1 \end{bmatrix}\]
\[ \begin{aligned} \vec{x} \in Span \left\{ \begin{bmatrix} 1 \\ 0 \\ -1 \end{bmatrix} \begin{bmatrix} -1 \\ 1 \\ 1 \end{bmatrix}  \begin{bmatrix} -1 \\ 1 \\ 1 \end{bmatrix} \right\}  \end{aligned} \]
\[ S_2 = Span \left\{ \begin{bmatrix} 1 \\ 0 \\ -1 \end{bmatrix} \begin{bmatrix} -1 \\ 1 \\ 1 \end{bmatrix}  \begin{bmatrix} -1 \\ 1 \\ 1 \end{bmatrix} \right\} = Span \left\{ \begin{bmatrix} 1 \\ 0 \\ -1 \end{bmatrix} \begin{bmatrix} -1 \\ 1 \\ 1 \end{bmatrix} \right\} \text{ which is linearly independant}\]
$$ \therefore \text{ this is a basis for } S_2 $$ 

\section{Dot Product}
\textbf{Recall : } 
\begin{itemize}
\item In \( \mathbb{R}^2 , \forall \vec{x} = \begin{bmatrix} x_1 \\ x_2 \end{bmatrix} , \vec{y} = \begin{bmatrix} y_1 \\ y_2 \end{bmatrix} , \vec{x} \cdot \vec{y} = x_1y_1 + x_2y_2 \)
\item \( \| \vec{x} \| = \sqrt{x_1 ^2 + x_2 ^2}  =  \sqrt{\vec{x} \cdot \vec{x}}\)
\item \( \vec{x} \cdot \vec{y} = \| x \| \| y\| \cos \theta \)
\end{itemize}

\begin{proof}
\[ \begin{aligned} \vec{x} - \vec{y} = \begin{bmatrix} x_1 - y_1 \\ x_2- y_2\end{bmatrix} \cos \theta & =  \frac{\|\vec{x}\|^2 + \|\vec{y}\|^2 - \| \vec{x} - \vec{y}\| ^2}{2\|\vec{x}\| \vec{y}\|} \\
& = \frac{x_1^2 + x_2^2 + y_2^2 + y_2^2 - ( (x_1 - y_1) ^2 + (x_2 - y_2) ^2)}{2\|\vec{x}\| \vec{y}\|} \\ 
\cos \theta & = \frac{2x_1y_1 + 2x_2y_2}{2\|\vec{x}\| \vec{y}\|} = \frac{x_1\theta_1 + x_2\theta_2}{\|\vec{x}\| \vec{y}\|}  \\
& =  \vec{x}\cdot \vec{y}  = x_1y_2+x_2y_2 \\
& = \|\vec{x}\| \|\vec{y} \| \cos \theta \end{aligned} \]
\end{proof}

\begin{definition} \( \forall \vec{x} = \begin{bmatrix} x_1 \\ \vdots \\ x_n \end{bmatrix} , \vec{y} = \begin{bmatrix} y_1 \\  \vdots \\ y_n \end{bmatrix} \in \mathbb{R}^n , \vec{x}\cdot \vec{y} = x_1y_1 + \ldots + x_ny_n = \sum\limits_{i=1}^{n} x_iy_i \)
\end{definition}

\begin{proposition}
\[\forall \vec{x} , \vec{y} , \vec{z} \in \mathbb{R}^n , \forall s, t \in \mathbb{R} \]
\begin{enumerate}
\item \(\vec{x} \cdot \vec{x} = 0  \text{ iff } \vec{x} = \vec{0} \)
\item \(\vec{x} \cdot \vec{y} = \vec{y} = \vec{x} \)
\item \(  \vec{x}  (t\vec{y} + s\vec{z}) = t(\vec{x} \cdot \vec{y}) + s(\vec{x} \cdot \vec{z}) \)
\end{enumerate}
\end{proposition}

\begin{definition} \( \forall \vec{x} \in \mathbb{R} \) The length (or norm) of \( \vec{x} \text{ is } \|\vec{x}\| = \sqrt{\vec{x}\cdot \vec{x}}\)\end{definition}
\textbf{Also , \( \vec{x} \) is called the unit vector if \( \|\vec{x}\| = 1\)}

\begin{theorem} \( \forall \vec{x} \vec{y} \in \mathbb{R}^n , t\in \mathbb{R} \) we have \\
\begin{enumerate}
\item \(\|\vec{x}\| \geq 0 \text{ iff } \vec{x} = \vec{0} \)
\item \(\|t\vec{x}\|  = \mid t \mid \|\vec{x}\| \)
\item Cauchy Schwarz inequality : \( \| \vec{x} \cdot \vec{y} \| \leq \|\vec{x}\| \|\vec{y}\| \)
\item Triangle inequality : \( \|\vec{x} + \vec{y}\| \leq \|\vec{x}\| + \|\vec{y}\| \)
\end{enumerate}
\end{theorem}

\begin{proof} Cauchy Schwarz inequality \\
\[ \begin{aligned} \text{If }  \vec{x} & = \vec{0} \text { it is clear,  suppose } \vec{x} \neq \vec{0} \\ 
0 & \leq \| t\vec{x} + t\vec{y} \| ^2 = (t\vec{x} + y) \cdot (t\vec{x} + y) = t^2 \vec{x} \cdot \vec{x} + 2t\vec{x}\cdot \vec{y} + \vec{y} \cdot \vec{y} \\
0 & \leq \|x \| ^2 t^2 + 2(\vec{x} \cdot \vec{y}) t + \|\vec{y}\| , \forall t \in \mathbb{R} \\
4 (\vec{x} \cdot \vec{y}) ^2 & \leq 4 \|\vec{x}\| ^ 2 \|\vec{y}\| ^ 2   \text{ Using the discriminant} \\
\|\vec{x} + \vec{y}\| & \leq \|\vec{x}\| + \|\vec{y}\| 
\end{aligned} \]
\end{proof}

\textbf{Remarks} : If \( \vec{x} = \begin{bmatrix} x_1 \\ \vdots \\ x_n \end{bmatrix} , \vec{y} = \begin{bmatrix} y_1 \\  \vdots \\ y_n \end{bmatrix}\) \\
\[\vec{x} - \vec{y}  = x_1y_1 + \ldots  + x_ny_n \]
\[ \| \vec{x} \| = x_1 ^2 + \ldots x_n^2 \ \ \ \| \vec{y} \| = y_1 ^2 + \ldots y_n^2 \|\]
Coordinate Form of C S 
\[\| x_1y_1 + \ldots + x_ny_n\| \leq \sqrt{x_1 ^2 + \ldots x_n^2} \sqrt{y_1 ^2 + \ldots y_n^2} \]
Try letting n = 3



\begin{center}
\textbf{End of Lecture Notes} \\
\textbf{Notes By : Harsh Mistry}
\end{center}
\end{document}
