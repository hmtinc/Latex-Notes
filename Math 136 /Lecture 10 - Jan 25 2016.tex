%Notes by Harsh Mistry 
%Math 136 
%Template Taken from : https://www.cs.cmu.edu/~ggordon/10725-F12/template.tex

\documentclass{article}
\setlength{\oddsidemargin}{0.25 in}
\setlength{\evensidemargin}{-0.25 in}
\setlength{\topmargin}{-0.6 in}
\setlength{\textwidth}{6.5 in}
\setlength{\textheight}{8.5 in}
\setlength{\headsep}{0.75 in}
\setlength{\parindent}{0 in}
\setlength{\parskip}{0.1 in}
\usepackage{amsfonts,graphicx, amssymb}
\usepackage[fleqn]{amsmath}
\newcounter{lecnum}
\renewcommand{\thepage}{\thelecnum-\arabic{page}}
\renewcommand{\thesection}{\thelecnum.\arabic{section}}
\renewcommand{\theequation}{\thelecnum.\arabic{equation}}
\renewcommand{\thefigure}{\thelecnum.\arabic{figure}}
\renewcommand{\thetable}{\thelecnum.\arabic{table}}
\newcommand{\lecture}[4]{
   \pagestyle{myheadings}
   \thispagestyle{plain}
   \newpage
   \setcounter{lecnum}{#1}
   \setcounter{page}{1}
   
   
%Info Box 
   \begin{center}
   \framebox{
      \vbox{\vspace{2mm}
    \hbox to 6.28in { {\bf Math 136 - Linear Algebra 
	\hfill Winter 2016} }
       \vspace{4mm}
       \hbox to 6.28in { {\Large \hfill Lecture #1: #2  \hfill} }
       \vspace{2mm}
       \hbox to 6.28in { {\it Lecturer: #3 \hfill Notes By: #4} }
      \vspace{2mm}}
   }
   \end{center}
   
   \markboth{Lecture #1: #2}{Lecture #1: #2}



 
}

\renewcommand{\cite}[1]{[#1]}
\def\beginrefs{\begin{list}%
        {[\arabic{equation}]}{\usecounter{equation}
         \setlength{\leftmargin}{2.0truecm}\setlength{\labelsep}{0.4truecm}%
         \setlength{\labelwidth}{1.6truecm}}}
\def\endrefs{\end{list}}
\def\bibentry#1{\item[\hbox{[#1]}]}

\newcommand{\fig}[3]{
			\vspace{#2}
			\begin{center}
			Figure \thelecnum.#1:~#3
			\end{center}
	}

\newtheorem{theorem}{Theorem}[lecnum]
\newtheorem{lemma}[theorem]{Lemma}
\newtheorem{ex}[theorem]{Example}
\newtheorem{proposition}[theorem]{Proposition}
\newtheorem{claim}[theorem]{Claim}
\newtheorem{corollary}[theorem]{Corollary}
\newtheorem{definition}[theorem]{Definition}
\newenvironment{proof}{{\bf Proof:}}{\hfill\rule{2mm}{2mm}}
\newcommand\E{\mathbb{E}}


%Start of Document 
\begin{document}

\lecture{10}{January 25, 2016}{Yongqiang Zhao}{Harsh Mistry}


\section{Reduced Row Echelon Form (RREF)}
\begin{enumerate}
\item All zero rows are "downstairs"
\item The first non-zero entry in each non-zero row is 1 and is called the leading one
\item The leading one in non-zero row is to the right of the leading one is the row above
\item A leading one is the only non-zero entry in its column
\end{enumerate}

\begin{theorem}
Every matrix is equivalent to a unique RREF 
\end{theorem}

\begin{ex}
Solve the following system \\
\[ \begin{aligned} \begin{cases} -x_2 + x_3 + x_4 = 4 \\ x_1 + x_2 + x_4 = 1 \\ 2x_1 + x_2 + x_3 + x_4 = -2 \end{cases} & \rightarrow \begin{bmatrix}0 & -1 & 1 & 1 &   4 \\ 1 &  1 &  0 &   1 &  1  \\ 2 &  1 &  1 &  1 &  -2 \end{bmatrix} & \rightarrow \begin{bmatrix}1  & 1 & 0 & 1 &   1 \\ 0 &  -1 &  1 &   1 &  4  \\ 2 &  1 &  1 &  1 &  -2 \end{bmatrix} \\ & \rightarrow \begin{bmatrix}1  & 1 & 0 & 1 &   1 \\ 0 &  -1 &  1 &   1 &  4  \\ 0 &  -1 &  1 &  -1 &  -4 \end{bmatrix}& \rightarrow \begin{bmatrix}1  & 1 & 0 & 1 &   1 \\ 0 &  -1 &  1 &   1 &  4  \\ 0 &  0 &  0 &  -2 &  8 \end{bmatrix} \\
& \rightarrow \begin{bmatrix}1  & 1 & 0 & 1 &   1 \\ 0 &  1 &  -1 &   -1 &  -4  \\ 0 &  0 &  0 &  1 &  4 \end{bmatrix}  &  \rightarrow  \begin{bmatrix}1  & 1 & 0 & 1 &   1 \\ 0 &  1 &  -1 &  0  &  0  \\ 0 &  0 &  0 &  1 &  4 \end{bmatrix} \\ 
& \rightarrow \begin{bmatrix}1  & 1 & 0 & 1 &   -2 \\ 0 &  1 &  -1 &   -1 &  -4  \\ 0 &  0 &  0 &  1 &  4 \end{bmatrix} & \implies \begin{cases} x_1 + x_3= -3 \\ x_2 - x_3 = 0 \\ x_4 = 4 \end{cases} \end{aligned} \]
\[ \text{Let t  = } x_3 \ \ \vec{x} = \begin{bmatrix} x_1 \\ x_2 \\ x_3 \\ x_4 \end{bmatrix} = \begin{bmatrix} -3 -t  \\ t \\ t \\ 4  \end{bmatrix} = \begin{bmatrix} -3 \\ 0 \\ 0 \\ 4 \end{bmatrix} + t \begin{bmatrix} -1 \\ 1 \\ 1 \\ 0 \end{bmatrix} , t \in \mathbb{R} \]
\end{ex}

\begin{definition}
Any variable whose column does not contain a leading one in the RREF of the coefficient matrix is called a free variable
\end{definition}

\begin{definition}
The rank of a matrix A is the number of leading one's in its RREF and is denoted as \( rankA \) , \( rank(A) \), or \( r(A) \) 
\end{definition}

\begin{ex}
$$  r \begin{bmatrix} 0 & 1 & 0 \\ 0 & 0 & 1 \end{bmatrix} = 2 \ \ \ \ \ r \begin{bmatrix} 1 & 0 & 0 \\ 0 & 1 & 0 \\ 0 & 0 & 1 \end{bmatrix} = 3   $$
\end{ex}

\begin{theorem}
\( A_{m \times n} , r(A) \leq min\{ m , n \} \) 
\end{theorem}

\begin{theorem}
Given a linear system (A \( \mid \vec{b} \) )  
\begin{enumerate}
\item (A \( \mid \vec{b} \) )  is consistent iff  r(A \( \mid \vec{b} \)) = r(A)   
\item if (A \( \mid \vec{b} \) )  is consistent, then it contains n - r(A) free variables
\item r(A) = m iff (A \( \mid \vec{b} \) ) is consistent for  for every \( \vec{b} \in \mathbb{R} ^m \) 
\end{enumerate}
\end{theorem}

\begin{center}
\textbf{End of Lecture Notes} \\
\textbf{Notes By : Harsh Mistry}
\end{center}
\end{document}
