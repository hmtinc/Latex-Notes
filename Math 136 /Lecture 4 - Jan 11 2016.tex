%Notes by Harsh Mistry 
%Math 136 
%Template Taken from : https://www.cs.cmu.edu/~ggordon/10725-F12/template.tex

\documentclass{article}
\setlength{\oddsidemargin}{0.25 in}
\setlength{\evensidemargin}{-0.25 in}
\setlength{\topmargin}{-0.6 in}
\setlength{\textwidth}{6.5 in}
\setlength{\textheight}{8.5 in}
\setlength{\headsep}{0.75 in}
\setlength{\parindent}{0 in}
\setlength{\parskip}{0.1 in}
\usepackage{amsfonts,graphicx, amssymb}
\usepackage[fleqn]{amsmath}
\newcounter{lecnum}
\renewcommand{\thepage}{\thelecnum-\arabic{page}}
\renewcommand{\thesection}{\thelecnum.\arabic{section}}
\renewcommand{\theequation}{\thelecnum.\arabic{equation}}
\renewcommand{\thefigure}{\thelecnum.\arabic{figure}}
\renewcommand{\thetable}{\thelecnum.\arabic{table}}
\newcommand{\lecture}[4]{
   \pagestyle{myheadings}
   \thispagestyle{plain}
   \newpage
   \setcounter{lecnum}{#1}
   \setcounter{page}{1}
   
   
%Info Box 
   \begin{center}
   \framebox{
      \vbox{\vspace{2mm}
    \hbox to 6.28in { {\bf Math 136 - Linear Algebra 
	\hfill Winter 2016} }
       \vspace{4mm}
       \hbox to 6.28in { {\Large \hfill Lecture #1: #2  \hfill} }
       \vspace{2mm}
       \hbox to 6.28in { {\it Lecturer: #3 \hfill Notes By: #4} }
      \vspace{2mm}}
   }
   \end{center}
   
   \markboth{Lecture #1: #2}{Lecture #1: #2}



 
}

\renewcommand{\cite}[1]{[#1]}
\def\beginrefs{\begin{list}%
        {[\arabic{equation}]}{\usecounter{equation}
         \setlength{\leftmargin}{2.0truecm}\setlength{\labelsep}{0.4truecm}%
         \setlength{\labelwidth}{1.6truecm}}}
\def\endrefs{\end{list}}
\def\bibentry#1{\item[\hbox{[#1]}]}

\newcommand{\fig}[3]{
			\vspace{#2}
			\begin{center}
			Figure \thelecnum.#1:~#3
			\end{center}
	}

\newtheorem{theorem}{Theorem}[lecnum]
\newtheorem{lemma}[theorem]{Lemma}
\newtheorem{ex}[theorem]{Example}
\newtheorem{proposition}[theorem]{Proposition}
\newtheorem{claim}[theorem]{Claim}
\newtheorem{corollary}[theorem]{Corollary}
\newtheorem{definition}[theorem]{Definition}
\newenvironment{proof}{{\bf Proof:}}{\hfill\rule{2mm}{2mm}}
\newcommand\E{\mathbb{E}}


%Start of Document 
\begin{document}

\lecture{4}{January 11, 2016}{Yongqiang Zhao}{Harsh Mistry}


\section{Basis Examples}
\textbf{Standard Basis}
$$ \mathbb{R}^n \left\{ \begin{bmatrix} 1 \\ 0 \\ \vdots \\ 0 \end{bmatrix} \ldots \begin{bmatrix} 0 \\ 0 \\ \vdots \\ 1 \end{bmatrix} \right\} $$
So, a standard basis is when all numbers in the vector except the corresponding component are 0 

\begin{ex} \textbf{Determine if the set is a basis} \\
\[\begin{aligned} \left\{ \begin{bmatrix} 1 \\ 2 \end{bmatrix} \begin{bmatrix} 2 \\ 1 \end{bmatrix} \right\} \textbf{ For } \mathbb{R}^2 \end{aligned} \]
\[ \forall \vec{x} \in \mathbb{R}^2 \text{ , } \vec{x} = \begin{bmatrix} x_1 \\ x_2 \end{bmatrix} \text{ ,  If }  \begin{bmatrix} x_1 \\ x_2 \end{bmatrix} = t_1 \begin{bmatrix} 1 \\ 2 \end{bmatrix} + t_1 \begin{bmatrix} 2 \\ 1 \end{bmatrix} \]
\[ \begin{cases} t_1 + 2t_2 = x_1 \\ 2t_1 + t_2 = x_2 \end{cases}  \implies \begin{cases} t_1 = \frac{2x_2 - x_1}{3} \\ t_2 = \frac{2x_1- x_2}{3} \end{cases}  \implies \vec{x} \in Span\left\{ \begin{bmatrix} 1 \\ 2 \end{bmatrix} \begin{bmatrix} 2 \\ 1 \end{bmatrix} \right\} \implies \mathbb{R}^2 = Span \left\{ \begin{bmatrix} 1 \\ 2 \end{bmatrix} \begin{bmatrix} 2 \\ 1 \end{bmatrix} \right\}\]
Suppose \( \vec{x} = \begin{bmatrix} 0 \\ 0 \end{bmatrix} \) , then we get \( t_1 = t_2 = 0 \) , so \( \left\{ \begin{bmatrix} 1 \\ 2 \end{bmatrix} \begin{bmatrix} 2 \\ 1 \end{bmatrix} \right\} \) is linearly independant
\end {ex}

\begin{definition} -\\
\begin{enumerate}
\item Let \(\vec{v} \ \vec{b} \in \mathbb{R} ^ n \) , \( \vec{v} \neq \vec{0} \) \\
The set  \(\{ \vec{x} \mid \vec{x} = t\vec{x} + t\vec{b} \text{ , } t \in \mathbb{R} \} \) is called a line in \( \mathbb{R} ^n \)
\item Let \(\vec{v} \ \vec{v_2} \in \mathbb{R} ^ n \) , \( \{ \vec{v_1} \vec{v_2} \} \) being  linear independant  and \(\vec{b} \in \mathbb{R} ^ n \) \\
then set \( \{ \vec{x} \mid \vec{x} = t_1\vec{v_1} + t_2\vec{v_2} + \vec{b} \) , \( t_1 \ t_2 \in \mathbb{R} \} \) is called a plane
\item Let \( \vec{v_1} \ldots \vec{v_{n-1}} \in \text{ and } \mathbb{R}^n  \{ \vec{v_1} \ldots \vec{v_{n-1}}\}\) be linearly independant \\
\( \vec{b} \in \mathbb{R} ^ n \) the set \( \{ \vec{x} \mid \vec{x} = t_1\vec{v_1} \ldots t_{n-1} \vec{v{n-1}} + \vec{b_1}  ,  
t_n \ldots t_{n-1} \in \mathbb{R} \} \)
\end{enumerate}
\end{definition}
\newpage
\section{Subspaces}
\begin{definition} -\\
A non-empty set S or \( \mathbb{R}^n \) is called a subaspace if S is closed under adddion and scaler multiplication
\begin{itemize}
\item \( \forall \vec{x} \ \vec{y} \in S , \vec{x} + \vec{y} \in S \)
\item \( \forall \vec{x} \in S , \forall C \in \mathbb{R} , C\vec{x} \in S \)
\end{itemize}
\end{definition} 

\textbf{Remarks}
\begin{itemize}
\item For any subspace S \( \vec{0} \in S \) 
\item \( \{ \vec{0}\} \text{ and } \mathbb{R}^n \) are subspaces of \( \mathbb{R} ^ n \) 
\end{itemize}

\begin{ex} Determine if the given set is a subspace \\
%split into 2 mini pages 
\begin{minipage}{.5\textwidth} %
\textbf{a)}
\[\begin{aligned} S_1 \left\{ \begin{bmatrix} x_1 \\ x_2 \end{bmatrix} \in \mathbb{R}^2 \mid x_1 - x_2 = 0 \right\} \end{aligned} \] 
\begin{enumerate}
\item \( \vec{0} \in S \)  , So \( S_1\) is non-empty
\item \( \forall \vec{x} = \begin{bmatrix} x_1 \\ x_2 \end{bmatrix} , \begin{bmatrix} y_1 \\ y_2 \end{bmatrix} \in S_1 \implies x_1 - x_2 = 0  \text{ and } y_1 - y_2 = 0 \implies (x_1 + y_1) - (x_2 + y_2) = 0 \implies \vec{x} + \vec{y} \in S_1 \)
\item \(\forall c \in \mathbb{R} , cx_1 - cx_2  = c(x_1 - x_2) = 0 \implies c\vec{x} \in S_1\)
\end{enumerate}
\( \therefore S_1\) is a subspace 
\end{minipage} %
\begin{minipage}{.5\textwidth} %
\textbf{b)}
\[\begin{aligned} S_1 \left\{ \begin{bmatrix} x_1 \\ x_2 \end{bmatrix} \in \mathbb{R}^2 \mid x_1 - x_2 = 1 \right\} \end{aligned} \] 
\[\vec{0} \notin \ S_2 \ \therefore S_2 \text{ is not a subspace } \]
\end{minipage}
\end{ex}

\begin{theorem}
Given \( S = \{ \vec{v_1} , \vec{v_2} , \vec{v_k} \} ,  Span_b  \) is a subspace  
\end{theorem}

\section{Basis of a subspace}
\textbf{Questions for Next Lecture} : Find the basis of the following subspaces 
\begin{itemize}
\item \( S_1 = \left\{ \begin{bmatrix} x_1 \\ x_2 \\ x_3 \end{bmatrix} \in \mathbb{R} \mid x_1 + x_2 - x_3 = 0 \right\} \)
\item \( S_2 = \left\{ \begin{bmatrix} a - b \\ b - c \\ c - a \end{bmatrix} \in \mathbb{R} \mid a , b , c \in \mathbb{R}  \right\} \)
\item \( \forall \vec{x} \in S_1 , \vec{x} \begin{bmatrix} x_1 \\ x_2 \\ x_1 + x_2 \end{bmatrix} = x_1 \begin{bmatrix} 1 \\ 0 \\ 1 \end{bmatrix} + x_2 \begin{bmatrix} 0 \\ 1 \\ 1 \end{bmatrix} \)
\end{itemize}



\begin{center}
\textbf{End of Lecture Notes} \\
\textbf{Notes By : Harsh Mistry}
\end{center}
\end{document}
