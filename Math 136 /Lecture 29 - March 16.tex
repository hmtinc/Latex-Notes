%Notes by Harsh Mistry 
%Math 136 
%based on Template from : https://www.cs.cmu.edu/~ggordon/10725-F12/template.tex

\documentclass{article}
\setlength{\oddsidemargin}{0.25 in}
\setlength{\evensidemargin}{-0.25 in}
\setlength{\topmargin}{-0.6 in}
\setlength{\textwidth}{6.5 in}
\setlength{\textheight}{8.5 in}
\setlength{\headsep}{0.75 in}
\setlength{\parindent}{0 in}
\setlength{\parskip}{0.1 in}
\usepackage{amsfonts,graphicx, amssymb}
\usepackage[fleqn]{amsmath}
\usepackage{fixltx2e}
\newcounter{lecnum}
\renewcommand{\thepage}{\thelecnum-\arabic{page}}
\renewcommand{\thesection}{\thelecnum.\arabic{section}}
\renewcommand{\theequation}{\thelecnum.\arabic{equation}}
\renewcommand{\thefigure}{\thelecnum.\arabic{figure}}
\renewcommand{\thetable}{\thelecnum.\arabic{table}}
\newcommand{\lecture}[4]{
   \pagestyle{myheadings}
   \thispagestyle{plain}
   \newpage
   \setcounter{lecnum}{#1}
   \setcounter{page}{1}
   
   
%Info Box 
   \begin{center}
   \framebox{
      \vbox{\vspace{2mm}
    \hbox to 6.28in { {\bf Math 136 - Linear Algebra 
	\hfill Winter 2016} }
       \vspace{4mm}
       \hbox to 6.28in { {\Large \hfill Lecture #1: #2  \hfill} }
       \vspace{2mm}
       \hbox to 6.28in { {\it Lecturer: #3 \hfill Notes By: #4} }
      \vspace{2mm}}
   }
   \end{center}
   
   \markboth{Lecture #1: #2}{Lecture #1: #2}



 
}

\renewcommand{\cite}[1]{[#1]}
\def\beginrefs{\begin{list}%
        {[\arabic{equation}]}{\usecounter{equation}
         \setlength{\leftmargin}{2.0truecm}\setlength{\labelsep}{0.4truecm}%
         \setlength{\labelwidth}{1.6truecm}}}
\def\endrefs{\end{list}}
\def\bibentry#1{\item[\hbox{[#1]}]}

\newcommand{\fig}[3]{
			\vspace{#2}
			\begin{center}
			Figure \thelecnum.#1:~#3
			\end{center}
	}

\newtheorem{theorem}{Theorem}[lecnum]
\newtheorem{lemma}[theorem]{Lemma}
\newtheorem{ex}[theorem]{Example}
\newtheorem{proposition}[theorem]{Proposition}
\newtheorem{claim}[theorem]{Claim}
\newtheorem{corollary}[theorem]{Corollary}
\newtheorem{definition}[theorem]{Definition}
\newenvironment{proof}{{\bf Proof:}}{\hfill\rule{2mm}{2mm}}
\newcommand\E{\mathbb{E}}


%Start of Document 
\begin{document}

\lecture{29}{March 16, 2016}{Yongqiang Zhao}{Harsh Mistry}

\section{More Determinants!}

\begin{corollary}
If A is an \(n \times n\) matrix and E is an \(n \times n\) elementary matrix, then \(det EA = det E \ det A\)
\end{corollary}

\begin{theorem} Addition to the Invertible Matrix Theorem \\
An \(n \times n \) matrix A is invertible if and only if \(det A \neq 0 \)
\end{theorem}

\begin{proof} 
Let R be the RREF of A, then there exists k elementary matrices \(E_1 \ldots E_k \) such that \(A = E_1 E_2 \ldots E_k R\)
Then, 
\[ A = det(E_1 E_2 \ldots E_k R ) = det(E_1) det(E_2) \ldots det(E_k) det(R) \]
Thus, 
\[ det A \neq 0 \iff det R \neq 0 \text{ since the determinant of an elementary marix is non zero} \] 
\[\therefore det R \neq 0 \iff rank R = n \iff \text{ A is invertible }\]
\end{proof}

\begin{theorem}
If A and B are \( n \times n\), then det(AB) = det(A) det (B)
\end{theorem}

\begin{proof}
Write A as \(A = E_1 E_2 \ldots E_k R\) such that R is the RREF of A. If A is invertible, then 
\[ R = I_n\]
\[det A = det(E_1) det(E_2) \ldots det(E_k) det(R) \]
\[det (AB) = E_1 E_2 \ldots E_k B = det(E_1) det(E_2) \ldots det(E_k) det(B) = det A det B\]
If AB is non-invertible, then E has at least one row of zeros and RB also contains one row of zeros which implies \(det(RB) = 0\)
\[det (AB) = E_1 E_2 \ldots E_k RB = det(E_1) det(E_2) \ldots det(E_k) det(RB) = 0 \]
While, 
\[det A = det(E_1) det(E_2) \ldots det(E_k) det(R) = det(E_1) det(E_2) \ldots det(E_k) det(R) = 0 \text{since det R = 0}\]
\[\implies det(AB) = 0 = detA det B\]
\end{proof}

\begin{corollary}
If A is an invertible matrix, then \(det A^{-1} = \frac{1}{det A} \)
\end{corollary}

\begin{theorem} False Expansion Theorem \\
If A is an \(n \times n\) matrix with cofactors \(C_{ij}\), then 
\[ \sum_{k=1}^{n} (A)_{ik}(C)_{jk} = 0, \text{ whenever } i \neq j\]
\end{theorem}

\begin{theorem}
If A is invertible, then \((A^{-1})_{ij} = \frac{1}{det A} C_{ij} \)
\end{theorem}

\textbf{Quick Fact : } for any two matricies \(A_{m\times n} \ B_{n \times s}\), If A contains a row of zeros, then AB also contains a row of zeros. (This is very useful in alot of proofs)


\begin{center}
\textbf{End of Lecture Notes}\\
\textbf{Notes by : Harsh Mistry}
\end{center}
\end{document}
