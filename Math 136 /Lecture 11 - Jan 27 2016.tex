%Notes by Harsh Mistry 
%Math 136 
%Template Taken from : https://www.cs.cmu.edu/~ggordon/10725-F12/template.tex

\documentclass{article}
\setlength{\oddsidemargin}{0.25 in}
\setlength{\evensidemargin}{-0.25 in}
\setlength{\topmargin}{-0.6 in}
\setlength{\textwidth}{6.5 in}
\setlength{\textheight}{8.5 in}
\setlength{\headsep}{0.75 in}
\setlength{\parindent}{0 in}
\setlength{\parskip}{0.1 in}
\usepackage{amsfonts,graphicx, amssymb}
\usepackage[fleqn]{amsmath}
\newcounter{lecnum}
\renewcommand{\thepage}{\thelecnum-\arabic{page}}
\renewcommand{\thesection}{\thelecnum.\arabic{section}}
\renewcommand{\theequation}{\thelecnum.\arabic{equation}}
\renewcommand{\thefigure}{\thelecnum.\arabic{figure}}
\renewcommand{\thetable}{\thelecnum.\arabic{table}}
\newcommand{\lecture}[4]{
   \pagestyle{myheadings}
   \thispagestyle{plain}
   \newpage
   \setcounter{lecnum}{#1}
   \setcounter{page}{1}
   
   
%Info Box 
   \begin{center}
   \framebox{
      \vbox{\vspace{2mm}
    \hbox to 6.28in { {\bf Math 136 - Linear Algebra 
	\hfill Winter 2016} }
       \vspace{4mm}
       \hbox to 6.28in { {\Large \hfill Lecture #1: #2  \hfill} }
       \vspace{2mm}
       \hbox to 6.28in { {\it Lecturer: #3 \hfill Notes By: #4} }
      \vspace{2mm}}
   }
   \end{center}
   
   \markboth{Lecture #1: #2}{Lecture #1: #2}



 
}

\renewcommand{\cite}[1]{[#1]}
\def\beginrefs{\begin{list}%
        {[\arabic{equation}]}{\usecounter{equation}
         \setlength{\leftmargin}{2.0truecm}\setlength{\labelsep}{0.4truecm}%
         \setlength{\labelwidth}{1.6truecm}}}
\def\endrefs{\end{list}}
\def\bibentry#1{\item[\hbox{[#1]}]}

\newcommand{\fig}[3]{
			\vspace{#2}
			\begin{center}
			Figure \thelecnum.#1:~#3
			\end{center}
	}

\newtheorem{theorem}{Theorem}[lecnum]
\newtheorem{lemma}[theorem]{Lemma}
\newtheorem{ex}[theorem]{Example}
\newtheorem{proposition}[theorem]{Proposition}
\newtheorem{claim}[theorem]{Claim}
\newtheorem{corollary}[theorem]{Corollary}
\newtheorem{definition}[theorem]{Definition}
\newenvironment{proof}{{\bf Proof:}}{\hfill\rule{2mm}{2mm}}
\newcommand\E{\mathbb{E}}


%Start of Document 
\begin{document}

\lecture{11}{January 27, 2016}{Yongqiang Zhao}{Harsh Mistry}


\section{Rank Continued}
r(A) \(\leq r(A\mid\vec{b})\)\\
r(A) \(= r(A\mid\vec{0})\)
\begin{corollary}
A constant linear system \((A\mid \vec{b})\) has a unique solution iff \( r(A) = n \) (no free variables)  \\
In addition, a homogeneous system (A\(\mid \vec{0}\)) only has a trivial solution iff r(A) = n
\end{corollary}

\begin{theorem}
A set of n vectors \( \{ \vec{v_1} \ldots \vec{v_n} \} \subset \mathbb{R}^n \) is linear independent iff it spans \( \mathbb{R}^n\)\\
\( \implies \{ \vec{v_1} \ldots \vec{v_n} \} \) is a basis iff it is linear independent 
\end{theorem}

\begin{proof}
Given \( \{ \vec{v_1} \ldots \vec{v_n} \} \subset \mathbb{R}^n\) , Let A = \( \{ \vec{v_1} \ldots \vec{v_n} \}\) \\
\(SpanA = \mathbb{R}^n \iff \forall \vec{x} \in \mathbb{R}^n , \vec{x} = t_1\vec{v_1}+\ldots+t_n\vec{v_n} \iff (A \mid \vec{x} ) \) is constant for any \( \vec{x} \in \mathbb{R}^n \iff  \\
 r(A) = n \iff (A \mid \vec{0}\) only has a trivial solution \( \iff t_1\vec{v_1} + \ldots + t_n \vec{v_n} = 0 \) only has a trivial solution  \\ \(\iff \{ \vec{v_1} \ldots \vec{v_n} \} \) is linear independent 
\end{proof}

\begin{theorem}
Let \( (A \mid \vec{b} ) \) be a constant linear system, r(A) = r then the solution set has he form 
\[ \{ \vec{x} \in \mathbb{R}^n \mid \vec{x} = \vec{x_0} + t_1\vec{v_1} + \ldots + t_{n-r} \vec{v_{n-r}} , t_i \in \mathbb{R}\} \}\] where  \(\{ \vec{v_1} \ldots \vec{v_{n-r}} \}\) is linear independent 
\end{theorem}

\begin{ex}
\[\begin{bmatrix} 1 & 2 & 0 & \mid & 1 \\ 0 & 0 & 1 & \mid & 1 \end{bmatrix} \rightarrow \begin{cases} x_1 + 2x_2 = 1 \\ x_3 = 1\end{cases} \text{let t = } x_3 \rightarrow \vec{x} = \begin{bmatrix} 1 - 2x \\ x_2 \\ 1 \end{bmatrix} = \begin{bmatrix}1 \\ 0 \\ 1 \end{bmatrix} + t \begin{bmatrix} -2 \\ 1 \\ 0\end{bmatrix} \]
\end{ex}

\section{Rank and Linear Independence}
\[ \begin{aligned} \{ \vec{v_1} \ldots \vec{v_n} \} \subset \mathbb{R}^n & \iff t_1\vec{v_1} + \ldots + t_k\vec{v_k} = \vec{0}\text{ has a trivial solution} \\
& \iff (\vec{v_1} \ldots \vec{v_k} \mid \vec{0}) \text{ only has a trivial solution } \\
& \iff r(\vec{v_1} \ldots \vec{v_k}) = k  \end{aligned} \]
 
\newpage

\begin{ex}
Show \( \left\{ \begin{bmatrix} 1 \\ 1 \\ 0 \end{bmatrix} \begin{bmatrix} 0 \\ 2 \\ 3 \end{bmatrix} \begin{bmatrix} 1 \\ 1 \\ 2 \end{bmatrix} \right\} \) is linear independent \\
\[\begin{bmatrix} 1 & 0 & 1 \\ 1 & 2 & 1 \\ 0 & 3 & 2\end{bmatrix} \rightarrow \begin{bmatrix} 1 & 0 & 1 \\ 0 & 2 & 1 \\ 0 & 3 & 2\end{bmatrix}\rightarrow \begin{bmatrix} 1 & 0 & 1 \\ 0 & 1 & 0 \\ 0 & 3 & 2\end{bmatrix}  \rightarrow \begin{bmatrix} 1 & 0 & 1 \\ 0 & 1 & 0 \\ 0 & 0 & 1\end{bmatrix} \]
\[ r \begin{bmatrix} 1 & 0 & 1 \\ 1 & 2 & 1 \\ 0 & 3 & 2\end{bmatrix}  = 3\]
Since there is 3 vectors and the rank is 3, the set is linear independent
\end{ex}

\begin{ex}
\( \left\{ \begin{bmatrix} 1 \\ 2 \\ 3 \end{bmatrix} \begin{bmatrix} -1 \\ 0 \\ 2 \end{bmatrix} \right\} \subset \mathbb{R}^3 \) is linear independent \\
\[ \begin{bmatrix} 1 & -1 \\ 2 & 0 \\ 3 & 2 \end{bmatrix} \rightarrow \begin{bmatrix} 1 & -1 \\ 0  & 2 \\ 0 & 5 \end{bmatrix} \rightarrow \begin{bmatrix} 1 & 0 \\ 0 & 1 \\ 0 & 0 \end{bmatrix} \implies Rank = 2  \implies \text{ the set is linear independent} \]
\end{ex}

\begin{ex}
Prove \( \left\{ \begin{bmatrix} 2 \\ -1 \\ 2 \end{bmatrix} \begin{bmatrix} -1 \\ 1 \\ 0 \end{bmatrix} \begin{bmatrix} 2 \\ 0 \\ 5 \end{bmatrix} \right\}  \) is a basis of \( \mathbb{R}^3 \) \\
\[ r\begin{bmatrix} 2 & -1 & 2 \\ -1 & 1 & 0 \\ 2 & 0 & 5 \end{bmatrix} = 3 \implies \text{The set is linear independent} \implies \text{The set spans }\mathbb{R}^3\]
\end{ex}



\begin{center}
\textbf{End of Lecture Notes} \\
\textbf{Notes By : Harsh Mistry}
\end{center}
\end{document}
