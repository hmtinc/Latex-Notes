%Notes by Harsh Mistry 
%Math 136 
%based on Template from : https://www.cs.cmu.edu/~ggordon/10725-F12/template.tex

\documentclass{article}
\setlength{\oddsidemargin}{0.25 in}
\setlength{\evensidemargin}{-0.25 in}
\setlength{\topmargin}{-0.6 in}
\setlength{\textwidth}{6.5 in}
\setlength{\textheight}{8.5 in}
\setlength{\headsep}{0.75 in}
\setlength{\parindent}{0 in}
\setlength{\parskip}{0.1 in}
\usepackage{amsfonts,graphicx, amssymb}
\usepackage[fleqn]{amsmath}
\newcounter{lecnum}
\renewcommand{\thepage}{\thelecnum-\arabic{page}}
\renewcommand{\thesection}{\thelecnum.\arabic{section}}
\renewcommand{\theequation}{\thelecnum.\arabic{equation}}
\renewcommand{\thefigure}{\thelecnum.\arabic{figure}}
\renewcommand{\thetable}{\thelecnum.\arabic{table}}
\newcommand{\lecture}[4]{
   \pagestyle{myheadings}
   \thispagestyle{plain}
   \newpage
   \setcounter{lecnum}{#1}
   \setcounter{page}{1}
   
   
%Info Box 
   \begin{center}
   \framebox{
      \vbox{\vspace{2mm}
    \hbox to 6.28in { {\bf Math 136 - Linear Algebra 
	\hfill Winter 2016} }
       \vspace{4mm}
       \hbox to 6.28in { {\Large \hfill Lecture #1: #2  \hfill} }
       \vspace{2mm}
       \hbox to 6.28in { {\it Lecturer: #3 \hfill Notes By: #4} }
      \vspace{2mm}}
   }
   \end{center}
   
   \markboth{Lecture #1: #2}{Lecture #1: #2}



 
}

\renewcommand{\cite}[1]{[#1]}
\def\beginrefs{\begin{list}%
        {[\arabic{equation}]}{\usecounter{equation}
         \setlength{\leftmargin}{2.0truecm}\setlength{\labelsep}{0.4truecm}%
         \setlength{\labelwidth}{1.6truecm}}}
\def\endrefs{\end{list}}
\def\bibentry#1{\item[\hbox{[#1]}]}

\newcommand{\fig}[3]{
			\vspace{#2}
			\begin{center}
			Figure \thelecnum.#1:~#3
			\end{center}
	}

\newtheorem{theorem}{Theorem}[lecnum]
\newtheorem{lemma}[theorem]{Lemma}
\newtheorem{ex}[theorem]{Example}
\newtheorem{proposition}[theorem]{Proposition}
\newtheorem{claim}[theorem]{Claim}
\newtheorem{corollary}[theorem]{Corollary}
\newtheorem{definition}[theorem]{Definition}
\newenvironment{proof}{{\bf Proof:}}{\hfill\rule{2mm}{2mm}}
\newcommand\E{\mathbb{E}}


%Start of Document 
\begin{document}

\lecture{14}{February 3, 2016}{Yongqiang Zhao}{Harsh Mistry}


\section{Linear Mappings}
\begin{definition}
If A is a \( m \times n \) matrix, then we can define the function \( f : \mathbb{R}^n \rightarrow \mathbb{R}^m \) as \( f (\vec{x} ) = A \vec{x} \)
 this is called a matrix mapping \end{definition}
 
 \textbf{Matrix Mapping Notation} : 
 \[f(\begin{bmatrix}x_1 \\ \vdots \\ x_n \end{bmatrix}) =  \begin{bmatrix}y_1 \\ \vdots \\ y_n \end{bmatrix} \]
 \[f(x_1 \ldots x_n ) = (y_1 \ldots y_n) \]
 
 
 \begin{theorem}
If A is a \( m \times n \) matrix and \( f : \mathbb{R}^n \rightarrow \mathbb{R}^m \)  is defined by \( f(\vec{x})  = A\vec{x} \) , then for all \( \vec{x}, \vec{y} \in \mathbb{R}^n and b, c \in \mathbb{R}\) we have :
\[ f (b\vec{x} + c\vec{y}) = bf(\vec{x}) + cf(\vec{y}) \] 
 \end{theorem}
 
 
 \begin{definition}
A function \(A  : \mathbb{R}^n  \rightarrow \mathbb{R}^m \text{ is said to be a } \textbf{ Linear mapping } is for every \vec{x} , \vec{y} \in \mathbb{R}^n \text{ and } b , c \in \mathbb{R} \) we have : 
\[ L (b\vec{x} + c\vec{y}) = bL(\vec{x}) + cL(\vec{y}) \] \end{definition}

\textbf{Remarks}
\begin{itemize}
\item If two linear mappings (L \& M) are said to be equal for all \( \vec{x} \in \mathbb{R}^n \), we write L = M
\item A linear mapping L : \( \mathbb{R}^n \rightarrow \mathbb{R}^n \) is called a linear operator
\end{itemize}

\begin{theorem}
Every linear mapping L : \( \mathbb{R}^n  \rightarrow \mathbb{R}^m  \) can be repersented as a matrix mapping with a matrix whose i-th column is the image of the i-th standard basis vector of \(\mathbb{R}^n \) under L for all \( 1 \leq i \leq n \). That is, \( L(\vec{x}) = \) [L]\(\vec{x}\) where, 
\[\begin{bmatrix}L\end{bmatrix} = \begin{bmatrix} L(\vec{e_1}) \ldots L(\vec{e_2})\end{bmatrix}\]
 \end{theorem}




\begin{center}
\textbf{End of Lecture Notes} \\
\textbf{Notes By : Harsh Mistry}
\end{center}
\end{document}
