%Notes by Harsh Mistry 
%Math 136 
%based on Template from : https://www.cs.cmu.edu/~ggordon/10725-F12/template.tex

\documentclass{article}
\setlength{\oddsidemargin}{0.25 in}
\setlength{\evensidemargin}{-0.25 in}
\setlength{\topmargin}{-0.6 in}
\setlength{\textwidth}{6.5 in}
\setlength{\textheight}{8.5 in}
\setlength{\headsep}{0.75 in}
\setlength{\parindent}{0 in}
\setlength{\parskip}{0.1 in}
\usepackage{amsfonts,graphicx, amssymb}
\usepackage[fleqn]{amsmath}
\newcounter{lecnum}
\renewcommand{\thepage}{\thelecnum-\arabic{page}}
\renewcommand{\thesection}{\thelecnum.\arabic{section}}
\renewcommand{\theequation}{\thelecnum.\arabic{equation}}
\renewcommand{\thefigure}{\thelecnum.\arabic{figure}}
\renewcommand{\thetable}{\thelecnum.\arabic{table}}
\newcommand{\lecture}[4]{
   \pagestyle{myheadings}
   \thispagestyle{plain}
   \newpage
   \setcounter{lecnum}{#1}
   \setcounter{page}{1}
   
   
%Info Box 
   \begin{center}
   \framebox{
      \vbox{\vspace{2mm}
    \hbox to 6.28in { {\bf Math 136 - Linear Algebra 
	\hfill Winter 2016} }
       \vspace{4mm}
       \hbox to 6.28in { {\Large \hfill Lecture #1: #2  \hfill} }
       \vspace{2mm}
       \hbox to 6.28in { {\it Lecturer: #3 \hfill Notes By: #4} }
      \vspace{2mm}}
   }
   \end{center}
   
   \markboth{Lecture #1: #2}{Lecture #1: #2}



 
}

\renewcommand{\cite}[1]{[#1]}
\def\beginrefs{\begin{list}%
        {[\arabic{equation}]}{\usecounter{equation}
         \setlength{\leftmargin}{2.0truecm}\setlength{\labelsep}{0.4truecm}%
         \setlength{\labelwidth}{1.6truecm}}}
\def\endrefs{\end{list}}
\def\bibentry#1{\item[\hbox{[#1]}]}

\newcommand{\fig}[3]{
			\vspace{#2}
			\begin{center}
			Figure \thelecnum.#1:~#3
			\end{center}
	}

\newtheorem{theorem}{Theorem}[lecnum]
\newtheorem{lemma}[theorem]{Lemma}
\newtheorem{ex}[theorem]{Example}
\newtheorem{proposition}[theorem]{Proposition}
\newtheorem{claim}[theorem]{Claim}
\newtheorem{corollary}[theorem]{Corollary}
\newtheorem{definition}[theorem]{Definition}
\newenvironment{proof}{{\bf Proof:}}{\hfill\rule{2mm}{2mm}}
\newcommand\E{\mathbb{E}}


%Start of Document 
\begin{document}

\lecture{16}{February 8, 2016}{Yongqiang Zhao}{Harsh Mistry}


\section{Linear Mapping Example}
\begin{ex} 
Write down the standard matrix for \(Proj_{\vec{a}}\) with \( \vec{a} = \begin{bmatrix} 1 \\ \vdots \\ 1 \end{bmatrix} \in \mathbb{R}^n \) \\
\(Proj_{\vec{a}} (\vec{e_n}) = \frac{\vec{e_n} \cdot \vec{a}}{\|\vec{a}\|^2}\vec{a} = \frac{1}{n} \begin{bmatrix} 1 \\ \vdots \\ 1 \end{bmatrix}\)
\end{ex}

\section{Rotation}
\begin{definition}
Let \( \mathbb{R}_{\theta} : \mathbb{R}^2 \rightarrow \mathbb{R}^2\) denote the function of the map that rortates a vector \( \vec{x} \in \mathbb{R}^2\) about the orgin counterclockwise through a angle \( \theta\), then we have 
\[ R_{\theta}(x_1 , x_2) = ( x_1 \cos \theta - x_2 \sin \theta, x_1\sin \theta + x_2 \cos \theta) \]
\end{definition}

\begin{ex} Let \( R(\vec{e_1}) = \begin{bmatrix} \cos \theta  \\ \sin \theta \end{bmatrix} \ \ \ \  R(\vec{e_2}) = \begin{bmatrix} -\sin \theta  \\ \cos \theta \end{bmatrix}\) then, 
\[ \begin{bmatrix} \mathbb{R}_{\theta}\end{bmatrix} = \begin{bmatrix} \cos \theta  & - \sin \theta  \\ \sin \theta  & \cos \theta\end{bmatrix}\]
\end{ex}
 
 
 \begin{theorem}
For \( \mathbb{R}_{\theta} : \mathbb{R}^2 \rightarrow \mathbb{R}^2 \ \forall \vec{x} , \vec{y} \in \mathbb{R}^2 \) we have. 
\begin{enumerate}
\item \( \|\mathbb{R}_{\theta} (\vec{x}) \| = \|\vec{x}\| \)
\item \( \mathbb{R}_{\theta} (\vec{x}) \cdot  \mathbb{R}_{\theta} (\vec{y}) = \vec{x} \cdot \vec{y} \)
\end{enumerate}
 \end{theorem}
 
 \section{Reflection}
 \begin{definition}
 Let \( Refl_p : \mathbb{R}^n \rightarrow \mathbb{R}^n\) denote the mapping that sends \(\vec{x} \) to its mirror image in hyperplane P with normal vector \(\vec{n}\) , then we define a reflection as :
 \[Refl_P (\vec{x}) = \vec{x} - 2Proj_{\vec{n}} (\vec{x}) \]
 \end{definition}

 \section{Special Subspaces} 
 
 \begin{definition}
 Let \( L  : \mathbb{R}^n \rightarrow \mathbb{R}^m \) be a linear mapping, then the range of L is defined by : 
 \[Range (L) = \{ L(\vec{x}) \mid \vec{x} \in \mathbb{R}^m \} \]
 \end{definition}
 
  \begin{theorem}
The range of a linear mapping L : \(\mathbb{R}^n \rightarrow \mathbb{R}^m \) is a subspace of \(\mathbb{R}^m\)
 \end{theorem}
 
 \begin{proof}
 \[\begin{aligned} & L(\vec{0}) = \vec{0} \in Range(L) , \text{ non- empty } \\ & \forall L(\vec{x}) , L(\vec{y}) \in Range(L) 
  \ \ \ \ L(\vec{x}) + L(\vec{y}) = L(\vec{x} + \vec{x}) \in Range (L) \ \forall L(\vec{x}) \in Range(L) \\
  & \forall L(\vec{x}) \in Range (L), \forall c \in \mathbb{R} \ \ \ cL(\vec{x}) = L(c\vec{x}) \in Range (L) \\
  & \implies Range(L) \text{ is a subspace of } \mathbb{R}^m\end{aligned} \]
 \end{proof}

\begin{center}
\textbf{End of Lecture Notes} \\
\textbf{Notes By : Harsh Mistry}
\end{center}
\end{document}
