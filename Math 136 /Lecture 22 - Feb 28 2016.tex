%Notes by Harsh Mistry 
%Math 136 
%based on Template from : https://www.cs.cmu.edu/~ggordon/10725-F12/template.tex

\documentclass{article}
\setlength{\oddsidemargin}{0.25 in}
\setlength{\evensidemargin}{-0.25 in}
\setlength{\topmargin}{-0.6 in}
\setlength{\textwidth}{6.5 in}
\setlength{\textheight}{8.5 in}
\setlength{\headsep}{0.75 in}
\setlength{\parindent}{0 in}
\setlength{\parskip}{0.1 in}
\usepackage{amsfonts,graphicx, amssymb}
\usepackage[fleqn]{amsmath}
\usepackage{fixltx2e}
\newcounter{lecnum}
\renewcommand{\thepage}{\thelecnum-\arabic{page}}
\renewcommand{\thesection}{\thelecnum.\arabic{section}}
\renewcommand{\theequation}{\thelecnum.\arabic{equation}}
\renewcommand{\thefigure}{\thelecnum.\arabic{figure}}
\renewcommand{\thetable}{\thelecnum.\arabic{table}}
\newcommand{\lecture}[4]{
   \pagestyle{myheadings}
   \thispagestyle{plain}
   \newpage
   \setcounter{lecnum}{#1}
   \setcounter{page}{1}
   
   
%Info Box 
   \begin{center}
   \framebox{
      \vbox{\vspace{2mm}
    \hbox to 6.28in { {\bf Math 136 - Linear Algebra 
	\hfill Winter 2016} }
       \vspace{4mm}
       \hbox to 6.28in { {\Large \hfill Lecture #1: #2  \hfill} }
       \vspace{2mm}
       \hbox to 6.28in { {\it Lecturer: #3 \hfill Notes By: #4} }
      \vspace{2mm}}
   }
   \end{center}
   
   \markboth{Lecture #1: #2}{Lecture #1: #2}



 
}

\renewcommand{\cite}[1]{[#1]}
\def\beginrefs{\begin{list}%
        {[\arabic{equation}]}{\usecounter{equation}
         \setlength{\leftmargin}{2.0truecm}\setlength{\labelsep}{0.4truecm}%
         \setlength{\labelwidth}{1.6truecm}}}
\def\endrefs{\end{list}}
\def\bibentry#1{\item[\hbox{[#1]}]}

\newcommand{\fig}[3]{
			\vspace{#2}
			\begin{center}
			Figure \thelecnum.#1:~#3
			\end{center}
	}

\newtheorem{theorem}{Theorem}[lecnum]
\newtheorem{lemma}[theorem]{Lemma}
\newtheorem{ex}[theorem]{Example}
\newtheorem{proposition}[theorem]{Proposition}
\newtheorem{claim}[theorem]{Claim}
\newtheorem{corollary}[theorem]{Corollary}
\newtheorem{definition}[theorem]{Definition}
\newenvironment{proof}{{\bf Proof:}}{\hfill\rule{2mm}{2mm}}
\newcommand\E{\mathbb{E}}


%Start of Document 
\begin{document}

\lecture{22}{February 28, 2016}{Yongqiang Zhao}{Harsh Mistry}

\section{Dimensions Continued}

\begin{theorem}
if  \(\mathbb{V}\) is an n-dimensional vector space and \( \{\vec{v_1} \ldots \vec{v_k}\}\) is a linearly independant set in  \(\mathbb{V}\) with k < n, then there exist vectors \( \vec{w_{k+1}} \ldots \vec{w_n } \) in  \(\mathbb{V}\) such that 
\(\{\vec{v_1}, \ldots , \vec{v_k}, \vec{w_{k+1}}, \ldots, \vec{w_n}\}\) is a basis for  \(\mathbb{V}\)
\end{theorem}

\begin{corollary}
if \(\mathbb{S}\) is a subspace of a finite dimensional  vector space \(\mathbb{V}\), then \(dim \mathbb{S} \neq dim \mathbb{V}\)
\end{corollary}

\section{Coordinates with respect to a basis}

\begin{theorem}
If \(\beta = \{\vec{v_1} \ldots \vec{v_n}\}\) is a basis for a vector space  \(\mathbb{V}\), then every \(\vec{v} \in \mathbb{V}
\) can be written as a unique linear combination of the vectors in \(\beta\)
\end{theorem}

\begin{definition}
If \(\beta = \{ \vec{v_1} \ldots \vec{v_n} \} \) is a basis for a vector space \(\mathbb{V}\)\\
if \(\vec{v} = b_1 \vec{v_1} + \ldots + b_n \vec{v_n}\), then \(b_1 \ldots b_1\) are called \(\beta\)-coordinates of \(\vec{v}\), and we define the \(\beta\)-coordinate vector by 
$$ \mid \vec{v}\mid _\beta = \begin{bmatrix} b_1 \\ \vdots \\ b_n\end{bmatrix} $$
\end{definition}


\begin{center}
\textbf{End of Lecture Notes}\\
\textbf{Notes by : Harsh Mistry}
\end{center}
\end{document}
