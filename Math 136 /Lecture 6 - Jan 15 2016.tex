%Notes by Harsh Mistry 
%Math 136 
%Template Taken from : https://www.cs.cmu.edu/~ggordon/10725-F12/template.tex

\documentclass{article}
\setlength{\oddsidemargin}{0.25 in}
\setlength{\evensidemargin}{-0.25 in}
\setlength{\topmargin}{-0.6 in}
\setlength{\textwidth}{6.5 in}
\setlength{\textheight}{8.5 in}
\setlength{\headsep}{0.75 in}
\setlength{\parindent}{0 in}
\setlength{\parskip}{0.1 in}
\usepackage{amsfonts,graphicx, amssymb}
\usepackage[fleqn]{amsmath}
\newcounter{lecnum}
\renewcommand{\thepage}{\thelecnum-\arabic{page}}
\renewcommand{\thesection}{\thelecnum.\arabic{section}}
\renewcommand{\theequation}{\thelecnum.\arabic{equation}}
\renewcommand{\thefigure}{\thelecnum.\arabic{figure}}
\renewcommand{\thetable}{\thelecnum.\arabic{table}}
\newcommand{\lecture}[4]{
   \pagestyle{myheadings}
   \thispagestyle{plain}
   \newpage
   \setcounter{lecnum}{#1}
   \setcounter{page}{1}
   
   
%Info Box 
   \begin{center}
   \framebox{
      \vbox{\vspace{2mm}
    \hbox to 6.28in { {\bf Math 136 - Linear Algebra 
	\hfill Winter 2016} }
       \vspace{4mm}
       \hbox to 6.28in { {\Large \hfill Lecture #1: #2  \hfill} }
       \vspace{2mm}
       \hbox to 6.28in { {\it Lecturer: #3 \hfill Notes By: #4} }
      \vspace{2mm}}
   }
   \end{center}
   
   \markboth{Lecture #1: #2}{Lecture #1: #2}



 
}

\renewcommand{\cite}[1]{[#1]}
\def\beginrefs{\begin{list}%
        {[\arabic{equation}]}{\usecounter{equation}
         \setlength{\leftmargin}{2.0truecm}\setlength{\labelsep}{0.4truecm}%
         \setlength{\labelwidth}{1.6truecm}}}
\def\endrefs{\end{list}}
\def\bibentry#1{\item[\hbox{[#1]}]}

\newcommand{\fig}[3]{
			\vspace{#2}
			\begin{center}
			Figure \thelecnum.#1:~#3
			\end{center}
	}

\newtheorem{theorem}{Theorem}[lecnum]
\newtheorem{lemma}[theorem]{Lemma}
\newtheorem{ex}[theorem]{Example}
\newtheorem{proposition}[theorem]{Proposition}
\newtheorem{claim}[theorem]{Claim}
\newtheorem{corollary}[theorem]{Corollary}
\newtheorem{definition}[theorem]{Definition}
\newenvironment{proof}{{\bf Proof:}}{\hfill\rule{2mm}{2mm}}
\newcommand\E{\mathbb{E}}


%Start of Document 
\begin{document}

\lecture{6}{January 15, 2016}{Yongqiang Zhao}{Harsh Mistry}


\section{Dot Product Continued}
\begin{definition} 
- For any two vectors \( \vec{x} \vec{y}\) , we define the angle between \( \vec{x} \vec{y}\) to be  \( \cos \theta  = \frac{\vec{x} \cdot \vec{y}}{\| \vec{x}\| \| \vec{y}\|} \)\\
\begin{itemize}
\item  If  \(\vec{x} \cdot \vec{y} = 0 \) , we say \(\vec{x} \text{ and } \vec{y} \) are orthogonal
\item Any any two vectors of , \( \{e_1, e_2 \ldots e_n\} \) standard basis for \( \mathbb{R} ^n \) , are orthogonal
\end{itemize}
\end {definition}

\section{Cross product}
Given two vectors \( \vec{x} \text{ and }  \vec{y} \text{ in } \mathbb{R}^3 \), find a third vectors which is orthongonal to \( \vec{x} + \vec{y} \) 

\begin{definition} 
- Given \(\vec{x} \text{ and } \vec{y} \ \in \mathbb{R}^3 , \vec{x} \times \vec{y} = \begin{bmatrix}
x_2y_3 - x_3y_2 \\ x_3y_1 - x_1y_3 \\ x_1y_1 - x_2 y_1 \end{bmatrix} \) \\
\textbf{geometrically} : \(\vec{x} \times \vec{y} = \|\vec{x} \| \| \vec{y} \| \sin \theta \)
\end {definition}

\begin{proposition} Suppose that \(\vec{v} , \vec{w} , \vec{x} \in \mathbb{R}^3 \) and \( c \in \mathbb{R} \) \\
\begin{enumerate}
\item If \( \vec{n} = \vec{v} \times \vec{w} \) then for any \( \vec{y} \in Span\{\vec{v} , \vec{w} \} \) we have \( \vec{y} \cdot \vec{n} = 0 \)
\item \(\vec{v} \times \vec{w} = -\vec{w} \times \vec{v}\)
\item \(\vec{v} \times \vec{v} = 0 \)
\item \(\vec{v} \times \vec{w} = \vec{0} \) iff  either \( \vec{v} = \vec{0} \text{ or } \vec{w} \) is a scaler multiple of \( \vec{w} \)
\item \(\vec{v} \times (\vec{x} + \vec{w}) = \vec{v} \times \vec{w} + \vec{v} \times \vec{x} \)
\item \( (c\vec{v}) \times \vec{w} = c(\vec{x} \times \vec{w}) \)
\end{enumerate}
\end{proposition}

\begin{ex} 
Let \( \vec{x} = \begin{bmatrix} 1 \\ 1 \\ 1\end{bmatrix}  \vec{z} = \begin{bmatrix} 1 \\ 2 \\ 3\end{bmatrix} \) \\
\( \vec{x} \times \vec{z} = \begin{bmatrix} 1 \\ 1 \\ 1 \end{bmatrix} \times \begin{bmatrix} 1 \\ 2 \\ 3 \end{bmatrix}  = \begin{bmatrix} 3-2 \\ 1-3 \\ 2-1 \end{bmatrix} = \begin{bmatrix} 1 \\ -2 \\ 1 \end{bmatrix}   \ \therefore \vec{x} \) is not orthogonal 
\end{ex}



\section{Scaler Equations}
\textbf{Recall : } A plane in \( \mathbb{R}^3 \) is given by \( \vec{x} = s\vec{v} + t\vec{w} + \vec{b}\) and \(\{\vec{v} , \vec{w} \}\) is linear independant

\begin{definition} 
- Given \(\vec{n} = \vec{v} \times \vec{w} \) , the scaler equation for a plan is \( (\vec{x} - \vec{b}) \cdot \vec{n} = 0 \)
\end {definition}

\begin{ex} - Finding a scaler equation \\
\[ \begin{aligned} \vec{x} & = s \begin{bmatrix} 1 \\ 0 \\ 1\end{bmatrix} + t\begin{bmatrix} 0 \\ 1 \\ 0\end{bmatrix} + \begin{bmatrix} 1 \\ 2 \\ 3\end{bmatrix} \\
\vec{n} & = \begin{bmatrix} 1 \\ 0 \\ 1 \end{bmatrix} \times \begin{bmatrix} 0 \\ 1 \\ 0\end{bmatrix} = \begin{bmatrix} -1 \\ 0 \\ 1\end{bmatrix} \\
0 & = ( \vec{x} - \begin{bmatrix} 1 \\ 2 \\ 3 \end{bmatrix}) \cdot \vec{n}  \\
\vec{x} \cdot \vec{n}  & = \begin{bmatrix} 1 \\ 2 \\ 3 \end{bmatrix} \vec{n} \\
\begin{bmatrix} x_1 \\ x_2 \\ x_3 \end{bmatrix} \begin{bmatrix} -1 \\ 0 \\ 1\end{bmatrix} & =  \begin{bmatrix} -1 \\ 0 \\ 1\end{bmatrix} \begin{bmatrix} 1 \\ 2 \\ 3 \end{bmatrix} \implies (x_3 - x_1 = z) \end{aligned}\]
\end{ex}



\begin{center}
\textbf{End of Lecture Notes} \\
\textbf{Notes By : Harsh Mistry}
\end{center}
\end{document}
