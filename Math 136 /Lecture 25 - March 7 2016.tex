%Notes by Harsh Mistry 
%Math 136 
%based on Template from : https://www.cs.cmu.edu/~ggordon/10725-F12/template.tex

\documentclass{article}
\setlength{\oddsidemargin}{0.25 in}
\setlength{\evensidemargin}{-0.25 in}
\setlength{\topmargin}{-0.6 in}
\setlength{\textwidth}{6.5 in}
\setlength{\textheight}{8.5 in}
\setlength{\headsep}{0.75 in}
\setlength{\parindent}{0 in}
\setlength{\parskip}{0.1 in}
\usepackage{amsfonts,graphicx, amssymb}
\usepackage[fleqn]{amsmath}
\usepackage{fixltx2e}
\newcounter{lecnum}
\renewcommand{\thepage}{\thelecnum-\arabic{page}}
\renewcommand{\thesection}{\thelecnum.\arabic{section}}
\renewcommand{\theequation}{\thelecnum.\arabic{equation}}
\renewcommand{\thefigure}{\thelecnum.\arabic{figure}}
\renewcommand{\thetable}{\thelecnum.\arabic{table}}
\newcommand{\lecture}[4]{
   \pagestyle{myheadings}
   \thispagestyle{plain}
   \newpage
   \setcounter{lecnum}{#1}
   \setcounter{page}{1}
   
   
%Info Box 
   \begin{center}
   \framebox{
      \vbox{\vspace{2mm}
    \hbox to 6.28in { {\bf Math 136 - Linear Algebra 
	\hfill Winter 2016} }
       \vspace{4mm}
       \hbox to 6.28in { {\Large \hfill Lecture #1: #2  \hfill} }
       \vspace{2mm}
       \hbox to 6.28in { {\it Lecturer: #3 \hfill Notes By: #4} }
      \vspace{2mm}}
   }
   \end{center}
   
   \markboth{Lecture #1: #2}{Lecture #1: #2}



 
}

\renewcommand{\cite}[1]{[#1]}
\def\beginrefs{\begin{list}%
        {[\arabic{equation}]}{\usecounter{equation}
         \setlength{\leftmargin}{2.0truecm}\setlength{\labelsep}{0.4truecm}%
         \setlength{\labelwidth}{1.6truecm}}}
\def\endrefs{\end{list}}
\def\bibentry#1{\item[\hbox{[#1]}]}

\newcommand{\fig}[3]{
			\vspace{#2}
			\begin{center}
			Figure \thelecnum.#1:~#3
			\end{center}
	}

\newtheorem{theorem}{Theorem}[lecnum]
\newtheorem{lemma}[theorem]{Lemma}
\newtheorem{ex}[theorem]{Example}
\newtheorem{proposition}[theorem]{Proposition}
\newtheorem{claim}[theorem]{Claim}
\newtheorem{corollary}[theorem]{Corollary}
\newtheorem{definition}[theorem]{Definition}
\newenvironment{proof}{{\bf Proof:}}{\hfill\rule{2mm}{2mm}}
\newcommand\E{\mathbb{E}}


%Start of Document 
\begin{document}

\lecture{25}{March 7, 2016}{Yongqiang Zhao}{Harsh Mistry}

\section{Matrix Inverse Theorems}

\begin{theorem}
If \(A\) and \(B\) are invertible matrices and \(c \in \mathbb{R}\) with \(c \neq 0\), then 
\begin{enumerate}
\item \((cA)^{-1} = \frac{1}{c}A^{-1}\)
\item \((A^T)^{-1} = (A^{-1})^T\)
\item \((AB)^{-1} = B^{-1}A^{-1}\)
\end{enumerate}
\end{theorem}

\begin{theorem}
If \(A\) is a \(n \times m \) matrix such that RankA = n, then A is invertible 
\end{theorem}

\begin{theorem}
\textbf{Invertible Matrix Theorem}\\
For any \(n \times m\) matrix \(A\), the following are equivalent:
\begin{enumerate}
\item A is invertible 
\item The RREF of A is I
\item rankA = n 
\item The system of equation \(A\vec{x} = \vec{b}\) is consistent with a unique solution for all \(\vec{b} \in \mathbb{R}^n\)
\item The nullspace of A is \(\{0\}\) 
\item The columns of A form a basis for \(\mathbb{R}^n\)
\item The rows of A form a basis for \(\mathbb{R}^n\)
\item \(A^T\) is invertible 
\end{enumerate}
\end{theorem}


\textbf{Note : } \\
A is invertible  \(\iff \) The system of equation \(A\vec{x} = \vec{b}\) is consistent with a unique solution for all \(\vec{b} \in \mathbb{R}^n\)
$$\begin{aligned} A\vec{x} & = \vec{b} \\ A^{-1}(A\vec{x}) & = A^{-1}\vec{b} \\
(A^{-1} A) \vec{x} & = A^{-1}\vec{b} \\ \vec{x} & = A^{-1} \vec{b}  \end{aligned}$$

\begin{center}
\textbf{End of Lecture Notes}\\
\textbf{Notes by : Harsh Mistry}
\end{center}
\end{document}
