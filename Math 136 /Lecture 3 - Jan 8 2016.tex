%Notes by Harsh Mistry 
%Math 136 
%Template Taken from : https://www.cs.cmu.edu/~ggordon/10725-F12/template.tex

\documentclass{article}
\setlength{\oddsidemargin}{0.25 in}
\setlength{\evensidemargin}{-0.25 in}
\setlength{\topmargin}{-0.6 in}
\setlength{\textwidth}{6.5 in}
\setlength{\textheight}{8.5 in}
\setlength{\headsep}{0.75 in}
\setlength{\parindent}{0 in}
\setlength{\parskip}{0.1 in}
\usepackage{amsfonts,graphicx, amssymb}
\usepackage[fleqn]{amsmath}
\newcounter{lecnum}
\renewcommand{\thepage}{\thelecnum-\arabic{page}}
\renewcommand{\thesection}{\thelecnum.\arabic{section}}
\renewcommand{\theequation}{\thelecnum.\arabic{equation}}
\renewcommand{\thefigure}{\thelecnum.\arabic{figure}}
\renewcommand{\thetable}{\thelecnum.\arabic{table}}
\newcommand{\lecture}[4]{
   \pagestyle{myheadings}
   \thispagestyle{plain}
   \newpage
   \setcounter{lecnum}{#1}
   \setcounter{page}{1}
   
   
%Info Box 
   \begin{center}
   \framebox{
      \vbox{\vspace{2mm}
    \hbox to 6.28in { {\bf Math 136 - Linear Algebra 
	\hfill Winter 2016} }
       \vspace{4mm}
       \hbox to 6.28in { {\Large \hfill Lecture #1: #2  \hfill} }
       \vspace{2mm}
       \hbox to 6.28in { {\it Lecturer: #3 \hfill Notes By: #4} }
      \vspace{2mm}}
   }
   \end{center}
   
   \markboth{Lecture #1: #2}{Lecture #1: #2}



 
}

\renewcommand{\cite}[1]{[#1]}
\def\beginrefs{\begin{list}%
        {[\arabic{equation}]}{\usecounter{equation}
         \setlength{\leftmargin}{2.0truecm}\setlength{\labelsep}{0.4truecm}%
         \setlength{\labelwidth}{1.6truecm}}}
\def\endrefs{\end{list}}
\def\bibentry#1{\item[\hbox{[#1]}]}

\newcommand{\fig}[3]{
			\vspace{#2}
			\begin{center}
			Figure \thelecnum.#1:~#3
			\end{center}
	}

\newtheorem{theorem}{Theorem}[lecnum]
\newtheorem{lemma}[theorem]{Lemma}
\newtheorem{ex}[theorem]{Example}
\newtheorem{proposition}[theorem]{Proposition}
\newtheorem{claim}[theorem]{Claim}
\newtheorem{corollary}[theorem]{Corollary}
\newtheorem{definition}[theorem]{Definition}
\newenvironment{proof}{{\bf Proof:}}{\hfill\rule{2mm}{2mm}}
\newcommand\E{\mathbb{E}}


%Start of Document 
\begin{document}

\lecture{3}{January 8, 2016}{Yongqiang Zhao}{Harsh Mistry}

%drop box info 
\begin{center}
\textbf{Dropbox Info} \\
Drop Box 1\\
Slot \#1 A - F  \\
Slot \#2 G - L \\
Slot \#3 M - R \\
Slot \#4 S - Z
\end{center}

%
\section{Geometry of Spanning}

\begin{ex}
\[Span\left\{\begin{bmatrix} 1 \\ 1 \end{bmatrix} \right\} \subseteq \mathbb{R}^2 \]
By Definition, 
\[\vec{x}\in Span\left\{\begin{bmatrix} 1 \\ 1 \end{bmatrix} \right\} \text{ , } \vec{x} = t \ \begin{bmatrix} 1 \\ 1 \end{bmatrix} \text{ , } t \in \mathbb{R} \]
\[\therefore \ \ \ Span\left\{\begin{bmatrix} 1 \\ 1 \end{bmatrix} \right\} 
\text{ is a line with the direction vector } \begin{bmatrix} 1 \\ 1 \end{bmatrix} \]
\end{ex}

\begin{ex}
\[S = Span \left\{ \begin{bmatrix} 1 \\ 0 \\ 0 \end{bmatrix} \begin{bmatrix} 0 \\ 1 \\ 0 \end{bmatrix} \right\} \]
No vectors in the set are scaler multiples, therfore the spanning setting can not be further simplified
\[ \vec{x} = x\begin{bmatrix} 1 \\ 0 \\ 0 \end{bmatrix} + y\begin{bmatrix} 0 \\ 1 \\ 0 \end{bmatrix} \]
\[ \therefore \ \ \ S \text{ is the xy plane in  } \mathbb{R}^3 \] 
\end{ex}

\section{Simplify Spanning Sets}
\begin{theorem}
\[ \text{If } \vec{v}_{k+1} \text{ is a linear combination of } \{\vec{v}_1\ldots \vec{v}_k\} \text{  then, } 
Span\{\vec{v}_1\ldots \vec{v}_{k+1} \}  = \{\vec{v}_1\ldots \vec{v}_k\} \]
\end{theorem}

\newpage

\begin{ex}
\textbf{Simplify}\\


%split into 2 mini pages 
\begin{minipage}{.5\textwidth} %
\textbf{a)}
\[\begin{aligned} S & = Span\left\{ \begin{bmatrix} 1 \\ 0 \\ 1 \end{bmatrix} \begin{bmatrix} 2 \\ 0 \\ 2 \end{bmatrix} \begin{bmatrix} 0 \\ 0 \\ 0 \end{bmatrix} \right\} \\ 
& =  Span\left\{ \begin{bmatrix} 1 \\ 0 \\ 1 \end{bmatrix} \begin{bmatrix} 2 \\ 0 \\ 2 \end{bmatrix} \right\} \\ 
& = x \begin{bmatrix} 1 \\ 0 \\ 1 \end{bmatrix} + 2y \begin{bmatrix} 1 \\ 0 \\ 1 \end{bmatrix} \\
& = (x + 2y)\begin{bmatrix} 1 \\ 0 \\ 1 \end{bmatrix} \text{ , Let c = x + 2y} \\
& = c\begin{bmatrix} 1 \\ 0 \\ 1 \end{bmatrix} \end{aligned} \]
\end{minipage} %
\begin{minipage}{.5\textwidth} %
\textbf{b)}
\[ \begin{aligned} S & = Span\left\{ \begin{bmatrix} 1 \\ 0 \\ 1 \end{bmatrix} \begin{bmatrix} 2 \\ 1 \\  0\end{bmatrix} \begin{bmatrix} -1 \\ -1 \\ 1 \end{bmatrix} \right\} \\ 
& = Span\left\{ \begin{bmatrix} 1 \\ 0 \\ 1 \end{bmatrix} \begin{bmatrix} 2 \\ 1 \\  0\end{bmatrix} \right\} \end{aligned} \]
\[ \text{ note : } \begin{bmatrix} 1 \\ 0 \\ 1 \end{bmatrix} -  \begin{bmatrix} 2 \\ 1 \\  0\end{bmatrix} = \begin{bmatrix} -1 \\ -1 \\ 1 \end{bmatrix} \]
\end{minipage}
\end{ex}

\section{Linear Independance}
\begin{definition}
\[ \{ \vec{v_1} \ldots \vec{v_k} \} \subseteq \mathbb{R} \text{ is called linear dependant if } \exists \ n \text{ (non-zero) solution to }  t_1 v_1 + \ldots + t_k v_k  = \vec{0}\]
\[ \text {If the only solution to }  t_1 v_1 + \ldots + t_k v_k  = \vec{0}  
\text{ is } t_1 = t_2  = t_k \text{, we say it is linearly independent} \]
\end {definition}

\begin{theorem}
\[ \{\vec{v_1} \ldots \vec{v_k} \} \text{ is linearly dependant iff }  \ \exists \ c_i \ \text{ , } 1 \leq i \leq  k  \text{ such that, } \]
\[ v_1 \in \ Span \{ \vec{v_1} \ldots \vec{v_{i-1} }, \vec{v_{i+1}} \ldots \vec{v_k} \} \]
\end{theorem}

\textbf{Idea : }
\[ \vec{v} = t\vec{v_1} + t_{i-1} \vec{v_{i -1}} + t_{i+1} \vec{v_{i+1}} + \ldots + t_k \vec{v_k} \]

\begin{corollary}
\[ \text{ If } \{\vec{v_1} \ldots \vec{k} \} \text{ contains the 0 vector, then it is linearly dependant} \] 
\end{corollary}

\newpage
\

\begin{ex}
\textbf{Determine whether the following  sets are linear dependant or independant}

\begin{minipage}{.5\textwidth} %
\textbf{a)}
\[\left\{ \begin{bmatrix} 1 \\ 0 \\ 1\end{bmatrix} \begin{bmatrix} 0 \\ 3 \\ 2 \end{bmatrix} \begin{bmatrix} 0 \\ -1 \\ 2\end{bmatrix} \right\}\]
\[ \begin{cases} t_1 = 0 \\ 3t_2 - t_3 = 0 \\ t_1 + 2t_2 + 2t_3 = 0 \end{cases} \]
\[t_1 = t_2 = t_3 = 0 \]
\[\therefore \text{set is linear independant} \]
\end{minipage} %
\begin{minipage}{.5\textwidth} %
\textbf{b)}
\[\left\{ \begin{bmatrix} 1 \\ 0 \\ -1\end{bmatrix} \begin{bmatrix} -1 \\ 1 \\ 2 \end{bmatrix} \begin{bmatrix} 1 \\ 1 \\ 0\end{bmatrix} \right\}\]
\[=  \begin{cases} t_1 - t_2 + t_3 = 0 \text{ (1) } \\ t_2 + t_3 = 0 \text{ (2) } \\ -t_1 + 2t_2 = 0 \text{ (3) } \end{cases}  \text{ (2) - (1) = (3) }\]
\[= \begin{cases} t_1 - t_2 + t_3 = 0 \\ t_2 + t_3 = 0 \end{cases} \]
\[ t_3 = -1 \text{ , } t_1 = 2 \text{ , } t_2 = 1\]
\[ 2 \begin{bmatrix} 1 \\ 0 \\ -1\end{bmatrix} +  \begin{bmatrix} -1 \\ 1 \\ 2 \end{bmatrix} - \begin{bmatrix} 1 \\ 1 \\ 0\end{bmatrix}  = \vec{0} \implies \text{ Linear dependant}  \]
\end{minipage}
\end{ex}

\begin{definition}
\[ \text{If } S = Span \{ \vec{v_1} \ldots \vec{v_k } \} \text{ and }  \{ \vec{v_1} \ldots \vec{v_k } \} \text{ are linearly independant} \]
\[ \text{Then, }  \{ \vec{v_1} \ldots \vec{v_k } \} \text{ is called a basis of S } \]
\end {definition}

\textbf{Practice :  Prove}
\[ \left\{ \begin{bmatrix} 1 \\ 0 \end{bmatrix} \begin{bmatrix} 0 \\ 1 \end{bmatrix} \right\} \text{ is a basis of } \mathbb{R}^2  \]

Check : 
\begin{enumerate}
  \item \[ \mathbb{R}^2 = span \left\{ \begin{bmatrix} 1 \\ 0 \end{bmatrix} \begin{bmatrix} 0 \\ 1 \end{bmatrix} \right\} \]
  \item Set is linear independant 
\end{enumerate}




\begin{center}
\textbf{End of Lecture Notes} \\
\textbf{Notes By : Harsh Mistry}
\end{center}
\end{document}
